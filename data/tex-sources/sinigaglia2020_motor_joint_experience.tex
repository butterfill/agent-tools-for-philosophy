% What Do We Experience of Actions When We Act Together with a Purpose?

% OR

% Motor Representation and Action Experience in Joint Action

% Sinigaglia \& Butterfill


% For Anika Fiebich (ed), Minimal Cooperation and Shared Agency


\maketitle

\begin{abstract}
\noindent
Acting together with a purpose is a familiar feature of everyday life. We jump
together, play music together and move tables together. But what do we
experience of action in acting together? It is perhaps tempting to suppose that
there is a special way in which we can experience our own actions, and that we
cannot experience the actions of others in this way. This view would imply that
in acting together, our own actions are experienced in a way that our partners’
actions are not. However recent research on motor representation suggests that,
in observing another act, it may be possible to experience her actions in
whatever sense we can experience our own actions. This makes it at least
conceivable that in acting together we can experience the actions each of us
performs in the same way. But the occurrence of a joint action involves more
than merely the occurrences of two individual actions. Are there experiences of
joint actions which involve more than merely two or more experiences of
individual actions? In this chapter we defend a positive answer. In some cases,
experiences associated with joint action are experiences of action in whatever
sense experiences of acting alone are.

\end{abstract}

%FINAL ABSTRACT (<201 words)

% TEXT (‘ there won't be a word limit for the chapters but to keep all the contributions at a similar length, taking 8000 words (including references) may be a good guideline’)





\section{Introduction}
\label{sec:introduction}
% [Plan #1 : introduce acting together]
Many of the things we do are done together with others.
We play duets, move pianos together and drink toasts together.
We also fill rooms with noise together, damage furniture together and spill drinks together.
As these examples hint, acting together is sometimes but not always done with a purpose.
Filling a room with noise is something that we typically do together (neither you nor I alone is speaking quite loudly enough to fill the room with noise), but it is not usually something done with a purpose.
By contrast, playing a piano duet and drinking toasts are paradigm cases of acting together with a purpose.

What is it to act together with a purpose?  Minimally, it seems necessary that there should be a single outcome to which the actions we perform in acting together with a purpose are directed. Further, our actions being directed to this outcome should not be, or should not only be, a matter of each of our actions being directed to that outcome. After all, acting together with a purpose is not merely a matter of me doing one thing and you doing another.

Of course philosophers have offered various more elaborate ways to characterise forms of acting together with a purpose \citep[e.g.][]{bratman:2014_book,gilbert:2014_book}. It is possible that our characterisation needs to be elaborated along some such lines. But for our purposes the above minimal characterisation of acting together with a purpose will be sufficient.

For the purposes of this chapter, a \emph{joint action} is an event involving two or more agents acting together with a purpose; similarly, an \emph{individual action} is an event involving a single agent acting alone with a purpose. Elsewhere the term ‘joint action’ has been defined in a variety of ways, some broader, others narrower \citep[e.g.][]{Sebanz:2006yq,ludwig_collective_2007}. Our definition can be treated as a terminological stipulation. Its purpose is only to fix the range of cases of interest for our questions about experience.

We face an immediate difficulty in talking about experiences associated with (joint) action.
What are the contents of such experiences?
Or, if experiences lack contents (we rely on no assumption about this in what follows), what do such experiences present to their subjects?
Do they present actions in whatever sense some visual experiences present colours and shapes?
Or are experiences associated with action phenomenologically no different from experiences associated with events that are not actions?
As things stand, we know of no way to answer such questions (see \citealp{sinigaglia:2015_goal_ascription} for some difficulties involved in deciding between the two views).
But we do know a way to duck these questions.
Experiences of action enable their subjects to make judgements about which action is being observed or performed---for example, about whether the goal of the action is to pull or push an object \citep{cattaneo:2010_ones}.
This suggests that at least some experiences associated with actions provide their subjects with reasons for judgements about those actions.
Let us stipulate that for an experience to \emph{reveal} an action is for it to provide the subject with a reason for judging that this action is being performed, and, further, that  an \emph{experience of action} is one that reveals an action.
The existence of experiences of action (in this sense) is both relatively uncontroversial and neutral on different views about the contents of experiences associated with actions.
Despite this, it is far from uninteresting for it raises deep and difficult issues.
In this chapter, our concern is with a question about experiences of joint actions.
The occurrence of a joint action involves more than merely the occurrences of two individual actions.
But what about experiences?
Are there experiences of joint actions which involve more than merely two or more experiences of individual actions?

To answer this question, we first need to consider some background issues concerning experiences of individual action.
Our plan is to consider how comparatively well established views about experiences of individual action might generalise to joint action.



\section{A Preliminary Distinction}
% Plan #3
As a preliminary, let us distinguish our question about experiences of actions from questions about experiences of acting. Experiences of action reveal what is being done whereas experiences of acting reveal who is acting.%
\footnote{%
\label{fn:reveal}
As mentioned in the Introduction, we stipulate that for an experience to \emph{reveal} an action is for it to provide the subject with a reason for judging that this action is being performed.
}

A body of research focuses on experiences which enable people to identify whether or not they are effective as agents of events. Here it is common to talk about a sense of agency, which is sometimes defined as the ‘experience of controlling one’s own actions’  \citep[p.~390]{haggard:2012_sense}. \citet{pacherie:2013feel} connects research on the sense of agency to joint action. However, our concern is different. The sense of agency is associated with experiences of acting, whereas our concern is with experience of action.

To see that this distinction matters, consider a patient with anarchic hand syndrome whose two hands appear to be fighting against each other (that is, there is intermanual conflict). Imagine that one is fastening buttons on her shirt while the other undoes them.%
\footnote{%
\citet[p.~197]{fisher:2000_alien} cites this case from \citep{bogen1993callosal}.
}
There is clearly a disruption in her sense of agency: she has a sense of being in control of some actions but not others.
But both actions are clearly purposive, and she can readily identify the goals to which they are directed.
As far as anyone can tell, what she experiences of action is barely, if at all, affected by the disruption to her sense of agency.
So while the two may be linked in all kinds of ways, cases of intermanual conflict in anarchic hand syndrome demonstrate the importance of distinguishing experiences of agency from experiences of action.
And whatever we experience of action, it cannot be reduced merely to our having experiences of agency.

%	Three possibilities
%	- Direct Hypothesis
%	- Metacognitive Hypothesis (cat.percp. of action; cf sense agency)
%	- Indirect Hypothesis

% Regardless of the extent to which one might sense control over actions, what can one experience of them?


\section{What Are the Causes of Experiences of Action?}
\label{sec:cause}
Having distinguished experiences of action from experiences of acting, a question arises. What are the causes of such experiences?  What factors could you change in order to alter those of an agent’s experiences which reveal what is being done?

Our ultimate aim to understand experiences associated with acting together. But it will be helpful to start with ordinary, individual action as there is a far wider body of evidence covering this case.

% Plan #4.a
One initially tempting assumption is that the causes of experiences of action are the configurations and movements of a body involved in acting, and perhaps also some of the perceptible consequences of these. On this view, experiences of action associated with brushing your hair (say) are a consequence of the ways your hand grips the brush and moves it through your hair.

This assumption appears to be wrong, however, because there are individuals who appear to experience actions without actually performing them at all.  Patients with anosognosia for hemiplegia are characterised as being unaware of suffering from paralysis \citep[e.g.][]{berti:2005_shared}. When requested to perform actions involving a limb which is in fact
 paralysed, they will report having done so. Further, indirect measures (which exploit bimanual interference effects and brain responses) indicate that this is not mere confabulation \citep{berti:2005_shared,berti:2008_motor,garbarini:2012_moving}. Although not decisive, this is a strong reason to suppose that experiences of action cannot all be merely consequences of bodily configurations and movements.
After all, the patient with anosognosia for hemiplegia reports experiences like those involved in actually acting in the absence of relevant bodily configurations and movements.

The idea that experiences of actions do not depend on actual bodily configurations or movements is strengthened by reflection on imagining acting. There is a way of imagining acting which is not simply a matter of thinking about acting but which involves motor imagery \citep{Jeannerod:1994oz,jeannerod:1995_mentala}. Imagining acting in this sense is phenomenally closer to actually acting than merely thinking about acting is. The similarity between experiences associated with actually acting and with imagining acting is a further reason to hold that experiences of action cannot all be merely consequences of bodily configurations and movements.

Given the experiences associated with imagining acting, and the experiences of patients with anosognosia for hemiplegia, it may be tempting to assume that the causes of experiences of action are intentions. After all, you can have an intention whether or not you act.
This can make it seem that intentions are common to all cases in which there is experience of action.

Against this possibility, consider the experiences of patients with anarchic hand syndrome.
Such patients may readily detect goals to which their various actions are directed even when these goals are in conflict with what they avowedly intend to achieve. And they appear to have whatever experiences are characteristic of actions rather than, say, experiencing the movements of an anarchic hand as mere events, as they would if, for example, they were suffering from a tremor or other loss of control over the body.%
\footnote{%
According to \citet[p.~750]{gallese:2010_bodily}, ‘the anarchic hand is still felt as being part of the experiencing and acting body and its actions’; even if these actions occur ‘outside the agent’s will, [they] are still lived by the patient as potentialities of his/her body.’
}
So even when actions run counter to intentions, experiences of action can persist.
(At least, this follows unless we are willing to suppose that all such patients have manifestly conflicting intentions, contrary to what they themselves say.)
These unusual cases reveal that whatever we experience of action, it cannot all be a consequence of our intentions.

Whatever causes experiences of action, it cannot be bodily manifestations  of action only because some experiences of action occur in the absence of such manifestations; and it cannot be intentions only because some experiences of action are contrary to what is intended. This motivates considering a third possibility: experiences of action are (at least in part) a consequence of motor representations.

Motor representations are the representations involved in control of very small-scale actions such as playing a chord, dipping a brush into a can of paint, placing a book on a shelf or cracking an egg. Attention to the ways these actions unfold reveals that, often enough, the early parts of an action anticipate the future parts in ways that cannot be determined from environmental constraints alone. This is a sign that even very small-scale actions are a consequence of representations concerning how actions will unfold in the future. These are what we refer to as \emph{motor representations}.%
\footnote{%
On what motor representations are and why they are necessary, key sources include \citet{rosenbaum:2010_human}, \citet{prinz:1990_cc}, \citet{wolpert:1995internal}, \citet{jeannerod_motor_2006}  and \citet{rizzolatti_mirrors_2008}.
}

Motor representations resemble intentions in some key respects. Like intentions, some motor representations specify outcomes, provide for the coordination of action, and normally do so in such a way as to increase the probability that the specified outcome will occur \citep[][]{rizzolatti_functional_2010,rizzolatti:2016_mirror}. This makes it easy to conflate motor representations with intentions. However, the two should be distinguished on several grounds. One is that motor representations exclusively concern present actions rather than potentially actions at some unspecified future time. Another is that motor representations, unlike intentions, are tied to the body and its capacities to act. But perhaps most importantly, intention and motor representation differ with respect to representational format and are therefore not inferentially integrated \citep{butterfill:2012_intention}. The claim that motor representations are a cause of experiences of action is therefore distinct from any claim about intentions.

Motor representations are good candidates for causes of experiences of action. They are present in ordinary cases as well as those involving anarchic hand syndrome and anosognosia for hemiplegia (on the latter, see \citealp{berti:2005_shared}). They also underpin imagining acting \citep{jeannerod:2003_mechanism}.  So whatever exactly we experience of action, at least some component of this experience is likely to be a consequence of how actions are represented motorically.
We can therefore be confident that motor representations are among the causes of experiences of action.



\section{Motor Representations Shape Experiences of Action}
\label{sec:shape}
In the previous section we considered causes of experiences of action.  In discussing these causes we were neutral on the nature of such experiences.  So far they have been characterised simply as experiences associated with actions which reveal what is being done. Our aim in this section is to take a small further step without getting entangled in complicated issues about the contents of experience.

There are experiences which provide their subjects with reasons for making judgements about actions (or so we have been assuming from the start).  Suppose, for example, that you have been lying in bed for some time worrying about various problems.  At some point you may find yourself getting up. Your experiences may provide you with reasons to judge that you are doing this, and that you are performing various actions associated with getting up. What things can influence whether your experience provides you with these reasons rather than with some quite different reasons (or with no reasons at all)?  We will say that these things, whatever they are, \emph{shape} your experience. So for something to {shape} an experience of action is just for it to influence which reasons this experience provides.

Note that a thesis concerning what shapes experiences of actions can be neutral on their contents. The contents of such experiences may only ever involve bodily configurations, joint displacements, sounds and other perceptible consequences of acting; or they may involve actions in some richer sense \citep[see][]{sinigaglia:2015_goal_ascription}. Our discussion of what shapes experiences of action is neutral between these and other possibilities.

In the previous section we argued that motor representations are a cause of experiences of action. In this section we shall argue, further, that motor representations also {shape} experiences of action.

As a first step, consider that motor representations can influence judgements about actions and their consequences. This is true both
for actions you are performing \citep[for example, ][]{costantini:2011_tool}
as well as for  actions you are merely observing (for example, \citealp{casile:2006_nonvisual,cattaneo:2010_ones}; see \citealp[pp.~56--7]{blake:2007_perceptiona} for a concise review of earlier research).
To illustrate, consider a pair of experiments by Repp and Knoblich. In each case, subjects were asked to make a judgement about the relative pitches of two tones. These tones were always played in the same order. However, the tones were carefully selected to be an ambiguous pair: that is, they could be heard either as ascending in pitch or as descending in pitch. Repp and Knoblich investigated how judgements were influenced by motor representations by exploiting the fact that expert pianists’ motor representations of certain keyboard actions specify not only finger movements but also chords \citep{haslinger:2005_transmodal}. As they found, performing \citep{repp:2009_performed} or observing \citep{repp:2007_action} soundless keyboard actions that expert pianists would typically represent in relation to a pair of tones which increase (or decrease) in pitch resulted in the expert pianists being correspondingly biased to judge that the tones they heard were increasing (or decreasing) in pitch. That is, the way you represent your own or another’s actions motorically can influence your judgements about those actions and their consequences.

Second, the influence of motor representation is thought to have a function: it enables you to make judgements which are more accurate or detailed than would otherwise be possible (\citealp{Wilson:2005qu}; \citealp[p.~72]{shiffrar:2010_people}).

Third, it is sometimes assumed that the effects of motor representation on judgements goes via experience. For example, in her discussion, \citet{shiffrar:2010_people} shifts without explicit argument from motor processes to ‘motor experience’ (her term). We take this assumption to be correct.  After all, motor representations and judgements are inferentially isolated (perhaps because of differences in their representational formats; see \citealp{butterfill:2012_intention}). The existence of any facilitatory connection between them stands in need of explanation. We suppose that such connections depend in one way or another on experience. That is, motor representations influence experiences associated with actions which in turn influence judgements. And if this is right, we have arrived at the conclusion that motor representations are not only causes, but also shape, experiences of action.


% Add agent-neutral motor representations -> experience of action does not yield reasons for judgements about who is acting, so doubly independent of experience of acting?



\section{Experiences of Joint Action}
\label{sec:two_questions}
Up to this point we have considered individual action only. This is because so much research on action focuses on this case. But our aim is to understand experiences associated with joint action. Suppose two people are performing a very small-scale joint action.  For example, they are clicking glasses, passing an object between them or playing a chord together. What could either of them, or an observer, experience of the joint action?

We have already removed one potential obstacle to answering this question.
Antecedent of any discoveries about motor representation and experience, it would perhaps have been tempting to suppose that there is a special way in which we can experience our own actions, and that we cannot experience the actions of others in this way. However, the twin discoveries that motor representations shape experiences of action and that motor representations occur in observing others act indicate that this may be incorrect.  Instead, it appears (in the absence of further discoveries about what shapes experiences of action with contrary implications) that experiences of your own and of another’s actions have a common element. Further, if we focus on how experiences provide their subjects with reasons for judgements about actions, there appears to be no ground for assuming that any difference in kind between your experience of an action of your own and your experience of the same action performed by another.
So the fact that a joint action involves more than one agent does not straightforwardly entail that there cannot be experiences of joint action.

Questions about experiences of joint action are nevertheless complicated by the fact that a joint action is not simply a composite of two ordinary, individual actions: it is the event of two or more people acting together with a purpose (see \cref{sec:introduction}). This requires, minimally, that there is a single outcome to which the individual actions are directed and this is not, or not only, a matter of each individual action being individually directed to that outcome.  It follows if experiences associated with joint action consisted in no more than experiences associated with component individual actions, then experiences associated with joint action would not reveal any difference between multiple actions being performed in parallel and genuinely joint actions.  We should therefore ask whether experiences of action only ever provide their subjects with reasons for judgements concerning the goals of individual actions? Or do they sometimes provide reasons for judgements concerning the goals of joint actions where this is not, or not only, a matter of them providing reasons for judgements concerning the goals of individual actions?

We stipulated earlier (in the Introduction) that an experience of action is one that reveals the action; that is, one which provides its subject with reasons for judging that the action is being performed.  Our question, then, is simply whether there are any experiences of joint actions which are not merely experiences of individual actions?

Given that motor representations shape experiences of action, an important step towards answering this question is to consider whether motor representations could also shape experiences of joint action.






\section{Motor Representation in Joint Action}
\label{sec:collective_goal}
% Collective goals are represented motorically.

What role do motor representations play in joint action?
In pursuing this question it is helpful to introduce some further terminology.
A joint action is an event involving two or more agents acting together with a purpose (as we stipulated in the Introduction).
In acting together with a purpose, there is an outcome to which the agents’ actions are directed where their being so directed is not, or not only, a matter of each individual action being directed to this outcome.
In what follows we will need to refer to outcomes with this property.
Let us therefore stipulate that for some actions to be \emph{collectively directed} to an outcome is for them to be directed to this outcome and for their being so directed not to be, or not only to be, a matter of each action being so directed.
Further, a \emph{collective goal} is an outcome to which two or more agents’ actions are collectively directed.
Note this notion of collective goal is neutral on mechanisms: it says nothing at all about the psychological (or other) states in virtue of which actions have collective goals.

Now consider a joint action in which you draw a line and another draws a circle where these actions, yours and the other’s, are collectively directed to producing a single design.
What is represented motorically in you?

Consider three possibilities: (1) the only outcomes ever represented motorically in you are those to which your own actions are directed; (2) the only outcomes ever represented motorically in you are those to which the other’s actions are directed and the outcomes to which your own actions are directed; and (3) the outcomes represented motorically in you sometimes include the collective goal of producing a single design.
To distinguish the first two possibilities experimentally, we can compare situations in which you are acting alone with situations in which you are performing a joint action \citep[e.g.][]{Sebanz:2003kf}.
Indeed, a range of evidence suggests that the first possibility can be excluded.%
\footnote{%
See, for example, \citet{kourtis:2012_predictive,meyer:2011_joint,meyer:2013_higher-order,loehr:2015_sound,loehr:2013_monitoring,Menoret:2013fk,baus:2014_predicting,schmitz:2017_corepresentation,novembre:2013_motor}.
For complications and opposing interpretations of some of the evidence, see
\citet{dolk:2014_joint,dittrich:2016_joint,wenke:2011_what,constable:2017_eye}.
}
But to distinguish the second and third possibilities we need a further comparison: between joint action and parallel but merely individual actions.

To this end, \citet{dellagatta:2017_drawn} had participants draw lines with their right hands while observing circles being unimanually drawn by a confederate.
To create a minimal contrast between acting together with a purpose and merely acting in parallel, participants were divided into two groups with different instructions.
In the ‘acting-together group’, participants were instructed to perform the task together with the confederate, as if their two drawing hands gave shape to a single design. In the ‘acting-in-parallel group’, participants were given no such instruction. Importantly, the groups differed only in the instructions given before the drawing started.
If participants were to follow the instructions, their actions would be collectively directed to the outcome of drawing a circle and a line in the acting-together group only.
Earlier research has established that when people have to perform incongruent movements simultaneously, such as drawing lines with one hand while drawing circles with the other, each movement interferes with the other and line trajectories tend to become ovalized \citep{franz:1991_spatial}.
This ovalization has been described as a bimanual coupling effect, suggesting that motor representations for drawing circles can affect motor representations for drawing lines \citep{garbarini:2012_moving}.
Accordingly, if the collective goal of producing the line-circle drawing is represented motorically (as only possibility 3 above allows), there should be an interpersonal motor coupling effect in the acting-together group only.
This would result in greater ovalization of the lines drawn in the acting-together group than in the acting-in-parallel group, which is actually what della Gatta et al found.

The findings of della Gatta et al indicate that collective goals can be represented motorically. We should be cautious about relying on evidence from a single source, of course: positive results from replications and extensions of their paradigm would strengthen confidence. Nevertheless, for the purposes of this chapter we will assume that when people act together with a purpose, collective goals are sometimes be represented motorically. What might this tell us about experiences of joint action?



\section{Motor Representations Shape Experiences of Joint Action}
\label{sec:shape_joint_action}
Our question (see \cref{sec:two_questions}) is whether there are any experiences of joint actions which are not merely experiences of individual actions.
An indirect way to answer this question is suggested by combining two pieces of evidence.
First, motor representations are not only causes of experiences of action (see \cref{sec:cause}) but also shape experiences of action (see \cref{sec:shape}).
So when experiences provide reasons for giving a particular answer to a question about which goal an action is directed to, which answer the experiences provide reasons for giving can depend on which goal is represented motorically.
Second, as we have just seen (in \cref{sec:collective_goal}), the collective goals of joint actions can be represented motorically.
We therefore conjecture that, when observing or performing a joint action, motor representations of a collective goal of the joint action sometimes both cause and shape your experiences.
This conjecture goes beyond the evidence discussed so far and, to our knowledge, is yet to be tested.
Were it correct, it would follow that experiences can provide reasons for judgements about the overall goal to which a joint action is directed in whatever ways experiences do for judgements about individual actions.
And, further, that experiences associated with joint action do not, or do not only, provide reasons for judgements concerning the goals of individual actions: they can also provide reasons for judgements concerning the collective goals of joint actions.

In short, there are motor representations of collective goals, and if (as we conjecture) these can shape experiences then there are experiences of joint action which are not merely experiences of individual actions.

It may be useful to note a limit on the conclusion we have drawn.
Can experiences of action also enable their subjects to make judgements not just about the goals of actions but also about whether they are performing or observing an individual or a joint action?
The ideas and discoveries we have reviewed in this paper do not provide an answer to this question either way.
Consider the motor representations which (we conjecture) shape experiences of joint actions.
For all we know, these are agent-neutral representations of outcomes.%
\footnote{%
For a representation to be \emph{agent-neutral} is for its content to not specify any particular agent or agents.
Some motor processes appear to involve agent-neutral representations \citep{ramsey:2010_understanding,jeannerod:2003_mechanism}.
\citet{gallese:2001_shared} and \citet{Pacherie:2006dl} argue for the agent-neutrality of some motor representations.
}
Now consider that an individual action and a joint action may be directed to just the same type of goal: filling a glass or playing a chord, say.
(In the joint action, the work is divided between two agents.)
Agent-neutral motor representations of the goals of such actions may in principle not differ between the individual action and the joint action.
It follows that the experiences shaped by such motor representations may in principle not provide reasons for judging that the action observed or performed is a joint or individual action.

This neutral conclusion makes sense given earlier reflection on the consequences of the agent-neutrality of motor representations (see \cref{sec:shape}). In the case of ordinary, individual action, the fact that experiences are shaped by motor representations may explain none of their powers (if any) to provide reasons for judgements about who is acting. It is only a small further step to the view that being shaped by motor representations may also fail to explain any power that experiences have to provide reasons about the number of those acting.


\section{Conclusion}
In this chapter our aim was to investigate whether there are experiences of joint actions which involve more than merely experiences of individual actions.  One complication we faced was uncertainty over the phenomenology of experiences associated with action. To avoid relying on any premise about the contents of experiences, we argued for a parallel between individual and joint action. Just as when acting individually  there are sometimes motor representations of the outcome to which the action is directed (see \cref{sec:cause}), so also when acting jointly there are sometimes motor representations of the collective goal (see \cref{sec:collective_goal}). In the individual case, these motor representations can shape experiences of action (see \cref{sec:shape}). That is, which actions the experiences reveal to their subject depends, at least in part, on which outcomes are represented motorically.  We conjecture that the same is true in the case of joint action.  If this is right, experiences associated with acting together are experiences of action in whatever sense experiences of acting alone are. Not all experiences associated with joint action are merely experiences of individual actions: there are irreducible experiences of joint action.
