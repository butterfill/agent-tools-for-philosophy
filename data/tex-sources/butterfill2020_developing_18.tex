%!TEX root = master.tex


\chapter{Conclusion to Part II}
\label{cha:conclusion-to-part-minds}


Abilities to track actions and mental states, and to perform joint actions, appear about as early in development as abilities concerning physical objects.
This should be no surprise given how social most humans are.
The real headline from the discoveries reviewed in \cref{part:minds-actions} is about limits, not successes.

These limits---patterns in what infants can and cannot do---mean that 
in attempting to characterise infants’ abilities in each domain, we are immediately confronted by developmental puzzles.
One kind of puzzle concerns abilities whose manifestation is response-dependent.
For goal tracking and mindreading, there are single scenarios in which infants will manifest a goal-tracking or mindreading ability when one response is measured but not when another response is selected.
For example, contrast pupil dilation with anticipatory looking in the case of goal-tracking \citep{gredeback:2010_infantsa}, or anticipatory looking with an explicit question in the case of mindreading (for example, \citealp{low:2014_quack}).
One puzzle facing any theory is to explain why such systematic discrepancies in performance should occur.

By itself, this is a shallow kind of puzzle.
After all, mere response-dependence may invite a methodological explanation.
A deeper puzzle arises because patterns of response-dependence interact with other limits.
Thus goal-tracking is sometimes, but not always, limited by an infant’s own abilities to perform actions; and within this own-action limit, response-dependence does not apply, or does not apply in the same way 
(see \cref{sec:tracking-is-acting-in-infancy}).
Equally, mindreading is sometimes, but not always, subject to a signature limit concerning numerical identity (see \cref{sec:signature-limits-mindreading}); and whether the limit is observed depends on which kind of response is measured (see \cref{sec:developmental-theory-mindreading}).
One challenge for any developmental theory of goal tracking or mindreading is to explain, ideally in a way that generates readily testable novel predictions, why there are these interacting limits on abilities.

Whereas puzzles about goal tracking and mindreading arise prior to accepting any theory, the main puzzle about joint action was linked to a particular theory.
On Bratman’s theory, all joint action involves shared intention and having shared intentions implies being able to coordinate planning, which in turn implies being able to know things about another’s mental states (see \cref{sec:Bratman-shared-intention}).
The puzzle about joint action arises because one- and two-year-olds appear capable of joint action, but incapable of coordinating planning (as we saw in \cref{sec:coordinating-planning}).

We should have low confidence that the solutions to puzzles about actions, minds and joint actions we have considered will turn out to be correct.
There is little prospect of firmly establishing a developmental theory on narrowly theoretical grounds, or by retrofitting it to past discoveries.
Ultimately everything turns on whether the theory’s predictions turn out to be broadly correct.
% So we are surely justified in continuing to explore theories which make competing predictions.
Nevertheless, some aspects of the theoretical approach could turn out to have lasting value even if many details turn out to be wrong.
It is therefore worth attempting to draw together what we have learnt about goal tracking, mindreading and joint action.


\section{Dual Process Theories}
\label{sec:dual-process-theories}

% [***this section is complete but maybe belongs in conclusion conclusion?]

The general approach to solving the puzzles about goal-tracking and mindreading we have followed leans on the idea of a \gls{dual process theory}.
The core claim of any dual process theory is just that humans’ abilities in a domain involve two or more processes; and the processes are distinct in this sense: the conditions which influence whether one process occurs differ from the conditions which influence whether another occurs.

The term ‘dual process’ can be misleading. 
It may be understood as meaning there are exactly two kinds of process, whereas we have often distinguished more than two.
I think the term remains apt, though, because the difficult thing is to get people past one. 
Once you get past the idea that goal-tracking, mindreading and anything else we pre-theoretically think about as a single ability has to be explained by appeal to a single kind of process, the step from two processes to three or more is not so hard.

The most important duality we have observed is linked to knowledge.
On the one hand, there are inferential processes involving knowledge states.
On the other hand, there are distinct motor and perceptual processes, and early-developing belief-tracking processes (whatever exactly those are).
In every domain, the inferential processes emerge later in development, months or even years after the early-developing processes are manifest.
This makes it theoretically coherent to appeal to the early-developing motor and perceptual processes in explaining the developmental emergence of knowledge.

There is an important difference in the dual process theories we have considered.
Some, but not all, involve a further distinction between two or more kinds of early-developing processes.
In the domain of action, we could distinguish two kinds of process which seem to occur even in the first year of life.
We have evidence for proper goal tracking based on motor processes (see \cref{sec:motor-theory-vs-teleological-stance}).
And we have evidence that infants can also track targets thanks to whatever broadly perceptual processes underpin \gls{perceptual animacy}.
Similarly, in the domain of physical objects, we distinguished the operations of a system of object indexes from motor processes (see \cref{cha:causation}): both are present in the first months of life.
By contrast, in the domain of minds we have yet to see evidence that distinct belief-tracking processes occur even in the first year of life.
Perhaps this is because less is currently known about the nature of these processes, or perhaps it is an artifact of our narrow focus on tracking beliefs rather than mental states generally.
Or maybe mindreading is just more unified than abilities in other domains.

\section{Pluralism about Models}
\label{sec:model_pluralism}

Alongside dual process theories, another key feature of our approach is pluralism about models.
A model is just a way some part or aspect of the world could be.

There is a striking contrast between developmental research on  physical objects and developmental research on minds and actions.

In the case of the physical, it is hardly questioned that processes for segmenting and tracking objects rely on distinctive models of the physical, models which are in some respects less accurate.
Indeed it was unnecessary to discuss models explicitly in \cref{part:physical-objects}.
It is too obvious to need emphasis that Spelke’s \gls{Principles of Object Perception} characterise a distinctive model of the physical, one that is less accurate but also less complex that those which adult humans typically rely on in ordinary thought and talk about physical objects.
And of course the models of the physical relied on in everyday life are in turn less accurate than models deployed by engineers or physicists.
The idea that distinct processes underlying abilities concerning physical objects could involve distinct models of the physical is hardly controversial and has attracted scarce attention \citep[but see][for a particularly elegant exception]{kozhevnikov:2001_impetus}.

In the case of the mental, I often encounter resistance to the idea that distinct mindreading processes might rely on different models of minds and actions.
It may be tempting to assume that 
there is just one model of the mental and all mindreading involves the use of that model.
Or, more carefully (to accommodate insights about development such as  \citealp{wellman:2011_sequentiala}), the tempting assumption is that
all models of the mental comprise a family in which one of the models, the best and most sophisticated model, contains everything contained in any of the models.

I suspect this assumption is tempting for at least two reasons.
One possible reason is lack of  familiarity with alternative models of minds and actions.
Even the slightest familiarity with the history of science enables us to get a fix on how impetus-based and Newtonian models of the physical differ, for example.
There is no comparably easy way to get a fix on the different models of minds and actions (but see \cref{sec:minimal-models-of-the-mental}).

The other possible reason it may be tempting to assume that there is just one model of minds and actions arises from self-understanding.
It may be quite difficult to accept that the model of minds and actions which underpins your everyday thinking about yourself and those around you is not especially accurate.
But note that this model of minds and actions has to serve a variety of purposes.
As well as enabling you to predict, it underpins normative, ethical and legal activities.%
\footnote{%
According to \citet{Gopnik:1997xq}, humans’ development is a consequence of their pursuing investigations in roughly the way scientists do (see \citealp{Gopnik:1996kl} for a shorter statement of the view for philosophers).
I suspect infants of doing a bit of ethics and legal work on the side.
}
You might judge a particular line of thought to be a blameless mistake; or you might blame someone for an emotional response which, although inappropriate, is not in any sense a mistake.
And of course legal questions can hinge on matters of belief and intention.
Insofar as a model of minds and actions serves all these purposes, should we expect it to be accurate?
My guess is that a model’s effectiveness in allowing us to knowingly reach normative, ethical and legal verdicts will often be in tension with its accuracy, given limits on our cognitive resources.
But even if a model of minds and actions were only used for prediction, the need to generate timely predictions requires speed--accuracy trade--offs, and these would likely mean that the model trades accuracy for simplicity (\cref{sec:speed-accuracy}).

The struggle to get from Aristotelian to Newtonian models of the physical---to get beyond the idea that informal observation and guesswork provides the only way to characterise physical phenomena---was notoriously long and bitter.
Cleaving to the idea that whatever model  underpins adults’ most reflective thinking is the one and only accurate model of minds and actions is a relic of this struggle.
%(The transition from Aristotelians models in the case of minds is still ongoing; this are more complicated because the model is itself part of the proper study of mind).
To understand development, we need to distinguish not only processes but also the models these processes rely on.
By characterising a model relied on by early-developing mindreading or goal-tracking processes we can see how minds and actions appear from the perspective of an infant.

\section{Goal Tracking Is the Foundation}
\label{sec:goal-tracking-is-foundation}
Tracking the goals of actions is indispensable for mindreading.
After all, it is predictions and retrodictions of action that provide the evidential basis for ascribing mental states---this holds even for the states involved in minimal theory of mind (see \cref{sec:minimal-models-of-the-mental}). 
And goal-tracking is, of course, essential whenever joint action (or any form of social interaction) involves monitoring another’s actions.
% So it is no exaggeration to say that infants’ abilities concerning action are the foundation of their social skills.

The foundation for research on goal tracking is provided by \citet{Gergely:1995sq}’s \gls{Teleological Stance} which states, roughly, that goals are those outcomes which an action is best suited to bringing about.
This formulation can be extended by adding further to principles that implicitly specify a relation between actions to outcomes.
Where the principles obtain then the outcomes will, often enough, be goals of the actions.

Given that the Teleological Stance is  \gls{formally adequate}, we can be sure that \gls{pure goal tracking} is possible in principle.
That is, goal tracking can occur independently of any information about mental states.
Theoretically, this matters because it shows that a theory of goal tracking can be independent of, and therefore foundational for, theories of mindreading and of joint action.
Practically, focussing on pure goal tracking allows us to formulate and evaluate conjectures about goal-tracking independently of conjectures about mindreading.

Adopting the Teleological Stance leaves open the question, Which processes enable infants to track goals?
(This is a counterpart of the \gls{Linking Problem} encountered in \cref{cha:linking-problem}; see \cref{sec:motor-theory-vs-teleological-stance}.)
Reflection on the limits of goal-tracking in adults and infants provides indirect support for the view that not all goal-tracking involves inferential processes operating on knowledge states: instead, at least some goal tracking involves motor processes only (see \cref{sec:motor-theory-goal-tracking}).
One problem for this view is that it is incompatible with many cases in which infants in the first year of life appear to adopt the Teleological Stance but could not be relying on motor processes.
But it is possible that in these cases the infants merely tracking targets and relying on heuristics rather than adopting the Teleological Stance.
If so, their abilities in these cases may be explained by whatever underpins \gls{perceptual animacy} in adults (see \cref{sec:perceptual-animacy}).

In this Part, we have travelled from goal-tracking to joint action.
This may give the impression, wrongly, that  a complete theory of goal-tracking could be independent of theories of joint action or mindreading.
Although goal-tracking is a foundation for mindreading and joint action, it is also possible that fully understanding goal-tracking may require us to travel in the opposite directed.
We may also need to understand how some forms of goal-tracking depend on joint action.


\section{When Joint Action Enables Goal Tracking}
\label{sec:joint-action-enables-goal-tracking}

The \gls{Teleological Stance} is limited.
Using the Teleological Stance hinges on you (or motor processes in you) computing means-ends relations.
But suppose that you are ignorant of the relevant means-ends relations.
Perhaps, for instance, you are observing someone using a novel tool and you have no idea what it is for or how it might works.
Or maybe the action you are observing involves multiple steps that do not form a familiar sequence, can occur in various orders and can be interspersed among other activities.
In such situations you cannot use the  \gls{Teleological Stance} because the means are opaque to you.

You will also run into the problem of opaque means if you have no prior experience of referential communication.
Communicative actions characteristically have  goals which the actions are means to realising only because others recognise them as means to realising those goals (that is, they involve a Gricean circle). 
This makes the means-ends relation involved to a noncommunicator, posing a significant problem for attempts to explain the developmental emergence of communicative abilities.

Just here joint action is relevant.
Abilities to perform joint actions provide you with a way to track the goals of other’s actions which does depend on having information about the relevant means-ends relations.
Joint action can therefore enable you to overcome the problem of opaque means.

How? According to \citet{butterfill:2012_interacting}, the following inference characterises a route to knowledge of others’ goals:
%
\begin{enumerate}
\item You are willing to engage in some joint action
or other with me
%(for example, because you have made eye contact with me while I was in the middle of attempting to do something).

\item I am not about to change the single goal to which my actions will be directed.

\end{enumerate}
%
Therefore:
%
\begin{enumerate}[resume]
%
\item A goal of your actions will be my goal, the goal I now envisage that my actions will be directed to.
\end{enumerate}
%
Call this inference \emph{your-goal-is-my-goal}.  
To say that it characterises a route to knowledge implies two things.  
First, in some cases it is possible to know the premises, 1–2, without already knowing the conclusion, 3.  
Second, in some of those cases knowing the premises would put one in a position to know the conclusion.  

Consider each point in turn.
Because another can signal that they are about to engage in joint action with you by, for example, smiling and making eye contact, it is possible to know that they are about to engage in joint action with you independently of any information about the goals of the joint action.
And because your goals are often enough obvious to people around you, and they are often enough willing to help you achieve them, the step from premises to conclusion is sometimes reasonable.
(That there is no guarantee: the inference is not supposed to be deductive, of course.)

If the above inference really is a route to knowledge about others goals, then it shows how abilities to perform joint actions enable a kind of goal tracking can avoid the problem of opaque means.
The idea is not, of course, that infants (or adults) who use this kind of  goal-tracking really need to know premises \#1 and \#2. 
No knowledge is needed at all,
just as the \gls{Teleological Stance} characterises a form of goal tracking which can be implemented with motor representations and no  knowledge at all (see \cref{sec:motor-theory-goal-tracking}).

The theory of goal tracking has been treated as an input to theories of mindreading and joint action in this Part.
This simplification makes sense given how fundamental goal tracking is.
But, as the problem of opaque means and the possibility of avoiding it by exploiting abilities to perform joint actions suggests,
we will eventually also need to consider how one- and two-year-olds’ abilities to perform joint actions and to track mental states can facilitate goal-tracking.


\section{Joint Action and the Developmental Emergence of Knowledge}
We have seen that infants in the first year of life can track actions and mental states and perform joint actions.
None of these abilities involve knowing facts about particular actions, mental states or intentions; nor do they require planning abilities.
It is therefore safe to work on the premise that these abilities, far from presupposing that knowledge is already in place, are somehow implicated in the emergence in development of knowledge.
Knowledge of particular minds and actions, like knowledge of particular physical objects, is not manifest in the very first months and years of life but somehow emerges from \gls{core knowledge} and abilities for joint action.


%%% Local Variables:
%%% TeX-master: "master"
%%% End:
