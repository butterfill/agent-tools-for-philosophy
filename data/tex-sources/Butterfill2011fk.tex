%!TEX TS-program = xelatex
%!TEX encoding = UTF-8 Unicode

%NB if you change paper size, change it in preamble too (where geometry is loaded)
\documentclass[12pt,a4paper]{extarticle}
% extarticle is like article but can handle 8pt, 9pt, 10pt, 11pt, 12pt, 14pt, 17pt, and 20pt text

\def \ititle {Joint Action and Development}
\def \isubtitle {}
\def \iauthor {Stephen A. Butterfill}
\def \iemail{s.butterfill@warwick.ac.uk}

%for anonymous submisison
%\def \iauthor {}
%\def \iemail{}
%\date{}

\input{$HOME/Documents/submissions/preamble_steve_paper}

\begin{document}

\setlength\footnotesep{1em}

\bibliographystyle{newapa} %apalike

%these two lines are for anonymous submission --- they remove author and date
%but don't forget to remove defs above as well --- otherwise it will be in the metadata
%\author{}
%\date{}


\maketitle
%\tableofcontents

\begin{abstract}
\noindent
Given the premise that joint action plays some role in explaining how humans come to understand minds, what could joint action be?  Not what a leading account, Michael Bratman’s, says it is.  For on that account engaging in joint action involves sharing intentions and sharing intentions requires much of the understanding of minds whose development is supposed to be explained by appeal to joint action.  This paper therefore offers an account of a different kind of joint action, an account compatible with the premise about development.  The new account is no replacement for the leading account; rather the accounts characterise two kinds of joint action.  Where the kind of joint characterised by the leading account involves shared intentions, the new account characterises a kind of joint action involving shared goals.


\ 

\noindent
\textbf{Keywords:}
Joint action, shared intention, plural activity, cognitive development, action

\

\noindent
\textbf{Word count:}
9600 (plus 1200 words in reference list)

\end{abstract}



\section{The Question}

On the assumption that joint action plays some role in explaining how humans develop an understanding of minds, what could joint action be?  This question needs a little background.  It is quite widely agreed that human adults’ reflections on thoughts and actions, their own and others’, involve a range of commonsense psychological concepts including belief, desire, intention, knowledge and perception.  Children’s abilities to deploy these concepts improve in fluency and sophistication over more than three years \citep[e.g.][]{en_610}.  Several psychologists have claimed that children first engage in joint action from around their first birthday and that engaging in joint action facilitates these early improvements \citep[]{en_1198, en_1421, en_1090, en_557}.  Joint actions that young children engage in include tidying up the toys together \citep[]{en_1204}, cooperatively pulling handles in sequence to make a dog-puppet sing \citep[]{en_1679}, bouncing a ball on a large trampoline together \citep[]{en_1421} and pretending to row a boat together.  The psychologists claim that engaging in joint actions like these plays some role in the early development of abilities to use concepts like belief, desire, intention, knowledge and perception, and in the development of higher forms of cognition more generally.  My question is what joint action could be given that some version of this claim is true.

The question arises because a leading account of joint action, Michael Bratman’s, is incompatible with the premise.  To anticipate what is explained in detail below, on the leading account engaging in joint action requires sharing intentions, and sharing intentions requires abilities to engage in reasoning about  propositional attitudes—reasoning of just the sort whose development was supposed to be explained by engaging in joint action.  So if the leading account were the whole truth about joint action, engaging in joint action would presuppose, and therefore could not explain, much of the development of reasoning about others’ mental states.  Given that the premise is true, the leading account cannot be the whole truth about joint action.  We need a further account of joint action, one that is compatible with the premise that joint action plays a role in explaining how humans develop abilities to think about minds and actions.  Sections 4–8 provide such an account.  Before that, Section 2 outlines the leading account of joint action and Section 3 explains why this account is incompatible with the premise about development.


\section{What is joint action?  The leading account}

Philosophers’ paradigm cases of joint action include painting the house together (Michael Bratman), lifting a heavy sofa together (David Velleman), preparing a hollandaise sauce together (John Searle), going to Chicago together (Christopher Kutz), and walking together (Margaret Gilbert).  One aim of an account of joint action is to identify features of some or all of these cases in virtue of which they count as joint actions.  In this paper I focus on Michael Bratman’s account because, despite the clarity of its presentation, no decisive objection to the parts of his account outlined below has yet been published, and also because this account has been most influential in psychology.\footnote{ 	Other notable contributions not discussed in this paper due to lack of space include 
	\citet[]{en_1369}, 
	\citet[]{en_1287}, 
	\citet[]{en_1403}, 
	\citet[]{en_1373},
	\citet[]{en_1383}, 
	\citet[]{en_1290} and 
	\citet[]{en_1427}.
}    

Bratman characterises a kind of joint action he calls ‘shared intentional activity’, which is activity explainable by shared intention.\footnote{ 	\citet[][p.\ 142]{en_1357}.  See \citet[][pp.\ 338-9]{en_1197} for further details on the relation between shared intentions and shared intentional activities.}  This immediately leads to the question of what shared intentions are.  Bratman’s answer has two parts, a specification of the functional role shared intentions play and a substantial account of what shared intentions could be.  On the first part, Bratman stipulates that the functional role of shared intentions is to: 
%
\begin{quote}
(i) coordinate activities; (ii) coordinate planning; and (iii) provide a framework to structure bargaining \citep[][p.\ 99]{en_1356}.
\end{quote}
%
To illustrate: if we share an intention that we cook dinner, this shared intention will (iii) structure bargaining insofar as we may need to decide what to cook or how to cook it on the assumption that we are cooking it together; the shared intention will also require us to (ii) coordinate our planning by each bringing complementary ingredients and tools, and to (i) coordinate our activities by preparing the ingredients in the right order.

Given this claim about what shared intentions are for, Bratman argues that the following three conditions are collectively sufficient\footnote{ 	In \citet[]{en_1356}, the following were offered as jointly sufficient \textit{and individually necessary} conditions; the retreat to sufficient conditions occurs in \citet[][pp.\ 143-4]{en_1357} where he notes that “for all that I have said, shared intention might be multiply realizable.”} for you and I to have a shared intention that we J.  This is his substantial account of what shared intentions could be:
%
\begin{quote}
\label{quote:bratman_account}
“1. (a) I intend that we J and (b) you intend that we J
 
“2. I intend that we J in accordance with and because of la, lb, and meshing subplans of la and lb; you intend that we J in accordance with and because of la, lb, and meshing subplans of la and lb
 
“3. 1 and 2 are common knowledge between us” \citep[][p.\ View 4]{en_1356}
\end{quote}
%
In arguing that these are collectively sufficient conditions for shared intention, Bratman combines two strategies.  He argues that these conditions collectively suffice to rule out certain cases where, intuitively, there is no shared intention (such as the case where we each intend to paint the house, I yellow and you blue).  And he argues that the attitudes specified in these conditions are collectively capable of playing the three roles shared intentions are supposed to play.



\section{Why shared intentional activity could not significantly foster an understanding of minds}

Suppose that joint action plays a role in explaining the early development of children’s abilities to think about minds.  Is Michael Bratman’s shared intentional activity a notion of joint action which could play this role?  Several psychologists have suggested that it is.  Thus Moll and Tomasello explicate their hypothesis that ‘the unique aspects of human cognition … were driven by, or even constituted by social cooperation’ (\citeyear[][p.\ 3]{en_1198}) by appeal to ‘a modified version of Bratman’s (1992) definition of “shared cooperative activities” …  [on which] the participants in the cooperative activity share a joint goal’ (\citeyear[][p.\ 3]{en_1198}); in this context ‘share a joint goal’ means ‘possess a shared intention’.  Similarly, Carpenter, in a discussion of joint action in infancy, writes:
%
\begin{quote}
‘I will … adopt Bratman’s (1992) influential formulation of joint action … [F]or an activity to be considered shared or joint each partner needs to intend to perform the joint action together ‘‘in accordance with and because of meshing subplans’’ (p. 338) and this needs to be common knowledge between the participants.’ \citep[][p.\ 281]{en_1682}
\end{quote}
%
Others who appeal to Bratman’s notion of shared intentional activity in characterising children’s first joint actions and their role in development include \citet[][p.\ 680]{en_1090} and \citet[][p.\ 1430]{en_1365}.

Recall that shared intentional activity requires shared intentions.  On Bratman’s substantial account, sharing intentions requires having intentions about intentions and even intentions about subplans of intentions (see Condition 2 in the quote \vpageref{quote:bratman_account}).  Bratman emphasises this feature of the account:
%
\begin{quote}
“each agent does not just intend that the group perform the […] joint action. Rather, each agent intends as well that the group perform this joint action in accordance with subplans (of the intentions in favor of the joint action) that mesh” \citep[][p.\ 332]{en_1197}.
\end{quote}
%
A natural thought at this point is that joint action might require only plans which \textit{in fact} mesh rather than \textit{intentions about} the meshing of plans.  Bratman considers this option and explains why this weakening of his account is not coherent \citep[][pp.\ 331-3]{en_1197}, so I shall not pursue this thought.  

The fact that shared intentions require intentions about intentions suggests a potential objection to the view that shared intentional activity explains early developments in children’s abilities to think about minds.  For it seems unlikely that 2- and 3-year-olds, who according to many findings are years away from being able to ascribe any propositional attitudes at all,%
\footnote{
The most widely discussed propositional attitude has been belief; see \citep[]{en_87, en_89} but also \citep[]{en_1789, en_1821, en_1092, en_1261}.  Apperly and Butterfill \citep[]{en_1686} argue for the possibility that while children in their first and second year have abilities to track beliefs, they cannot ascribe beliefs or other propositional attitudes as such.
}
can form intentions about others’ intentions.%
\footnote{
\label{fn:goals}
There is evidence that children of this age have difficulty understanding intentions \citep[]{en_69, en_70}.  A range of researchers have argued that infants form expectations about goal-directed activity \citep[]{en_1438, en_1207, en_717, en_1439}.  It may be that the understanding of goal-directed activity examined by these studies falls short of an understanding of intention.
}
This would mean they cannot meet the sufficient conditions Bratman lays out for sharing intentions.  

This potential objection is weak because it depends on controversial empirical claims about the absolute time in development at which abilities to ascribe, and to form intentions about, intentions might emerge.  A more promising objection avoids this dependence.  The ability to form intentions about intentions involves a sophisticated kind of propositional attitude ascription (as explained below).  This ability is required for sharing intentions in accordance with Bratman’s substantial account.  So meeting the sufficient conditions for joint action given by this account could not significantly \textit{explain} the development of an understanding of minds because it already \textit{presupposes} too much sophistication in the use of psychological concepts.\footnote{ 	A related objection may apply to Claire Hughes’ appeal to “reciprocal exchanges”.  She specifies that such exchanges “depend on … modelling the other’s intentions/desires (i.e. reflecting on the other’s inner states) and monitoring the others’ understanding of one’s own intentions and desires (i.e. detecting mistaken beliefs about one’s own inner states)” \citep[][p.\ 56]{en_1300}.  Thus engaging in reciprocal exchanges appears to require fluid and sophisticated ascriptions of mental states.  Accordingly, engaging in such exchanges cannot significantly explain how children develop abilities to think about minds.}  

Note that, as it stands, this objection does not establish much.  It concerns conditions imposed by the substantial account of shared intention which are sufficient but not necessary conditions.%
\footnote{
\citet[]{en_1358} objects to Bratman’s account of shared intention on the grounds that it requires common knowledge.  This objection also fails because common knowledge is explicitly required by what I am calling Bratman’s substantial account only, which gives sufficient but not necessary conditions for shared intention.  (A second potential problem for Tollefsen’s argument is that it requires the premise that young children engage in joint actions of the kind Bratman’s account aims to characterise, the kind whose paradigms involve coordinating potentially long-term plans.)
}  
The substantial account is supposed to characterise one—perhaps one among many—ways in which the functional role of shared intentions can be realised.  So the objection serves only to raise a question.  Are there in fact alternative sufficient conditions for shared intention, conditions that can be met without already having abilities to use psychological concepts whose development was supposed to be explained by joint action?

The answer to this question is not entirely straightforward.  We must begin with the functional roles of shared intention,  for these provide necessary conditions.  One of the roles of shared intentions is to coordinate planning.  What does coordinating planning involve?  Intuitively the idea is that just as individual intentions serve to coordinate an individual’s planning over time, so shared intentions coordinate planning between agents.  (I use the terms ‘individual intention’ and ‘individual goal’ to refer to intentions and goals explanatory of individual actions; an ‘individual action’ is an action performed by just one agent such as that described by the sentence ‘Ayesha repaired the puncture all by herself’.)  A second role for shared intentions is to structure bargaining concerning plans.  To understand these roles it is essential to understand what ‘planning’ means in this context.  The term ‘planning’ is sometimes used quite broadly to encompass processes involved in low-level control over the execution of sequences of movements, as is often required for manipulating objects manually \citep[e.g.][]{en_1535}, as well as processes controlling the movements of a limb on a single trajectory \citep[e.g.][]{en_1681}.  In Bratman’s account and this paper, the term ‘planning’ is used in a narrower sense.  Planning in this narrow sense exists to coordinate an agent’s various activities over relatively long intervals of time; it involves practical reasoning and forming intentions which may themselves require further planning, generating a hierachy of plans and subplans.  Paradigm cases include planning a birthday party or planning to move house.  

Given the functional roles of shared intention, when (if ever) must the states which realise shared intentions include intentions about others’ intentions?  Coordinating plans with others does not seem always or in principle to require specific intentions about others’ intentions.  It is plausible that in everyday life some of our plans are coordinated largely thanks to a background of shared preferences, habits and conventions.  Consider, for example, people who often meet in a set place at a fixed time of day to discuss research over lunch.  These people can coordinate their lunch plans merely by setting a date and following established routine; providing nothing unexpected happens, they seem not to need intentions about each other’s intentions.  Within limits, then, coordinating plans may not always require intentions about intentions.  The same may hold for structuring bargaining.  But when the background of shared preferences, habits and conventions is not sufficient to ensure that our plans will be coordinated, it is necessary to monitor or manipulate others’ plans.  And since intentions are the basic elements of plans (in the special sense of ‘plan’ in terms of which  Bratman defined shared intention), this means monitoring or manipulating others’ intentions.  The background which makes for effortlessly coordinated planning is absent when our aims are sufficiently novel, when the circumstances sufficiently unusual (as in many emergencies), and when our co-actors are sufficiently unfamiliar.  In all of these cases, coordinating plans and structuring bargaining will involve monitoring or manipulating others’ intentions.  Now this does not necessarily involve forming intentions about their intentions because, in principle, monitoring and manipulating others’ intentions could (within limits) be achieved by representing states which serve as proxies for intentions rather than by representing intentions as such, much as one can (within limits) monitor and manipulate others’ visual perceptions by representing their lines of sight.  But possession of general abilities to monitor and manipulate others’ intentions does require being able to form intentions about others’ intentions.

The question was whether there are sufficient conditions for shared intention which do not presuppose abilities to use psychological concepts whose development is supposed to be explained by joint action.  As promised, the answer is not straightforward.  In a limited range of cases, coordinating plans and perhaps structuring bargaining does not appear to require insights into other minds.  But in other cases, particularly cases involving novel aims or agents unfamiliar with each other, intentions about others’ intentions are generally required.  

The main question for this section was whether Bratman’s account captures a notion of joint action suitable for explaining the early development of children’s abilities to think about minds.  Some of the joint actions which young children engage in involve novel aims, and some involve unfamiliar partners.  So if these joint actions did involve coordinating planning and structuring bargaining, they could not rest on a shared background but would require abilities to form intentions about others’ intentions.  It follows that joint action would presuppose much of the sophistication in the use of psychological concepts whose development it was supposed to explain.  So given the premise that joint action plays a role in explaining early developments in understanding minds, it cannot be the case that the joint actions children engage in as soon as they engage in any joint actions involve shared intentions as characterised by Bratman.

This conclusion rests on the assumption that having intentions about intentions involves some of the psychological sophistication whose development is supposed to be explained by appeal to joint action.  One might object that the ability for form intentions about intentions is somehow less sophisticated than abilities to form other kinds of representation of other kinds of mental states.   To answer the objection it is sufficient to clarify what ‘intention’ means in this context.  For the term ‘intention’, like ‘planning’, is used to mean different things by different researchers.  Sometimes ‘intention’ and ‘goal’ are used interchangeably in describing behaviour which is somewhat flexibly organised around some outcome \citep[e.g.][p.\ 14]{en_1433}.  But in this context we need a different notion of intention, one on which intentions are elements of plans.  Such intentions play a role in coordinating an agent’s activities over time.  Their role is characterised in part by normative constraints expressed in terms of the propositional contents of intentions.  For instance, one norm characteristic of the role of intentions in plans requires an agent to avoid ways of realising one intention that will make it impossible for her to realise other intentions she has (all things being equal).  This norm requires someone who intends both to visit an aunt and to buy some shoes in a single evening to limit time spent on each activity to allow for the other.  Another norm characteristic of intention concerns the compatibility of having multiple intentions simultaneously: it is not rational to have multiple intentions unless it is rational to have a single intention agglomerating them all \citep[]{en_1543}.  Norms such as these, together with the role of intentions in practical reasoning, are what characterise the role of intentions in planning.  Given that intentions are characterised in this way as elements in plans, it seems necessary that understanding intentions will involve some grasp both of the role that intentions as propositional attitudes play in practical reasoning and also of the norms relating intentions to planning.  Of course this does not mean that individuals who understand intentions as elements in plans can articulate or list the relevant roles or norms.  But it does mean that they should sometimes be sensitive to some of the requirements these norms impose and also that they would be able to recognise some of the norms as correct in optimal conditions.  This is why requiring intentions about intentions presupposes significant psychological sophistication.

Given that joint action facilitates the development of mental understanding, and that (as just argued) Bratman’s notion of shared intentional activity is not a kind of joint action which could play this role, does it follow that Bratman’s account is incorrect?\footnote{ 	\citet[]{en_1358} argues that Bratman’s account of joint action is incorrect on the grounds that children engage in joint action but not in shared intentional activity as characterised by Bratman.  The considerations below identify a missing premise in her argument.}  Drawing this conclusion would require the further assumption that there is just one kind of joint action.  This is not obviously true.  Compare individual action.  It is sometimes accepted that there is a distinction between intentional action and other kinds of action such as response behaviours and merely purposive activities \citep[]{en_496, en_194}.  Because there may be an analogous distinction between kinds of joint action, the developmental considerations do not directly bear on the correctness of Bratman’s account.  Perhaps what we need is not a modified version of Bratman’s account but an account of a different kind of joint action.

To sum up, Bratman’s account does not characterise a kind of joint action which could play a role in explaining how children come to understand minds.  In the next section I consider whether this undermines the claim that the interactions highlighted by developmental psychologists are really joint actions before offering a new account of joint action in the following sections.  



\section{Joint action, shared intention and coordinated planning}

The argument of the previous section establishes that not all of the following claims are true:
%
\begin{quote}
(1) joint action fosters an understanding of minds;

(2) all joint action involves shared intention; and

(3) a function of shared intention is to coordinate two or more agents’ plans.
\end{quote}
%
These claims are inconsistent because if the second and third were both true, abilities to engage in joint action would presuppose, and so could not significantly foster, an understanding of minds.  For all that has been said so far, any of these claims could be rejected.  In what follows I characterise a form of joint action which involves what I call ‘shared goals’ and no shared intentions.  The aim is to show by construction that there are forms of joint action which require minimal cognitive sophistication.  In doing this I shall provide grounds for rejecting the claim that joint action always involves shared intention, (2), strengthening the case for accepting the premise about development, (1), while remaining neutral on whether shared intentions function to coordinate planning, (3). 

Some researchers assert that all joint actions involve shared intentions.  For instance, Tomasello writes that ‘[t]he sine qua non of collaborative action is a joint goal and a joint commitment’ (\citeyear[][p.\ 181]{en_1828}).  
Here ‘joint goal’ refers to shared intention in Bratman’s sense and ‘collaborative action’ includes joint actions early in development.  Similarly, Gilbert writes ‘I take collective action to involve a collective intention’ (\citeyear[][p.\ 5]{en_1287}).
Perhaps, then, a better strategy than the one I propose would be to reject the first claim, (1), above and conjecture that joint action cannot significantly foster an understanding of minds (although some lesser form of interaction may do so).  But it is striking that none of the researchers who assert that all joint action involves shared intention provide an argument, and narrowly semantic considerations provide no support for this assertion \citep[][p.\ 367]{en_2394, en_2395}.  In fact other researchers have assumed without argument that not all joint actions involve shared intention (\citealp[][p.\ 330]{en_1197};  \citealp[][p.\ 2010]{en_1860}; \citealp{en_1778}).  On what grounds could we accept this assumption or its negation?

Let us step back.  What features other than shared intention indicate that the actions of two or more agents constitute a joint action as opposed to any other kind of interaction?  Here are several indicators of joint action:
%
\begin{quote}
i.	this case seems to fit with paradigm examples of joint action such as walking, cooking or playing the piano together \citep[]{en_1861, en_2392, en_25};

ii.	the candidate joint action differs from a case in which the agents perform the same type of activity (such as walking or cooking) in parallel rather than together \citep[][p.\ 150]{en_1768, en_1861, en_1365};

iii. for each agent, acting together rather than individually is voluntary in this sense: in so far as they control which means they adopt in pursuing a goal, such as whether to move an object by lifting it or by dragging it, they can also control whether their actions are individual or joint;

iv. there is a sense in which all of the agents’ actions taken together are directed to a single goal, and this is not just a matter of each agent’s action being individually directed to that goal;

v. there is a description of the interaction with a plural subject and an action verb, such as ‘they are bouncing the ball on the trampoline’ (\citealp{en_1290} emphasises this indicator);

vi.	each agent is disposed to modify her actions in accordance with what is needed to achieve the goal given how the other agents’ actions are unfolding \citep[][p.\ 328]{en_1197}.
\end{quote}
%
I am not suggesting that any of these features is necessary for joint action, nor that they are collectively sufficient.  My claim is just these features are relevant to deciding whether an interaction is a joint action, and that where an interaction has many or all of these features we have (defeasible) grounds to infer that it is a joint action.  In short, I propose that, in the absence of a deeper analysis, we should take these features as a rough and provisional explication of one theoretically significant way of using the term ‘joint action’.\footnote{ 	There may be other notions of joint action.  For instance, Ludwig asserts that “[t]he concept of a joint action as such is just that of an event of which there are multiple agents” (\citeyear[][p.\ 366]{en_2394}).  As he notes, it is a straightforward consequence of this view that if there are joint actions then there are joint actions without shared intentions.}

So what about the objection that all joint action involves shared intention?  The following will provide (defeasible) grounds for rejecting it.  For I shall show that some interactions have the features listed above, (i)-(vi), but do not involve shared intention.  



\section{Plural activities}

Our question is what joint action could be on the assumption that it fosters an understanding of minds.  We have seen (in Section 3) that the answer cannot involve appeal to shared intention.  Joint actions involving shared intention presuppose, and so cannot significantly foster the development of, sophisticated uses of psychological concepts.  What we need, then, is to identify a form of joint action that requires as little psychological sophistication as possible; by presupposing less we make it possible to explain more.  

I shall approach this task indirectly by first considering forms of action involving multiple agents more basic than any kind of joint action.  Some ants harvest plant hair and fungus in order to build traps to capture large insects; once captured, many worker ants sting the large insects, transport them and carve them up \citep[]{en_1292}.  The ants’ behaviours have an interesting feature distinct from their being coordinated: each ant’s behaviours are individually organised around an outcome—the fly’s death—which occurs as a common effect of many ants’ behaviours.  We can say that there is a single activity—killing a fly—which several ants performed.  In general, a plural activity is one involving two or more agents.  As I shall use the term ‘plural activity’, for agents to be engaged in a plural activity it is sufficient that each agent’s activities are individually organised around a single outcome which occurs as a common effect of all the agents’ activities.

Note that nothing controversial is assumed in stating these sufficient conditions for plural activity.  The first ingredient is the notion that an individual’s behaviours can be organised around an outcome.  This is shorthand for an open-ended disjunction of cases; it means that there is an intention, habit, biological function or other behaviour-organizing circumstance connecting the individual’s behaviours to the outcome.  The second ingredient is the notion of a common effect, which is not specific to action.  The fly’s capture is a common effect of the ants’ individual behaviours in just the sense that the fly’s death is a common effect of the multiple doses of poison it received: none of the doses was individually deadly, each had its murderous effect only in concert with some of the others.  As characterised here, the notion of a plural activity depends on nothing more controversial than the cogency of these two ingredients.   

My use of the term ‘plural activity’ is different from other natural uses of this term and may be too broad to pick out an intuitive category.  If Helen and Ayesha individually aim to smooth a section of pavement by shuffling their feet when they walk over it and if it does become smooth partly as a consequence of both their efforts, then they are engaged in the plural activity of smoothing the pavement.  This is true even in the absence of any intention to act together.  It is true even if their lives do not overlap at all, so that there may be no intuitive sense in which their smoothing the pavement is a joint action.  Allowing this case to count as a plural activity does no harm for present purposes and makes it possible to characterise a useful notion of plural activity in uncontroversial terms.

Another potentially unnatural feature of this definition of plural activity is that it requires success.  By definition, where two or more agents’ actions constitute a plural activity the outcome to which they are directed must occur.  This simplifies the definition in ways that will shortly be useful.

Humans sometimes perform activities that are plural in the minimal sense that some ants’ behaviours are.  In a particularly sulky mood Thomas pulls on one end of a large boat in order to move it; he does not realise that Illaria is pushing the other end and that without her contribution the boat would not move.  He succeeds in moving the boat, as does Illaria.  So there is a single activity—moving the boat—which they both perform.  Even though Thomas’ action is goal-directed (its goal is to move the boat to the sea), his activity is plural only in the minimal sense that the ants’ fly-trapping behaviour is: it is organised around an outcome which occurs as a common effect.  The plural nature of such activities need not show up in intentions, desires or beliefs.  That an activity is performed by two individuals does not require that they intend, believe or desire this to be so. 

What most philosophers mean by joint action is not, or not only, this minimal notion of plural activity.  Certainly this minimal notion is inadequate for our present purpose, which is to understand what joint action could be on the assumption that it plays some role in explaining how children come to understand minds.  Nevertheless, the notion of plural activity is a useful starting point for understanding a kind of joint action relevant to explaining development.  


\section{What is the function of shared goals?}

In giving an account of one kind of joint action I shall first identify something I call ‘shared goals’.  Before starting it will be helpful to fix terminology.  As I use the term ‘goal’ it refers to an outcome, actual or possible, and not to a state.  I make no direct use of the notion that agents can have goals (as in ‘Sam’s goal was to topple the president’) and focus on relations between goals and actions (as in ‘the goal of Marvin’s action was to upset Ayesha’).  An action is \textit{goal-directed} where it makes sense to ask which of its possible and actual outcomes are goals to which the action was directed.  One paradigm case of goal-directed action involves intention: where an agent acts on an intention, the intention’s content specifies a goal to which her action is directed.  In addition, an action can arguably be directed to a goal which is not specified in the content of any of the agent’s intentions \citep[]{en_1359}.  There may also be forms of action which are goal-directed but do not involve intention at all.\footnote{ 	This is suggested by Velleman’s (\citeyear[][pp.\ 10-30]{en_25}) discussion of ‘purposeful activity’.}  Certainly there are ways of representing actions as goal-directed which do not involve representing intentions or any other propositional attitudes of agents.  

The term ‘shared’ is used loosely.  Just as, on most accounts, shared intentions are neither literally intentions nor literally shared (no single intention is mine and yours), so shared goals are not goals (they are complexes of states and relations) and do not by definition alone involve anything which is literally shared.  I have used the otherwise infelicitous label ‘shared goal’ in order to highlight the basic intuition behind the positive account of joint action I shall offer: some cases of joint action involve structures which bind not the agents’ intentions but the goals to which their activities are directed.\footnote{ 	This intuition provides a loose connection between the notion of shared goals and Miller’s notion of a collective end \citep[]{en_1827}.  While it would be useful to discuss differences and similarities in substance and motivation, there is no space to do that here.}   

Having fixed terminology we can now turn to the primary issue, which is to identify shared goals.  Following the model provided by Bratman’s account of shared intentions, my account of shared goals has two parts: a specification of their function role and a substantial description of states that could realise them.  This section is about the functional role of shared goals, the following section concerns their realisation.  To emphasise, questions about the mental states involved in sharing goals will be deferred until the next section; in this section the question is only what shared goals are for. 

Successful plural activity generally requires coordination.  How is this coordination achieved?  In the case of ants such coordination may be achieved hormonally.  In humans, who can voluntarily engage in plural activities with novel outcomes, coordination can usually only be achieved psychologically.  This is what shared goals are for.  Shared goals coordinate multiple agents’ goal-directed activities around an outcome to be achieved as a common effect of their efforts.  That is, their function is to coordinate plural activities.  

An illustration may help to clarify what performing this function amounts to.  There is a fallen tree lying across the road.  Several people each want it moved, but none of them can move it by themselves and none of them can control the others’ actions.  The tree’s movement can only be secured as common effect of several people’s actions.  In this situation, there is a need for several people to coordinate their goal-directed activities.  They need to lift in complementary directions and at suitably related times; and if lifting doesn’t work, they need to change strategy and try pushing or something else that might achieve the outcome.  The role of shared goals is to coordinate these goal-directed activities.

In the above illustration, plural activity is necessary to achieve an outcome.  Shared goals also play a role in situations where plural activity occurs even though it is not necessary.  For instance, Amin and Bertram each individually aim to put a large barrel into a boat.  Either of them could move the barrel into the boat alone or their doing this could be a plural activity; the choice is theirs.  The sequence of activities Amin would need to perform to put the barrel in the boat differs depending on whether he is acting alone or with Bertram.  Acting alone, Amin would position himself so that the barrel and boat are in front of him, throw his arms around the middle of barrel, raise it, tilt back and then push up and forwards.  If he chose to act with Bertram, Amin would need to take an entirely different approach.  It is this need that shared goals answer.  

The notion that a function of shared goals is to coordinate goal-directed activities needs qualifying because all action involves coordination at several levels.  Any goal-directed action, individual or joint, will be realised by a collection of simple object-directed actions such as pushing, pulling and tearing; and these in turn will be realized by some kind of motor actions, and so on until at some point we reach continuous bodily movements.  Plainly most tasks require coordination at several of these levels; for instance, passing an object from one hand to another requires precise timing of releases and grasps as well as appropriate positioning in space.  At some levels, coordination is largely independent of which goals agents’ actions are directed to (for example, it can be hard to use one’s hands in an uncoordinated way even when doing so would be advantageous).  This is true even for coordination of multiple agents’ activities.  We may coordinate with others without being aware of how we are coordinating or even that we are coordinating \citep[]{en_1693}.  In fact there seem to be several forms of \textit{emergent coordination}, that is, coordination which is independent of the goals of an agent’s actions \citep[]{en_1812}.  Clearly, then, shared goals are not the only factor in coordinating plural activity.  The role of shared goals is limited to coordinating goal-directed actions and not their non-purposive components, and it may be that shared goals can play this role only thanks to the existence of other mechanisms of emergent coordination.  

Shared goals resemble shared intentions insofar as both exist to coordinate activities.  They differ in that structuring bargaining and coordinating planning are not functions of shared goals.  On some views, the distinction I have drawn between shared intentions and shared goals parallels a distinction between individual intentions and more primitive states connecting individuals’ activities with the goals to which they are directed.  For individual intentions are sometimes held to be intrinsically elements in agents’ plans and therefore absent from the lives of any agents incapable of planning \citep[]{en_1694}.  Such agents (if there are any) may need to act when faced with equally desirable alternatives and to coordinate their activities around goals despite fluctuations in desire.  This need might be met by states which resemble intentions in that they exist in part to coordinate a single agent’s activities and in that they connect the agent’s activities to a goal, but differ from intentions in lacking planning functions. Given this distinction, shared intentions would stand to shared goals roughly as individual intentions stand to their more primitive counterparts.

The limited function of shared goals makes them better suited than shared intentions for characterising the cases studied in developmental research.  Many, perhaps all, of these cases would not normally require coordinated planning.  As mentioned above, these cases include tidying up the toys together and cooperatively pulling handles in sequence in order to make a puppet sing.  As coordinated planning is not needed in such cases, nor are shared intentions.  What is needed, though, is for the agents’ goal-directed actions to be coordinated.  This is what shared goals are for.  


\section{Which states could realise shared goals?}

In the previous section I identified shared goals in terms of their function, which is to coordinate plural activities.  The next step is to characterise states capable of realising this function.  

To start with an illustration, suppose that a goal of Amin’s actions in the near future will be move a large barrel into a boat.  Amin anticipates that some of Bertram’s future actions will have the same goal, and Amin expects the barrel’s moving into the boat to occur as a common effect of his own goal-directed actions and Bertram’s.  For his part, moving the barrel into the boat will also be a goal of Bertram’s actions and Bertram has expectations mirroring Amin’s.  In favourable circumstances their goal-directed contributions to a plural activity of moving the barrel into the boat could be coordinated in virtue of this pattern of goal-relations and expectations.  Accordingly, the existence of such goal-relations and expectations are sufficient for Amin and Bertram to share the goal of getting the barrel into the boat.

Here are the key features of this case expressed in general terms: 
%
\begin{quote}
(a) \textit{one goal, two or more agents—}

there is a single goal, G, to which each agent’s actions are, or will be, individually directed;

(b) \textit{identification—}

each agent can identify each of the other agents in a way that doesn’t depend on knowledge of the goal or actions directed to it;

(c) \textit{expectations about goal-directed actions—
}

on balance\footnote{ 	The ‘on balance’ qualification in conditions (c) and (d) rules out cases where agents do have the specified expectations about goals and outcomes but also have further, conflicting expectations which outweigh them.} each agent expects each of the other agents she can identify to perform an action directed to the goal; and

(d) \textit{expectations about a common effect—
}

on balance each agent expects this goal to occur as a common effect of all of their actions directed to the goal, her own and the others’.
\end{quote}
%
In favourable circumstances and in concert with emergent coordination, these goal-relations and expectations could serve to coordinate the goal-directed plural activities of two or more agents (I shall say which circumstances are favourable below).  Since ‘shared goal’ was defined in terms of this coordinating function, (a)-(d) are collectively sufficient for possessing a shared goal. 

Let us consider these four features in turn.  The first feature, (a), is required just because we are concerned with plural activities; by definition, actions comprising a plural activity are directed to a single goal such as moving a particular barrel onto a certain boat (see Section 4).  The second feature, (b), was not explicit in the description of Amin and Bertram’s barrel moving.  It excludes the following sort of case.  Mia and Sobani are in a crowded space.  Each intends to move a table and, thanks to her background knowledge, expects that exactly one other agent intends the same.  But neither Mia nor Sobani can identify who else she expects to be involved in moving the table, except trivially as the other table-mover.   In this case, the pattern of goal-relations and expectations in (a), (c) and (d) could have at most a limited effect on Mia and Sobani’s ability to coordinate their efforts.   So the identification requirement, (b), is included because the coordinating effect of (a), (c) and (d) seem to depend on it.  Identification may not feature whenever agents have a shared goal; (a)-(d) collectively provide only sufficient conditions.  

The third and fourth features, (c) and (d), involve expectations.  Knowledge states and beliefs both count as expectations but it is not necessary to have either.  In developmental research, looking times and eye movements are regularly used as measures of infants’ expectations concerning goals \citep[for example][]{en_1434, en_1436, en_1207, en_1437, en_1439}.  It is an open question whether these sorts of expectations are beliefs.  Where such expectations do not merely control looking times and eye movements but also inform a range of goal-directed actions in ways that are rational given their contents, then expectations of this type are sufficient for sharing goals whether or not they amount to beliefs.  This marks one contrast between requirements on sharing goals and sharing intentions.  Given that intentions function to coordinate planning in Bratman’s intellectual sense of planning and given some plausible norms governing the rationality of planning \citep[][pp.\ 29-31]{en_1774}, sharing an intention will require knowledge of others’ intentions and their relations to one’s own.  By contrast, sharing a goal does not require knowledge.   because several kinds of expectation which fall short of knowledge are sufficient for coordinating plural activities. 

The third feature, (c), concerns expectations about other agents’ goal-directed actions.  This is a minimal counterpart of the requirement that agents who share an intention represent each other’s intentions.  Possessing a shared goal requires representing only goal-directed actions.  It is possible to represent an action as goal-directed without representing (or even being able to represent) intentions or any other propositional attitudes.  For instance, an agent might represent goals as functions of actions.%
\footnote{
On goals as functions of actions see, for example, Wright \citep[]{en_161} and Price \citep[]{en_139}.  A variety of research supports the claim that young children, non-human primates and corvids track the functions of things (including Rakoczy and Tomasello 2007; Casler and Kelemen 2007; Csibra and Gergely 2007; Kelemen 1999; German and Defeyter 2000; Hauser 1997; Emery and Clayton 2004).  On the abilities of these groups to represent goals specifically, see further footnote \vref{fn:goals}.
}
There are strong theoretical and empirical grounds to hold that representing goal-directed actions requires less conceptual sophistication, and may be less cognitively demanding, than representing intentions as such.

The fourth feature, (d), concerns the agents’ expectations that the goal to which their actions are directed will occur as a common effect of their efforts.  This and the third feature are jointly equivalent to requiring that the each agent expects that she and the other agents are engaged in a plural activity with goal G.  (The agents may not actually be engaged in a single plural activity because, as noted earlier, plural activities are by definition successful whereas it is possible to possesses a shared goal without succeeding.)  The claim that features (a)-(d) are sufficient for agents to possess a shared goal is the claim that this combination of features could function to coordinate plural activities.  In essence, the claim is this: an expectation, on the part of each agent concerned, that she is or will be involved in a plural activity with the others, will, in favourable circumstances and in concert with emergent coordination, normally enable them to coordinate their actions.

Shared intention is sometimes thought to involve common knowledge in such a way that agents who share an intention can know that they share an intention.%
\footnote{
This view is endorsed by \citet[][p.\ 103]{en_1356} and rejected by   \citet[][pp.\ 387-8]{en_2394}.
}
By contrast, it is possible to have a shared goal without knowing that one does.  Agents can have, and act on (see Section 8 below), a shared goal without understanding their actions as comprising anything more than a plural activity.

The above pattern of goal-relations and expectations, (a)-(d), can play its coordinating role only in favourable circumstances.  What makes circumstances \textit{un}favourable?  One factor is a lack of freedom.  To illustrate, suppose that Hendrik and Arch are instructed to tidy the toys away.  Arch would not normally obey this instruction but Hendrik convincingly threatens reprisals unless Arch tidies all the toys away.  For the goal of tidying the toys away, (a)-(d) above could all obtain in this case (Hendrik fulfils (a) by the act of threatening).  But any coordinating effect this pattern of goal-relations and expectations might have had is trumped by Hendrik’s control over Arch’s actions.  Arch’s lack of freedom is an unfavourable circumstance, that is, one in which the coordinating role of shared goals may be blocked.  Another unfavourable circumstance is antagonism to plural activity.  Suppose that Ella and Cohen have been tasked with wiping a table clean.  Ella is desperate to clean the table without Cohen, and Cohen is desperate to clean it without Ella.  Neither thinks this will be possible, and they satisfy (a)-(d) above.  But because they are desperate to act alone, each tries to sabotage the other’s efforts.  Any coordinating effect the shared goal might have had is overridden by the agents’ antagonism to plural activity. In short, then, favourable circumstances are those in which factors that would defeat the coordinating tendency of the pattern of goal-relations and expectations in (a)-(d) are absent; paradigm defeating factors are a lack of freedom and antagonism towards plural activity.



\section{Shared goals characterise one form of joint action}

So far I have stipulated that the function of shared goals is to coordinate plural activities and argued that this function could be realised by a certain pattern of goal-relations and expectations.  Finally I shall use this to characterise a form of joint action which, lacking the cognitive and conceptual demands associated with shared intentional action, could be used to explicate the premise that engaging in joint action fosters an understanding of minds.

Shared goals are characteristic of a form of joint action but the relation between possession of a shared goal and performing a joint action is not straightforward.  Compare ordinary individual action.  Acting intentionally is not just a matter of acting and simultaneously intending; nor is it even just a matter of being caused to act by an intention \citep[][pp.\ 136-7]{en_1211}.  Relatedly, we should not suppose that the mere presence, or even the mere efficacy, of a shared goal is sufficient for joint action.  Take any collection of actions directed to a single goal, G, involving two or more agents.  Let us say that these actions, taken collectively, are \textit{driven by a shared goal} when G is a shared goal of the agents, when, in performing actions directed to this goal, they are acting on the associated expectations and any other attitudes in ways that are rational, and when their so acting functions to coordinate their actions in a way that would normally facilitate the goal’s occurring as a common effect of all their efforts. (I assume that, although the expectations mentioned in Section 7 may not be necessary for possessing a shared goal, some such expectations are always involved.)  I claim that actions driven by shared goals are joint actions.

Why accept this?  Recall the six indicators of joint action identified earlier (Section 4).  Where these three conditions hold of an interaction, most of these indicators will be present.  The interaction will be distinct from a case in which the agents pursue the goal in parallel and without a shared goal (this was the second indicator).  The action will be voluntary with respect to its jointness insofar as jointness is partly due to agents acting on their expectations in ways that are rational, as contrasted with interactions where coordination involves only involuntary forms of emergent coordination (third indicator).  If the goal occurs, this will normally be in part because of the coordination provided by the shared goal; in this sense, the coordination serves to direct the agents’ actions, taken together, to the goal and this amounts to more than each agent’s actions being individually directed to the goal (fourth indicator).  Finally, the agents’ dispositions to adapt their actions to each other’s is built into the requirement that the shared goal function to coordinate their actions by means of their acting on the associated expectations (sixth indicator).  In short, actions driven by shared goals have many features indicative of joint action.  This is reason to hold that they are in fact joint actions. 

Not every case in which actions are driven by shared goals fits intuitively with paradigm examples of joint action.  Consider two drivers on a collision course in a narrow street.  Suppose (perhaps unrealistically) that each acts with the goal of avoiding a collision between their cars, expects the other to do the same and expects that they will avoid collision thanks to their combined efforts.  This is sufficient for avoiding a collision to be a shared goal.  (Note that specifying the goal requires care: the drivers’ actions would have different goals if, for instance, the only goal of each driver’s action were to avoid hitting the other.)  Suppose also that their actions are driven by a shared goal in the sense defined above.  So, on the above account, their avoiding collision is a joint action.  (Not all cases of avoiding a collision are joint actions, only those, if any, which are driven by shared goals.)  But intuitively this case may not seem to fit with paradigms of joint action because the interaction is so minor. If this counts as joint action, then, given the right goal-relations and expectations, so could passing someone in a corridor.  Should we modify the account of joint action in order to exclude this sort of case?  There is an obstacle to doing that.  We could elaborate a series of interactions driven by the shared goal of avoiding collision where each interaction is slightly less minor than its predecessor in the series.  Whether or not intuitions support drawing a boundary, it seems that no such boundary is theoretically significant for understanding the role of joint action in development.  Since our aim is to characterise joint action as it fosters development, we should risk deviating from intuition to avoid otherwise unnecessary complexity. 

There is another sort of case in which actions driven by a shared goal may not intuitively fit with paradigm joint actions.  Consider again the two drivers whose goal is to avoid collision.  Now suppose, in addition, that the first driver hawkishly accelerates while covertly preparing to brake if necessary, causing the second driver to brake hard.   Given the present account, their actions nevertheless constitute joint action.  This may not fit intuitively with paradigm joint actions because the first driver dominates the second (and does so by means of deception).\footnote{ 	This example is can be modelled as a Hawk-Dove game and is adapted from a discussion of Gold and Sugden \citep[][pp.\ 304-8]{en_2393}.  As these authors note, the combination of actions is not a rational consequence of team reasoning \citep[on team reasoning see]{en_1373} and so does not involve the associated notion of group agency \citep[]{en_1383}.}    Again it is possible to elaborate a series of cases involving gradually varying degrees of domination.  While outright coercion is incompatible with joint action on the account I have offered (see Section 7 above), neither domination nor other failures to be cooperative are excluded.  This may conflict with intuitions about joint action but reduces the complexity of the account.  That the account is nevertheless an account of joint action is shown by the presence of the other indicators of joint action mentioned above.

Accounts of joint action sometimes invoke special kinds of mental state \citep[]{en_1426, en_1369}, special kinds of reasoning \citep[]{en_1383}, special kinds of interdependence \citep[]{en_1356, en_1827} or, apparently, special kinds of agent \citep[]{en_2391}.  The present account, if successful, shows that there is a simple kind of joint action characterising which requires no such special ingredients.  The simple kind of joint action involves only ordinary individual goal directed-actions being coordinated in part by expectations about others’ goal-directed actions and their common effects.  

There are, of course, questions and puzzles about joint action which the simple account offered here does not address and which may call for greater complexity.  These include issues about commitment \citep[]{en_1287}, the coordination of decisions to act jointly \citep[]{en_1297}, the kind of reasoning needed for coordinating choices \citep[]{en_1373}, and the possibilities of intending others’ actions \citep[]{en_1369, en_1692} and of acting on others’ intentions \citep[]{en_1427}.  Failure to address these questions or tackle these puzzles might be an objection if the simple account were meant to be the whole story about joint action.  In fact the simple account is not supposed to apply to every case of joint action.  For instance, painting a house together will probably involve intentionally coordinating plans and so require sharing intentions rather than merely sharing goals.  On the other hand, in some cases joint action does not require planning, commitment, coordinating decisions to act or special kinds of reasoning. Knoblich and Sebanz offer an example: 
%
\begin{quote}
“the way people lift a two-handled basket depends on whether they lift it alone or together. When alone, a person would normally grasp each handle with one hand. When together, one person would normally grasp the left handle with his/her right hand and the other person would grasp the right handle with his/her left hand.” \citep[][p.\ 2026]{en_1429}
\end{quote}
%
Because handles provide an obvious way for two people to lift the basket (these authors even postulate a joint affordance) and people are often skilled at coordinated lifting, planning is typically unnecessary and it is plausible that shared goals are sufficient for joint actions of this sort.\footnote{ 	Knoblich and Sebanz claim that lifting the basket together requires “joint intentionality” which in turn requires shared intentions in roughly Bratman’s sense: “[t]here needs to be an intentional structure that allows an actor to relate his/her own intention and the other’s intention to an intention that drives the joint activity” (\citeyear[][p.\ 2025]{en_1429}).  I reject this claim for the reasons given above.}  



\section{Conclusion}

The question was this.  Given the premise that joint action plays some role in explaining how children come to understand minds, what could joint action be?  The negative point was that it couldn’t involve sharing intentions for reasons connected to the fact that sharing intentions involves coordinating planning and so requires sophistication in ascribing propositional attitudes.  The positive claim was that there is a simple account of joint action which is compatible with the developmental premise.  On the simple account, joint action involves sharing goals and sharing goals requires only an understanding of goal-directed actions and their common effects.  
	
This simple account of joint action is not offered as a replacement for Bratman’s account or any accounts competing with his.  Bratman’s account assumes that joint action involves shared intention where the functions of shared intention include coordinating paradigmatically long-term plans.  Such an account may be required to characterise complex cases where success demands that agents’ plans mesh.  But some cases of joint action (such as carrying a two-handled basket together) do not involve plans in the relevant sense of planning.  The agents need to coordinate their activities but not their plans.  The simple account applies only in such cases.  In philosophical accounts of individual action, actions explainable by intending are sometimes distinguished from other kinds of individual action including response behaviours, arrational actions and merely purposive activities \citep[]{en_181, en_1683, en_25}.  Comparable distinctions are certain to be needed for understanding joint action; the differences between shared intentions and shared goals mark one such distinction between kinds of joint action. 












\small
\bibliography{$HOME/endnote/phd_biblio_en_record_num_keys}

\end{document}