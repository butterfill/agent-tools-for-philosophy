%!TEX root = master.tex


\chapter{Object Indexes and Motor Representations of Objects}
\label{cha:causation}

% thoughts after Warwick discussion 2015-06-03

% 1. specify that there is a system of object indexes and don't rely too much on the evidence:

      % The interesting thing about object indexes is that a system of object
      % indexes (at least one, maybe more)
      % appears to underpin cognitive processes which are not
      % strictly perceptual but also do not involve beliefs or knowledge states.
      % While I can’t fully explain the evidence for this claim here,
      % I do want to mention the two basic paradigms that are used to
      % investigate the existence of
      % a system of object indexes between perception and thought \ldots\

% 2. say what object indexes are for: to guide ongoing action; and what they do: influence the allocation of attention  \citep{flombaum:2008_attentional}


% n.b. : must return to the issue from chapter 2 that infants can search in milk but not behind barriers.  Can this pattern of findings be derived from the proposal?

% \section{Reading}
% On perception of causation: \citep{hubbard:2012_phenomenal,hubbard:2012_phenomenala}

% Perception of causation and RM: \citep{choi:2006_measuring} updated by a later Hubbard study that follows up on this (and reaches a conflicting conclusion)


Four-month-old infants have unexpectedly rich abilities concerning physical objects.
They can segment objects, represent them as persisting and track their causal interactions;
and their abilities to do all three things can be described by a single set of principles about how physical objects behave, the \gls{Principles of Object Perception} (as we saw in \cref{cha:principles-object-perception,cha:simple-view}).
How are these early abilities  related to knowledge of  physical objects?
An initially promising conjecture with radical implications for understanding the origins of knowledge is that the principles are things infants know or believe.
But this conjecture systematically generates incorrect predictions (as we saw in \cref{cha:linking-problem}).
% According to the \gls{Simple View}, these abilities rest on their knowing, or believing, some principles about how objects behave.
% In \cref{cha:linking-problem} we saw that the Simple View generates incorrect predictions and so must be rejected.
On rejecting it we are immediately confronted  with the \gls{Linking Problem}.
If not knowledge or belief, what does link principles describing infants’ abilities concerning physical objects to their minds, and thereby explains their abilities?
% We also saw that there are significant obstacles to improving on the Simple View by invoking a very general theory about infants’ representations, regardless of whether that theory concerns modularity, core knowledge or graded representations.

The most influential and best developed attempt to answer this question invokes core knowledge (see \cref{cha:core-knowledge}). But as introduced so far, merely invoking core knowledge does not seem to generate the kind of predictions that would enable us to know whether it is true, nor does it explain apparently contradictory patterns in infants’ abilities.
There may be nothing wrong with invoking core knowledge, but doing so does not enable us to know we have solved the Linking Problem.

In this chapter we will develop a conjecture which may explain the apparently contradictory patterns of evidence about infants’ abilities concerning physical objects, and which might almost enable us to solve the \gls{Linking Problem}.

As a first step we will consider a brilliant conjecture about the cognitive mechanisms underpinning infants’ abilities concerning physical objects.
Several researchers have conjectured that infants’ abilities
concerning physical objects are a consequence of something called
\glspl{object index}
\citep{Leslie:1998zk,Scholl:1999mi,Carey:2001ue,scholl:2007_objecta}.
In developing a version of their view,
this chapter will also introduce the method of  \glspl{signature limit}.
By the end of this chapter you should have a sense of how these ingredients, object indexes and signature limits, can be used to construct developmental theories with readily testable predictions which provide an alternative to, or refinement of, theories relying on knowledge, belief or \gls{core knowledge}.
But first, what are object indexes and why should we believe they exist?


\section{Object Indexes in Adult Humans}
\label{sec:object-indexes-adults}


Start with an analogy.
An old-fashioned logistician is keeping track of supply trucks by sticking pins in a map.
She updates the positions of the pins regularly, based on her information about the terrain and new information about the trucks’ fates.
Because she only has partial information, much of this is guesswork.
But she does not guess at random.
Instead she relies on certain principles.
For instance, she operates in ways that presuppose the trucks will move along continuous paths, and that they will tend to keep going in roughly the direction they are currently moving.

The system of pins and principles provides a way of tracking trucks.
The pins are not representations in the sense that cargo manifests or photographs of the trucks are.
Instead the idea is that behaviours of the pins correspond to the behaviours of the trucks.
For each pin there is a corresponding truck and, ideally, the movements of the pin reflect the movements of the truck it corresponds to.
(Of course the logistician will sometimes stop tracking a truck and assign its pin to a new truck.)

Object indexes are a mental counterpart of the pins: they are things that point to, or index, objects.
Like the pins, object indexes are part of a system for tracking objects’ movements.
And like the logistician, this system is able to function well enough on the basis of partial information because it operates in ways that presuppose principles about physical objects.
For instance, it operates in ways that presuppose objects move along continuous paths.

Why believe that object indexes exist in adult humans?
One reason is that they can track at least four moving objects simultaneously.
(There is debate about exactly how many objects can be tracked simultaneously.  Some say the limit is four while others claim eight; see \citealp{alvarez:2007_how}.)
Suppose you are shown a display involving eight stationary circles (see \cref{fig:pylyshyn_2001_fig6}).
Four of these circles flash, indicating that you should track these circles.
All eight circles now begin to move around rapidly, and keep moving unpredictably for some time.
Then they stop and one of the circles flashes.
Your task is to say whether the flashing circle is one you were supposed to track.
Adults are good at this task \citep{pylyshyn:1988_tracking}, indicating that they can use at least four object indexes simultaneously.


%
\begin{figure}
\begin{center}
\includegraphics[width=0.8\textwidth]{fig/pylyshyn_2001_fig6.png}
\caption{
  \label{fig:pylyshyn_2001_fig6}
  A multiple object tracking task.  You first see eight circles (at t=1), four of which flash (at t=2).  Then all circles move around unpredictably for some time (at t=3).  Finally, one circle flashes (at t=4).  Your task is to say whether this last circle is one of the four which flashed at the start.
  Source: \citet[][figure 6]{Pylyshyn:2001hl}.
}
\end{center}
\end{figure}
%

The task just mentioned is called \emph{multiple object tracking}, or \emph{MOT} for short.  It has been widely studied and there are several theories about the mechanisms which underlie humans’ abilities to track multiple, simultaneously moving objects.
If you look into this research you will find people talking about ‘FINSTs’, ‘object files’ and the like.
Although the details are daunting and the differences between theories subtle, we need consider only points on which most theories agree.
Most importantly for us, there is quite wide agreement that multiple object tracking is made possible by a system of object indexes analogous to the logistician’s pins.

Another kind of evidence for the existence of a system of object indexes in adult humans arises from something called the \gls{object-specific preview benefit}.
Suppose that once again you are shown an array of objects (see \cref{fig:kahneman_1992_fig3}).
At the start a letter appears briefly on each object.
(It is not important that letters are used; in theory, any readily distinguishable features should work.)
The objects now start moving.
At the end of the task, a letter appears on one of the objects.
Your task is to say whether this letter is one of the letters that appeared at the start or whether it is a new letter.
Consider just those cases in which the answer is yes: the letter at the end is one of those which you saw at the start.
Of interest is how long this takes you to respond in two cases: when the letter appears on the same object at the start and end, and, in contrast, when the letter appears on one object at the start and a different object at the end.
It turns out that most people can answer the question more quickly in the first case.
That is, they are faster when a letter appears on the same object twice than when it appears on two different objects
\citep{Kahneman:1992xt}.
This difference in response times is the
% $glossary: object-specific preview benefit
\emph{\gls{object-specific preview benefit}}.
Its existence shows that, in this task, you are keeping track of which object is which as they move.
This is why the existence of an object-specific preview benefit is taken to be evidence that object indexes exist.

\addFigure{kahneman_1992_fig3}{%
A demonstration of the object-specific preview benefit.
  Source: \citet[][figure 3]{Kahneman:1992xt}}

Is the system of object indexes that explains adults’ abilities to track multiple objects the same system of object indexes that explains the object-specific preview benefit?
I think this is an open question.%
\footnote{%
\citet[p.~216]{Kahneman:1992xt}, \citet{scholl:1999_tracking} and
  \citet[p.~333]{noles:2005_persistence} all propose a positive answer, but alternative possibilities are sometimes mentioned \citep[for instance in][p.~1078]{odic:2012_relationship}.
}
Fortunately we do not need to answer it.
Having seen what object indexes are and the evidence that one or more systems of object indexes exist in adults, our question is how this might bear on infants’ abilities concerning physical objects.




\section{Object Indexes and the Principles of Object Perception}
\label{sec:object-indexs-princ}


Infants have sophisticated abilities concerning physical objects which are manifest from four months of age or earlier (see \cref{cha:principles-object-perception,cha:simple-view}).
But what kind of mechanism is responsible for these abilities?
% Because infants’ abilities and their limits can be described by a set of principles which approximately characterise how physical objects behave, the Principles of Object Perception, it is perhaps natural to suppose that it is knowledge of, or belief in, these principles that explains infants’ abilities concerning physical objects.
% But this supposition generates incorrect predictions and so must be rejected (see \cref{cha:linking-problem}).
% What else might underlie abilities to segment objects, represent them as persisting and track their causal interactions?
%
Could it be object indexes?
Could infants’ abilities concerning physical objects be wholly or in part a consequence of the fact that a system of object indexes exists in humans from the first few months of life?

To answer this question, we need to consider infants’ various abilities individually.
Infants can segment objects, represent them as persisting and track their causal interactions.
Which if any of these are even potentially a consequence of having a system of object indexes?

Take segmentation first.
Four-month-olds can segment objects, that is, they can identify where one object ends and another begins.
Moreover they make appropriate use of both featural information and objects’ movements in segmenting objects (see \cref{sec:segmentation,sec:principles-object-perception}).
Consider the possibility that this ability to segment physical objects is a consequence of two things.
First, insofar as infants use featural information they are relying on visual and other perceptual processes that may be only indirectly related to any system of object indexes.
Second, infants’ use of information about movements to segment physical objects is a consequence of constraints on how object indexes are assigned: things that appear to be moving independently cannot be assigned the same index, whereas things that move together are typically assigned the same index.
% \footnote{%
% For considerations which suggest there may be processes in addition to the operations on object indexes at work in segmentation, see \citet{gao:2009_are} and \citet[§5.4, pp.~194ff]{Carey:2001ue}.
% Note also that a system of object indexes can make use of featural information \citep{hollingworth:2009_object}; this suggests that the use of featural information to segment objects may sometimes be a consequence of object indexes.
% }
To illustrate, recall \citeauthor*{spelke:1989_reaching}’s experiments depicted in \vref{fig:spelke_1989_fig3}.
In these experiments, infants who see two shapes moving together treat them as parts of a single object, whereas infants who see the two shapes moving in opposite directions treat them as parts of distinct objects.
This might in principle be because infants have a system of object indexes which assigns one index to the shapes when they move together and two indexes to the shapes when they move in opposite directions.

So much for segmentation.
What about infants’ abilities to represent objects as persisting?
For instance, consider what happens when infants see an object disappear completely behind a barrier.
As mentioned in \cref{sec:persistence}, four-month-old infants, like adults, manifest an expectation about the object’s reappearance by looking in anticipation to the far side of the barrier.
This suggests that infants continue to represent the object (and its velocity) even while it cannot be seen.
Could this and other ways in which infants can represent objects as persisting likewise be a consequence of a system of object indexes?

% *** this paragraph not well written:
To answer this question we need to ask one about object indexes.
What happens to an object index when the object it indexes disappears behind a screen?
To find out, we can use the object-specific preview benefit.
Whereas the object-specific preview benefit was initially used in arguing that object indexes exist, we can now reverse direction and use the presence of a particular object-specific preview benefit as a marker of a particular object index.
Suppose a shape disappears on one side of a screen as if travelling behind it and then, some time later, a shape appears on the other side of the screen as if emerging from it.
We want to know whether the shapes are assigned the same object index.
We can find this out by determining whether there is an object-specific preview benefit for the two shapes.
If there is, we can conclude that they are assigned the same object index.
Using this method it is possible to determine that object indexes can index objects while they are completely hidden by a screen.
Since an object can have the same object index before and after disappearing, it is conceivable that some abilities to represent objects as persisting could be a consequence of these object indexes.


Is it plausible that infants’ abilities to represent objects as persisting could be entirely a consequence of their having a system of object indexes?
This would require not merely that object indexes could in principle allow infants to represent objects as persisting, but also that infants represent an object as persisting exactly when it is indexed by a single object index.
Infants’ abilities to represent objects as persisting are partially characterised by a set of principles, the \gls{Principles of Object Perception} (these are listed in \vref{table:principles-of-object-perception}).
So if infants abilities to represent objects as persisting are entirely a consequence of their having a system of object indexes, the same principles should characterise the operation of object indexes.
Do they?

It turns out  that they probably do \citep{mitroff:2009_staying,scholl:2010_persistence}.
Let me explain by returning to the old-fashioned logistician who is keeping track of supply trucks.
In doing this she has only quite limited information to go on.
She receives sporadic reports that a supply truck has been sighted at one or another location.
But these reports do not specify which supply truck is at that location.
She must therefore work out which pin to move to the newly reported location.
In doing this she might rely on assumptions about the trucks’ movements being constrained to trace continuous paths, and about the direction and speed of the trucks typically remaining constant.
These assumptions allow her to use the sporadic reports that some truck or other is there in forming views about the routes a particular truck has taken.
A system of object indexes faces the same problem when the indexed objects are not continuously perceptible.
% At times they cannot be seen or heard.
What assumptions or principles are used to determine whether this object at time $t_1$ and that object at time $t_2$ have the same object index pinned to them?

The answer is not completely straightforward.
Suppose object indexes are being used in tracking four or more objects simultaneously and one of these objects---call it the \emph{first object}---disappears behind a barrier.
Later two objects appear from behind the barrier, one on the far side of the barrier (call this the \emph{far object}) and one close to the point where the object disappeared (call this the \emph{near object}).
If the system of object indexes relies on assumptions about speed and direction of movement, then the first object and the far object should be assigned the same object index.
But this is not what typically happens.
Instead it is likely that the first object and the near object are assigned the same object index.%
\footnote{%
See \citet{franconeri:2012_simple}.  Note that this corrects an earlier argument for a contrary view \citep{scholl:1999_tracking}.
}
If this were what always happened, then we could not fully explain how infants represent objects as persisting by appeal to object indexes because, at least in some cases, infants do use assumptions about speed and direction in interpolating the locations of briefly unperceived objects.
There would be a discrepancy between the Principles of Object Perception which characterise how infants represent objects as persisting and the principles that describe how object indexes work.

But this is not the whole story about object indexes.
It turns out that object indexes behave differently when just one object is being tracked and the object-specific preview benefit is used to detect them.
In this case, it seems that assumptions about continuity and constancy in speed and direction do play a role in determining whether an object at $t_1$ and an object at $t_2$ are assigned the same object indexes \citep{flombaum:2006_temporal,mitroff:2007_space}.
In the terms introduced in the previous paragraph, in this case where just one object is being tracked, the first object and the far object are assigned the same object index.
This suggests that the principles which govern object indexes may match the principles which characterise how infants represent objects as persisting.

%haven’t discussed the \citep{wang:2004_young} with the wide and narrow occluders.  This would be explained by information about shape and size being attached to object indexes.  Don’t mention because it’s more complex; do mention because it shows that the Object Index Hypothesis is actually an improvement on the Simple View.

The question we are considering is whether infants’ abilities concerning physical objects could be a consequence of their having a system of object indexes.
We have seen that infants’ ability to segment objects might be the joint upshot of perceptual processes and object indexes.
Here the idea is that perceptual processes can explain much of infants’ use of featural information in segmenting objects, and object indexes can explain their use of motion information.
We have also seen that object indexes can be used in tracking objects which are briefly unperceived, for example because they have disappeared behind a barrier.
Because the principles describing how object indexes operate appear to match the \gls{Principles of Object Perception}, which characterise how infants represent objects as persisting,
it seems that the existence in infants of a system of object indexes could explain their abilities to represent objects as persisting.

We have not considered whether infants’ abilities to track causal interactions among objects might also be a consequence of their having a system of object indexes.
I want to skip this question in favour of another.%
\footnote{%
There is some evidence for the view that some abilities to track causal interactions may depend on the errors (or error-like patterns) in assignments of object indexes and the \glspl{metacognitive feeling} these give rise to  \citep[see][pp.~420ff]{Butterfill:2009vs}.
\citet[p.~91ff]{rips:2011_causation} offers an objection to this view based on an the Pulfrich double pendulum illusion \citep{wilson:1986_impossibly}.
This objection could be overcome if assignments of object indexes can diverge from verbal reports of what is seen, or if physical properties such as solidity may sometimes constrain operations on object indexes; and the results of \citet{Mitroff:2004pc} suggest that both possibilities obtain.
% discussed this in https://philosophical-psychology.butterfill.com/lecture_04.html#slide-140
As things stand, any link between object indexes and tracking causal interactions must nevertheless be regarded as highly tentative since, as \citet[p.~108]{choi:2006_measuring} note, the most direct experimental evidence linking abilities to track causal interactions with object indexes comes from a study by \citet{Krushke:1996ge} which, although beautifully designed, is yet to be followed up in print.
% \citep{choi:2006_measuring} on Kruschke and Fragassi: ‘one previous study suggested that causal events gave rise to divergent patterns of object-specific priming than non-causal events (Kruschke \& Fragassi, 1996). Unfortunately, this preliminary study has not received any published follow-up work, to our knowledge, so it is not clear if it would also generalize beyond highly specific contrasts’  \citep[p.~108]{choi:2006_measuring}.
}
So far we have seen some evidence that a system of object indexes could in principle underlie infants’ abilities concerning physical objects.
Is there any evidence that it does actually underlie them?

\section{The CLSTX Conjecture}
\label{sec:CLSTX-conjecture}

Reflection on object indexes and the \gls{Principles of Object Perception} (in \cref{sec:object-indexs-princ}) has led us to a conjecture:
%
\begin{quote}
%\emph{The CLSTX Conjecture} \\
Five-month-olds’ abilities to segment objects, to represent them while briefly unperceived and to track their causal interactions
are not grounded on belief or knowledge:
instead
they are consequences of the operations of
a system of object indexes.
\end{quote}
%
I will call this the \emph{\gls{CLSTX Conjecture}} because versions of it were arrived at, perhaps independently, by Carey, Leslie, Scholl, Tremoulet and Xu.\footnote{%
See \citet[]{Leslie:1998zk,Scholl:1999mi,Carey:2001ue,scholl:2007_objecta}.
These researchers probably hold differing views on exactly which abilities concerning physical objects can be explained by invoking the \gls{CLSTX Conjecture}, and are unlikely to endorse my version of the conjecture in all details.
}
%Certainly the idea that this conjecture might explain infants’ abilities to track causal interactions is controversial (which is one reason for postponing discussion of it until later).
% In evaluating the \gls{CLSTX Conjecture} we should be asking to what extent infants’ abilities concerning physical objects can be explained by their having a system of object indexes.

% Should I restore reference to \gls{Core Knowledge View} here?  No: CLSTX Conjecture could be an elaboration of the Core Knowledge View.
The \gls{CLSTX Conjecture} is a step towards constructing an alternative to the \gls{Simple View}. %, and the \gls{Core Knowledge View} (which was introduced in \cref{sec:core-knowledge-view}) can be construed as consistent with this conjecture.
The hope is that the \gls{CLSTX Conjecture} will not generate the incorrect predictions that follow from the Simple View. %; and that, unlike the Core Knowledge View, it will explain puzzling discrepancies in how abilities concerning physical objects develop.
Even better, if true it will provide us with a way of solving the \gls{Linking Problem} (see \cref{sec:the-challenge}) and perhaps enable us to identify representations that are not knowledge or belief, and so eventually illuminate how humans first come to know simple facts about particular physical objects.

How does the CLSTX Conjecture promise to solve the \gls{Linking Problem}?
According to this conjecture, what links the \gls{Principles of Object Perception} to an individual mind is a system of object indexes.
The Principles are \gls{explanatorily adequate} and not merely \gls{descriptively adequate} because they characterise how the system of object indexes operates.
This is important for understanding development.
Whereas the Simple View entails that infants’ earliest abilities already involve beliefs about, or knowledge of, general principles about physical objects and their interactions, the CLSTX Conjecture allows that these early abilities need not involve belief or knowledge at all.
Rather than presupposing knowledge, they may therefore provide a foundation for its emergence in development.

What is the evidence for the \gls{CLSTX Conjecture}?
In \cref{sec:object-indexs-princ}, we saw that the  \gls{Principles of Object Perception}, which characterise infants’ abilities concerning physical objects, resemble principles describing how object indexes work.
This indicates \glslink{formally adequate}{formal adequacy}: object indexes could in principle explain infants’ abilities.
But it is not evidence that they actually do so.
After all, the resemblance of the \gls{Principles of Object Perception} to principles describing how object indexes work
could be a consequence of the fact that, given the way physical objects actually behave, the same principles will describe many entirely different systems for tracking them.

Is there more direct evidence for the \gls{CLSTX Conjecture}?
Object indexes are probably present in infants by six months of age at the latest  \citep{richardson:2004_multimodal}.
It is also likely that in infants, as in adults, object indexes are typically maintained when an object disappears behind a barrier.
How do we know?
In adults there is a pattern of brain activity which appears to be characteristic of processes involved in maintaining an object index for an object that is briefly hidden from view.
\citet{kaufman:2005_oscillatory} asked whether the same is true of infants.
They measured brain activity in six-month-olds infants as they observed a display typical of an object disappearing behind a barrier.
They found the pattern of brain activity characteristic of maintaining an object index.
This suggests that in infants, as in adults, object indexes can attach to objects that are briefly unperceived.
So it is not just conceivable that infants’ abilities to represent objects as persisting could be a consequence of object indexes: we also know that object indexes are at least sometimes present when infants are representing objects as persisting.

The evidence we have so far gets us as far as saying, in effect, that someone capable of committing a murder was in the right place at the right time.
Can we go beyond such circumstantial evidence?
The key to doing so is to exploit signature limits.

\section{Signature Limits}
\label{sec:signature-limits}
In general, a \emph{\gls{signature limit} of a system} is a pattern of behaviour the system exhibits which is both defective given what the system is for and peculiar to that system.
%a consequence of the particular way the system is constructed.
To give a simple example, suppose there is a machine for performing addition which generally works well but gives incorrect answers when asked to add twin primes.
This is a signature limit of the system.
By contrast, the system’s giving incorrect answers in very hot conditions, or when very large numbers are input, is not a signature limit.
After all, lots of systems fail in these conditions.

Signature limits are useful for determining whether apparently disparate effects are the effects of a single type of system.
To continue the simple example, suppose you want to know whether two people are using the same type of machine to do addition.
If you ask them both what $2+2$ is and they both give you the same correct answer, this tells you almost nothing because lots of different machines can reliably give you correct answers.
But if you ask them both what the sum of some twin primes is ($137+139$, say) and they both give you the same incorrect answer, you have evidence that they are using the same type of machine.

Perhaps we can use signature limits to identify  evidence for the \gls{CLSTX Conjecture}.
Are any signature limits of a system of object indexes also signature limits of whatever underpins infants’ abilities concerning physical objects?

\citet{carey:2009_origin} argues that the signature
limits of object indexes in adults are reflected in corresponding limits on
infants’ abilities to track briefly occluded objects.
What is her argument?

One signature limit of a system of object indexes is that featural information sometimes fails to influence how objects are assigned in ways that seem quite dramatic.
Consider \vref{fig:scholl_2007_fig4}.
A patterned square disappears behind the barrier; later a plain black ring emerges.
If you consider speed and direction only, these movements are consistent with there being just one object.
But given the distinct shapes and textures of these things, it seems all but certain that there must be two objects.
Yet in many cases these two objects will be assigned the same object index \citep{flombaum:2006_temporal,mitroff:2007_space}.
So one signature limit of systems of object indexes is that information about speed and distance can override information about shape and texture.

Is this also a signature limit of infants’ abilities to represent objects as persisting?
\citet{xu:1996_infants} showed 10-month-olds the sequence illustrated in \vref{fig:carey_2001_fig3}.
Infants first see a yellow rubber duck appear from behind a screen and then disappear behind it again.
They then see a white foam ball appear from behind the screen and return behind it.
If infants can use featural information like colour, shape and texture in representing objects as persisting then they should expect there to be two objects behind the screen, just as adults do.
In fact even 10-month-olds (who have probably been representing objects as persisting for at least six months) appear to have no such expectation.
This is not a function of the particular objects used; changing them to entirely different objects makes no difference.
\citeauthor{xu:1996_infants} offer an impressively deadpan report on a failure that is astonishing from the point of view of an adult:
%
\begin{quote}
‘babies failed to demonstrate that they could use the differences between a yellow rubber toy duck emerging from one side of the screen and a white Styrofoam ball emerging from the other side of the screen to infer that there must be at least two objects behind the screen’ (\citeyear[p.~129]{xu:1996_infants}).
\end{quote}
%
This failure is striking because infants do generally show an ability to make inferences like this is the paradigm \citeauthor{xu:1996_infants} used.
Apparently infants’ early developing abilities to represent objects as persisting have a signature limit matching a signature limit of a system of object indexes, and that this signature limit is not easily overcome (\citealp{Carey:2001ue,scholl:2007_objecta}; \citealp[pp.~77–83]{carey:2009_origin}).

%
\begin{figure}
\begin{center}
\includegraphics[width=0.9\textwidth]{fig/scholl_2007_fig4.png}
\caption{
  \label{fig:scholl_2007_fig4}
  A patterned square disappears behind the barrier; later a plain black ring emerges.  How likely is it that they the same object?
  Source: \citet[][figure 4]{scholl:2007_objecta}.
}
\end{center}
\end{figure}
%

%
\begin{figure}
\begin{center}
\includegraphics[width=0.9\textwidth]{fig/carey_2001_fig3.png}
\caption{
  \label{fig:carey_2001_fig3}
  A duck appears from behind a screen and then disappears behind it again; later a ball emerges from the screen and then also disappears behind it.  How many objects are there behind the screen?
  Source: \citet[][figure 4]{Carey:2001ue}.
}
\end{center}
\end{figure}
%

How strong is this argument?
As it hinges on evidence from a single study, we should be cautious in putting too much weight on it.
More interestingly,
the argument is complicated by evidence that infants around 10 months of age do not always fail to use featural information appropriately in representing objects as persisting \citep{wilcox:2002_infants}.
In fact \citet{mccurry:2009_beyond} report evidence that even five-month-olds can make use of featural information in representing objects as persisting \citep[see also][]{wilcox:1999_object}.
%they use a fringe and a reaching paradigm.  NB the reaching is a problem for the simple interpretation of looking vs reaching!
Likewise, object indexes are not always updated in ways that amount to ignoring featural information \citep{hollingworth:2009_object,moore:2010_features}.
It remains to be seen whether there is really an exact match between the signature limit on object indexes and the signature limit on four-month-olds’ abilities to represent objects as persisting.
The \gls{CLSTX Conjecture} is a bet on the match being exact.
%It is also a bet that where older infants manifest abilities to use featural information in representing objects as persisting that cannot be explained by .

Other signature limits characterise the operation of object indexes.
For instance, there are limits on the number of objects that can be tracked simultaneously (perhaps because there are a limited number of pins or object indexes); it is possible that there is a matching limits in infants’ abilities to represent objects as persisting \citep{cheries:2006_interrupting}.
There are also limits on how long object indexes can be maintained. % without new perceptual information concerning the object indexed.
For instance, object indexes associated with the object-specific preview benefit can last for up to eight seconds in adult humans \citep{noles:2005_persistence}.
It is conceivable that four-month-olds’ abilities to represent objects as persisting are also subject to a similar limit.%
\footnote{%
See, for example, \citet{Johnson:2003bl}.
(These experiments involve changes in both distance and temporal duration, and were not designed to support a conclusion about how long infants can maintain object indexes for.)
}
We also know that infants’ initial abilities to represent objects as persisting can be affected by having objects follow untypical trajectories \citep[p.~5]{bremner:2014_perception} and by the way objects disappear from view \citep[p.~1004]{charles:2009_object}.
%\citep[p.~5]{bremner:2014_perception} ‘our work indicates perceptual constraints (such as limited time or distance out of sight, difficulty with oblique trajectories) on 4-month-olds’ detection of trajectory continuity that speaks for a perceptual account of object persistence.’
%Relatedly, \citet[p.~1004]{charles:2009_object} ‘the looking behavior of young infants is, in fact, sensitive to the way an object leaves sight.’
 It is possible that there are signature limits here, and conceivable that these correspond to signature limits of a system of object indexes.
As more and more evidence concerning signature limits appears, we can be increasingly confident that infants’ abilities concerning physical objects are at least in part a consequence of a system of object indexes, as the \gls{CLSTX Conjecture} claims.


\section{Knowledge or Core Knowledge or \ldots?}
\label{sec:is-CLSTX-core-knowledge}

The \gls{CLSTX Conjecture}  has an advantage over the \gls{Simple View} which is not widely
recognised.
This is that object indexes are independent of beliefs and knowledge states.
Having an object index pointing to a location is not the same thing as believing that an object is there.
And nor is having an object index pointing to a series of locations over time is the same thing as believing or knowing that these locations
are points on the path of a single object.
Further, the assignments of object indexes do not invariably give rise to beliefs; nor need such assignments match your beliefs \citep{Mitroff:2004pc,scholl:2007_objecta}.
To emphasise this point, consider once more this scenario
in which a patterned square disappears behind the barrier; later a plain black ring emerges (as depicted in \vref{fig:scholl_2007_fig4}).
You probably don't believe that they are the same object, but they probably do get assigned the same object index.
Your beliefs and assignments of object indexes are inconsistent in this sense: the world cannot be such that both are correct.

The independence of object indexes from beliefs  means that the \gls{CLSTX Conjecture} does not entail that four- and five-month-olds have beliefs or knowledge about physical objects. 
It does not generate the \gls{Simple View}’s incorrect predictions about four- and five-month-olds’ abilities concerning physical objects.
In between mindless behaviour and propositional attitudes like belief and knowledge there are things such as object indexes.

Compare the CLSTX Conjecture with the \gls{Core Knowledge View}.
We saw that mere appeal to core knowledge fails to generate useful predictions (in \cref{cha:core-knowledge}). 
By contrast, invoking object indexes and their signature limits generates many readily testable predictions.
But should we think of the CLSTX Conjecture as a competitor to the \gls{Core Knowledge View}? Or is it better understood as a development of this View?

\citet[chapter 3]{carey:2009_origin} presents the CLSTX Conjecture as a development of a view about Core Knowledge.
This makes sense insofar as a system of object indexes may have some of the features used in characterising core knowledge, such as information encapsulation.
And a system of object indexes is probably largely unchanging through development, just as core knowledge is supposed to be.

There is a cost to identifying object indexes with core knowledge, however.
According to Carey, core knowledge is a ‘type of conceptual structure \ldots\ that differs systematically from \ldots\ sensory/perceptual representation’ \citep[][p.~10; see \cref{sec:core-knowledge} for discussion]{carey:2009_origin}.
Since object indexes are a broadly perceptual phenomenon, this claim appears to be in tension with thinking of the CLSTX Conjecture as a development of the Core Knowledge View.

Whether or not we should regard a system of object indexes as a core system (see \cref{sec:core-knowledge-minimal-view} for more on this), we should not neglect a key virtue of the CLSTX Conjecture: 
like the Simple View, it does not require us to postulate novel kinds of representation or knowledge.
Instead it is an attempt to explain infant cognition and its development by appeal to systems that are already required for understanding adults’ abilities.
As we will see in every domain of knowledge, this strategy appears to be the winning one.
Much of the progress researchers have made in understanding the emergence of knowledge in development comes from identifying signature limits on infants’ capacities and making connections between these capacities and  cognitive systems that are already relatively well understood in human adults.

The discussion so far assumes that the \gls{CLSTX Conjecture} is correct. 
Just here we face an important complication: the Conjecture is false.


\section{Against the CLSTX Conjecture}
\label{sec:clstx-conjecture-incorrect}
We have seen that there is some evidence for the \gls{CLSTX Conjecture}, according to which infants’ abilities concerning physical objects are a consequence of their having a system of object indexes (see \cref{sec:signature-limits}).
% A currently open issue is whether infants’ abilities to track causal interactions could be a consequence of their having a system of object indexes.
% But we must face a more pressing issue.
But can the CLSTX Conjecture explain the puzzling pattern of discrepancies in four- and five-month-olds’ abilities concerning physical objects?

This puzzling pattern was identified in \cref{cha:linking-problem} and is summarized in  \vref{table:occlusion-vs-endarkening}.
Consider first occlusion events and the difference  between violation-of-expectation and manual search measures.
Why do five-month-olds fail to manifest their ability to track briefly occluded objects by initiating searches for them after they have been
fully occluded?
This is consistent with the \gls{CLSTX Conjecture} because object indexes are independent of beliefs
and do not by themselves support the initiation of action.
But what about the difference between occlusion and endarkening on \gls{violation-of-expectation} experiments?
Why do infants show evidence of tracking objects that disappear by occlusion but not by endarkening?
Whereas \glspl{object index} can be maintained through occlusion events,
it may be that endarkening a whole scene interferes with the maintenance of object indexes, perhaps because it involves destroying not only the object but the whole frame of reference.
This is merely a speculation (as far as I know).
But unless something like this is true, the CLSTX Conjecture will fail to explain the difference between occlusion and endarkening on \gls{violation-of-expectation} experiments.
And even assuming for the sake of argument that endarkening impairs the maintenance of object indexes, there is a more pressing challenge to the CLSTX Conjecture.

Why do infants succeed in searching for momentarily endarkend objects?
This finding seems to run directly against the CLSTX Conjecture.
Object indexes do not survive endarkening (or so we are assuming); and even if they did, they don’t enable you to initiate purposive actions.
So the CLSTX Conjecture provides two independent reasons to predict
that five-month-olds will {not} search for endarkened objects.
And yet they do.

In short, the \gls{CLSTX Conjecture} generates predictions we already know to be incorrect.
This fact requires us to abandon or revise the CLSTX Conjecture.
Considering how objects can be represented motorically will lead us to  a novel way of revising it.


\section{Motor Representations of Objects}
\label{sec:motor-representation-objects}
For adults, representing objects is not always a matter of knowledge, belief or \glspl{object index}.
They can also represent objects \glslink{motor representation}{motorically}.
(See \cref{sec:crude-picture} on the notion of motor representation.)

How an object is represented motorically depends on its affordances.
In general, you can only represent  an object motorically if you
can interact with it \citep{cardellicchio:2011_space}.
Putting an impenetrable barrier between you and an object---even a translucent barrier---means that
you can’t interact with it, and so the object is unlikely to be
represented motorically (\citealp{costantini:2010_where}; see \cref{fig:costantini_2010_fig1b}).

\addFigure{costantini_2010_fig1b}{Barriers, even when translucent, impair adults’ abilities to represent objects (such as this mug) motorically.
Source: \citet[figure 1B]{costantini:2010_where}}

Object indexes and motor representations  of objects plausibly have complementary features with respect to different modes of occlusion.
Whereas object indexes can survive occlusion but (we are assuming) not endarkening, motor representations survive endarkening but not occlusion events in which an occluder is an impenetrable barrier.%
\footnote{%
This assumption is not justified by experimental findings (as far as I know).
However, the mundane experience of being able to reach for and grasp things after the lights go off indicates that motor representations of objects survive endarkening.
And the fact that endarkening involves destroying the frame of reference indicates that object indexes are unlikely to survive it. 
}
This is illustrated in \cref{table:index-vs-motor}, which resembles \vref{table:occlusion-vs-endarkening}.


\begin{table}

  \begin{center}
  \footnotesize	%shrink for better spacing
  
  % add space between rows
  \extrarowsep=7pt
  
  \begin{tabu} to 0.8\linewidth {X[3,l] X[2,c] X[2,c]}
  
  \toprule
  
  & survive occlusion & survive endarkening
  \\
  \cmidrule(r){2-3}
  object indexes & \checkmark & $\times$
  \\
  motor representations & $\times$ (barrier) & \checkmark
  \\
  %
  \bottomrule
  %
  \end{tabu}
  \caption{Complementary features of object indexes and motor representations. Compare \vref{table:occlusion-vs-endarkening}.}
  \label{table:index-vs-motor}
  \end{center}	%careful -- position of this affects distance between table and caption(!)
  
  \end{table}
  
  \normalsize
  

Object indexes and motor representations of objects also plausibly differ in the kinds of response they enable.
As broadly perceptual phenomena, object indexes function to support the allocation of attention and so plausibly guide looking behaviours, including anticipatory looking; but they probably do not to enable the initiation of purposive actions.
By contrast, motor representations of objects function to support purposive actions; but there is no reason to suppose that they would cause an infant (or adult) to look in anticipation of an object’s reappearance (unless it were the target of an action, perhaps).
So where only object indexes underpin an infant’s (or an adult’s) responses, we would expect to observe anticipatory looking and perhaps success on  violation-of-expectation tasks (but see \cref{sec:objection-to-conjecture-o}); but we would not expect to observe manual searching.
Conversely, where only motor representations underpin an infant’s responses, we would expect to observe manual searching but not necessarily anticipatory looking or success on violation-of-expectation tasks (see \cref{table:index-vs-motor-b}).



\begin{table}

  \begin{center}
  \footnotesize	%shrink for better spacing
  
  % add space between rows
  \extrarowsep=7pt
  
  \begin{tabu} to 0.8\linewidth {X[3,l] X[2,c] X[2,c]}
  
  \toprule
  
  & enable success in violation-of-expectation tasks & enable initiation of purposive action
  \\
  \cmidrule(r){2-3}
  object indexes & \checkmark & $\times$
  \\
  motor representations & $\times$  & \checkmark
  \\
  %
  \bottomrule
  %
  \end{tabu}
  \caption{Object indexes and motor representations  support complementary types of response. Again, compare \vref{table:occlusion-vs-endarkening}.}
  \label{table:index-vs-motor-b}
  \end{center}	%careful -- position of this affects distance between table and caption(!)

\end{table}

\normalsize




The contrasts between motor representations and object indexes (as summarised in \cref{table:index-vs-motor,table:index-vs-motor-b}), and the relation between these and the pattern in infants’ abilities concerning physical objects,
 invites us to consider the possibility that four- and five-month-old infants’ capacities to track briefly unperceived objects may involve not only  object indexes but also motor representations of objects.




\section{Conjecture O}
\label{sec:revised-CLSTX-conjecture}

Given that objects can be represented motorically, what might we guess about infants’ earliest abilities concerning physical objects?
Consider  \emph{\gls{Conjecture O}}, a revision of the \gls{CLSTX Conjecture}:
%
\begin{quote}
  Five-month-olds’ abilities to segment objects, to represent them while briefly unperceived and to track their causal interactions
  are not grounded on belief or knowledge:
instead
they are consequences of the operations of
a system of \glspl{object index} \ldots\ \\
  \ldots\ and of a further, independent capacity to track physical objects which involves \glspl{motor representation} and processes.%
  \footnote{%
This conjecture was suggested by Krisztina Orban (personal communication, December 14, 2015).
%She would not endorse all the details of my presentation.
}
\end{quote}
%
Conjecture O generates some readily testable predictions.
It predicts that impairing or boosting four- and five-month-old infants’ abilities to represent objects motorically will modulate performance when they are tasked with initiating searches for briefly unperceived objects.
Such manipulations should, however, have no effect on these infants’ performance on standard anticipatory looking or violation-of-expectation tasks.

How could this prediction be tested?
Occlusion events often involve screens which are also barriers to action, so that occluding an object is confounded with removing it from the space of action.
We need barriers which do not occlude (as in \vref{fig:costantini_2010_fig1b}) and occluders which are not barriers to action.
Suppose we showed five-month-olds an object disappearing behind an occluder which was no barrier to action.
\gls{Conjecture O} predicts that these infants should search for the object in this case
(and, further, that their search behaviours should not be subject to signature limits associated with \glspl{object index}).

If a philosopher makes a prediction, she will invariably be able find a psychological study from the past which appears to support it.
I am no exception.

\citet{mccurry:2009_beyond} tested five-month-olds using a fringed screen through which you can reach but not see (see \vref{fig:mccurry_2009_fig1}).
This is the perfect way to separate barriers to action from occluders.
In a familiarization phase, infants were encouraged to reach through the screen.
If they did not, their hands were forcibly moved through the screen.
% ‘In the first familiarization trial, infants were shown the fringed-screen and were encouraged to reach through the fringe. If necessary, the experimenter gently guided the infant’s hand through the fringed-screen. Once the infant placed his or her hand through the fringed-screen twice, the trial ended.’
In the test phase, they contrasted two events.
In one, a red cube was moved behind the screen and, shortly afterwards, a red cube was moved out from the other side of the screen.
This was done in such a way as to suggest to  adults that a single red cube had moved behind the screen and back out again.
In the other event, a red cube was moved behind the screen and, shortly afterwards, a green spotty was moved out from behind the screen.
This should indicate to naive adults that the red cube is still behind the screen.
Following each event, \citeauthor{mccurry:2009_beyond} gave infants an opportunity to search behind the screen.
They were interested in whether infants would reach more frequently towards the screen after the red-cube/green-ball event.
And so they did.

\addFigure{mccurry_2009_fig1}{Deconfounding obstacles to action from obstacles to perception.  Source: \citet[figure 1]{mccurry:2009_beyond}}

% Here's the authors' description of their procedure: 'Once the ball came to rest at the right edge of the platform, the platform was pushed forward until the edge of the platform was directly in front of, and within easy reach of, the infant. In the second phase, the infant was allowed to search for 20 s. ' \citep{mccurry:2009_beyond}

Why did infants do this?
On the \gls{CLSTX Conjecture}, this pattern of behaviour makes no sense.
\Glspl{object index} are not sufficient to initiate purposive actions.
And, as we saw, a signature limit of object indexes is their disregard for featural information \citep{Carey:2001ue}.
So the CLSTX Conjecture provides two reasons for predicting, incorrectly, that five-month-olds should not search longer after the red-cube/green-ball event.
But understood in terms of \glspl{motor representation}, infants’ performance makes perfect sense.
The screen is an occluder that is no barrier for action, so no obstacle to representing the objects motorically.
And of course motor processes do not typically disregard featural information: they care deeply about the shapes of things, so we should expect a difference between the ball and the cube.
\citeauthor{mccurry:2009_beyond} have designed the perfect experiment to test Conjecture O.

We should not take \citeauthor{mccurry:2009_beyond}’s findings as providing evidence for \gls{Conjecture O}, of course. 
My interpretation of them is entirely post hoc.
\citeauthor{mccurry:2009_beyond} do not interpret them in this way.%
\footnote{%
Their ironically named paper misses an opportunity to take us ‘Beyond the Search Barrier’ by interpreting their fascinating findings as showing that 
‘when task demands are minimal \ldots\ young infants search reliably for hidden objects’ \citep[p.~435]{mccurry:2009_beyond}.
Earlier research provides plenty of evidence that task demands are not what prevent five-month-olds from searching for briefly unperceived objects (see \cref{sec:against-simple-view}).
Further, \citeauthor{mccurry:2009_beyond}’s experiment involves no direct manipulation of task demands and so could at most indirectly measure their effects.
}
But their experiment does illustrate how predictions of Conjecture O could be tested.

\gls{Conjecture O} generates many further predictions, meaning it should be readily refutable if false.
It predicts that, in the first six months of life, infants’ abilities concerning physical objects should show \glspl{signature limit} of object indexes and signature limits of motor representations.
Where objects cannot be represented motorically (because they are behind an impenetrable barrier, for example), infants’ abilities should be insensitive to featural information to the extent that object indexes are.
And where objects cannot be tracked using object indexes (because their disappearance is due to endarkening, for example), infants’ abilities should be sensitive to whether the objects reachable to the extent that motor representations are.


Can \gls{Conjecture O} explain the puzzling pattern of discrepancies in four- and five-month-olds’ abilities to track briefly unperceived objects (see \vref{table:occlusion-vs-endarkening})?
The gist of how it might do so is suggested by comparing these discrepancies with the complementary features of object indexes and motor representations summarised in \vref{table:index-vs-motor}.
Object indexes should survive objects’ disappearing behind occluders (including barriers) but not events like endarkening which disrupt perceptual frames of reference.
By contrast, motor representations should survive objects’ disappearing due to endarkening (you can reach for, and grasp, objects in the dark) but not their displacement behind barriers.

While Conjecture O is yet to be tested, it does generate some readily testable predictions and has the potential to explain the puzzling pattern of discrepancies in four- and five-month-olds’ abilities to track briefly unperceived objects.







\section{Conclusion: Paradox Lost}
\label{sec:paradox-lost}
Recall the \glslink{Linking Problem}{problem} we face.
Many studies provide evidence that, from around four months of age, infants can represent objects as persisting, and track their causal interactions (see \cref{cha:principles-object-perception,cha:simple-view}).
These studies involve a variety of methods, including \gls{habituation}, \gls{violation-of-expectation}, anticipatory looking, and manual search (as we saw in  \cref{sec:segmentation,sec:principles-object-perception,sec:persistence,sec:causal-interactions}).
The studies suggest that infants’ abilities can be \glslink{descriptively adequate}{described} by a set of principles describing how objects behave, the \gls{Principles of Object Perception}.
But what links these principles to infants’ minds?

An initially tempting position is the \gls{Simple View}: the principles specify things which infants know or believe, and they acquire beliefs about particular objects by making inferences from these principles.
Yet there is compelling evidence, also from studies involving a variety of methods, that infants do not know about unperceived objects’ locations or about causal interactions among objects until months or years later (as we saw in \cref{sec:against-simple-view,sec:furth-evid-against,sec:even-worse-for-the-simple-view}).
This evidence arises from cases in which groups of infants fail to do things which anyone with their abilities who had the relevant knowledge could hardly fail to do.
It provides strong reasons to reject the \gls{Simple View}.
The problem we face is to find an alternative (see \cref{cha:linking-problem}).
This is the \gls{Linking Problem}.
If not belief or knowledge, what does link the principles describing infants’ abilities to their minds?

\gls{Conjecture O} suggests  a solution to the \gls{Linking Problem}.
Four- and five-month-old infants do not have knowledge of, or beliefs about, particular physical objects.
Instead they have two things, motor representations and a system of object indexes, which enable them to segment objects, represent them as persisting and  track their causal interactions.
So whereas it might look as if an infant knows or believes that a particular object is behind an impenetrable screen, the truth may be that a system of object indexes in the infant has assigned an index to that object and the index points to a location behind the screen.
Or, if the screen is no barrier to action, it may be that she represents the object behind it motorically.
What connects the the \gls{Principles of Object Perception} to an individual mind is just this: a system of object indexes in the individual, and her motor system, operate broadly in conformity with the principles.

Our new view, \gls{Conjecture O}, is quite different from the \gls{Simple View}.
Whereas on the Simple View infants have beliefs about, or knowledge of, the \gls{Principles of Object Perception}, the new view does not entail that any general principles are represented at all.
And whereas on the Simple View four-month-old infants believe or know simple facts about particular physical objects, the new view attributes infants no such states.
This is why the new view does not generate the incorrect predictions which all but force us to reject the Simple View.


If  untrue, \gls{Conjecture O} should not be difficult to refute because it generates many testable predictions (as we saw in \cref{sec:revised-CLSTX-conjecture}).
It predicts that \glspl{signature limit} of object indexes, and of motor representation, will be manifest in infants’ abilities concerning physical objects in roughly the first six months of life.
% It predicts that, in the first six months of life, infants’ abilities concerning physical objects should show \glspl{signature limit} of object indexes and signature limits of motor representations.
% Where objects cannot be represented motorically (because they are behind an impenetrable barrier, for example), infants’ abilities should be insensitive to featural information to the extent that object indexes are.
% And where objects cannot be tracked using object indexes (because their disappearance is due to endarkening, for example), infants’ abilities should be sensitive to whether the objects reachable to the extent that motor representations are.


The pattern of findings that gives rise to the Linking Problem is sometimes called a paradox \citep[for example, ][p.~202]{Meltzoff:1998wp}.
Strictly speaking there is no paradox here.
To get a paradox we at least need to add the further assumption that only knowledge or belief could underpin infants’ abilities concerning physical objects.
This assumption is related to one about adults:
all mental phenomena and their intentional effects can be explained by appeal to belief, knowledge and other propositional attitudes.
You can see this assumption at work in Davidson’s philosophical inquiries.
 He attempts to characterise mental phenomena only by appeal to propositional attitudes like belief, desire and intention.
Thus pride is a propositional attitude \citep{Davidson:1976qd},
and perception is either a form of belief
or a merely causal process  \citep[p.~730]{Davidson:1999il}.
% *ref: on perception being causal, non-epistemic.
But the discovery of object indexes shows that recognising things in between merely mindless behaviour and propositional attitudes is essential not only for understanding infants’ minds but also for understanding adults’ minds too.
We do not ‘lack \ldots\ a satisfactory vocabulary’ for describing phenomena in between mindless nature and propositional attitudes (\citealp[contra][pp.~127--8]{Davidson:2001sm}, discussed in \cref{sec:the-challenge}).
Far from it.
Discoveries about object indexes, and about motor representation, have led not only to the vocabulary but even to theoretically coherent and empirically testable hypotheses.
Belief, knowledge and other propositional attitudes are merely a fraction of the mental phenomena.
Only ignoring this makes it tempting to think there might be a paradox.

Accepting \gls{Conjecture O} does not entail rejecting the \gls{Core Knowledge View}.
The characterisation of core knowledge (see \cref{sec:core-knowledge}) is open enough to allow that having core knowledge of objects might consist in having two things, namely a system of object indexes and a capacity to represent objects motorically.
As we saw (in \cref{sec:is-CLSTX-core-knowledge}), this means abandoning Carey’s idea that core knowledge is a novel type of conceptual structure.
Instead, postulating core knowledge does not (for all we have yet seen) require going beyond the crude picture on which the mind comprises epistemic, perceptual and motoric states (see \cref{sec:crude-picture}).
More importantly, accepting Conjecture O without rejecting the Core Knowledge View entails recognising that core knowledge of objects lacks unity: it is a hybrid phenomenon, comprising at least two distinct systems.
It also entails recognising that core knowledge generally is likely to lack uniformity across domains: 
whereas core knowledge of physical objects consists in object indexes plus motor representations of objects, core knowledge in other domains will surely involve quite different kinds of representations and processes.

What does \gls{Conjecture O} imply about how humans first come to know simple facts about particular physical objects?
On the \gls{Simple View}, such knowledge is already in place by four months of age.
By contrast, if \gls{Conjecture O} is right then it may be months or years later that knowledge about particular physical objects first appears in development.
This makes it coherent to guess that there are multiple foundations  for such knowledge.
One is having a system of object indexes, another is having a capacity to represent objects motorically. 
Getting from these to knowledge of physical objects may involve social interaction about objects, perhaps including learning to use tools.
%
% [*Aside on tool use:]
% Basic forms of tool use may not require understanding how objects interact
% (Barrett, Davis, & Needham; Lockman, 2000), and may depend on core cognition of
% contact-mechanics (Goldenberg & Hagmann, 1998; Johnson-Frey, 2004).
% Experience of tool use may in turn assist children in understanding notions of manipulation,
% a key causal notion (Menzies & Price, 1993; Woodward, 2003). Perhaps non-core capacities for
% causal representation are not innate but originate with experiences of tool use.

Can we discover more about the transition from infants’ earliest abilities concerning physical objects to the acquisition of knowledge of simple facts about physical objects?
Doing so depends on identifying a role for \glspl{metacognitive feeling}.

% There is just one small problem.
% Studying the \gls{CLSTX Conjecture} has shown us how the \gls{Linking Problem} might be solved.
% It has even given us some insight into when and how knowledge of physical objects may emerge in development.
% Much of this remains valid despite the fact that the CLSTX Conjecture is incorrect.


%%% Local Variables:
%%% TeX-master: "master"
%%% End:
