%!TEX TS-program = xelatex
%!TEX encoding = UTF-8 Unicode

%\def \papersize {a5paper}
\def \papersize {a4paper}
%\def \papersize {letterpaper}

\documentclass[12pt,\papersize]{extarticle}
% extarticle is like article but can handle 8pt, 9pt, 10pt, 11pt, 12pt, 14pt, 17pt, and 20pt text

\def \ititle {Motor Representation in Goal Ascription}
\def \isubtitle {}
\def \iauthor {}
\def \iemail{}
\def \iauthor {C. Sinigaglia* \& S. Butterfill**
\\ 
**Dipartimento di Filosofia, Università degli Studi di Milano, Italia
\\ 
***Department of Philosophy, University of Warwick, UK}
\def \iemail{corrado.sinigaglia@unimi.it}
%\date{}

\input{$HOME/Documents/submissions/preamble_steve_paper3}
%\author{}
%\author {S. Butterfill* \& C. Sinigaglia**
%\\ 
%**Department of Philosophy, University of Warwick, UK
%\\ 
%***Dipartimento di Filosofia, Università  degli Studi di Milano, Italia}
%\date{}
\usepackage{fancyhdr}
\pagestyle{fancy}
\lhead{\footnotesize \sc Sinigaglia \& Butterfill}
\rhead{\footnotesize \sc Motor Representation in Goal Ascription}

%comment these out if not anonymous:
%\author{}
%\date{}


%%for e reader version: small margins
%% (remove all for paper!)
%\geometry{headsep=2em} %keep running header away from text
%\geometry{footskip=1.5cm} %keep page numbers away from text
%\geometry{top=1cm} %increase to 3.5 if use header
%\geometry{bottom=2cm} %increase to 3.5 if use header
%\geometry{left=1cm} %increase to 3.5 if use header
%\geometry{right=1cm} %increase to 3.5 if use header

%for phil review
%\usepackage{setspace}
%\doublespacing
%\renewcommand{\footnotelayout}{\doublespacing}





\begin{document}

\setlength\footnotesep{1em}

\bibliographystyle{newapa} %apalike

%custom title to make abstract fit on first page
\maketitle
%{
%\begin{center}
%\LARGE
%{Motor Representation in Goal Ascription}
%
%\ % blank line
%
%\end{center}
%}

%\tableofcontents
%\title{}



% --- paste from google docs after here ---
% Motor Representation in Goal Ascription.tex
% To appear in Jan Coello & Martin Fisher (eds) ...

\begin{abstract}
\noindent
Goal ascription, the process of identifying outcomes to which purposive actions are directed, is indispensable for predicting others’ behaviours and understanding their minds.  
But which mechanisms underpin goal ascription?
This chapter examines several ways in which motor representations and processes are involved in different forms of goal ascription.
We argue that motor representations and processes matter for goal ascription in two ways.
They provide for capturing the directedness of an action to an outcome, and they shape the observers’ experiences of actions in such a way that these experiences reveal the goals of actions.
The occurrence of motor representations in action observation thereby makes available, independently of any prior knowledge of others’ mental states, a route to knowledge of the goals of their actions.
 

\ % blank line

\noindent
Key words: goal ascription, knowledge of action, motor representation, social cognition

\ % blank line

% \noindent
% Word count: *

\end{abstract}

\tolerance=5000


\begin{comment}
Understanding others -> goal ascription is part of (key feature) … (goal ascription can be independent from UO)
Goal ascription---goal-state ascription (like Csibra and some others)
Goal ascription may have the form of judging about goals (this means that not all propositional goal ascription is a form goal-state ascription)
Goal-state ascription: is it exclusively propositional?  No!  See implicit vs. explicit goal-state attribution. 

----- 3 steps: GA is not necessarily JaboutG: MR is relevant for both
from goal identification <<--->> to judgment about goals
		from goal identification to goal-state ascription
from goal-state identification to judgment about goal-state	


\end{comment}



\section{Introduction}
\label{sec:introduction}

Goal ascription is the process of identifying outcomes to which purposive actions are directed. 
Lucina is waving her arms in the street.
Her movements have many actual and possible outcomes, from attracting your attention through hailing a taxi to exercising her body.
Among these outcomes, one or more is a goal to which her action is directed.
You engage in goal ascription when you identify one or more outcomes as among those to which the movements are directed.
This identification may be key to understanding what Lucina believes about the street around her and what she desires and intends; it may also reveal opportunities to help or exploit her.
Which mechanisms underpin goal ascription?

In investigating this question it is useful to start with some distinctions.
The term ‘goal’ can be used in two ways.
It be used to refer to outcomes to which actions are directed.
This is the familiar, everyday sense of the term in which we talk about the goal of someone’s struggles.
The term ‘goal’ is also sometimes used to refer to a mental state of an agent in virtue of which her actions might be directed to an outcome. %\citep[p.\ 239]{goldman:2009_mirroring}.
In characterising goal ascription we used ‘goal’ as a term for outcomes to which actions are directed; we shall always use the term in this way and never for intentions or other mental states.
This makes it coherent to conjecture that some forms of goal ascription are more primitive than, and foundational for, mental state ascription.


%*this paragraph is too compressed; mention that Csibra on the teleological stance can be thought of as specifying functional goal ascription; also, emphasise that representational goal ascription is not (necessarily) mindreading, i.e. not a matter of ascribing mental states (representation comes in twice: do you represent directedness, and is the directedness grounded in a representation); maybe make use of Pierre Jacob & Goldman on goal ascription involving representation of (1) action, (2) outcome and (3) relation between them.
A second distinction concerns two forms of goal ascription, representational and functional (\citealp{gallese:2011_what}).
In \emph{representational goal ascription}, three things must be represented: an action, an outcome and the relation between this outcome and the action in virtue of which the outcome is a goal of the action.
% jacob:2012_sharing: ‘Ascribing a goal to an agent consists in forming a belief (or judgment) about an agent that he or she has a goal or is performing some goal-directed action.’
In \emph{functional goal ascription}, the relation between action and outcome is captured without being represented.
To say that this relation is \emph{captured} is to say that there is a process which ensures that the outcome represented is a goal of the action.
As both representing and capturing are ways of identifying goals, representational and functional ascription are both forms of goal ascription.

The two forms of goal ascription are important in different ways.  While philosophers have tended to focus on representational goal ascription \citep[see, for example,][]{jacob:2012_sharing,goldman:2009_mirroring}, developmental research is mostly focussed on functional goal ascription.  Under investigation is typically infants’ abilities to represent outcomes to which actions are directed, and not their abilities to represent the relation between an outcome and an action in virtue of which the outcome is a goal of the action \citep[e.g.][]{Gergely:1995sq, Woodward:1998dm, Southgate:2008el, Csibra:2003kp}. It is possible that functional goal ascription is a developmental precursor to representational goal ascription.  It is also possible that abilities to rapidly anticipate others’ actions and their precise timings generally require functional goal ascription, whereas representational goal ascription may be indispensable for understanding others as planning agents who need to get several things done \citep[compare][]{gallese:2009_motor,rizzolatti_functional_2010}.  Whether or not these conjectures are right, it is plausible that both forms of goal ascription, representational and functional, play a role in humans’ abilities to understand and interact with each other.


%One leading hypothesis about goal ascription is that it involves adopting a teleological stance \citep{Gergely:2003gb,Csibra:2003kp}.  To adopt this teleological stance is to follow a procedure for capturing the goal of an action; it is not necessary to represent the relation between the action and the goal in virtue of which the action is directed to that goal.  This is one account of functional goal ascription.


In this chapter we are concerned with both forms of goal ascription, their relations and the mechanisms that underpin them.
Our first aim is to argue that functional goal ascription can be achieved motorically; that is, there are cases of functional goal ascription in which the only representations involved are motor representations.  The argument will involve two key findings.  First, motor representations are not merely representations of bodily configurations and joint displacements, but also of outcomes such as grasping a particular ball, reaching for something and eating a certain food.  So some outcomes which feature in goal ascription can be represented motorically.  Second, there are motor processes in action observation which reliably ensure that outcomes represented motorically are outcomes to which observed actions are directed. So motor processes can capture the directedness of an action to an outcome. These two findings establish that functional goal ascription can be achieved motorically (see section \ref{sec:capture}).

Our second aim in this chapter concerns representational goal ascription. It may be tempting to assume that representational goal ascription works entirely independently of motor processes and representations.  However a body of evidence shows that, in some cases, motor representations facilitate representational goal ascription (see section \vref{sec:facilitate}). What this evidence does not yet explain, though, is how motor representations could facilitate representational goal ascription.  The answer, we shall propose, hinges on how actions are experienced.  Motor representations shape certain experiences, which in turn provide their subjects with reasons for judgements about the goals of particular actions (see section \vref{sec:experience}).  Understanding which mechanisms underpin representational goal ascription therefore requires understanding relations between judgement, experience and the motoric.  We shall discuss two distinct hypotheses about these relations.  On one hypothesis, motor representations enhance perceptual experiences of bodily configurations, joint displacements and sensory effects of actions (see \vref{sec:two_hypotheses}).  On the other, more radical hypothesis, motor representations enable experiences of actions as directed to particular outcomes (see \vref{sec:direct_hypothesis}). Whichever hypothesis is correct, motor representations can shape humans’ experiences and thereby help them to understand others’ actions and minds.



\section{Goals Are Sometimes Captured Motorically}
\label{sec:capture}

What can be represented motorically?
Some motor representations concern bodily configurations and joint displacements.
But this is not all that can be represented motorically. 
Accurate control of actions which involve interacting with objects requires motor representations of outcomes such as the grasping of a box or the kicking of a ball (\citealp{rizzolatti:1988_functional}; for a review, see \citealp{Jeannerod:1995bb}).
So some outcomes ascribed in goal ascription can also be represented motorically.
Further, such outcomes are represented motorically 
not only in performing actions
but also in observing actions directed to bringing them about 
	\citep{rizzolatti_functional_2010,rizzolatti_mirrors_2008}.
This tells us  that some motor representations represent the very outcomes which are ascribed in goal ascription.

% capturing ...
How does it ever come about that an outcome represented motorically in observing an action is an outcome to which that action is directed?
First consider a parallel question about performing rather than observing actions.
Suppose you are alone and not observing or imagining any other actions.
When performing actions in this situation, outcomes represented motorically in you will normally be among the goals of your actions; that is, they will be outcomes to which your actions are directed.  What ensures this correspondence between outcomes represented and goals?  It is the role of the representation in controlling how the action unfolds.  Representations of outcomes trigger planning-like motor processes whose function is to cause actions that will bring about the outcomes represented \citep{miall:1996_forward,arbib:1985_coordinated,rosenbaum:2012_cognition}.  Now return to observing rather than perform actions.  What ensures the correspondence between outcomes represented motorically and goals when you are merely observing another act?  

The answer, we suggest, is roughly that planning-like processes can be used not only to control actions but also to predict them.
Let us explain. 
There is evidence that a motor representation of an outcome can cause a determination of which movements are likely to be performed to achieve that outcome \citep[see, for instance,][]{kilner:2004_motor, urgesi:2010_simulating}. Further, the processes involved in determining how observed actions are likely to unfold given their outcomes are closely related, or identical, to processes involved in performing actions. 
This is known in part thanks to studies of how observing actions can facilitate performing actions congruent with those observed, and can interfere with performing incongruent actions \citep{
	brass:2000_compatibility, 
	craighero:2002_hand, 
	kilner:2003_interference, 
	costantini:2012_does}. 
Planning-like processes in action observation have also been demonstrated by measuring observers' predictive gaze.  If you were to observe just the early phases of a grasping movement, your eyes might jump to its likely target, ignoring nearby objects \citep{ambrosini:2011_grasping}. These proactive eye movements resemble those you would typically make if you were acting yourself \citep{Flanagan:2003lm}. 
Importantly, the occurrence of such proactive eye movements in action observation depends on your representing the outcome of an action motorically; even temporary interference in the observer's motor abilities will interfere with the eye movements \citep{Costantini:2012fk}.
These proactive eye movements also depend on planning-like processes; requiring the observer to perform actions incongruent with those she is observing can eliminate proactive eye movements \citep{costantini:2012_does}.  This, then, is further evidence for planning-like motor processes in action observation. 

So observers represent outcomes motorically and these representations trigger planning-like processes which generate expectations about how the observed actions will unfold and their sensory consequences.  
Now the mere occurrence of these processes is not sufficient to explain why, in action observation, an outcome represented motorically is likely to be an outcome to which the observed action is directed.
To take a tiny step further, we conjecture that, in action observation, motor representations of outcomes are weakened to the extent that the expectations they generate are unmet \citep[compare][]{Fogassi:2005nf}. 
A motor representation of an outcome to which an observed action is not directed is likely to generate incorrect expectations about how this action will unfold,
and failures of these expectations to be met will weaken the representation.
This is what ensures that there is a correspondence between outcomes represented motorically in observing actions and the goals of those actions (see figure \vref{fig:capturing}).



\begin{figure}
\begin{center}
\includegraphics[width=0.8\textwidth]{fig_capturing.png}
\caption{
\label{fig:capturing}
	How motor processes can capture goals to which observed actions are directed.
}
\end{center}
\end{figure}



Let us return to comparing performing with observing actions.  In both cases, motor representations of outcomes trigger planning-like processes which generate predictions.  Failure of these predictions requires different responses in observing and acting, however.  In performing actions the failure of predictions is a signal that the movements selected are inappropriate to achieving the outcome represented: the movements should be corrected to better fit this outcome, for example by adjusting the trajectory of a limb in reaching (or, in more complicated cases, subplans should be revised).  By contrast, in observing actions the failure of predictions is a signal that the outcomes represented are not goals of the action observed: the representation of the outcome should be adjusted to better fit the movements observed.%
\footnote{
\citet[p.\ 207]{jacob:2012_sharing} notes that goal ascription requires representations of outcomes which have mind-to-world directions of fit.  What we have just seen, in effect, is that motor representations considered as part of system involving processes that capture the directedness of an action to an outcome have mind-to-world directions of fit, as required.  (Note that we are not asserting outright that motor representation has a mind-to-world directions of fit, of course.  Representations have directions of fit relative to the systems in which they occur.)
}  

Our question is, Which mechanisms underpin goal ascription?  We have just seen that functional goal ascription can be achieved motorically: that is, in some cases of functional goal ascription the only representations involved are motor representations.  
It is not just that in observing an action there is sometimes a motor representation of an outcome which is a goal of the action.  It is also that planning-like motor processes capture the directedness of the action to this outcome; that is, they ensure that the outcome represented is a goal of the action.  This is functional goal ascription.%
\footnote{
Note that we are not claiming that all functional goal ascription involves motor representations and processes; our argument shows only that functional goal ascription can be achieved motorically.
}

Not all goal ascription is functional goal ascription, of course.  Sometimes people make judgements, perhaps expressed verbally, to the effect that some outcome is among the goals of a particular action.  This is an instance of representational goal ascription.  Could representational goal ascription also be achieved motorically?


\section{Why Representational Goal Ascription Cannot Be Motoric}
Goal ascription involves more than than representing an outcome to which an action is in fact directed: it involves identifying that outcome \emph{as} a goal of the action.
In representational goal ascription, identifying means representing.
Because there is no motor representation of the directedness of an action to an outcome, motor representations cannot suffice for representational goal ascription.

Or can they?
We saw that outcomes are represented motorically; 
why couldn't the directedness of an action to an outcome also be represented  motorically?
One way to explain the notion of goal-directedness is in terms of intention: for an action to be directed to an outcome is for the action to be appropriately related to an intention whose content specifies that outcome (see, for instance, \citealp{Searle:1983tx}), or else to an intention specifying an appropriately related outcome \citep{Bratman:1984jr}. Given this explanation, no motor representation could represent the directedness of an action to an outcome.  After all, no motor representation represents an intention.  

There are, of course, other views about how actions are related to their goals.  For instance, some have argued that the directedness of some actions to particular outcomes can be explained in terms of motor representation and not only in terms of intention \citep[e.g.][]{butterfill:2012_intention}.  This changes nothing, however.  For motor representations do not represent any representations at all.  So the directedness of an action to an outcome still cannot be represented motorically.

The directedness of an action to a goal can be understood in ways that involve neither intention nor any representation at all.  For instance, consider the idea that an action's being directed to an outcome consists in its having the function of bringing about that outcome, where function might be construed teleologically.%
\footnote{
Teleological accounts of function, and of the application of this notion to understanding goal-directed action, have been extensively developed (see 
	\citealp{Godfrey-Smith:1996ln}; 
	\citealp{Millikan:1984ib}; 
	%\citealp{Neander:1991to}, 
	\citealp{Price:2001hs}; and  
	%\citealp{Stout:1996le}, 
	\citealp{Wright:1976ls}).
} 
Even someone who accepted this idea would still have to conclude that motor representations do not represent the directedness of actions to outcomes. After all, motor representations no more represent functions than they represent representations.  

This is why motor representations cannot suffice for representational goal ascription. 
It is true that, in someone observing an action there can be motor representations of outcomes which, non accidentally, are the goals of the observed action.
But this is not enough.
There would have to be, in addition, a motor representation of an intention, or of a motor representation or of some other goal-state, or of a function. 
But there are no such motor representations.
This may make it tempting, initially, to suppose that motor representations have nothing at all to do with representational goal ascription.
The truth, however, is more interesting.




\section{Motor Representations Facilitate Representational Goal Ascriptions} 
\label{sec:facilitate}

Motor representations sometimes enable us to make faster or more accurate judgements about the goals of actions, as a variety of evidence shows.  
Some of this evidence comes from studies which manipulate motor expertise  \citep[e.g.][]{casile:2006_nonvisual} or
temporarily  lesion part of the motor cortex \citep{urgesi:2007_representation, michael:2014_continuous, Costantini:2012fk, pobric:2006_action, candidi:2008_virtual}: 
these interventions significantly impair goal ascription.
Further evidence that motor representations can facilitate goal ascription comes from research on apraxia. 
In one study subjects were asked to identify goals such as the cutting of some paper or the use of a straw for drinking on the basis of the sounds actions produced. 
Subjects with limb apraxia showed an impairment in recognising the goals of hand-related actions whereas subjects with buccofacial apraxia were impaired in recognising the goals of mouth-related actions; but no subjects showed a general impairment in recognising sounds and their significance (\citealp{pazzaglia:2008_sound_}; see also \citealp{rizzolatti:2014_cortical}). 
These links between motor deficits and judgements about the goals of actions provide evidence that motor representations can facilitate goal ascription.

But how do motor representations ever facilitate goal ascription?
We aim to answer this question in the rest of this chapter.
Note that in asking this question we are not assuming that motor representation always occurs when goals are representationally ascribed.
Even if motor representations only rarely facilitated representational goal ascription, 
fully understanding how representational goal ascription works would require understanding how motor representations ever facilitate it.  

We conjecture that where motor representations facilitate representational goal ascription,  planning-like motor processes normally sustain a motor representation of the outcome which is the goal, or of an outcome matching%
\footnote{
Two outcomes \emph{match}  in a particular context just if, in that context, either the occurrence of the first outcome would normally constitute or cause, at least partially, the occurrence of the second outcome or vice versa.
}
 the goal. 
This conjecture is the key to understanding how motor representation can facilitate goal ascription.
But how exactly does it help?

Accepting the conjecture means we are confronted with an obstacle.
In the above-mentioned experiments demonstrating that motor representation can facilitate representational goal ascription, the goal ascriptions often take the form of judgements, verbally articulated, that some outcome is the goal of a particular action.
The conjecture implies not merely that motor representations influence these judgements but that the motor representations have content-respecting influences on them.  
To illustrate, it is motor representations of grasping outcomes that, according to the conjecture, facilitate ascription of goals involving grasping.
How could motor representations have content-respecting influences on 
judgements?

One familiar way to explain content-respecting influences is to appeal to inferential relations. To illustrate, it is no mystery that your beliefs have content-respecting influences on your intentions, for the two are connected by processes of practical reasoning. But motor representation, unlike belief and intention, does not feature in practical reasoning. Indeed, there is no inferential process which combines motor representations and judgements---that is, motor representation is inferentially isolated from judgement. 
How else could motor representations sometimes have content-respecting influences on judgements about the goals of observed actions? 


\section{Experiences Revelatory of Action}
\label{sec:processes}
\label{sec:experience}

Broadly, our proposal will be that content-respecting influences of motor representations on judgements go via experience.
Motor representations sometimes influence our experiences when we observe actions, and these experiences in turn provide their subjects with reasons for judgements about the actions.  
Further, which reasons an experience provides depends in part on the contents of the motor representations influencing it.  (To save words, let us say that a representation \emph{shapes} an experience to mean that the representation influences the experience in such a way that which reasons the experience provides depends in part or whole on what the representation represents.)  
It is thus experience that ties judgement to motor representation.
This, anyway, is the proposal we shall elaborate and defend here.

Observing actions sometimes involves \emph{experiences revelatory of action}, that is, experiences which provide the subject of experience with reasons for judgements about the goals of actions someone (another or herself) is performing.
Suppose, for instance, that you are observing someone in motion.  You may be able to judge on the basis of observation that she is reaching for a particular box, or that she is attempting to move the box to somewhere else. You may be able to make such judgements about the goals of her actions because your experiences provide you with reasons for them.  It is such experiences that we are calling revelatory of action.
When an experience provides its subject with reasons for judging that an action is directed to a particular goal, we will say  that the experience \emph{reveals} this goal.

Might experiences revelatory of action be shaped by motor representations of outcomes?
Motor representations can certainly influence perceptual processes (\citealp{bortoletto:2011_action}; 
%for a recent review of some of the evidence concerning motor influences on perceptual processing, see 
\citealp{halasz:2012_unconscious}). 
Further, motor representations can influence what you experience when you observe an action.  
This has been shown by investigating how such experiences are affected  both by your expertise \citep{repp:2009_performed} and also by what you are doing while observing \citep{zwickel:2010_interference}.  

To show that those experiences which are revelatory of action can be shaped by motor representations of outcomes, consider some further evidence.  
Suppose someone observes a hand in motion.  
How can we tell whether she has an experience revelatory of a hand action?
One way to show that she does not would be to show that her experience involves no sensitivity to biomechanical constraints on hand movements.
Accordingly it is possible to determine whether someone has an experience revelatory of a particular action by measuring their sensitivity to such constraints \citep{shiffrar:1990_apparent}.  This opens up the possibility of investigating what happens when the capacity to represent an action motorically is impaired.  \citet{funk:2005_hand} did just this.  They compared individuals who could not represent a hand action motorically with individuals who could.  They found that only those who could represent the hand action motorically were sensitive to biomechanical constraints on hand movements.  This is evidence that the occurrence of an experience revelatory of action depends on the capacity to represent the action’s goal motorically.  Which outcomes are represented motorically can influence which goals are revealed in experiences revelatory of action.

We propose that experience is the key to explaining how motor representation can have a content-respecting influence on judgement. It is experience that connects what is represented motorically to what is judged.  You may object that even if this is right, it hardly counts as explaining how those content-respecting influences arise.  Surely the relation between motor representation and  experience is no easier to understand than the relation between motor representation and judgement was?  


\section{On the relation between motor representations and experiences}
\label{sec:two_hypotheses}
\label{sec:indirect_hypothesis}
In the previous section, we argued that some motor representations are related to some experiences revelatory of action in this way: which goals the experiences reveals depends, wholly or in part, on which outcomes are represented motorically.
This takes us one small step towards understanding how motor representations could have content-respecting influences on judgements about the goals of actions despite the inferential isolation of motor representations from judgements. 
But it raises more questions than it answers.
First, how do motor representations shape experiences?
And, second, how do these experiences provide reasons for judgements?  

In what follows we shall elaborate two hypotheses which give conflicting answers to these questions.  Our aim is not to decide between the hypotheses but rather to defend our proposal about the role of experience in linking motor representation to judgement by showing that it is plausible on either hypothesis.  

On one hypothesis, \emph{the Indirect Hypothesis}, experiences revelatory of action are all experiences of bodily configurations, of joint displacements and of effects characteristic of particular actions.  Some such experiences are influenced by motor representations in ways that reliably improve veridicality.  And such experiences can provide reasons for judgements about the goals of actions providing that the subject knows, or is entitled to rely on, certain facts about which bodily configurations, joint displacements and  sensory effects are characteristic of which actions.%
\footnote{
The Indirect Hypothesis is inspired by, and consistent with, views defended in \citet{Csibra:2007fy} and \citet{Wilson:2005qu}.  However, these papers do not discuss our question about content-respecting influences of motor representation on judgement and we are not suggesting that their authors would endorse the Indirect Hypothesis.  
}

On the other hypothesis, \emph{the Direct Hypothesis}, some experiences revelatory of action are experiences of actions as directed to particular outcomes.  In observing action we experience not only bodily configurations, joint displacements, sounds and the rest but also goal-directed actions.  Further, such experiences stand to motor representations somewhat as perceptual experiences stand to perceptual representations.  These experiences provide reasons for judgements in something like the way that, on some views, perceptual experience of a physical object might provide a reason for a judgement about that object; or so the Direct Hypothesis claims.%
\footnote{
In formulating the Direct Hypothesis we are inspired by \citet{Rizzolatti:2001ug} and \citet{rizzolatti_mirrors_2008}.  
%That there are experiences of goal-directed actions is suggested by Garbarini and colleagues when they interpret interference effects due to bimanual coordination in patients with anosognosia for hemiplegia as due to (perhaps illusory) experiences of action (\citeyear[][pp.\ 1, p.\ 3]{garbarini:2012_moving}). 
}

Consider the Indirect Hypothesis first. According to this hypothesis, experiences revelatory of action are not special with respect to what they are experiences of: they are experiences of sounds, joint displacements and the rest. But how is this consistent with the claim that motor representations sometimes make possible experiences revelatory of action by virtue of having content-respecting influences on experiences?   A possible answer appeals to a view about the control of action.
In planning one's own actions it is sometimes useful to be able to identify and predict configurations of one's body parts, joint displacements and the likely sensory consequences of these. 
One can then use this information in monitoring action and perhaps also in identifying constraints and opportunities in planning what to do next. Accordingly, motor representations may be inputs to mechanisms that generate sensory expectations \citep{wolpert:1995internal}; and these expectations may affect how one experiences one's own bodily configurations, displacements of one’s own joints and the sensory consequences of these (\citealp{blakemore:2002_abnormalities}).  
Now, as we have seen, motor representations and the associated planning-like processes also occur when observing an action; and these generate expectations which influence how one experiences others' bodily configurations and joint displacements and their consequences.
Perhaps, then, one role for motor representations is to generate sensory expectations, and thereby to enhance perceptual experiences---sometimes enhancing them in such a way that they would not otherwise have provided reasons for judgements about the goals of actions.  

But how could motor representations enhance perceptual experiences?  Occlusion and other factors mean that observers typically have limited perceptual information about others' bodily configurations, joint displacements and the sensory consequences of these.  Further, the effects of observed actions are often partially obscured, surrounded by distractors or otherwise difficult to identify. As Wilson and Knoblich argue, these reflections suggest that the influences of motor representations on perceptual experiences may play a significant role in helping `to fill in missing or ambiguous information' (\citeyear{Wilson:2005qu}, p.\ 463).  
Since the effects of motor representations on experiences are the effects of an additional source of information,
it is plausible that in filling in information they reliably increase the probability that perceptual experiences concerning observed actions are veridical.  


This allows us to see that the Indirect Hypothesis is consistent with motor representations having content-respecting influences on judgements about goals.  
A motor representation of a particular outcome enables an observer to perceptually experience another's bodily configurations or joint displacements or some consequences of these more accurately than would otherwise be possible: it enhances perceptual experience.   
Crucially for our purposes, the enhancement is not general.  Rather, the motor representation normally enhances only aspects that are predictable given that the observed action is directed to the outcome represented.  Now the experience of these aspects does not, all by itself, provide the subject with reasons for judgements about the goal of an action.  Taken in isolation, the experience provides reasons only for judgements about bodily configurations and joint displacements and their sensory effects.  However, the subject of experience may know that some or all of these things are associated with actions directed towards a particular goal. In this case, the experience provides its subject with reasons for a judgement about a goal of the observed action; and it does so in something like the way that an experience of smoke coming from the kitchen may give you a reason to judge that there is a fire  (see Figure \vref{fig_indirect_hypothesis}). So the Indirect Hypothesis.

\begin{figure}
\begin{center}
%*todo: straighten arrow
\includegraphics[width=\textwidth]{fig_indirect_hypothesis.png}
\caption{
\label{fig_indirect_hypothesis}
	How motor representation could facilitate judgements about the goals of actions given the Indirect Hypothesis.
}
\end{center}
\end{figure}

In considering the Indirect Hypothesis we have encountered a possible explanation of how some motor representations could have content-respecting influences on judgements. But perhaps there is more to the story.

\section{The Direct Hypothesis}
\label{sec:direct_hypothesis}

The Direct Hypothesis is a simpler and more radical alternative to the Indirect Hypothesis. 
(It is more radical in that it is inconsistent with widely held views about what can be experienced.)
The Direct Hypothesis starts from the premise that in observing an action we sometimes experience not only bodily configurations and joint displacements and their sensory effects but also the action as directed to a particular outcome.  
In observing someone grasp a box, for instance, we might experience not only the movements of hand and mug but also the grasping of the mug.%
\footnote{
Note that the Direct Hypothesis does not entail that all goal-directed actions can be experienced.  The range of actions that can be experienced may be quite narrow, and may vary from subject to subject and context to context depending on factors such as subjects' expertise, current activities and bodily configuration 
(compare \citet{aglioti_action_2008}, \citet{costantini:2012_does} and \citet{ambrosini:2012_tie}).
}
Further, the experience of an action as directed to a particular outcome is made possible by a motor representation of that outcome.
Finally, this experience typically provides a reason for judging that the action is directed to that goal (see Figure \vref{fig_direct_hypothesis}).  So the Direct Hypothesis.

If the Direct Hypothesis is correct, there is a partial analogy between perceptual and motor representations.  Some representations involved in perceptual processes have content-respecting influences on judgements. This is arguably possible in part because some of these perceptual representations enable or constitute experiences of the very objects and properties that they represent. Similarly, some motor representations have content-respecting influences on judgements in part because they enable or constitute experiences of the very goal-directed actions represented motorically. Of course, proponents of the Direct Hypothesis should not claim that motor representations stand to experiences exactly as perceptual representations do.  But their view is that motor representations stand to experiences of actions somewhat as perceptual representations of persisting physical objects stand to experiences of objects. 


\begin{figure}
\begin{center}
\includegraphics[width=\textwidth]{fig_direct_hypothesis.png}
\caption{
\label{fig_direct_hypothesis}
	The Direct Hypothesis.
}
\end{center}
\end{figure}

Note that accepting the Direct Hypothesis requires rejecting the Indirect Hypothesis as we have formulated it.  However, a proponent of the Direct Hypothesis can allow that the mechanism identified by the Indirect Hypothesis does partially explain how motor representations facilitate judgements about the goals of actions.  Her claim is only that a full explanation will require, in addition, experiences of particular actions as directed to certain outcomes which are underpinned by motor representations of those outcomes.

Is it possible to distinguish experimentally between the Direct and Indirect hypotheses? One approach would be to test whether it is possible to vary which action someone experiences while holding fixed her perceptual experiences of bodily configurations and  joint displacements and their sensory effects.  The Indirect Hypothesis predicts that no such variation is possible.  After all, on that hypothesis there is nothing action-related in experience over and above the bodily configurations,  joint displacements and their sensory effects.
By contrast, the Direct Hypothesis is consistent with the existence of such variation: on this hypothesis, experience of action is distinct from experience of bodily configurations and the rest.
Accordingly, evidence that it is possible to vary which action someone experiences while holding fixed her perceptual experiences of bodily configurations and  joint displacements and their sensory effects would be evidence that favoured the Direct Hypothesis over the Indirect Hypothesis.  So the two hypotheses are not merely conceptually distinct but also empirically separable.

Our aim was to understand how motor representations could have content-respecting influences on judgements despite their inferential isolation (see section \vref{sec:facilitate}).  Our proposal is that the influence goes via experience.  Motor representations shape experiences revelatory of action which provide reasons for judgements.  This proposal may initially have seemed to complicate rather than explain because it raises two questions: one about relations between motor representations and experiences, and another about relations between experiences and judgements.  However, we have just shown that the proposal is compatible with either of two hypotheses about what can be experienced.  
So while the proposal leaves many questions open, it seems clear that understanding the role of motor representation in facilitating representational goal ascription will require understanding relations between motor representations, experiences and judgements.



\section{Conclusion}
\label{sec:conclusion}


Our question was, Which mechanisms underpin goal ascription?
We approached this question by distinguishing functional from representational goal ascription (see section \ref{sec:introduction}).
Our first step was to argue, in section \ref{sec:capture}, that functional goal ascription 
can be achieved motorically; that is, there are cases of functional goal ascription in which the only representations involved are motor representations.
Here our argument hinged on two claims for which, as we saw, there is an impressive variety of evidence.
First, sometimes goals to which observed actions are directed are represented motorically in the observer.
Second, such representations trigger, and are sustained by, planning-like processes in the observer.
These processes generate predictions about how the observed actions will unfold, and about their characteristic sensory consequences.
Because the motor representations of outcomes are sustained only to the extent that these predictions are met, it is not accidental that outcomes which are goals of observed actions are represented motorically in the observer.
This is exactly what functional goal ascription requires: the planning-like processes capture the directedness of the observed action to the outcome represented motorically.

%The existence of functional goal ascription that is achieved motorically shows that goal ascription can sometimes involve merely subdoxastic states. You can be engaged in functional goal ascription without knowing that you are, and there could be discrepancies between your functional goal ascriptions and your verbally expressed judgements about the goals of actions.  

How, if at all, are mechanisms underpinning functional goal ascription involved in representational goal ascription?  We are forced to ask this question by several pieces of evidence, cited in section \ref{sec:facilitate}, that motor representations and processes sometimes facilitate verbally expressed judgements about the goals of observed actions.  
This facilitation plausibly involves motor representations of particular outcomes facilitating judgements about outcomes matching, or identical to, those very outcomes.
That is, motor representations have content-respecting influences on judgements.
But there is an obstacle to understanding how this could occur.
Among states of types that could all feature in a single inferential process (such as beliefs, desires and intentions),
such content-respecting influences are readily explicable.  However, motor representations and judgements do not feature in a single inferential process: the former are inferentially isolated from the latter.
If no process of inference connects judgements to motor representations, what does?

Our answer was:  experience.  Sometimes a motor representation of a particular outcome influences an experience in such a way that the experience provides its subject with reasons for judgement about a goal of action, where this goal is, or matches, the outcome represented motorically.  This, we proposed, is how experience connects judgements to motor representations.  

As we noted, this proposal raises more questions than it answers.  How do motor representations shape experiences, and how do experiences provide reasons for judgements?  We considered two ways of answering these questions.  It may be that motor representations of outcomes enhance aspects of perceptual experiences which, independently, their subjects know are characteristic of actions directed to the outcomes represented motorically.  We called this the Indirect Hypothesis (see section \ref{sec:indirect_hypothesis}).  On this Hypothesis, bodily configurations, joint displacements and their sensory effects can be experienced, but actions as directed to particular outcomes cannot.  So motor representations can influence experience but there is nothing characteristically motor in the phenomenology of experiences of action.  What we called the Direct Hypothesis implies the contrary.  On this Hypothesis, there are experiences of actions as directed to particular outcomes. These experiences are underpinned by motor representations of matching outcomes; and they provide reasons for judgements about to which outcomes actions are directed simply by being experiences of actions as directed to those outcomes. So, on the Direct Hypothesis, the relation between motor representations and judgements partially resembles that between perceptual representations of persisting physical objects and judgements.  We did not aim to argue for one of these hypotheses over the other.  Our claim is only that the proposal about experience linking motor representation to judgement is plausible because it is compatible with either hypothesis.  

If this is right, motor representation is not only a mechanism for functional goal ascription but also provides a basis for representational goal ascription.  What are the consequences?

One consequence is that knowledge of the goals of others’ actions does not necessarily require  knowledge of the particular contents of intentions, beliefs, desires and other mental states.  
Because motor processes in action observation can capture the directedness of an action to an outcome, and because motor representations can shape experiences, there is a route to knowledge of the goals of others' actions which can be taken independently of individuating knowledge of the contents of their mental states.


Our has focus has been goal ascription but the view we have arrived at also has consequences for mental state ascription.  This is straightforward for intentions.  Since the goals of actions are more likely than other outcomes to be things that the agent of the action intends, identifying the goals of actions is often useful for understanding agents’ intentions.  
But goal ascription can also be indispensable for identifying beliefs and other epistemic states.
What someone believes or knows or is ignorant of generates predictions about what she will do.
These predictions are not typically about mere bodily configurations and joint displacements; they are predictions about which goals her actions will be directed to.  
To evaluate these predictions and so confirm or reject a hypothesis about an epistemic state, goal ascription is necessary.
We speculate that motorically achieved functional goal ascription plays a foundational role in mental state ascription too.


\bibliography{$HOME/endnote/phd_biblio}



\end{document}


