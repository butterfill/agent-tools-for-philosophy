\documentclass[review]{elsarticle}



%%%%%%%%%%%
\usepackage[utf8]{inputenc}
\usepackage{natbib}
\usepackage{varioref}
%\setcitestyle{aysep={}}  %philosophy style: no comma between author & year
%\bibliographystyle{references/chicago-16th} 
%%%%%%%%%%%


\usepackage{lineno,hyperref}
\modulolinenumbers[5]

\journal{}

%%%%%%%%%%%%%%%%%%%%%%%
%% Elsevier bibliography styles
%%%%%%%%%%%%%%%%%%%%%%%
%% To change the style, put a % in front of the second line of the current style and
%% remove the % from the second line of the style you would like to use.
%%%%%%%%%%%%%%%%%%%%%%%

%% Numbered
%\bibliographystyle{model1-num-names}

%% Numbered without titles
%\bibliographystyle{model1a-num-names}

%% Harvard
%\bibliographystyle{model2-names.bst}\biboptions{authoryear}

%% Vancouver numbered
%\usepackage{numcompress}\bibliographystyle{model3-num-names}

%% Vancouver name/year
%\usepackage{numcompress}\bibliographystyle{model4-names}\biboptions{authoryear}

%% APA style
\bibliographystyle{model5-names}\biboptions{authoryear}

%% AMA style
%\usepackage{numcompress}\bibliographystyle{model6-num-names}

%% `Elsevier LaTeX' style
%\bibliographystyle{elsarticle-num}
%%%%%%%%%%%%%%%%%%%%%%%

\begin{document}

\begin{frontmatter}

\title{Perceiving Expressions of Emotion: What evidence could bear on questions about perceptual experience of mental states?}

%% Group authors per affiliation:
%\author{Stephen A. Butterfill*}
%\address{Department of Philosophy, University of Warwick, CV4 7AL, UK}

%% or include affiliations in footnotes:
\author[mymainaddress]{Stephen A. Butterfill\corref{mycorrespondingauthor}}
\cortext[mycorrespondingauthor]{Corresponding author}
\ead{s.butterfill@warwick.ac.uk}

\address[mymainaddress]{Department of Philosophy, University of Warwick, CV4 7AL, UK}

\begin{abstract}
What evidence could bear on questions about whether humans ever perceptually experience any of another's  mental states,
and how might those questions be made precise enough to test experimentally?
This paper focusses on emotions and their expression.
%The aim is to identify evidence that could bear on hypotheses about perceptually experiencing others' mental states, and, in so doing, to formulate a modest version of the hypothesis that humans can sometimes perceptually experience others' mental states.
It is proposed that research on perceptual experiences of physical properties provides one  model for thinking about 
what evidence concerning expressions of emotion might reveal about 
perceptual experiences of others' mental states.
This proposal motivates consideration of the hypothesis that  
categorical perception of expressions of emotion 
occurs,
can be facilitated by information about agents' emotions, 
and
gives rise to phenomenal expectations.
It is argued that the truth of this hypothesis would support a modest version of the claim that humans sometimes perceptually experience some of another’s mental states.
Much available evidence is consistent with,
but insufficient to establish,
the truth of the hypothesis. 
We are probably not yet in a position to know whether humans ever perceptually experience others' mental states.
\end{abstract}

\begin{keyword}
Perceptual Experience\sep Social Cognition\sep Categorical Perception\sep Phenomenal Expectations
%\MSC[2010] 00-01\sep  99-00
\end{keyword}

\end{frontmatter}

\linenumbers


\section{Introduction}
The trainer's mind had wandered so far from the match that he had no idea who was winning until Blanche's howl of victory rang through the stadium, seizing his attention.
Her howl and accompanying aerial contortions revealed much about the category and intensity of her emotions.
Afterwards he said he could see she'd won, could see her ecstasy in winning.
How could we find out whether this is all merely a way of speaking or in part a literal description of a  perceptual experience?
More generally, what evidence might bear on questions about whether humans ever perceptually experience any of another's  mental states?

%In aiming to identify evidence for or against a hypothesis 
In asking what evidence might bear on such questions
I am presupposing, of course, that they have not yet been answered decisively.
This is reasonable given recent interest in arguments whose modest aim is to show only 
that the view
that humans can perceptually experience some mental states of subjects other than themselves
is not obviously false \citep{smith:2010_seeing,mcneill:2012_embodiment,mcneill:2012_seeing}.

In asking what evidence might bear on questions about perceptual experience of mental states, I am also presupposing that answering such questions will involve some people doing experiments.
This may initially seem controversial given what may appear to be narrowly philosophical arguments for the view that humans can perceptually experience some mental states.
However, those arguments quite often rest on unargued conjectures about the existence of certain perceptual states, looks, visual similarities or the like.
To illustrate, \citeauthor{smith:2013_phenomenology} offers an argument which hinges on the conjecture that there are `visual states ...\ [which] possess content that matches the causal profile' of states such as a person's happiness (\citeyear[p.\ 17]{smith:2013_phenomenology}; on what this amounts to, see \citealp[§4]{smith:2010_seeing}).
%It would be good to know how we could test the conjecture that such visual states exist.
His argument and some other careful narrowly philosophical discussions might be charitably interpreted as providing frameworks for understanding claims about perceptual experience and mental states rather than as offering grounds for accepting or rejecting them.%
\footnote{
Compare \citet[p.\ 18]{smith:2013_phenomenology}: `I have not offered a robust defence of the phenomenological claims …, motivating them rather on intuitive grounds.'
}
There may much that can be discovered without seeking experimental evidence. 
But to know whether the claims that matter most are true or false we will probably have to find experimental evidence that bears on questions about whether any humans ever perceptually experience others' mental states.
%will probably require identifying ways to test them.

%
%identifying existing evidence relevant to hypotheses about perceptual experience of mental states
%and 
%generating predictions testing which would provide further evidence.

%
%
%I shall compare this hypothesis about mental properties with 
%a corresponding hypothesis about physical properties such as momentum or solidity.
%Whereas the hypothesis about  mental states has yet to be investigated in detail, hypotheses about physical properties have been tested and there is now some evidence in their favour (see Section \vref{sec:physical_properties}).
%Comparing the two is useful because, as we shall see, breakthroughs in the physical case provide a model for investigating the mental.
%

%***The attempt to link hypotheses about perceptual experience to evidence about perceptual mechanisms has two virtues; first, challenges some assumptions (e.g. about modality); second, allows us greater understanding of limits on what can be perceived and how they arise.

Note that the claims under consideration here are about perceptual experience; they are not claims about psychological mechanisms, nor about epistemology.  (These distinctions are explained in the Introduction to this Special Issue, \citealp{michael:2015_intro}.)
These topics are probably related, however.
One lesson from research on physical properties is that discoveries about psychological mechanisms can inform views about what can be perceptually experienced (see Section \vref{sec:physical_properties}).
And it is possible that 
answering questions about perceptual experience will somehow inform investigation of 
epistemological questions about whether perceiving is sometimes a way of knowing facts about another's mental states.
Nevertheless, 
my focus in this paper is perceptual experience.

%Distinguishing these is necessary because 
%some schools of philosophy treat questions about experience as if they were unconstrained by discoveries about mechanisms,
%and some follow Dretske  questions about whether perceiving is a way of knowing as if they were un
%Perhaps the hypothesis about perceptual experience sits in the middle.
%Possessing a psychological mechanism whose function concerns Xs is necessary (but not sufficient) for perceptually experiencing Xs, and such experiences are necessary for perceiving to be a way of knowing facts about Xs.
%Successes in answering questions about whether humans perceptually experience certain physical properties indicates that discoveries about psychological mechanisms can bear on questions about perceptual experience, as we will see.
%(You might suppose this too uncontroversial to mention, but some schools of philosophy treat questions about experience as if they were unconstrained by discoveries about mechanisms.)

% CUT - another way of saying the same thing.
%Before starting it is necessary to specify the views under discussion.
%In both cases, the physical and the mental, we need to distinguish a psychological claim, a phenomenological claim and an epistemic claim.
%The psychological claim is that representations of these properties, anger or mass, occur in perceptual processes.
%The phenomenological claim is that humans (or some other group) can perceptually experience such properties.
%And the epistemic claim is that, for humans (or whoever), perceiving such properties is a way of knowing .



The aim of this paper is to identify evidence linked to expressions of emotion that bears on claims about perceptually experiencing others' mental states, and, in so doing, to formulate a modest interpretation of the view that humans can sometimes perceptually experience others’ mental states.
I  start by describing the problem which motivates this work (in section \ref{sec:how_not_to}) 
and then outline some research on perceptual experiences of physical properties (in section \ref{sec:physical_properties})
which will serve  as a model for thinking about 
what evidence concerning expressions of emotion might reveal about 
perceptual experiences of others' mental states (in sections\ref{sec:cp_emotion} onwards).


\section{The Problem}
\label{sec:how_not_to}
Some researchers appear to hold that mere verbal reports and explicit ratings are sufficient to show that 
humans can perceptually experience some mental properties of subjects other than themselves 
(see, for example, \citealp[p.\ 299]{Scholl:2000eq}; 
\citealp[p.\ 135]{Schlottmann:2006dp}).
This would make it easy to confirm the hypothesis about the perception of mental states.
After all, \citet[p.\ 257]{Heider:1944ts} famously demonstrated that people  will, in describing what they see, spontaneously attribute  motives and needs to animated polygons.
As this suggests, many people are disposed to say things which, if literally true, would imply that they sometimes perceptually experience others' mental states.  Indeed, people will spontaneously say such things even when presented with stimuli which manifestly do not involve subjects of mental states.

But can we really support claims about perceptual experience merely by measuring verbal reports?
In the right situations people will also talk about seeing properties related to the market values or historical origins of physical objects.
To infer from this that what humans can perceptually experience extends beyond the narrowly physical to include features related to an object's scarcity, or that it extends beyond the present to include the past, would make sense only given an extremely broad notion of perceptual experience.
Heider and Simmel were clearly operating with such a broad notion, for they stipulated that they use the word `perception' `in the sense of cognitive response, i.e.\ as covering all cognitive processes which follow the exposure of a set of receptors to stimulation' \citep[p.\ 243, footnote 1]{Heider:1944ts}.
This is a way of saying that they are not concerned with perception at all.

Compare the question, can humans perceptually experience categorical colours of physical objects in addition to particular shades?
Can humans, for example, perceive the greenness of an unripe tomato where greenness is a property the tomato shares with a blade of grass and a leaf?
Answering this question depends in part on complex issues about when and why verbal interference affects discrimination between categorical colour properties \citep{Roberson:2000ge,Wiggett:2008xt}.
Clearly what people say about their experiences is not decisive here.
But if such verbal reports alone are not sufficient to decide questions about perceptual experience of colour, they surely cannot decide questions about perceptual experience of mental states either.

This leaves us with a problem.
Given that mere verbal reports and explicit ratings alone are not sufficient to establish claims about perceptual experience,
how else could the hypothesis that humans sometimes perceptually experience others' mental states be tested?





\section{Phenomenal Expectations (A Preliminary Step)}
\label{sec:phenomenal_expectations}
In working out what kind of evidence might be relevant to answering questions about whether humans ever perceptually experience others’ mental states, it is useful to consider physical properties. 
%Comparison with certain physical properties is useful because, we shall see, 
%there is more evidence relevant to hypotheses about perceptual experience and physical properties. 
%Successes in the physical case also provide some hints about where to look for perceptual experiences of mental states.
%
In tracking objects' movements, which physical properties of them can humans perceptually experience?
In particular, can they perceptually experience properties such as solidity, velocity or momentum?

As a preliminary to investigating both this question and its counterpart about mental states, it is useful to distinguish two ways in which things can feature in perceptual experience.
Suppose you saw the scene depicted in the right panel of Figure \vref{fig:expectations}.
Do you perceptually experience the parts of the shape behind the thumb?
In one sense you do not: after all, the thumb is blocking your view of them.
So accurately characterising your perceptual experience requires distinguishing some parts of the shape from others.
But we cannot simply say that you do not perceptually experience the occluded parts of the shape 
because your perceptual experience is not neutral on these parts of the shape.
If the thumb were removed to reveal the scene depicted in the left panel of Figure \ref{fig:expectations}, an expectation would be violated.
Such expectations are plausibly perceptual rather than being a matter of believing or knowing
in part because of the laws that govern them
and in part because they are judgement-independent.
As \citet{kellman:1983_perception} report, Michotte, Thines and Crabbe observed that people typically continue to report seeing a single large triangle behind the thumb even when they know that there isn't one there.  
You can cover and reveal the shape repeatedly without losing the incorrect expectation that there is  a triangle behind the thumb.
As this illustrates, things can be unperceived in one sense---because they are, for example, occluded---while being perceived in another sense.
To describe cases like this---cases where perceptual experience is not neutral concerning things which are in some sense unperceived---I will use the term \emph{phenomenal expectation}.%
\footnote{
 \citet[pp.\ 736--9]{smith:2010_seeing}   uses the term `perceptual anticipation' for phenomenal expectation.
In what follows I build on his idea that invoking phenomenal expectations is useful in interpreting views about perceptually experiencing mental states---I would add certain physical properties of objects too.
While terminological continuity would make my debts to Smith's discussion more obvious, 
his term `perceptual anticipation' is quite naturally used by others to refer to anticipation in perceptual processes irrespective of whether this has consequences for the phenomenal character of an experience 
\citep[e.g.][]{turk-browne:2010_implicit}.
The distinction between phenomenal expectations and anticipation in perceptual processes matters for reasons explained below.
}
%
When encountering the scene depicted in the right panel of Figure \vref{fig:expectations}, people typically have phenomenal expectations concerning the occluded parts of the shape.

\begin{figure}
\begin{center}
\includegraphics[scale=0.3]{fig/kellman_1983_fig2.jpg}
\end{center}
\caption{Phenomenal expectations.  \emph{Source}: Michotte et al (1964) via \citet[][figure 2]{kellman:1983_perception}
\label{fig:expectations}
}
\end{figure}
 
Phenomenal expectations are not limited to static features of objects.
To illustrate, imagine being shown a video consisting of two static frames, first the frame on the left in Figure \vref{fig:apparent_motion} and then the frame on the right.
You would typically see two objects moving horizontally.
This is apparent motion: objects which appear successively at two locations sometimes result in a perceptual experience as of an object moving between those locations  \citep{burt:1981_time}.
In some but not all cases, the perceptual experiences associated with apparent motion are readily distinguishable from the perceptual experiences associated with encountering actually moving objects.
You do have a sense of the things moving, and the sense is perceptual, but your experience is quite different from what it would be if the things were actually moving.
In these cases there are phenomenal expectations concerning the objects' movements. 
It is your phenomenal expectations which specify that a particular object is located midway between the two endpoints of its movement at a certain time.
As this illustrates, some phenomenal expectations concern objects’ movements. % and have implications for numerical identity. (implicit in movements, especially given the example of apparent motion I’m using)
%NB: apparent motion is probably a consequence of low-level motion detectors rather than of the sort of object indexes that underpin OSPBs \citep[p.\ 1078]{odic:2012_relationship}

\begin{figure}
\begin{center}
\vspace{1cm}
\includegraphics[scale=0.5]{fig/odic_2012_fig2_part.jpg}
\end{center}
\caption{Apparent motion.  \emph{Source}: \citet[part of figure 2 on p.\  1060]{odic:2012_relationship}
\label{fig:apparent_motion}
}
\end{figure}

Note that the existence of a phenomenal expectation requires more than anticipation in a perceptual process.
A variety of mechanisms, extending to statistical learning and motor processes, modulate perceptual processes in such a way as to enable a range of features of objects to be perceptually represented in advance of the objects actually having those features \citep[e.g.][]{kandel:2000_perceptual,turk-browne:2010_implicit,Wilson:2005qu}.%
\footnote{
Here I am emphasising prediction as evidence for anticipation in perceptual processes.
While such prediction is important methodologically for establishing anticipation, 
not all anticipation in perceptual processes is future-directed.
This is nicely illustrated by an early discussion of apparent motion, Roget's study of a cartwheel viewed through a Venetian blind \citep{roget:1825_explanation}.
}
Anticipation in perceptual processes is surely necessary for there to be phenomenal expectations.
But not all anticipation in perceptual processes need result in phenomenal expectations.
% $revision example removed: I'm not sure it was right
%Consider, for instance,  that anticipation in perceptual processes can alter experiences of pitch \citep{repp:2009_performed}: phenomenal expectations are not obviously involved.

 
 
How are phenomenal expectations relevant to our questions
about the possibility of perceptually experiencing physical properties and mental states?
As we will see, 
much of the evidence on whether humans can perceptually experience physical properties of objects such as solidity and momentum is evidence that such properties influence phenomenal expectations.
I shall also suggest, later (in section \vref{sec:emotions_influence_cp}), that in asking whether humans ever perceptually experience others’ mental states it is useful to focus on how, if at all, others’ mental states can influence phenomenal expectations.









\section{Physical Properties influence Phenomenal Expectations}
\label{sec:physical_properties}

Which physical properties of objects feature in, or influence, phenomenal expectations?
When an object moves, for instance, do humans ever have phenomenal expectations concerning its spatio-temporal trajectory?
If they do, can an object's solidity or momentum somehow influence their phenomenal expectations concerning its movements?

I shall take a perverse approach to answering these questions.
It is arguably impossible to adequately answer them without considering research on the perception of force 
(see \citealp{jones:1986_perception} and  \citealp{wolff2013causation} for reviews), and related illusions \citep[e.g.][]{diedrichsen:2007_illusions}.
But I shall ignore all such research in what follows,
with the result that I am addressing the question almost as if it were not about actual humans but about imaginary incorporeal humans who are incapable of movement and never intervene on objects.
The result is an artificially narrow view of the evidence concerning whether humans can perceptually experience properties such as solidity or momentum.
But there is a reason for adopting this perverse restriction rather than considering a wider body of evidence.
Adhering to the perverse restriction on evidence for claims about perceptually experiencing solidity or momentum 
will yield a useful model for parallel questions about perceptually experiencing mental states.%
\footnote{
Note that the perverse restriction is not equivalent to considering vision in isolation from other senses.  
The operation of object indexes may turn out not to be specifically visual, as \citealp{jordan:2010_seeb}’s finding that acoustic stimuli can cause object indexes to be updated hints.
The perverse restriction allows us to consider research on object indexes even if this is so.
%the operation of these turns out not to be specifically visual (and \citealp{jordan:2010_seeb}’s finding that acoustic stimuli can cause object indexes to be updated hints that they may not be).
%(While this does not provide that object indexes are not specifically visual, \citet{jordan:2010_seeb}’s finding that acoustic stimuli can cause object indexes to be updated; as they note (p.~501), this does not imply that object indexes are not specifically visual, but it does hint that they might not be.)
}

 
%How can we determine which physical properties feature in phenomenal expectations? 
The first thing we need is to know something about how perceptual systems track objects.
In principle we could imagine a perceptual system that is concerned exclusively with answering questions about the locations of features at particular times.
Such a system would be concerned with whether there is a red square over there now, 
but not with whether this red square is the same as the thing that was over there a moment ago.
It is plausible, though, that many perceptual systems, including those found in primates, are concerned not just with the locations of features but also with objects and their movements.


Consider the claim that 
perception involves a system (at least one) of indexes which attach to objects;
this claim is common to a range of theories about aspects of object perception including those offered by 
	\citet{Kahneman:1992xt}, 
	\citet{pylyshyn:1994_multiple} 
	and \citet{alvarez:2007_how}.
These object indexes can be thought of, roughly, as mental analogues of the pins that an old fashioned logistician sticks into a map in keeping track of supply trucks.
When things go well, the movements of trucks on the ground are mirrored by the movements of pins on the map.
% The pins themselves are not representations in any ordinary sense: they are pointers.
% Perhaps the pins can be used to organise incoming information about the trucks, but this doesn't make the pins themselves into representations.
The key characteristic of the pins is just this: ignoring re-use, if you have the same pin at two times, then the trucks it points to at those times are one and the same truck.

Two famous discoveries 
provided early evidence that perception involves a system  (at least one, perhaps more) of indexes for objects analogous to these pins
and 
introduced the paradigms used to investigate how object indexes function.
\citet{pylyshyn:1988_tracking} showed that humans are able to track multiple moving objects simultaneously;
and it seems they do this by means of parallel processes  \citep{howe:2010_distinguishing}.%
% $revision : clarified that Odic et al, 2012 do not provide evidence for the distinctness of the mechanisms underpinning MOT and OSPB.
\footnote{
While several current models of multiple object tracking  differ in essential respects from Pylyshyn and Storm's first version (\citeyear{pylyshyn:1988_tracking}), many retain commitment to the existence of indexes for objects \citep[e.g.][]{alvarez:2007_how,franconeri:2010_tracking}.
}
And \citet{Kahneman:1992xt} showed that in a display with moving objects, people are faster to re-identify features when they are positioned on the same object: that is, there is an \emph{object-specific preview benefit}.%
\footnote{
Do the mechanisms underpinning object-specific preview benefits and multiple object tracking involve a single system of object indexes?
\citet[p.\ 216]{Kahneman:1992xt}, \citet{scholl:1999_tracking} and \citet[p.\ 333]{noles:2005_persistence} all propose that they might.
Although there is to my knowledge no evidence against this proposal, others have considered alternative possibilities (for example \citealp[p.\ 1078]{odic:2012_relationship}).
% consider the possibility that perception involves multiple, hierarchically related, systems of object indexes: `The visual system may contend with the problem of object correspondences at every stage in processing, and higher level mechanisms may rely on the outputs of lower level ones when appropriate, but also possibly overriding them when incorporating additional evidence leads to a different solution.'
}
While much remains to be learnt about the mechanisms involved (see, for example, \citealp{xu:2009_selecting}),
these and other findings indicate that perception involves a system of object indexes (see \citealp{scholl:2010_persistence} for a brief overview).

How are object indexes relevant to our question about which if any physical properties of objects influence phenomenal expectations?
Consider two claims.
First, there is a system of object indexes for tracking objects' movements whose operations can be facilitated 
by information about solidity or momentum in accordance with certain principles such as impetus.
(Let me abbreviate this first claim by saying that the operations of object indexes \emph{reflect} solidity or momentum.) \label{df:reflect}
%solidity or momentum \emph{shape} how object indexes behave.)
Second, this system of object indexes sometimes gives rise to phenomenal expectations concerning the movements of objects.%
\footnote{
\label{fn:mitroff}%
The claim that some system of object indexes sometimes gives rise to phenomenal expectations does not imply that object indexes and phenomenal expectations are always aligned.
\citet{Mitroff:2004pc} construct a situation involving two objects which simultaneously undergo temporary occlusion.
In this situation, perceivers' verbal reports imply the objects' paths crossed whereas measuring an object-specific preview benefit implies that the objects bounced off each other.
The object indexes underpinning  object-specific preview benefits are unlikely to be informing phenomenal expectations about objects' movements in this situation.
Object indexes and phenomenal expectations can come apart in some situations.
%Should we conclude that  object indexes and phenomenal expectations are entirely unrelated?
%Surely not: the fact that object indexes and phenomenal expectations can come apart in some situations hardly warrants such a conclusion (as \citealp[p.\ 87]{Mitroff:2004pc} note).
}
These claims jointly provide one way---not the only way---of making precise the rough idea that one or both of these physical properties, solidity and momentum, can be perceptually experienced.
Evidence for these claims would be evidence that humans sometimes perceptually experience physical properties like solidity or momentum.
But is there any such evidence?





Consider how object indexes are maintained over time in cases where the indexed objects  are not continuously perceptible (in terms of the analogy, the question is how the logistician moves pins when she has partial information about the supply trucks).
What determines whether this object at time-1 and that object at time-2 have the same object index pinned to them?
One factor may be the similarity of their features \citep{hollingworth:2009_object}.
Another factor is the objects' spatio-temporal trajectories: objects' whose movements are consistent with the movement of a single object tend to have a single object index pinned to them \citep{flombaum:2006_temporal,mitroff:2007_space}. 
In fact, spatio-temporal constraints sometimes mean that even two objects with completely different shapes and colours and no common features are assigned the same object index
\citep{odic:2012_relationship}.%
\footnote{\label{fn:mot_proximity}
The findings cited in this paragraph all involve measuring object-specific preview benefits. 
Some researchers have argued that in multiple object tracking with at least four objects,
motion information is not used to update indexes during the occlusion of the corresponding objects \citep{keane:2006_motion,horowitz:2006_how}; rather, `MOT through occlusion seems to rely on a simple heuristic based only on the proximity of reappearance locations to the objects’ last known preocclusion locations' (\citealp{franconeri:2012_simple}, p.\ 700).
However information about motion is sometimes available \citep{horowitz:2010_direction} and used in tracking multiple objects simultaneously \citep{howe:2012_motion, clair:2012_phd}.
One possibility is that, in tracking four objects simultaneously, motion information can be used to distinguish targets from distractors but not to predict the future positions of objects \citep[p.\ 8]{howe:2012_motion}.
}
There is no comparably substantial, direct evidence that physical properties like solidity or momentum also influence how object indexes are maintained.%
\footnote{
The findings of \citet{Mitroff:2004pc} mentioned in footnote \vref{fn:mitroff} might be interpreted as providing such evidence.  
However this interpretation would go beyond anything those authors have claimed in describing their findings (either in that paper or in later discussions of the findings), and these findings are currently isolated.
}
However there are indirect routes to the conclusion that they do.

A conjecture about development provides one route to the conclusion that the operations of some object indexes reflect solidity.
From around three months of age, infants manifest abilities to track occluded objects, and do so in ways that reflect some understanding of physical properties like solidity (e.g.~\citealp{spelke:1992_origins}; \citealp{baillargeon:1987_object}; \citealp{durand:2002_object}; \citealp{saxe:2006_five}).
% $revision : dropped reference to Baillargeon's review; the reviewer is right that this is poorly chosen because,  although it discussed much relevant research, it does so in the context of a framework which involves many claims I would not want to endorse here.
Several researchers have conjectured that infants' abilities are based not on knowledge or thought about objects but rather on the very system of object indexes that underpins object-specific preview benefits in adults \citep{Leslie:1998zk,Scholl:1999mi,Carey:2001ue,scholl:2007_objecta}.
Of course, this conjecture alone is unlikely to explain the full range of abilities to track physical objects that infants manifest throughout their first year of life \citep{cacchione:2013_foundations}.
But several considerations count in favour of the conjecture.
First, it is plausible that object indexes are involved when six month olds encounter an object undergoing occlusion \citep{kaufman:2005_oscillatory}.
Second, infants' abilities to track objects seem to prioritise spatio-temoporal cues over featural information in much the way observed when measuring object-specific preview benefits \citep{Xu:2004xn}.
And, third, the conjecture suggests an elegant way of making sense of some otherwise puzzling discrepancies between tests of infants' abilities to represent unperceived objects using different measures (e.g. \citealp{Shinskey:2001fk,moore:2010_numerical,Hood:2000bf,Hood:2003yg}).%
\footnote{
Some researchers hold views which are (or appear to be) incompatible with the conjecture that object indexes underlie infants’ abilities to track occluded objects in ways that show sensitivity to physical properties like solidity.
For instance, \citet{Baillargeon:2002hb} reviews evidence for her view that innate concepts, causal explanation and learnt rules underlie infants’ abilities.
}

How is the conjecture that infants' abilities to track occluded objects rely on object indexes relevant to our question about perceptual experience?
Infants' abilities to track the locations of temporarily occluded objects show sensitivity to physical properties like solidity from around four months of age or earlier 
\citep{baillargeon:1987_object,spelke:1992_origins,durand:2002_object}.
%no: these experiments are about velocity not momentum (even if the authors sometimes suggest momentum, the manipulations involve changing speeds).
%Further, patterns in their anticipatory looking to where moving objects might reappear after occlusion suggest that they are sensitivity to kinematic properties including objects' velocities \citep{rosander:2004_infants,vonhofsten:2007_predictive}.
The conjecture that these abilities depend on the very object indexes which underpin object-specific preview benefits therefore supports the two claims that are jointly sufficient for us to conclude that solidity or momentum is perceptually experienced.
There is a system of object indexes whose operations reflect solidity or momentum and which gives rise to phenomenal expectations concerning objects’ movements.


An independent route to the same conclusion involves psychophysics rather than development.
Sometimes when adult humans observe a moving object that disappears, they will misremember the location of its disappearance in way that reflects its momentum; this effect is called \emph{representational momentum} \citep{freyd:1984_representational,hubbard:2010_rm}.
The trajectories implied by representational momentum reveal that the effect reflects impetus mechanics rather than Newtonian principles \citep{freyd:1994_representational,kozhevnikov:2001_impetus,hubbard:2001_representational,hubbard:2013_launching}.
And these trajectories are independent of subjects' scientific knowledge
\citep{freyd:1994_representational,kozhevnikov:2001_impetus}.
Representational momentum therefore reflects judgement-independent expectations about objects’ movements which 
track momentum in accordance with a principle of impetus.%
\footnote{
Note that momentum is only one of several factors which may influence mistakes about the location at which a moving object disappears \citep[p.\ 842]{hubbard:2005_representational}.
%:
%\begin{quote}
%`The empirical evidence is clear that (1) displacement does not always correspond to predictions based on physical principles and (2) variables unrelated to physical principles (e.g., the presence of landmarks, target identity, or expectations regarding a change in target direction) can influence displacement.'
%
%...
%
%`information based on a naive understanding of physical principles or on subjective consequences of physical principles appears to be just one of many types of information that could potentially contribute to the displacement of any given target'
%\end{quote}
}
But are these expectations  phenomenal expectations?
We should be cautious here because 
the relation between representational momentum and object perception is not straightforward \citep[compare][p.\ 433]{freyd:1987_dynamic},
and because there are  currently several competing models of representational momentum and related phenomena involving misremembered location \citep{hubbard:2010_rm}.
However it is perhaps tempting to conjecture that the expectations manifested in representational momentum are a consequence of some perceptual system of object indexes.%
\footnote{
The way that object-specific preview benefits line up with representational momentum when Michotte's launching stimuli are observed (compare \citealp{Krushke:1996ge} with \citealp{hubbard:2001_representational} and \citealp{hubbard:2013_launching})
perhaps even hints  that the very object indexes which underpin object-specific preview benefits could also be responsible for representational momentum.
%
%\begin{quote}
%`The decrease in the forward displacement of targets in a launching effect display is not consistent with the hypotheses that observers are sensitive to the implied dynamics or that observers possess an accurate implicit knowledge of physical principles. However, the decrease in the forward displacement of targets in a launching effect display is consistent with the hypothesis that observers appeal to the heuristic that the launcher imparts an impetus or influence to the target that will gradually dissipate.
%...
%Representational momentum and the launching effect may both reflect the expectations of observers regarding the influence of other stimuli on a target and the future behavior of that target.'
%\citep[p.\ 300]{hubbard:2001_representational}
%See further \citep{hubbard:2013_launching}.
%\end{quote}
%
}
This conjecture implies that there is a system of object indexes whose operations reflect momentum as characterised by impetus mechanics, and which gives rise to phenomenal expectations concerning objects’ movements.
% removed footnote: I’m not convinced I fully understand what to conclude from \citet{perry:2008_representational} 
% \footnote{
% Can we combine the earlier conjecture with this one, yielding the conjecture that a single system of object indexes underpins both  infants' abilities to track occluded objects and representational momentum?  
% This combined conjecture 
% should probably be rejected on the grounds that it
% is inconsistent with the conjunction of
% the conclusion of a  study of representational momentum in two-year-olds \citep{perry:2008_representational}
% and 
% well-supported claims about infants' sensitivity to solidity \citep{spelke:1992_origins}.
% }
That is, it implies that, in a sense, humans can perceptually experience not only a physical object’s movements but also its momentum or solidity.

% $revision: cut to shorten (repitition)
% We have just seen two routes to defending this claim about the perception of physical properties:
% There is a system of object indexes whose behaviours reflect  properties such as solidity or momentum and principles such as impetus; and the behaviours of these object indexes sometimes give rise to phenomenal expectations which concern objects’ movements.


%
%Object indexes of some kind are the basis of phenomenal expectations about where and when temporarily occluded moving objects  will re-appear.
%Imagine watching as one ball moves behind a wide tree trunk and, moments later, another ball emerges from the other side of the tree trunk.  
%If the two balls' movements are related in the right way,
%you are likely to have the impression of a single object moving.
%This is true even if the balls differ saliently in colour, so that you know they could not be one and the same.
%The impression of a single object moving persists even if you know a scientist has concealed assistants behind the trunks to carefully coordinate the movements of several balls.
%This impression is perceptual, and it is a consequence of the way object indexes work \citep{flombaum:2006_temporal}.
%Good continuity of movement ensures that a single object index attaches to the balls, generating the perceptual impression that there is a single object. 

The evidence I have reviewed  for this claim is indirect and leaves many questions  open.
As emphasised  at the  start of this section, I have imposed an artificial restriction on the range of evidence considered by imagining incorporeal human perceivers who neither intervene on objects nor move.
Even under this restriction, evidence other than that reviewed here may be relevant; for instance, 
studies of the launching effect and related sensitivities to causal interactions among perceived objects might \citep{Butterfill:2009vs} or might not \citep{rips:2011_causation} yield evidence.
But my aim here was not to establish the claim:
it was to identify an approach that can serve as a model for investigating a parallel claim about perceptual experience of mental states.

Reflection on how the claim that humans sometimes perceptually experience physical properties like solidity or momentum might be tested supports three conclusions which can guide our thinking about perceptual experience and mental states.
First, even if scientists rarely explicitly mention claims about  perceptual experience, such claims can be supported by, or inconsistent with, experimental evidence;
 and there are rich bodies of evidence that do bear on them.
Second, working out how to evaluate such claims involves making connections between perceptual processes and phenomenology.
These connections  can inform views about the nature of experience and generate new predictions. 
%%New predictions? Yes!  (1) OSPB in infants lines up with success on habituation and V-of-E tasks (e.g. OSPB using colour change detection, as \citep{flombaum:2006_temporal} method?); (2) OSPB in representational momentum
Third, 
it is not necessary to start with general stipulations about, or criteria governing, where perceptual experience ends and other forms of cognition begin.
We can often be more confident %that a particular case involves perceptual experience 
in detailed claims about particular cases 
than we can in ambitious generalisations \citep[compare][pp.\ 11--44]{smortchkova:2014_phd}.
% In short, progress both in 
% understanding what it would be to perceptually experience things like solidity or momentum 
% and in
% working out whether humans ever actually do so 
% requires considering experimental evidence.

Before going further, consider an objection to the way I have been interpreting the view that humans sometimes perceptually experience properties like solidity or momentum. 
(Or skip over the objection if you like.) 


\section{An Objection}
\label{sec:an-objection}
In the previous section I identified a modest interpretation of the view that humans can perceptually experience physical properties like solidity or momentum.  
This view does not imply that such properties ever feature in perceptual experience in the same way that shapes, tones, colours or odours can.
Nor is there even commitment to the claim that  phenomenal expectations are expectations about solidity or momentum.
The view is merely that these properties, solidity and momentum, may make a difference to the overall phenomenal character of perceptual experiences because of the guiding role of information about them in  processes involving object indexes that give rise to phenomenal expectations.
As far as the view I have defended goes, such expectations may concern only the movements of objects.

You may object that this way of interpreting the view about perceptual experience of properties like solidity or momentum is too modest.
I suppose that any adequate view must permit a distinction between what is represented in perceptual processes and what is perceptually experienced.
But, you might object, isn’t this distinction obliterated by my modest interpretation of the claim about perceptual experience of solidity or momentum?
Doesn’t accepting such a modest interpretation amount to accepting that anything which influences phenomenal expectations in any way at all is perceptually experienced?

It does not.
Consider one indicator that the relation between solidity or momentum and phenomenal expectations about the movements of objects is special.
In a world lacking solidity and momentum, there could still be moving things; whereas in a world lacking movement there could not be momentum, and solidity would have no functional role.
Relatedly, phenomenal expectations could not be influenced by information about solidity or momentum unless they concerned objects’ movements; and the converse is false.
%(Indeed, as noted in footnote \vref{fn:mot_proximity}, some phenomenal expectations about movement appear not to be influenced even by information about direction.)
Let me abbreviate this by saying that information about momentum \emph{counterfactually enhances} phenomenal expectations concerning movement.
Not everything that influences an aspect of perceptual experience also counterfactually enhances it.
Information about contrasts in luminance may influence perceptual experiences concerning edges, but luminance contrasts are not thereby perceptually experienced.
Or, to take another case, information about junctions may influence perceptual experiences concerning three-dimensional shapes, but junctions are not thereby perceptually experienced.
What indicates that these influences differ in their significance for phenomenology from those involving information about solidity or momentum?
Information about luminance contrasts does not counterfactually enhance perceptual experiences of edges, and information about junctions does not counterfactually enhance perceptual experiences about three-dimensional shape.
(After all, there could be luminance contrasts in a world without edges, and there could be junctions in a two-dimensional world.)
So accepting the modest interpretation of the view that humans can perceptually experience physical properties like solidity or momentum is consistent with holding that there is a distinction between what is represented in perceptual processes and what is perceptually experienced.

Counterfactual enhancement is only one indicator that the relation between solidity or momentum and phenomenal expectations about objects’ movements is special.%
\footnote{
Another indicator is suggested by Smith’s discussion of ‘latching on’ (see \citealp[pp.~741ff]{smith:2010_seeing} and the refinement in \citealp[p.~17]{smith:2013_phenomenology}).
}
%I am committed to the claim that momentum is perceptually experienced in virtue of its influence on phenomenal expectations about objects’ movements.
I am not suggesting that counterfactual enhancement explains why it is possible to interpret the view that solidity or momentum are perceptually experienced in the modest way I have proposed whereas similar ways of interpreting corresponding views about other properties would be unacceptable.
My suggestion is just that accepting the modest interpretation of the view about perceptual experience of some physical properties does not commit you to accepting the same for arbitrary physical properties.
% In that case, why take the case of solidity as a model for the case of mental states?  Because they have indicators like counterfactual enhancement in common.


% In observing that information about momentum {enhances} phenomenal expectations concerning movement I am not making a general claim about which things are perceptually experienced.
% It strikes me as more plausible that momentum’s influence on phenomenal expectations about objects’ movements 

% It may be that enhancement (in the special sense defined here) is neither necessary nor sufficient.

% Without any role for information about solidity or momentum in shaping phenomenal expectations (or some other aspect of perceptual experience), it would be hard or impossible to accept that either solidity or momentum is perceptually experienced in any sense at all.
% [*But I’m not happy to step back to this because I want to show that experimental evidence can bear more directly on questions about phenomenology.]


%After all, to claim that there are phenomenal expectations about the solidity or momentum of an object would already be a fairly modest way of interpreting the view  that these can be perceptually experienced.
% But I propose to interpret this view yet more modestly, as requiring only phenomenal expectations about objects’ movements.
% Surely the view that humans can perceptually experience physical properties like solidity or momentum requires at least that humans have phenomenal expectations about such properties?

% In reply to this objection, ***different grades of claim ***weakest claim is probably necessary for stronger claim [*relink final question]  ***If not perceptual experience, then it is something which involves perceptual processes and influences judgements via phenomenology (would be delighted to discover that such things exist other than perceptual experience).

% *Further objection: the modestly of the view obliterates the distinction between perceptual processes and perceptual experience.

% *Reply: solidity and momentum are not necessary to the 
% maintenance of object indexes and the tracking of movements; indeed in some cases object indexes are maintained without regard even to information about velocity (see the point about ‘nearest to point of disappearance’ above).
% Because these are enhancements rather than basic requirements for the existence of the phenomenal expectations about objects’ movements, claiming perceptual experience does not require obliterating the distinction between perceptual processes and perceptual experience.

% *Even if you think this is not any degree of perceptual experience, the claim about perceptual processes is surely necessary for a claim about perceptual experience.
% When we turn to mental states we will see how hard it is to establish the weaker claim …

In what follows I evaluate a similarly modest interpretation of the view that humans can perceptually experience some mental properties of subjects other than themselves.
What evidence is there, and what evidence could there be, for this view?

%**Also modest with respect to visual vs perceptual; object files may be multi-modal \citep{jordan:2010_seeb}.

%
%\section{*Lessons [cut?]}
%We have just explored
%how hypotheses about the perceptual experience of physical properties can be tested.
%Two observations guide what follows.
%First, understanding perceptual experience need not depend entirely on guessing, introspecting or making bold assumptions about the accuracy of verbal reports.
%Science  can be useful in making precise and supporting  hypotheses about perceptual experience; and, if sufficiently precise, such hypotheses can generate testable predictions.
%%Predictions? Yes!  (1) OSPB in infants lines up with success on habituation and V-of-E tasks (e.g. OSPB using colour change detection, as \citep{flombaum:2006_temporal} method?); (2) OSPB in representational momentum
%Second, it is not necessary to start with general stipulations about, or criteria governing, where perceptual experience ends and other forms of cognition begin.
%We can often be more confident that a particular case involves perceptual experience than we can in ambitious generalisations \citep[compare][pp.\ 11--44]{smortchkova:2014_phd}.




\section{Categorical Perception and Emotion}
\label{sec:cp_emotion}
We have been asking what kinds of evidence could bear on questions about the perceptual experience of physical properties.
How can this guide us in working out what kinds of evidence might bear on questions about the perceptual experience of others' mental states?

When asking about physical properties like momentum or solidity, it is useful---and perhaps even essential---to start from what we know about how perceptual processes track objects  (as we saw in Section \ref{sec:physical_properties}).
Similarly, when asking about mental states it makes sense to start by thinking about how perceptual processes track the subjects of mental states.
The subjects of mental states pursue goals and they express emotions.
This suggests two lines of enquiry.
One line focusses on animacy and object- or goal-directed behaviours, another on expressions of emotion.
Here I shall consider the second line only.%
\footnote{
When discussing physical properties I restricted consideration to evidence that would bear on an imaginary incorporeal human incapable of intervening on the things around her.
In now discussing mental states, I shall no longer apply that restriction.
}


%
%In this section I shall review some of the evidence on whether humans enjoy categorical perception of some expressions of emotion.
%While the evidence is far from decisive and many complex questions remain open,
%I shall suggest that, on balance, the evidence provides some support for the view that  expressions of emotion  feature in phenomenal expectations.
%
%One source of evidence which bears on such hypotheses 


Do any perceptual processes in humans discriminate stimuli according to the expressions of emotion they involve?
That is, do humans have \emph{categorical perception} of expressions of emotion?%
\footnote{
This working definition of categorical perception is not intended to be controversial but differs from some definitions given in the literature.
On my working definition, to say that someone has categorical perception of expressions of emotion is to say that there are perceptual processes  in her which discriminate a certain range of stimuli according to the expressions of emotion they involve.
The most detailed attempt to define categorical perception (which focusses on speech, the best studied case) differs in being framed in terms of subjects’ abilities to discriminate rather than in terms of perceptual processes; it is also more restrictive in some respects \citep[see][pp.~251--4]{Repp:1984ko}.
Categorical perception is also occasionally defined phenomenologically, that is in terms of perceptual appearances or experiences of similarity and difference (e.g.~\citealp[p.~190]{Matthen:2005sc}; \citealp[p.\ 288--9]{Bornstein:1987vv}; \citealp[p.~115]{Kotsoni:2001ph}.)
For my purposes it is essential to distinguish claims about discrimination from claims about phenomenology.
}
To answer this question we first need to consider findings about the discrimination of expressions, leaving open for now whether discrimination involves perceptual processes.
Assume that we as theorists have a system which allows us to categorise static pictures of faces and other stimuli according to which emotion we think they are expressing: some faces are happy, others fearful, and so on.
From five months of age, 
or possibly much earlier \citep{field:1982_discrimination}, 
through to adulthood, humans are better at distinguishing faces when they differ with respect to these categories than when they do not \citep{Etcoff:1992zd,Gelder:1997bf,Bornstein:2003vq,Kotsoni:2001ph,cheal:2011_categorical,hoonhorst:2011_categoricala}.
To illustrate, consider Figure \vref{fig:faces}.
The idea is this.
With respect to all features apart from the expression of emotion, each face picture differs from its neighbours no more than any other picture differs from its neighbours.
Most neighbouring pairs of face pictures would be relatively hard to distinguish, especially if they were not presented side-by-side. 
But most people find one pair of neighbouring face pictures  relatively easy to distinguish---you may notice this yourself.


\begin{figure}
\begin{center}
\includegraphics[scale=0.4]{fig/kotsoni_2001_fig1.jpg}
\end{center}
\caption{The faces at either end have been morphed with each other in differing degrees in order to produce a sequence of faces where each differs physically from its neighbours by a fixed amount.  \emph{Source}: \citet[figure 1]{Kotsoni:2001ph}\label{fig:faces}
}
\end{figure}


What underlies these patterns of discrimination?
Several possibilities that would render them uninteresting for our purposes can be ruled out.
The patterns of discrimination do not appear to be an artefact of linguistic labels (\citealp{sauter:2011_categorical}; see also \citealp{laukka:2005_categorical}, p.\ 291),%
%
\footnote{
Puzzlingly, experiments by \citet{fugate:2010_reading} using photos of chimpanzee faces with human subjects are sometimes cited as evidence that categorical perception of expressions of emotion depends on, or can be modulated by, the use of verbal labels for stimuli (e.g.\ \citealp[p.\ 288]{barrett:2011_context}; \citealp[p.\ 315]{gendron:2012_emotion}).
Caution is needed in interpreting these findings
given that there may be differences in the ways  humans process human and chimpanzee faces.
In fact, what \citeauthor{fugate:2010_reading}'s findings show may be simply that `human viewers do not show [categorical perception] for the chimpanzee facial configurations used in their study' \citep[p.\ 1482]{sauter:2011_categorical}.
}
% 
nor of the particular choices subjects in these experiments are presented with \citep{bimler:2001_categorical,fujimura:2011_categorical}.
Nor are the patterns of discrimination due to narrowly visual features of the stimuli used \citep{sato:2009_detection}.
We can be confident, then, that the patterns of discrimination probably reflect one or more processes which categorises stimuli by expression of emotion.

What kinds of process are these?
Although linguistic information can have a top-down effect on categorical perception \citep{cheal:2013_contexta},
the processes of categorisation underpinning the patterns of discrimination just observed 
are unlikely to rely entirely on conceptual thought about the stimuli. 
At least some of them are rapid (occurring within roughly 200 milliseconds of a stimulus' appearance), pre-attentive \citep{vuilleumier:2001_emotional} and automatic in the sense that whether they occur is to a significant degree independent of subjects' tasks and motivations \citep{batty:2003_early}.%

But are any of the processes that categorise stimuli by expression of emotion perceptual?
%That is, are the observed abilities to discriminate  expressions of emotion ever based on perceptual processes?
Answering this question is complicated by the fact that many parts of the brain are involved \citep{adolphs:2002_recognizing,vuilleumier:2007_distributed}.  
There is evidence that both the amygdala \citep{harris:2012_morphing,harris:2014_dynamic} and also some cortical structures \citep{batty:2003_early}
respond categorically to  expressions of emotion;
and that intervening in the operations of the somatosensory cortex can impair categorisation (\citealp{pitcher:2008_transcranial}; see also \citealp{banissy:2011_superior}).
%Neurophysiological measurements indicate the amygdala may be involved as fMRI reveals that it distinguishes stimuli according to categories of emotion \citep{harris:2012_morphing,harris:2014_dynamic}.
To my knowledge, so far it is only for happy and fearful stimuli that we have direct evidence 
from both neurophysiological \citep{Campanella:2002aa} and behavioural measures \citep{williams:2005_looka}
of categorisation occurring in perceptual processing.
So while the evidence is not conclusive,
there is converging evidence that some perceptual processes categorise stimuli including faces by  expression of emotion.
Humans may have categorical perception not only for speech, colour, orientation and other properties but also for expressions of emotion.

How is any of this relevant to our question about whether any mental states can be perceptually experienced?
The evidence for categorical perception is evidence that there are perceptual processes which track expressions of emotion and which give rise to phenomenal expectations.
Having identified evidence for the existence of such processes,
we need to ask whether the mental states of the thing perceived influence how expressions of emotion are tracked.
(This is analogous to asking whether properties like solidity or momentum affect how perceptual processes track physical objects' movements.)
Before doing this we must face up to a complication, however.




\section{What Are the Perceptual Processes Supposed to Categorise?}
\label{sec:what_is_categorised}
We have just seen evidence that humans have categorical perception for expressions of emotion.
To work out how this might bear on hypotheses about perceptual experience of mental states, we need to ask:
What are these processes supposed to categorise?
One answer is obvious, of course: expressions of emotion are what they are supposed to characterise.
But what are these?
%Until now I have been following much of the literature in saying nothing about what they are.
%But, as we will see, getting from categorical perception to a hypothesis about perceptual experience depends on thinking more carefully about the nature of expressions of emotion as categorically perceived.

It is perhaps tempting to assume that 
categories of emotion like happiness, sadness, fear and surprise are each associated with a category of facial configurations,
and that the relation between the emotions and the categories of configurations is merely contingent.
(So that the expression associated with happiness might just have well been associated with surprise.)
%these categories of configurations can be individuated without specifying any emotions.
This might make it plausible to assume, further, that the things  perceptual processes are supposed to categorise---the `expressions of emotion'---are facial configurations.
If this were right, the evidence we have been reviewing on perception and emotion would support the view that humans have phenomenal expectations concerning certain characteristic bodily effects of some mental states.
It would not support the view that humans can perceptually experience mental states in the sense in which, arguably, they can perceptually experience physical properties like solidity and momentum.  


But are the things categorised by perceptual processes facial configurations?
This view faces a problem.
There is evidence that
the same facial configuration can express intense joy or intense  anguish depending on the posture of the body it is attached to,
and, relatedly, that humans cannot accurately determine emotions from spontaneously occurring
%spontaneously occurring --- i.e. as opposed to acted out
facial configurations \citep{motley:1988_facial,aviezer:2008_angry,aviezer:2012_body}.
These and other findings, while not decisive, cast doubt on the view that categories of emotion are associated with categories of facial configurations \citep{hassin:2013_inherently}.
This evidence makes the findings we have reviewed on categorical perception puzzling.
Given that the facial configurations are not diagnostic of emotion,
why are they categorised by perceptual processes?%
\footnote{
Compare \citet[p.\ 1228]{aviezer:2012_body}:
`although the faces are inherently ambiguous, viewers experience illusory affect and erroneously report perceiving diagnostic affective valence in the face.'
}
This question appears unanswerable as long as we retain the assumption---for which, after all, no argument was given---that the things categorical perception is supposed to categorise are facial configurations.
But if we reject this assumption, what is the alternative?

Compare expressing an emotion by, say, smiling or frowning,  with articulating a phoneme.
Both have a communicative function (on expressions of emotion, see for example \citealp{blair:2003_facial,sato:2007_spontaneous})
and both are categorically perceived, 
but the phonetic case has been more extensively investigated.
Variations due to coarticulation, rate of speech, dialect and many other factors mean that isolated acoustic signals are not generally diagnostic of phonemes:
in different contexts, the same acoustic signal might be a consequence of the articulation of any of several phonemes.
So here there is a parallel between speech and emotion.
Much as isolated facial expressions are not diagnostic of emotions (as we saw a moment ago), isolated acoustic signals are plausibly not diagnostic of phonetic articulations.
% And much as a side-effect of categorical perception of expressions of emotion is plausibly responsible for the mistaken impression that facial expressions are diagnostic of emotion,
% so also categorical perception of speech probably makes it natural for people to guess, incorrectly, that they could identify phonemes on the basis of  isolated acoustic signals.  
Why then are isolated acoustic signals---which rarely even occur outside the lab---categorised by perceptual or motor processes at all?
To answer this question we first need a rough idea of what it is to articulate a phoneme.
Articulating a phoneme involves making coordinated movements of the lips, tongue, velum and larynx.
How these should move depends in complex ways on numerous factors including phonetic context \citep{Browman:1992da,Goldstein:2003bn}.
In preparing for such movements, it is plausible that the articulation of a particular phoneme is an outcome represented motorically,
where this motor representation coordinates the movements and normally does so in such a way as to increase the probability that the outcome represented will occur. 
This implies that the articulation of a particular phoneme,
although probably not an intentional action,
is a goal-directed action whose goal is the articulation of that phoneme.
(On the link between  motor representation and goal-directed action, see \citealp{butterfill:2012_intention}.)
Now some hold that the things categorised in categorical perception of speech are not sounds or movements (say) but rather these outcomes---the very outcomes in terms of which speech actions are represented motorically (\citealp{Liberman:2000gr}; see also \citealp{Browman:1992da}).%
\footnote{
Note that this claim does not entail commitment to other components of the motor theory of speech perception.
}
%
%On this view, 
%categorical perception of speech is a process which takes as input the bodily and acoustic effects of actions---in this case, speech actions---and attempts to identify which outcomes the actions are directed to bringing about---in this case, which phonemes the speaker is attempting to articulate.
On this view, 
categorical perception of speech is a process which takes as input the bodily and acoustic effects of speech actions and attempts to identify which outcomes the actions are directed to bringing about, that is, which phonemes the speaker is attempting to articulate.
That isolated acoustic signals can engage this process and thereby trigger categorical perception  is merely a side-effect, albeit one with useful methodological consequences.

How is this relevant to understanding categorical perception of expressions of emotion?
A problem arose from the perhaps natural assumption that the things categorical perception is supposed to categorise  are facial configurations.
The problem, as we saw, is that this assumption conflicts with evidence that facial configurations are not diagnostic of emotions.
We can resolve the conflict by rejecting the assumption in favour of an alternative inspired by the view that the things categorised in categorical perception of speech are not acoustic signals but actions of a certain type, phonetic articulations,
where the actions are categorised by the outcomes to which they are directed.
Whether or not the things categorised in categorical perception of speech are  actions, maybe this is true of categorical perception of expressions of emotions. 

This wild conjecture requires that some expressions of emotion---such as smiling or frowning---be goal directed actions in roughly the sense that the articulation of a phoneme is a goal-directed action.
This may initially strike you as implausible given that such expressions of emotion can be spontaneous, unintentional and involuntary.
But note that expressing an emotion by, say, smiling or frowning,
whether intentionally or not,
involves making coordinated movements of multiple muscles
where exactly what should move and how can depend in complex ways on  contextual factors.
That such an expression of emotion is a goal-directed action follows just from its involving motor expertise
and being coordinated around an outcome (the goal) in virtue of that outcome being represented motorically.%
\footnote{
To increase the plausibility of the conjecture under consideration, we should allow that some categorically perceived expressions of emotion are not goal-directed actions but events grounded by two or more goal-directed actions.  
For ease of expoistion I shall ignore this complication.
}
Recognising that some expressions of emotion are goal-directed actions in this sense makes it possible to explain 
what distinguishes a genuine expression of emotion of this sort, a smile say, from something unexpressive like the exhalation of wind which might in principle resemble the smile kinematically. 
Like any goal-directed actions, genuine expressions of emotion of this sort are distinguished from 
their kinematically similar doppelgänger
in being directed to outcomes by virtue of the coordinating role of motor representations and processes.

Recall that the wild conjecture under consideration is that the things categorical perception is supposed to categorise, the ‘expressions of emotion’, are actions of a certain type, and these are categorised by which outcomes they are directed to.
Let me explain the increasingly bold commitments involved in accepting this conjecture.
First, the things categorised in categorical perception of expressions of emotion are events rather than configurations or anything static.
(Note that this is consistent the fact that static stimuli can trigger categorical perception; after all, static stimuli can also trigger motor representations of things like grasping \citep{borghi:2007_are}.)
Second, these events are not mere physiological reactions (as we might intuitively take blushing to be) but things like frowning and smiling, whose performance involves motor expertise.
% removed footnote: true but fiddly
%\footnote{
%To emphasise, one consequence of this is that not everything which might intuitively be labelled as an expression of emotion is relevant to understanding what is categorised by perceptual processes.
%%For example, in the right context a blush may signal emotion without requiring motor expertise.
%}
Third, these events are perceptually categorised  by the outcomes to which they are directed.
That is, outcomes represented motorically in performing these actions are things by which these events are categorised in categorical perception.

Should we accept the wild conjecture? 
It goes well beyond the available evidence and currently lacks any reputable endorsement.
In fact, we lack direct evidence for even the first of the increasingly bold commitments just mentioned (namely, the claim that the things categorically perceived are events).
A further problem is that we know relatively little about the actions which, according to the wild conjecture, are the things categorical perception is supposed to categorise  (\citealp[p.\ 47]{scherer:2013_understanding}; see also \citealp{scherer:2007_are} and \citealp{fernandez-dols:2013_advances}).
However, 
the wild conjecture is less wild than the only published responses to the problems that motivate it%
% (which, admittedly, are wilder than an acre of snakes).%
.\footnote{
See
\citet[p.\ 15]{motley:1988_facial}: ‘particular emotions simply cannot be identified from psychophysiological responses’;
and
\citet[p.\ 289]{barrett:2011_context}: ‘scientists have created an artifact’.
}
And, as I shall now explain, several considerations make the wild conjecture seem at least worth testing.

Consider again the procedure used in testing for categorical perception.
Each experiment begins with a system for categorising the stimuli (expressions).
This initial system  is either specified by the experimenters or, in some cases, by having the participants first divide stimuli into categories using verbal labels or occasionally using non-verbal decisions.
The experiment then seeks to measure whether this initial system of categories predicts patterns in discrimination.
But what determines  which category each stimulus is assigned to in  the initial system of categories?
You might guess that it is a matter of how likely people think it is that each stimulus---a particular facial configuration, say---would be associated with a particular emotion.
In fact this is wrong.
Instead, 
each stimulus is categorised in the initial system according to how suitable people think such an expression would be to express a given emotion: this is true whether the stimuli are facial \citep{horstmann:2002_facial} or vocal \citep{laukka:2011_exploring} expressions of emotion (see also \citealp[pp.\ 98--9]{parkinson:2013_contextualizing}).
To repeat, 
in explicitly assigning an expression to a category of emotion, people are not making a judgement about the probability of someone with that expression having that emotion:
they are making a judgement about which category of emotion the expression is most suited to expressing.
Why is this relevant to understanding what perceptual processes categorise?
The most straightforward way of interpreting the experiments on categorical perception is to suppose that they are testing whether perceptual processes categorise stimuli in the same ways as the initial system of categories does.
But we have just seen that the initial system categorises stimuli according to the emotions they would be best suited to expressing.
So on the most straightforward interpretation, 
the experiments on categorical perception of expressions of emotion 
are testing whether there are perceptual processes whose function is to categorise  actions of a certain type by the outcomes to which they are directed.
So the wild conjecture is needed for the most straightforward interpretation of these experiments.
This doesn't make it true but it does make it worth testing.

So far I have focussed on evidence for categorical perception from experiments using faces as stimuli.
However, there is also evidence that perceptual processes categorise vocal and facial expressions alike (\citealp{grandjean:2005_voices,laukka:2005_categorical}; see also \citealp{jaywant:2012_categorical}).
We also know that 
%judgements about which emotion an observed person is expressing in a photograph can depend on the posture of the whole body and not only the face \citep{aviezer:2012_body}, and that 
various contextual factors can affect how even rapidly occurring perceptual processes discriminate expressions of emotion \citep{righart:2008_rapid}.
There is even indirect evidence that categorical perception may concern whole bodies rather than just faces or voices \citep{aviezer:2008_angry,aviezer:2011_automaticity}.
In short, categorical perception of expressions of emotion plausibly resembles categorical perception of speech in being a multimodal phenomenon which concerns the whole body and is affected by several types of contextual feature.
This is consistent with the wild conjecture we are considering.
The conjecture generates the further prediction that the effects of context on categorical perception of expressions of emotion will resemble the the myriad effects of context on categorical perception of speech so that `every potential cue ... is an actual cue'
 (\citealp[p.\ 11]{Liberman:1985bn}; for evidence of context effects see in categorical perception of speech, for example, \citealp{Repp:1987xo}; \citealp{Nygaard:1995po} pp.\ 72--5; \citealp{Jusczyk:1997lz}, p.\ 44).



%We must therefore revise the view under consideration:
%if the things categorised by perceptual processes are not facial configurations,
%could they be  gestures involving the whole body and voice?%
%\footnote{
%\citet{barrett:2011_context} make a more radical claim: according to them, findings using depictions or photos of faces as stimuli are an `artefact' of experimental method.
%For reasons given below, 
%I think this is a mistake analogous to suggesting that the many ways in which context affects categorical perception of speech shows that findings arrived at by using isolated acoustic stimuli are artefacts.
%}

How would the conjecture under consideration, if true, bear on our question about perceptual experience of mental states?
Recall that the  conjecture is this: 
the expressions of emotion categorical perception is supposed to categorise are actions of a certain type, and these are categorised by the outcomes to which they are directed.
Since  outcomes are not mental states (of course), the view might initially appear to imply that evidence from categorical perception cannot bear on whether humans can perceptually experience mental states.
This is also what the comparison with categorical perception of speech might be taken to indicate: in both cases, perceptual processes involve information about outcomes to which actions are directed rather than information about mental states.
This suggests that the phenomenal expectations these processes give rise to concern action outcomes and not mental states.
However things are not quite so straightforward.





\section{Do Humans Ever Perceptually Experience Emotions in Categorically Perceiving Their Expressions?}
\label{sec:emotions_influence_cp}

According to the wild conjecture under consideration, 
categorical perception of expressions of emotion 
is supposed to categorise actions of a certain type 
by the outcomes to which they are directed.
What evidence could take us from this conjecture to a conclusion about perceptual experience of emotions?

In outline we seek a view parallel to the one about physical properties considered earlier (in Section \vref{sec:physical_properties}).
There the idea was this.
A perceptual mechanism for tracking objects' movements exists;
its operations can be facilitated 
by information about solidity or momentum in accordance with certain principles such as impetus;
and 
it can give rise to phenomenal expectations about objects' movements.
Here is a parallel view:
(a) a perceptual mechanism for tracking actions of a certain type exists;
(b) its operations can be facilitated by information about agents' emotions in accordance with principles describing aspects of their functional or normative roles;
and
(c) it gives rise to phenomenal expectations about the outcomes to which agents’ actions are directed, or about agents' bodily configurations or movements.%
\footnote{
Note that if this view is correct, information about an agent’s emotions counterfactually enhances phenomenal expectations concerning outcomes to which her actions are directed or her bodily configurations or movements.
(As explained in section \ref{sec:an-objection}, this makes it possible to reply to an objection to the claim that (a)–(c) provide a modest interpretation of the view that some mental states can be perceptually experienced.)
}
So far I have been examining the case for the claim that categorical perception of expressions of emotions is a perceptual mechanism with the features specified in (a) and (c).  
Is there any evidence that it also has the feature specified in (b)?
That is, can categorical perception of facial expressions of emotion be facilitated by information about others’ emotions?

First consider a theoretical objection.
Assume that categorical perception of facial expressions of emotion exists in part because it enables 
information about others' emotions
to be discovered.
(I relied on this assumption in Section \ref{sec:what_is_categorised} when arguing that the things categorical perception is supposed to categorise are not facial configurations.)
If discovering information about others' emotions is supposed to be an upshot of categorical perception, how could such information also facilitate it? 
The objection is that it cannot:
 the information about others’ emotions that might facilitate categorical perception is the very information  that  it could enable you to discover.
 
In reply to the objection consider a simplistic model of how categorical perception of facial expressions of emotion works.
Observed configurations and movements are identified as a possible expression of emotion.
Two or more competing hypotheses about which emotion is being expressed are formulated.
Each hypothesis generates different predictions about the outcomes to which the candidate expression of emotion is directed,
and these in turn generate different predictions about how the action will unfold and about its sensory consequences.
These predictions are tested against perceptual information, and  hypotheses are scored according to how well their predictions stand up.
If one hypothesis pulls far enough ahead in  this contest, the observed configurations and movements are categorised as an action directed to the predicted outcomes,
and thereby as a particular expression of emotion.
This model, although obviously simplistic, illustrates how 
accepting that categorical perception enables information about others' emotions to be discovered 
is in principle compatible with 
supposing that such information also facilitates the process of categorisation.
Acquiring information about another's emotions and categorically perceiving her expressions of emotion may be aspects of a single process.

It is just conceivable that the simplistic model is not entirely misguided.
A range of evidence suggests that some of the processes that would be involved in your experiencing certain emotions can also occur when you merely observe expressions of those emotions  
(\citealp{wicker:2003_both}; \citealp{Gallese:2004dm}; \citealp{vandergaag:2007_facial}; \citealp[Chapter 7]{rizzolatti_mirrors_2008}; \citealp{bastiaansen:2009_evidence}),
and that the occurrence of these processes can facilitate  perceptual discrimination of expressions of emotion \citep{adolphs:2000_role,pitcher:2008_transcranial}.
There is also  evidence indicating that such processes might influence sensorimotor processes responsible for preparing or monitoring actions \citep{hill:2013_modulation},
and that sensorimotor processes can in turn facilitate perceptual discrimination of expressions of emotion \citep{oberman:2007_face,enticott:2008_mirror,banissy:2011_superior}.%
\footnote{
This suggests one point at which the analogy with physical properties (see section \ref{sec:physical_properties}) breaks down.
Perceptual experience of another’s mental states, if it exists,  may involve a match in the perceiver between some of the sensorimotor processes involved in having an emotion and some of the sensorimotor processes involved in perceptually experiencing another’s emotion.
As far as the evidence considered here goes, nothing comparable occurs in the case of perceptually experiencing physical properties like solidity.
}

These findings inspire a revision to the simplistic model just considered.
When observed configurations and movements are identified as a possible expression of emotion,
a ‘hypothesis’ about which emotion is being expressed 
is actually the occurrence of processes in the observer  which would normally occur were she to  have that emotion.
This results in various outcomes being represented motorically.
And these representations in turn generate predictions about the action and its sensory consequences.
The predictions' successes or failures influence which outcomes continue to be represented motorically and selectively modulate the occurrence of the emotion-related processes in the observer.
These representations and processes influence whether and how the expression of emotion is perceptually categorised.  
This is one way in which information about  others' emotions might facilitate categorical perception.

That something like this model might describe part of what is involved in detecting expressions of emotion is not a new idea.
A key presupposition of the model 
is that familiar ideas about the role of motor representations and processes in identifying others' actions \citep[e.g.][]{Wolpert:2003mg,rizzolatti_functional_2010}
can be extended to the special case where the actions are expressions of emotion.
Several researchers have already defended detailed views along roughly these lines (including \citealp{adolphs:2001_neurobiology} as well as those already cited).
What is novel is just the claim that 
visceromotor processes involved in observing others' actions  
might influence phenomenal expectations by virtue of facilitating the categorical perception of expressions of emotion.


%\citep[p.\ 189]{rizzolatti_mirrors_2008}:
%‘The insula is the centre of this mirror system, not only because it is the cortical region in which the internal states of the body are represented, but because it is the visceromotor integra­tion centre, which, when activated, provokes the transfor­mation of sensory input into visceral reactions.’


%The phenomenal expectations are all about movements, but having phenomenal expectations about *those* particular movements depends on having represented solidity or momentum.
%So we might say that solidity is not perceptually experienced we can see the effect of solidity in facts about what we perceptually experience.
%How can I make the parallel point about emotion?

The model provides an answer to the question I started with.
One way to find evidence for or against claims about humans' abilities to perceptually experience others' mental states is to test the commitments and predictions of this model.
These commitments and predictions are not so strongly supported by the available evidence that everyone will find the model compelling. 
However, the model does have two virtues.
It makes a claim about perceptual experience of mental states precise enough to refute,
and 
links it the claim to a growing body of research.




\section{Conclusion}
What evidence might bear on questions about whether humans ever perceptually experience any of another's  mental states,
and how could these questions be made precise enough to test experimentally?
In this paper I have defended a partial answer based on the discovery that humans have categorical perception of expressions of emotion (see section \ref{sec:cp_emotion})
and 
the further conjecture that the expressions of emotion categorical perception is supposed to categorise are actions of a certain type, where these are categorised by the outcomes to which they are directed (see section \ref{sec:what_is_categorised}).
Categorical perception of expressions of emotion clearly has effects on the overall phenomenological character of experiences, for it facilitates explicit perceptual judgements of sameness and difference (as we saw in section \ref{sec:cp_emotion}).
It does not follow that expressions of emotion are perceptually experienced in the same sense in which the shapes or heights of physical objects are.
More plausibly, categorical perception gives rise to phenomenal expectations about 
action outcomes 
or about 
the bodily configurations, articulations and movements which are involved in some expressions of emotion.
On the face of it, this might suggest that evidence about categorical perception could not support a claim about the perceptual experience of mental states.
However, reflection on a parallel claim about physical properties suggests otherwise. 
In the physical case, there is some evidence that 
properties like solidity and momentum are sometimes perceptually experienced in the sense that information about them has predictable and beneficial effects on phenomenal expectations concerning the movements of objects (see section \ref{sec:physical_properties}).
Analogously we might conjecture that 
information about others' emotions sometimes 
facilitates categorical perception of expressions of emotion
and thereby 
influences phenomenal expectations. % concerning the configurations and movements of others' bodies.
This conjecture is partially supported by 
evidence that some processes involved in having certain emotions also occur in observing  them,
and that such processes facilitate perceptual discrimination of expressions of emotion either directly or by way of influencing motor processes 
(see section \ref{sec:emotions_influence_cp}).

If all of this is right there is at least one sense in which humans can sometimes perceptually experience certain of another's mental states:
information about others' emotions 
can facilitate categorical perception of their expressions of emotion
which in turn gives rise to phenomenal expectations concerning outcomes to which their actions are directed or concerning their bodily configurations, articulations and movements. 
This is not to say that humans can perpetually experience another's mental states in the sense that they can perceptually experience a physical object's shape (Section \vref{sec:phenomenal_expectations}).
But others' mental states can have predictable and beneficial effects on the 
overall phenomenal characters of perceptual experiences. 


%
%Given how speculative the argument has been, let me offer, by way of a recap, a list of increasingly tentative claims that we have considered:
%%
%\begin{enumerate}
%\item Humans' abilities to discriminate some expressions of emotion, whatever exactly expressions of emotion turn out to be, are sometimes underpinned by perceptual processes which categorise stimuli.
%\item The things categorised by these perceptual processes are not static configurations: they are events.
%\item The things categorised by these perceptual processes are  outcomes to which certain actions, `expressions of emotion', are directed.
%%Alternative may be worth considering: emotions have systematic effects on configuration and kinematics and CP of expressions of emotion is a process which works back from configurations and joint displacements to the emotions which influence them (so on this view understanding action is a matter of identifying two factors, goals and emotions, which have interdependent effects on bodily configurations and joint displacements).
%\item As categorised perceptually, the outcomes to which  expressions of emotion are directed specify mental states. 
%\item Categorical perception of expressions of emotion is a mechanism by which information about others' mental states sometimes influences the perceivers' phenomenal expectations concerning the configurations and movements of others' bodies. 
%\end{enumerate}
%

%I do not claim that the view I have discussed in this paper must be right if humans can perceptually experience others' emotions.
%Given how little is known about expressions of emotion and how they are perceived, it is surely worth considering alternatives.
%The view I have discussed  is  worth considering because
%it is 
%consistent with the available evidence and 
%refutable by new findings.

It is worth emphasising that there are gaps in the evidence.
The view we have been discussing  is  worth considering because
it is 
consistent with the available evidence and 
refutable by new findings,
not because there is decisive evidence for it.
In fact,
at present it appears that there is less evidence for 
the claim that humans can sometimes perceptually experience any of another's emotions
than for the claim that an imaginary incorporeal human could perceptually experience physical properties of objects like solidity or momentum. 
But emotions are among the mental states most likely to be perceptually experienced,
and research on how perceptual processes track expressions of emotion is most likely to provide evidence relevant to evaluating claims about perceptual experience and emotions.
The gaps in the evidence therefore reveal something important.
We are probably not yet in a position to know whether humans ever perceptually experience others' mental states.


\section*{References}

\bibliography{references/phd_biblio}

\end{document}