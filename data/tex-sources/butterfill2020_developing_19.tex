%!TEX root = master.tex

 % ∞todo Richard Moore comments: integrate!

\chapter{Referential Communication}
\label{cha:communication}

Although you and I inhabit a world of words, much of our communication does not involve language at all.
The human face is articulate in ways we are often barely aware of.
Imagination allows the hands to describe almost anything given a suitable context.
And humans can refer to objects and events, present and past or even merely imaginary, by pointing with fingers, eyes, shoulders or almost any body part.
In our quest to understand communication and its role in explaining the developmental emergence of knowledge, we will first focus on the kind that does not require language (in this chapter) before turning to the additional complexities language involves in the next chapter.


As a topic, communication is far too complex to be taken whole.
Instead we will focus on particular kinds of communication.
This chapter is all about communicative actions which, like pointing to inform someone about the location of a missing puzzle piece,  are nonlinguistic, purposive and referential.
How does nonlinguistic, purposive referential communication develop?
And what understanding drives the production and comprehension of such  communication in infancy?

Answering these questions will eventually prove essential for understanding the nature and developmental emergence of linguistic communication, and for understanding how humans come to know simple facts about colours and minds, among other things.

But first let me explain the topic.
\emph{Nonlinguistic} communication is communication that does not involve language.
(Or, more carefully, it is communication that does not depend on expressions having syntactic structure---we shall consider syntax in \cref{cha:words}.)
Almost any nonlinguistic action can be communicative in the right context.
That yawn reveals Ayesha’s tiredness, while the slight quickening of Beatrice’s movements hint at her impatience.
As these examples indicate, not all communicative actions are purposively communicative.
In yawning, Ayesha accidentally reveals her tiredness.
This goes against her intention to give the impression that she is alert and captivated by what Beatrice is telling her.
But when Ayesha later narrows her eyes and frowns to signal her dissent, her actions are \emph{\gls{purposively communicative}}.
It is part of her purpose in acting to communicate.
Even so, her actions do not fall into the category that we shall focus on in this chapter.
That is because Ayesha’s frowning does not involve reference to an object.
Contrast what happens when she points at her watch to indicate that time is pressing.
This is \emph{\gls{referential communication}}.
She is referring to her watch in order to communicate with Beatrice, without relying on language.

% You might wonder why we are considering nonlinguistic communication at all.
% Part of the answer is that doing so will turn out to be essential for understanding linguistic communication, which in turn is critical for understanding the developmental emergence of knowledge.
% As we will see (in \cref{cha:words}), there are objections to currently dominant accounts of how children acquire capacities to communicate with language.
% Resolving these objections will involve recognizing that linguistic communication is rooted in, and continuous with, nonlinguistic communication.
% Relatedly, Tomasello  argues that capacities to communicate by language are acquired ‘as a kind of by-product of social interaction with adults’ \citep[p.~90]{Tomasello:2003fk}.
% To make sense of this view, we need to think first about how nonlinguistic communication emerges from social interaction, and then how linguistic communication develops from nonlinguistic communication.



\section{Infants’ Pointing Is Referential}
\label{sec:infants-pointing-is-referential}

Infants spontaneously point from around 12 months of age.
When, for example, they are with a caregiver and see a novel object appear, they will spontaneously point to it.
But why do infants point in this kind of situation?

\citet{Liszkowski:2006ec} consider four possibilities.
One is that infants’ pointing actions are merely responses drawn out of them by novel events.
If this were right, their pointing actions would serve no purpose at all.
So infants’ pointing would not be purposively communicative.
A second possibility is that infants point merely to draw attention to themselves.
This would make sense insofar as infants are generally attention seeking and pointing generally causes others to look at them.
But it would also imply that infants’ pointing is not referential.
A third, converse, possibility is that infants point merely to draw attention to the thing they are pointing to.
Perhaps they regard the novel event as something that may need action from an adult, and so want to be sure adults around them have noticed it.
Fourth and finally, it is possible that infants point to an object to initiate joint engagement in that object with another.
This possibility is attractive insofar as we have seen that infants of this age are capable of initiating joint actions (see \cref{sec:joint-action-first-years-of-life}).
And if this possibility obtained, we could also be confident that infants’ pointing involves both an addressee and an object or event that is referred to.

But how could we decided which of these four possibilities best characterises how infants point?
Consider how an adult might respond to an infants’ pointing action.
She might ignore it.
Or she might look at the infant and produce some positive infant-directed facial gestures.
If infants point merely to draw attention to themselves (which is the second of the above four possibilities),
then their pointing actions will fail when adults ignore them but succeed when adults attend to them.
By contrast, if the first possibility is correct, if infants’ points are merely responses drawn out of them by novel events, then their pointing actions will be equally successful whether the adult ignores them or attends to them.
As this illustrates, the different possibilities have different implications about  which responses from an adult are necessary for infants’ pointing actions succeed.

We can exploit this fact to generate predictions about infants’ behaviours.
To see how, consider what happens if a communicative action fails.
In general, failure is likely (or more likely than success) to dispose you to attempt a further communicative action.
At least, this is true in the short run.
But if you continually fail to communicate with someone, you are likely (or more likely than when you succeed) to attempt fewer communicative acts.
Given that twelve-month-olds are like adults in this respect, we can expect successful pointing to be distinguished from unsuccessful pointing in two ways.
Successful pointing should result in fewer repeated attempts to point in the short term, and it should result in more repeated attempts to point in the long term.
So measuring repeated attempts to point will enable us to work out when twelve-month-olds’ pointing actions succeed and when they fail.

This is exactly what \citet{Liszkowski:2006ec}  did.
They created a situation in which novel objects appeared unexpectedly from behind a large sheet, causing infants
to spontaneously point.
How the adult reacted differed across four conditions:
%
\begin{enumerate}
 \item the adult ignored pointing actions;
 \item the adult looked at the infant (but not the thing pointed to);
 \item the adult looked at the thing pointed to (but not the infant);
 \item the adult looked backwards and forwards between the thing pointed to and the infant.
  \end{enumerate}
%
\citet{Liszkowski:2006ec} observed that twelve- and eighteen-month-olds’ responses barely differed between the first three conditions.
However, infants in the fourth condition repeated pointing actions less often in the short term, and they pointed more often than infants assigned to the other conditions in the long term.
This indicates that it was only the fourth kind of response,  looking both at the object pointed at and  at the infant, which resulted in infants’ pointing actions succeeding.
And this in turn suggests that infants cannot be pointing for no reason, nor in order to draw attention merely to themselves or merely to the object they are pointing at.
Instead twelve-months-olds appear to point in order to initiate joint engagement in an object with another.

This tells us that pointing actions by twelve-month-olds are referential.
Rather than merely drawing attention to themselves, infants are performing a communicative action which involves both an addressee and an event or object.

\citeauthor{Liszkowski:2006ec}’s findings also hint that infants’ pointing actions may be purposively communicative.
It is perhaps hard to imagine that they are merely accidentally communicative in the way that an unsuppressed yawn or groan can be---if they were merely accidentally communicative, why would infants expect a certain kind of response?
But arguments from what you or I find hard to imagine are not compelling: sometimes the difficulty in imagining something arises from our limits rather than from how the facts lie.
Is there more direct reason to believe that twelve-month-olds’ pointing is purposively communicative?




\section{Infants’ Pointing Is Purposively Communicative}
\label{sec:purposively-communicative}
If an action is purposively communicative, then whether it occurs should depend to some extent on how strong the reasons to communicate are.
Imagine that Ayesha, without seeing what she is doing, accidentally knocks a fork off the table, so that it falls under the table and out of her view.
Contrast this situation with one that is as similar as possible except that Ayesha sees the fork fall.
There is more reason to communicate with Ayesha about the fork in the former situation just because it is likely that she does not know where the fork is.
Do infants distinguish the two situations appropriately by pointing more in the former?

\citet[experiment 2]{Liszkowski:2008al} answered this question.
They contrived pairs of situations of just the kind described (see \cref{fig:liszkowski_2008_fig3}).
In both situations of each pair, the protagonist performed a task which clearly involved a particular object.
She then nudged the object off the table, so that it slid down and away.
The difference was just that in one situation she did not see the object’s departure, whereas in the other situation she did.
The twelve-month-old infants who observed these situations were never involved in the action, and it was clear that they would not be able to interact with the objects or the protagonist.
Some infants nevertheless pointed to the nudged object, but nearly all the pointing occurred in the situation where the protagonist did not see the object slide off the table.
This indicates that giving infants more reason to communicate means they point more, just as the conjecture that infants’ pointing is purposively communicative predicts.

\addFigure{liszkowski_2008_fig3}{%
Two situations in which someone knocks a container from the table; in the first she does not see what she has done.
Source: \citet[figure 3]{Liszkowski:2008al}}

Further support from this conjecture comes from a study on pointing to warn.
Using similar minimally different, contrasting situations,
\citet{knudsen:2013_oneyearolds} showed again that twelve-months-olds tended to point more when there was a stronger reason to communicate about a problem.
In this case, the stronger reason to communicate was the protagonist’s unawareness of the problem:
infants pointed less when the protagonist was aware of the problem (and when there was no problem) than when she was unaware of it.

Pointing actions are not the only referential, purposively communicative actions.
An adult might reach for something that is clearly out of her reach in order to communicate that they would like it.
Can infants do this too?
\citet{ramenzoni:2016_social} used contrasting situations similar to those used in pointing studies to investigate eight-months-olds’ reaching actions to out-of-reach objects.
They observed that infants reached more when doing so would plausibly serve some communicative purpose.
This indicates that reaching actions may be purposively communicative even for infants as young as eight months old.

Overall, then, there is much evidence that infants can perform actions, including pointing and reaching, that are purposively communicative.
But just how sophisticated is infants’ pointing?



\section{The Block--Slab Model}
\label{sec:block-slab}
To say that twelve-month-olds’ pointing is purposively communicative and referential, as we have done, is not yet to say that it is particularly sophisticated.
Wittgenstein famously describes a four-word ‘language’ used by builders:
%
\begin{quote}
‘%Let us imagine a language for which the description given by Augustine is right. \ldots\
A is building with building-stones: there are blocks, pillars, slabs and beams.
B has to pass the stones, and that in the order in which A needs them.
For this purpose they use a language consisting of the words ‘block’, ‘pillar’, ‘slab’, ‘beam’.
A calls them out; B brings the stone which he has learnt to bring at such-and-such a call’
\citep[§2]{Wittgenstein:1953mm}.
\end{quote}
%
This story says everything there is to say about the two builders’ use of language.
Each word has one purpose and is never used for anything else.
Can we extend this \gls{block--slab model} to characterise infants’ pointing too?

An immediate objection to doing so is that twelve-month-olds will point for a variety of reasons.
They will point at something because they want it,
or because they want to inform you about it,
or because they want to initiate joint engagement with it \citep{Tomasello:2007fi}.
But perhaps this objection is not very informative.
It seems we could extend the block--slab ‘language’ so that each word had three uses instead of one.
Theoretically what matters is that we can codify rules specifying the use, or uses, of each word.

Why does the block--slab model fail to characterise adult pointing?
Imagine that we are doing a puzzle and I point to one of the puzzle pieces.
What I am trying to achieve in pointing will depend on the context.
It might be that I am trying to get you to pass me the piece, or I might be trying to draw your attention to a piece that fits in the bit you are working on.
Or, if we are tidying up, I might be trying to get you to put away a piece.
Instead of a fixed use or set of uses, there is an open-ended range of things I might be trying to achieve.
And to know what I am trying to achieve, you will have to take the context into account.

This is potentially confusing.
Given that my pointing action is purposively communicative, what I am trying to achieve is to communicate about the puzzle piece.
So don’t all pointing actions have a single purpose?
At this point it is vital to distinguish between communicative and \glspl{extra-communicative purpose}.
In pointing at a puzzle piece, you or I perform an action whose purpose is to communicate about that puzzle piece.
But this communicative purpose is not the only purpose we have in pointing, of course.
There is always some further, extra-communicative purpose, such as getting me to tidy up the piece (say).
This purpose is extra-communicative in the sense that it might in principle be achieved without communication at all.
It is a purpose you could rationally have even if you were entirely incapable of communication and knew this.
And the extra-communicative purpose is related to the communicative purpose as means to end.
Communicating about the puzzle piece is a means to of getting me to tidy it up.
Contrast the block--slab model.
We might say that on this model there is a fixed range of extra-communicative purposes.
But that understates the contrast.
On the block--slab model there is no need at all to distinguish communicative and extra-communicative purposes.
Each word has a fixed use (or uses) and that is all there is to say about it.
It is because adults’ pointing can serve an open-ended range of ends that we have to distinguish communicative and extra-communicative purposes.
As we will see, this is a source of theoretical complexity.

Can infants’ pointing be characterised by the block--slab model?
To address this question, \citet[][experiment 1]{Liebal:2010lr} created a situation in which two adults were involved in two activities with an infant.
One was playing a puzzle game, the other was playing a clean up game.
The researchers set things up so that at a critical moment, there was a single puzzle piece to which either adult could point (see \vref{fig:liebal_2009_fig1}).
Would infants respond differently to the two adults, in ways appropriate given the differences in their activities?

\addFigure{liebal_2009_fig1}{%
Layout for an experiment on infants’ abilities to respond to pointing actions in ways appropriate to the context of the pointing action.
Source: \citet[][figure 1]{Liebal:2010lr}}

The results showed that 18-month-olds did, but 14-month-olds did not.
This may be because 14-month-olds have difficulty keeping two activities in mind, or because there are limits on their willingness to cooperate with different experimenters.
In a second experiment, \citeauthor{Liebal:2010lr} therefore simplified their procedure.
There were still two adults, but this time infants only performed an activity with one of them.
Now even the 14-month-olds responded differently depending on context.
They responded to pointing actions by tidying up the piece pointed to only when the person pointing was someone they had recently been involved in a tidying up activity.
These findings indicate that infants’ pointing cannot be characterised by the \gls{block--slab model}.
Like adults, infants are sensitive to context in identifying the \glspl{extra-communicative purpose} of a pointing action.
The pointing actions they comprehend can serve an open-ended range of extra-communicative ends.


\section{Communicative Intention}
\label{sec:communicative-intention}
What model of communication characterises infants’ production and comprehension of pointing gestures?
We have just seen that the \gls{block--slab model} fails.
The model requires that communicative actions are treated as having a fixed set of uses.
But infants’ responses to pointing actions can vary appropriately depending on context.
Is there a better model of communication?

First step back.
What is a \gls{model of communication} ?
It is a way that communicative activities could be.
In searching for a model of communication we are not attempting to provide a true account of the nature of communication.
Rather, we are attempting to characterise communication from the point of view of a twelve-month-old infant.
A model of communication provides the basis for a \gls{computational description} of infant communication.

\citet{Moll:2007gu} as well as \citet{csibra:2010_recognizing} independently suggest that the model of communication must incorporate  \gls{communicative intention}.
But what is a communicative intention?
Suppose you are faced with a choice between two containers.
One has a reward in it, the other holds nothing for you.
Not knowing which is which, you are glad when a friend points to one of the containers for you.
Your friend doing the pointing plausibly has two intentions (at least).
She intends to get you to choose a particular container,
and she intends to communicate with you about this.
The first is an \gls{extra-communicative intention}, one that is not intrinsically tied to communicative action and which could be achieved in any number of ways not involving communication.
But the second intention is a communicative intention.
It is an intention to communicate.

Why suppose that communicative intentions are needed in a model of infant communication?
\citet{Moll:2007gu}  answer this question by highlighting an experiment by \citet{hare_chimpanzees_2004}.%
\footnote{%
A related argument and complementary findings are reported in \citet{tomasello:1997_comprehension}. \citet{moore:2015_twoyearold} present an interesting development.
And \citet{csibra:2010_recognizing} offers a different line of argument.
}
 The experiment contrasts comprehension of two actions, a failed reach and a pointing action.
Both a failed reach and a pointing action could be used to indicate which of two containers an object is in
(see \vref{fig:hare_2004_fig4}).
\citeauthor{hare_chimpanzees_2004} therefore used a very simple task.
Subjects were placed in front of two containers, one empty and the other containing a reward.
Their task was to choose a container; they could have the reward only if they chose the correct container.
But before they chose, they observed either a failed reach towards one of the containers or a point towards it.
In every case the failed reach and the point indicated the container with the reward in it.
Infants can do this sort of task given either a failed reach or a pointing action from fourteen months of age or earlier \citep{Behne:2005qh}.
%(And, incidentally, they distinguish communicative points from similar but non-communicative bodily configurations.)
But contrast chimpanzees.
They had no problem getting the reward when observing a failed reach to the correct container.
But they were significantly less good in getting the reward when offered a pointing action.%
\footnote{%
The difference between the two conditions is not due merely to the fact that one involves a human and the other a chimpanzee.
Participants were also successful when the failed reach was executed by a human rather than another chimpanzee \citep[][experiment 1]{hare_chimpanzees_2004}.
}
 \citet[p.~578]{hare_chimpanzees_2004} suggest that the chimpanzees did not understand the pointing action as a communicative gesture but did gradually learn to exploit it as a cue to the location of the reward over the course of several trials.
If this is right, chimpanzees failed to understand something about the pointing action.
%(Actually the apes were above chance in using the point, just better in the failed reach condition.  Hare et al comment ;chimpanzees can learn to exploit a pointing cue with some experience, as established by previous research (Povinelli et al. 1997; Call et al. 1998, 2000), and so by the time they engaged in this condition they had learned to use arm extension as a discriminative cue to the food’s location' \citep[p.~578]{hare_chimpanzees_2004}.)
But what?
%
\addFigure{hare_2004_fig4}{%
A failed reach (left) and a pointing action (right) can both be used to indicate which of two containers an object is in.
Source: \citet[figure 4]{hare_chimpanzees_2004}}
%
In principle it might be that chimpanzees failed to associate the pointing action with any target at all.
But in fact chimpanzees did follow the point to a container \citep[see][p.\
6]{Moll:2007gu}.
So what did the chimpanzees fail to understand?
\citet{Moll:2007gu} suggest that it was \gls{communicative intention}.

In short, \citeauthor{Moll:2007gu}’s idea is that infants’ understanding of communicative intention is what explains the difference between their performance and that of the chimpanzees.
Accepting this claim leaves us with a problem.
The problem is to explain what communicative intention is, from the point of view of an infant.
So far I have said that a communicative intention is an intention to communicate.
But this takes us in a tight circle, of course.
We will not get far by giving a model of communication simply by saying that it involves communicative intentions where these are in turn characterised as intentions to communicate.
If we accept \citeauthor{Moll:2007gu}’s suggestion that
a model of communication capable of characterising infant communication must incorporate  {communicative intention}, then we will need a more substantive account of communicative intentions as formed and comprehended by one-year-olds.%
\footnote{%
This is not to say that there is anything incorrect about the claim that a communicative intention is an intention to communicate.
As \citet[p.~11]{Davidson:1994ol} writes, ‘The intention to be taken to mean what one wants to be taken to mean is \ldots\ clearly the only aim that is common to all verbal behaviour.’
What matters here is just that recognising this is not sufficient to provide a model of communication with the potential to illuminate what communication is from the point of view of a one-year-old.
}


\section{The Gricean model}
\label{sec:gricean-model}
Our aim in this chapter is to explore how nonlinguistic, purposive referential communication develops and what understanding underpins such communication in infancy.
To this end we need a model of communication that characterises communication from the point of view of a one-year-old.
So far we have seen that a minimally complicated option, the \gls{block--slab model}, fails.
Instead, the view we are currently considering is that a model of communication for infants will involve communicative intention.
We therefore need some handle on communicative intentions as formed and comprehended by infants.



\citet[p.~706]{Tomasello:2007fi} propose that
%
\begin{quote}
     ‘infant pointing is best understood---on many levels and in many ways---as depending on uniquely human skills and motivations for cooperation and shared intentionality.’
\end{quote}
%
What do these researchers have in mind by appealing to ‘shared intentionality’?
In developing this idea, \citet{tomasello:2014_natural} appeals to Grice’s theory of communication (see also \citealp[pp.~89ff]{tomasello:2008origins}).

What is Grice’s theory?
A full answer involves many details, and the theory was elaborated and reworked over many years (see \citealp{Neale:1992uw} for a guide).
But, following Tomasello, we need consider only the gist of the theory.
Although elegant and simple, Grice’s core ideas can be difficult to get your head around at first.
Start with an example.
Suppose Ayesha waves at Ben to get him to come over.
Ayesha’s goal is to get Ben to come over.
Her means of achieving this is to get Ben to recognise that this is what she intends.
So when she waves, her intention is that waving will let Ben know that she intends him to come over.
As this illustrates, there are sometimes things you can achieve just by letting people know that you intend to achieve them.

Grice’s view, put very roughly, is that to achieve things in this way is to perform an act of communication.
What, then, is a communicative action?
It is an action done with an intention to provide someone with evidence of an intention with the further intention of thereby fulfilling that intention \citep[chapter 14]{Grice:1989ha}.

To illustrate, let us apply this view to \citeauthor{hare_chimpanzees_2004}’s study contrasting failed reaches with pointing (see \vref{fig:hare_2004_fig4}).
As the subject of this experiment, you are required to choose between two containers.
Helpfully, an adult points to the left container (say).
What is she doing, according to Grice?
She is intending that:
%
\begin{enumerate}
  \item you open the left container
  \item you recognize that she intends (1), that you open the left container; and
  \item that your recognition that she intends (1) will be among your reasons for opening the left box.%
  \footnote{%
  Compare \citet[p.~151]{Grice:1967we} and  \citet[p.~544]{Neale:1992uw}.
  For more on Grice’s view and development, see \citet{moore:2016_gricean}.
}
\end{enumerate}
%
This suggestion is attractive insofar as it provides an informative, noncircular way of distinguishing communicative intentions from extra-communicative intentions.
The first intention, (1), that you open the left container, is extra-communicative because it is something you could intend outside the context of communication.
By contrast, the second and third intentions, (2) and (3), jointly comprise your communicative intention.
These are the intentions in virtue of which your action is purposively communicative.

Grice aimed to given an account of what communication is.
But we are not concerned with the nature of communication.
Instead our concern is with characterising infants’ abilities to communicate purposively.
Grice’s account seems well suited to this purpose insofar as it applies not just to linguistic communication but also to nonlinguistic communication.
In fact, one of Grice’s aims was to understand how linguistic communication could emerge from nonlinguistic communication.
So although Grice himself was not attempting to model infants’ communication, it seems a good bet that his account might be used to specify such a model.

Here, then, is the \gls{Gricean model} of communication.
To produce a communicative action involves acting with an intention to provide someone with evidence of an intention with the further intention of thereby fulfilling that intention.
And to comprehend a communicative action is to know that the communicator has such intentions.
As far as I know, the Gricean model, unlike the \gls{block--slab model}, does not generate predictions that conflict with the available evidence about infants’ nonlinguistic communication.
But there is a reason for considering alternatives.

If you read \cref{cha:joint-action}, you may notice that Grice’s ideas about purposive communication resemble Bratman’s about shared intention.
Both thinkers are challenged by an narrowly circular idea, either that a communicative intention is an intention to communicate or that a shared intention is some intentions to perform a joint action.
And both respond to the challenge by layering intentions and knowledge to create nested structures.
In both cases, the layering response imposes hard limits on what could be explained by appeal to joint action or communication.
The hypothesis that the Gricean model characterises infants’ nonlinguistic communication is incompatible with ambitions to invoke abilities to communicate in explaining the developmental emergence of certain kinds of knowledge.


\section{Presupposing Prevents Explaining}
\label{sec:presupposing-prevents-explaining}
On the \gls{Gricean model}, to communicate something to me involves you having intentions about my recognition of your intentions.
So if the Gricean model correctly characterises infants’ communication, we cannot hope to explain the developmental emergence of knowledge of others’ minds.
Instead, their abilities to communicate presuppose that infants have such knowledge.
Their abilities to communicate also presuppose, indirectly, knowledge of actions, physical objects and causal interactions.
It would therefore be valuable to identify an alternative model, one that does not presuppose such knowledge.

It may be tempting to follow \citet{tomasello:2014_natural} in regarding this as a nonissue.
On his view knowledge of minds, actions, physical objects and causal interactions provides the basis for the emergence of communicative skills, which he in turn takes to explain the emergence of larger scale social norms and conventions, and of communication by language.
But, like nearly everyone else, Tomasello can offer no explanation for how knowledge of minds, actions, physical objects and causal interactions emerges in development.
In attempting to find such an explanation, it would be useful to know whether it is really impossible to invoke one-year-olds’ communicative abilities.
After all, the main reason for accepting the \gls{Gricean model} of infant communication is the lack of a viable alternative.

The position we have reached can be summarised with an inconsistent tetrad:
 \begin{enumerate}
   \item 11- or 12-month-old infants produce and understand nonlinguistic, referential actions which are purposively communicative.
 \end{enumerate}
Evidence for this first claim comes from studies by Liszkowski and others (see \cref{sec:infants-pointing-is-referential,sec:purposively-communicative}).
\begin{enumerate}[resume]
  \item Producing or understanding pointing gestures involves understanding communicative intention.
\end{enumerate}
This claim was motivated by rejection of the \gls{block--slab model} and a contrast between chimpanzees and infants (see \cref{sec:communicative-intention}).
\begin{enumerate}[resume]
  \item A communicative action is
  an action done with an intention to provide someone with evidence of an intention with the
  further intention of thereby fulfilling that intention.
\end{enumerate}
This third claim is the claim that Gricean model correctly characterises infants’ nonlinguistic, referential actions  which are purposively communicative.
I take it to be endorsed by \citet{Tomasello:2007fi} and \citet{tomasello:2014_natural} among others.
\begin{enumerate}[resume]
  \item Abilities to perform  nonlinguistic, referential actions that are purposively communicative facilitate the developmental emergence of knowledge of minds, actions, physical objects or causal interactions.
\end{enumerate}
%
Which claim should we reject?
As already implied, Tomasello and some of his collaborators accept claims (2) and (3), so must reject claim (4).
I want to pursue an alternative line.
I would like to reject claims (2) and (3), in order to be able to hold on to (4) and leave open the possibility of appealing to those communicative abilities in explaining the developmental emergence of knowledge.
But of course doing this requires an alternative model of communication in infancy, one that doesn’t involve
appeal to intentions about, or knowledge of, minds, actions, physical objects or causal interactions.

\citet{moore:2016_gricean} grapples with this challenge.
His strategy is to retain the second claim, (2), while providing a less demanding account of communicative intention.
He provides an alternative to (3) and a new model of communication in infancy that is in some ways simpler than the Gricean model while retaining many of its virtues.
 (Tomasello also considers Moore’s model plausible; see \citeyear[p.~73]{tomasello:2014_natural}).
However, accepting Moore’s model would appear to require rejecting claim (4).
% p. 18: ‘attributing communicative intentions to others requires no more than entertaining a pair of first-order metarepresentations of the goals that S has with respect to H.’
% see also footnote 7, p. 10
His primary concern is to defend the idea that infants’ communicative abilities ‘play a role in cognitive development’, but not necessarily in the development of knowledge of minds, actions, physical objects or causal interactions.
Can we go further and provide a model of communication that is both consistent with what we know about infants and does not presuppose knowledge of minds, actions, physical objects or causal interactions at all?


%\citep[p.~73]{tomasello:2014_natural}: ‘This may not be one multiply embedded communicative intention, as in the Gricean analysis, but rather, as argued by Moore (in press) two singly embedded intentions: I intend that you notice that this communicative act is for you, plus I intend that you know that the banana is in that bucket. Nevertheless, the single embedding in this second intention is already more than great apes can do, and so it represents a new form of recursive inference’
% \citep{moore:2016_gricean}: ‘an alternative has been thought to be a non-starter (Sperber 2000). This is the possibility that the abilities (a)—(c) often thought necessary for Gricean communication were themselves enabled—or at least greatly facilitated—by language. On the account defended here, it is possible to envisage this situation.’


\section{From Joint Action to Communication}
\label{sec:from-joint-action-to-communication}
A model of communication needs to characterise both comprehension and production of communicative actions.
Start with comprehension.
Suppose someone attempts to reach an obviously out-of-reach toy (\citealp{ramenzoni:2016_social}; see \cref{sec:purposively-communicative}).
You might suppose that she is acting on an intention to communicate.
But  in order to respond appropriately to her action, it is not actually necessary that you do so.
You might equally recognise that she wants to obtain the toy and pass it to her for that reason.
Sometimes recognising an \gls{extra-communicative purpose} is sufficient for responding appropriately to a purposively communicative action.

Can we generalise this idea to other cases of communication?
Consider pointing rather than failed reaching.
There are two obstacles to the idea that recognising an \gls{extra-communicative purpose} of a pointing action is sufficient for responding appropriately to a purposively communicative action.
The first obstacle is this: pointing actions, unlike failed reaches, do not seem to have a single extra-communicative purpose.
And, secondly, it is unclear how infants might detect the extra-communicative purpose of an action other than by identifying a communicative purpose.
After all, communicative actions like pointing are typically only means to achieving things in virtue of successful communication occurring.
Consider someone with no understanding or experience of communication applying the \gls{Teleological Stance} (see \cref{sec:teleological-stance}) in an attempt to identify the goal of a pointing action.
She asks herself, Which outcome is this pointing movement a most rational means of achieving?
As a noncommunicator, she has little hope of finding an answer that will enable her to respond appropriately to the communicative action.

Reflection on joint action shows that there is a way around these obstacles.
Infants in the second year of life are capable of participating in, and initiating, simple forms of joint actions (see \cref{cha:joint-action}).
Further, there is a model of joint action in infancy which does not require knowledge or shared intentions but only expectations and collective goals (see \cref{sec:expectations-about-collective-goals}).
Being able to initiate, and participate in, joint actions provides infants with a way to track goals which is independent of, and complementary to, the \gls{Teleological Stance}.
There are various cues which can signal, in the right context, that another is about to engage in joint action with you, such as eye contact and smiles.
These cues let you know that another will engage in joint action with you without specifying what the collective goal will be.
But sometimes it is obvious to you what the collective goal must be simply because you are acting on some goal and unwilling to switch to another goal.
To illustrate, suppose that you have been tasked with choosing between two containers, only one of which contains a reward.
Another provides some cue or other to indicate that she is about to be involved in a joint action with you.
She then points to one of the containers.
As someone with no understanding or experience of communication, using the \gls{Teleological Stance} would not enable you to identify the goals of her pointing action.
But you reason that since she is engaged in joint action with you, a goal of her action must be the goal of your action, namely to retrieve the reward.%
\footnote{%
\citet{butterfill:2012_interacting} offers more detail on how this inference works.
Of course the inference can often go wrong, not least because those who point for us do not always have the same goal as us. But the inference is supposed to be useful and need not always, or even often, preserve truth.
}

So abilities to perform joint actions provide infants with a way of tracking goals that is independent of, and complementary to the \gls{Teleological Stance}.
And this way of tracking goals enables them to identify an extra-communicative goal of pointing actions (and other purposively communicative actions)  even without any understanding or experience of communication.

But how would this enable someone to respond appropriately to a communicative action?
We saw earlier that even chimpanzees who do not respond appropriately to pointing actions may nevertheless associate a pointing action with its target (\citealp[][p.\
6]{Moll:2007gu}; see \cref{sec:communicative-intention}).
As this illustrates, associating a pointing gesture with its target does not imply understanding it as a communicative action, nor being able to respond appropriately.
We can therefore assume, without introducing circularity, that humans likewise can associate pointing actions with objects independently of understanding any communicative intention.
Now consider encountering a pointing action.
You understand nothing of communication, but there are cues that the communicator is about to engage in some joint action with you.
Since you are dedicated to your task, you know that the goal of the joint action is to retrieve the reward; and so you can infer that a goal of the communicator’s action is to retrieve the reward.
Further, you associate the communicator’s action with one of the containers.
So in performing an action with the goal of retrieving the reward she has done something associated with that container.
This provides you with a reason, not decisive but probably compelling in the situation, to choose that container.
And so it is that you can respond to her pointing action without any understanding of communication or communicative intentions.

The strategy for responding appropriately to communicative actions without any understanding of  communicative intentions can be used for a wider range of communicative gestures including single-word utterances.
The basic requirements are three.
First, in a particular context, you must associate a communicative gesture with its referent.
You must associate the pointing gesture with the object indicated, and  your must  associate an utterance of `daddy' with the daddy.
Twelve-month-olds appear to meet this requirement, as
some associations between words and objects  appear to be in place by six months of age or earlier \citep{tincoff:1999_beginnings_,tincoff:2011}.
Second, you must be capable of joint action and able to detect some cues indicating that another is about to engage in joint action with you.
And, third, you must be able to anticipate the goal of the joint action without already being able to track the goals of the communicator’s actions.
When these  requirements are met, you can respond appropriately to communicative actions which occur, or seem to you to occur, in the context of joint action using this inference:
%
\begin{enumerate}
  \item We are about to engage in some joint action or other.
  \item A goal of this joint action must be to retrieve the reward (either because I am not about to change the single goal to which my actions are directed, or else because there is no other comparably salient goal)
\end{enumerate}
Therefore:
  \begin{enumerate}[resume]
  \item A goal of your action is to retrieve the reward.
\end{enumerate}
But:
\begin{enumerate}[resume]
\item Your action is associated with the blue container.
\end{enumerate}
  Therefore:
  \begin{enumerate}[resume]
  \item I should act on the blue container in whatever ways will best achieve the goal.
\end{enumerate}
%
In order to exploit the route to responding appropriately to communicative actions this inference provides, it is necessary that you can track the goals of actions.
But it is not necessary that you understand communicative intentions.
In describing an inference, I am not suggesting, of course, that infants or other communicators have to infer a response.
The inference provides us as theorists with a partial specification of a model of communication.

In many cases, following the above inference would involve misunderstanding the communicative action.
Arguably many communicative actions are not joint actions with collective goals like that of retrieving the reward.
It is, after all, possible to provide information through communication while not caring at all about the goals of the recipient’s actions.
But such misunderstandings arising from following the inference are productive: they do no harm, and they enable you to respond appropriately to a communicative action in almost the way you would if you understood intentions to communicate.


\section{Producing Communicative Actions}
\label{sec:producing-communicative-actions}
We have just seen that abilities to participate in joint actions could in principle enable you to respond appropriately to  communicative actions without understanding communicative intention.
But what about the production of communicative actions?
Could abilities to initiate, and to participate in, joint actions enable you to produce apparently communicative actions.

Recall that infants will sometimes reach for an obviously out-of-reach toy in a way that appears to be purposively communicative  (\citealp{ramenzoni:2016_social}; see \cref{sec:purposively-communicative}).
Why might someone do this?
One possibility, of course, is that she might intend you to recognise that she intends you to pass her the toy because you recognise that she intends this.
But we are attempting to characterise communication without invoking any such communicative intentions.
Why else might someone do this?

In some situations, performing an action directed to a goal where it is manifest that you will fail can be a rational way of initiating joint action.
To illustrate, first consider a slightly different scene.
Ayesha is at a loud party and recognises that several heavy tables need to be moved.
As conversation is impossible over the music, she simply takes one end of a nearby table and starts to lift, perhaps looking around to make eye contact with potential helpers.
You might think that the goal of Ayesha’s action cannot be to lift the table as it is manifestly impossible for her to lift the table alone.
But as a matter of fact the party is one of those situation in which other people are likely to track the goals of Ayesha’s action and to anticipate that she cannot lift the table alone.
Often enough this will motivate them to perform actions that are also directed to the goal of lifting the table.
Ayesha need not be aware of these facts.
Even if she is unaware of them, this is a situation in which it is rational for Ayesha to perform actions directed to lifting the table.
Her success will depend on the contribution of others, but Ayesha need not think of herself as initiating joint action.
It is sufficient that her actions do as a matter of fact nonaccidentally initiate a joint action.

In many situations where others are around and the costs of failure would not be too great, if you perform actions directed to a goal and your actions manifestly cannot succeed, others will help you.
This can  make it rational for you to perform actions directed to a goal even if you know full well that your actions alone cannot bring that goal about.
So when others are around, someone might reach  for  an obviously out-of-reach toy without any expectation of succeeding in reaching it alone, but with a reasonable expectation that success will ensue somehow.

Can we extend this idea to pointing and other referential communicative actions?
There are at least three obstacles.
First, pointing is unlike reaching in that there is apparently no \gls{extra-communicative purpose} which all pointing  actions serve and pointing actions are not ideal means to achieving anything except by means of communicating with others.
This makes it difficult to understand what might motivate someone who had no understanding or experience of communication to perform a pointing action.
Second, we know that infants point with an expectation that there will be joint action involving the thing they point to (see \cref{sec:infants-pointing-is-referential}).
It seems this feature of pointing cannot be captured by a model on which communication is just about achieving extra-communicative purposes.
Third, we also know that infants point to warn others about lost items or potential dangers even when there is manifestly no prospect that the infant will be given the item or be able to do anything further about the danger (see \cref{sec:purposively-communicative}).
This also would appear to be incompatible with a model on communicating is a matter of performing actions directed to some extra-communicative goal.

We can get around the second two obstacles by a further appeal to joint action.
Suppose you point to a puzzle piece and I respond by picking it up and adding it to the jigsaw puzzle.
In some situations it would be reasonable for us to say, ‘We completed the puzzle.’
Likewise, if I point to warn you of some danger, either of us might say ‘We avoided the danger.’
There is, then, no obstacle to thinking of such actions as your pointing to an object and my doing something with it as joint actions in the minimal sense (see \cref{cha:joint-action}).
That is, our actions can be thought of as collectively directed to the completing the puzzle or avoiding danger.%
\footnote{%
Of course, you could also think of our actions as quite separate---I am not suggesting that it is necessary to think of them as involving joint action, only that it is coherent and harmless to do so.
What matters is how the infant sees things, not how they actually are.
}
When an infant points, she may be initiating joint actions directed to such collective goals as these.
And since we are already committed to infants’ being able to initiate joint actions (see \cref{sec:joint-action-first-years-of-life}), there is no additional theoretical cost in adding to our model of communication the principle that pointing gestures and one-word utterances are attempts to initiate joint action.

But recall the first obstacle.
Our model of communication needs to provide motivation to perform pointing actions (and to make one-word utterances).
What extra-communicative goal might all pointing actions have?
The answer is so obvious that it is easy to miss.
One extra-communicative goal of all pointing actions is to point.
For comparison, consider a grasping action.
To grasp something is not merely a matter of it becoming somehow secured in your fingers, hands, toes, elbows, knees or mouth: it is to perform an action directed to the goal of grasping that thing.
The same applies to kicking, reaching, throwing and other actions.
What distinguishes grasping a ball from it merely ending up secured in your fingers (say) is that to grasp it is to perform an action directed to the goal of grasping that ball.
Similarly, to point at something is to perform an action directed to the goal of pointing to that object.
And to produce a one-word utterance is to perform an action directed to articulating that word.



Here then is a minimal model of the production of communicative actions, one on which \glspl{communicative intention} play no role.
There is some goal which you manifestly cannot achieve alone, such as Ayesha’s avoiding a dangerous item.
It seems as if no action of yours could sensibly be directed to this goal: you cannot reach the dangerous item or otherwise act on it.
Except that you can point at it, or produce a one-word utterance directed to it (or do both).
Because Ayesha is around and the costs of failure would not be too great, you can rationally perform such an action with the further (collective) goal of avoiding the dangerous item.
Because your actions will manifestly not succeed without some contribution from Ayesha, and because Ayesha can track the goal of your action, she will likely respond by contributing to a joint action.
You therefore point to the dangerous item, and in doing so attempt to initiate joint action directed to the collective goal of Ayesha avoiding a dangerous item.
This, in barest outline, is
how possessing abilities to initiate joint actions
could enable an individual to perform what appear to be  actions that are referential and purposively communicative.%
\footnote{%
For an alternative, more minimal model of the production of communicative actions, see \citet{townsend:2016_exorcising}.
}

This model implies that when an infant points to something, she has a goal concerning that object.
Is this right?
Even twelve-month-olds seem satisfied if, in response to a pointing gesture, you merely look at them and the object (see \cref {sec:infants-pointing-is-referential}).
This may appear incompatible with the idea that the infant represents the situation in terms of any goal, let alone a collective goal.
But looking at something can be a purpose action, and it also among the things that we can do together.
As long as we avoid thinking about goals in a very narrow way, there is nothing obviously wrong with the idea that pointing is purposive.

We have now constructed a rough model of communication.
The model needs to be revised and elaborated in various ways, of course---twelve-month-olds are too sophisticated, and too quirky, to be tied down quite so simply.
But the model is at at least descriptively adequate to much pointing behaviour and so might capture some core facts about humans’ earliest pointing behaviours.
On this model communication requires abilities to track the goals of others’ actions (see \cref{cha:action}) and to participate in, and  initiate, joint actions (see \cref{cha:joint-action}).
But it does not require an ability to form, nor to ascribe, \glspl{communicative intention}.


A true Gricean is likely to say that the actions characterised by this model are not communicative at all.
This is no objection to any claim that we need to maintain.
What matters for our purposes is not whether the model accurately characterises communication,
but whether it characterises one-year-olds’ production and comprehension of pointing gestures and other apparently communicative actions such as one-word utterances.
And the evidence we have reviewed in this chapter provides no reason to deny this.



\section{Conclusion}
From around twelve months of age, infants can perform, and respond appropriately to, a range of nonlinguistic, referential actions that are purposively communicative (\cref{sec:infants-pointing-is-referential,sec:purposively-communicative}).
What drives the production and comprehension of such communicative actions?
Could it be simply a matter of assigning each each word or gesture  a fixed use or fixed set of uses, as on a
\gls{block--slab model}?
Apparently not.
Infants’ abilities to respond appropriately to variations in the context of communication provide evidence that we cannot characterise their communication with a \gls{block--slab model}.
Instead we must recognise that gestures like pointing are used by infants, as by adults, for an open-ended range of \glspl{extra-communicative purpose}.

Rejecting the block--slab model of communication means we are confronted with a choice between two alternative models.
The leading, best developed model is a \gls{Gricean model}, which turns on the notion of \gls{communicative intention} (see \cref{sec:communicative-intention,sec:gricean-model}; \citealp{moore:2016_gricean}).
The hypothesis that this model correctly characterises twelve-month-olds’ communicative activities implies that they have knowledge of minds and of actions.
It is therefore incompatible with attempts to explain the developmental emergence of such knowledge by appeal to communicative abilities (see \cref{sec:presupposing-prevents-explaining}).

This motivated the construction of an alternative model of communication, one on which communicative intentions play no role at all.
Instead, communication requires abilities to track the goals of others’ actions and to participate in, and  initiate, joint actions (see \cref{sec:producing-communicative-actions,sec:from-joint-action-to-communication}).

The aim of constructing this model of communication was to show that a surprisingly rich and flexible range of communicative actions can be characterised by appeal to joint action and goal tracking without any need for communicative intentions.
It is unlikely that, as it stands, the model we have constructed entirely succeeds in characterising one-year-olds’ referential, purposively communicative actions.
But the model could be elaborated in various ways without violating the requirement that it must not involve communicative intentions or anything else that would presuppose knowledge of minds or actions.

Which model most accurately characterises twelve-month-olds communicative activities?
Answering this question would require further experiments.
Evidence that twelve-month-olds distinguish failures of communicative intentions from failures of extra-communicative intentions would support a model featuring communicative intention.
On the other hand, evidence that twelve-month-olds’ abilities to communicate are linked to, and limited by, their abilities to initiate, and participate in, joint actions would support  a model featuring joint action and goal tracking.

The two models have different implications for how nonlinguistic, referential actions that are purposively communicative emerge in development.
If we invoke \glspl{communicative intention}, we need to explain how these emerge in development.
How does an infant first come to understand the possibility of getting others to do something by means of their recognising the infant’s intention that they do this thing?
On the alternative, joint action model of communication, the capacity to have and ascribe communicative intentions appears in development some time after infants can perform and respond appropriately to a range of referential, purposively communicative actions.
It would therefore be possible, if the joint action model were correct, to invoke successful communication in explaining the developmental emergence of communicative intentions.
And if the joint action model is correct, the question about how referential,  purposively communicative actions  emerge in development simply does not arise.
From the twelve-month-olds’ point of view these actions are much like any other purposive actions.
 They are simply attempts to get things done.


%%% Local Variables:
%%% TeX-master: "master"
%%% End:
