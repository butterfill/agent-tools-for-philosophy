%!TEX root = master.tex

\chapter{A Theory of Goal Tracking}
\label{cha:theory-goal-tracking}

As understanding actions is fundamental for profoundly social animals, it is should come as no surprise that goal tracking appears early in development, perhaps from around three months of age.
But when we attempt even so much as to describe infant goal tracking, we immediately encounter two puzzles (see \cref{cha:action}).
We need to understand why infants sometimes rely on statistical regularities and sometimes on the \gls{Teleological Stance}; and we need to understand how, and why, their goal tracking is linked to their abilities to act.
To this end we need a theory of goal tracking.

As in the case of tracking physical objects and their interactions (see \cref{part:physical-objects}), it makes sense to pick an uncomplicated theory as our starting point and only complicate matters as needed.


\section{The Simple View}
\label{sec:simple-view-goal-tracking}
If infants use the Teleological Stance, do they thereby come to know facts about the goals of actions?
Csibra and Gergely appear to hold that they do.
In their view, applying the Teleological Stance is a matter of reasoning explicitly about actions, about the constraints under which they are performed and about their goals.
They stress  continuities between goal tracking in infants and explicit reasoning in adults \citep{Gergely:2003gb,Csibra:1998cx},
and they describe applying the Teleological Stance as a matter of using knowledge in drawing inferences \citep{csibra:2013_teleological}.
% ‘when taking the teleological stance one-year-olds apply the same inferential principle of rational action that drives everyday mentalistic reasoning about intentional actions in adults’
% `Such calculations require detailed knowledge of biomechanical factors that determine the motion capabilities and energy expenditure of agents. However, in the absence of such knowledge, one can appeal to heuristics that approximate the results of these calculations on the basis of knowledge in other domains that is certainly available to young infants. For example, the length of pathways can be assessed by geometrical calculations, taking also into account some physical factors (like the impenetrability of solid objects). Similarly, the fewer steps an action sequence takes, the less effort it might require, and so infants’ numerical competence can also contribute to efficiency evaluation.’
\citet[p.~31]{Woodward:1998dm} also holds that infants’ goal-tracking abilities involve knowledge of action.
% ‘early in life, infants begin to set up a system of knowledge of human action that has features in common with more mature understandings, and that is distinct from their knowledge of inanimate object motion’

Let us recycle a label and call Csibra and Gergely’s view \emph{The \gls{Simple View}}:
%
\begin{quote}
 The principles comprising the Teleological Stance are things we know or believe, and we are able to \glspl{track a goal} by making inferences from these principles.
 \end{quote}
 %


 This view is theoretically coherent.
 Recall figure \vref{fig:teleological_stance_diy}.
 You or I can use the principles comprising the Teleological Stance to reason explicitly about the goal of the movements represented in this figure.
 According to Csibra and Gergely, what we are doing in this case is what infants (and adults) are doing whenever they apply the Teleological Stance.
 It follows that infants who apply the Teleological Stance thereby come to know facts about the goals of actions.
 
 One consequence of this view is that we cannot appeal to abilities to communicate with language, nor to rich forms of social interaction in explaining how humans first come to know simple facts about the goals of actions.
 Instead such an explanation would have to draw on experiences available in the first three months of life, and on innate capacities.
 
But is the Simple View correct?
Unlike in the case of physical objects, I know of no case in which the Simple View concerning the goals of actions generates incorrect predictions.
(Perhaps because there is less research.)
There is, however, an alternative to the Simple View and one that is arguably better supported.
As in the case of physical objects, identifying a compelling alternative to the Simple View is best done by considering how adults  track the goals of actions.


\section{The Motor Theory of Goal Tracking}
\label{sec:motor-theory-goal-tracking}

Adults’ eyes show that they are capable of tracking the goals of others’ actions extremely rapidly, in mere fractions of second.
How do we know?
When an adult reaches for an object, her eyes are not typically on her hand but on the object she is reaching for.
The same is true of someone observing an action: the observer’s eyes are typically on the object someone is reaching for, rather than on the reaching hand  \citep{Flanagan:2003lm}.
These eye movements are proactive: they occur in advance of the action and reveal anticipation of how the action will unfold in the observer.
We know that these proactive eye movements reflect goal tracking because they can be influenced by information about the kind of action being observed.
For consider that whether someone will grasp an object with her whole hand or between her finger and thumb can often be seen in the shape of the hand, even while the hand is still some distance from the object (see \cref{fig:ambrosini_2013_fig1ii}).
When observing someone reaching to grasp one of two different sized objects, adults will proactively look to the object indicated by the grasping hand’s shape \citep{ambrosini:2011_grasping}.
This proactive look indicates goal tracking.

%
\begin{figure}
\begin{center}
\includegraphics[width=0.6\textwidth]{fig/ambrosini_2013_fig1ii}
\caption{
	\label{fig:ambrosini_2013_fig1ii}
Which of the two balls is this person about to grasp?
Source: \citet[][figure 1 (part)]{ambrosini:2013_looking}}
\end{center}
\end{figure}
%


How are adults able to track the goals of actions they observe so rapidly?
Part of the answer involves \glspl{motor representation}.
These are the representations characteristically involved in preparing, performing and monitoring sequences of small actions such as grasping, transporting and placing a fragile egg.
We know that performing such actions involves representations of some kind because facts about how subsequent parts of the action will eventually unfold influence how earlier parts of the action are performed \citep[for example,][]{kawato:1999_internal,zhang:2007_planning}.
We also know that these representations are not any familiar kind of representation such an intentions or knowledge states because doing things like grasping, transporting and placing an egg involves satisfying constraints not normally considered in explicit practical reasoning.
This is the reason for postulating another kind of representation, the \emph{\gls{motor representation}}.

Although they were postulated to explain facts about the preparation and performance of actions, it turns out that \glspl{motor representation} live a double life.
Motor representations concerning a particular type of action are involved not only in performing an action of that type but also sometimes in observing one.
That is, if you were to observe someone grasp one of the two objects in \cref{fig:ambrosini_2013_fig1ii}, motor representations would occur in you much like those that would also occur in you if it were you---not her---who was doing the grasping \citep{rizzolatti_mirrors_2008,rizzolatti:2016_mirror}.

Why do motor representations live a double life?
Why might they occur in someone who is not acting other than in observing an action?
Consider again someone observing another reaching for a ball in order to grasp it, as in \cref{fig:ambrosini_2013_fig1ii}.
Suppose you were to interfere with her ability to represent actions involving the hands motorically, either by tying her hands (as \citealp{ambrosini:2012_tie} did) or by using transcranial magnetic stimulation to temporarily suppress neural activity in bits of the brain most linked to motor representation (as \citealp{costantini:2013_how} did).
In both cases you would drastically reduce or even eliminate the proactive gaze.
This suggests that motor representations concerning the observed action can facilitate rapid goal tracking in adults.
Which is puzzling.
Motor representations were postulated to explain action performance.
How could such representations also facilitate goal tracking?

An answer is given by what I shall call the \emph{\gls{Motor Theory of Goal Tracking}}.
In short, this theory says that some \gls{pure goal tracking} is acting in reverse (see \cref{fig:sinigaglia_2015_fig1}).
More carefully, the idea is this.
In observing an action, all kinds of outcomes may be represented motorically in you  more or less simultaneously.
These motor representations trigger processes associated with preparing for, performing and monitoring actions, just as they would if it were you, not her, who was acting.
As when you are actually acting, these processes lead to expectations concerning which movements you will observe and the sensory effects of these movements.
The extent that these predictions are incorrect determines the probability that the motor representation triggering them will be dropped.
In many (but not all) cases, this will ensure that motor representations of outcomes other than the goals of the observed action are less likely to be sustained in you than motor representations of the action’s goal.
 And so it is that motor processes in you, as the observer of an action, can ensure, in a limited but useful range of circumstances, that outcomes represented motorically in you are goals of the action you are observing \citep{sinigaglia:2015_goal_ascription}.%
\footnote{%
Versions of the Motor Theory of Goal Tracking have been defended under various names by many researchers including \citet{Falck-Ytter:2006dg}, \citet{kanakogi:2011_developmental}, \citet{ambrosini:2013_looking},  \citet[p.~204]{gredeback:2010_infantsa} and \citet{green:2016_culture}.
\citet{gredeback:2015_eye} offer a detailed review of research with adults and infants.
Note that  researchers differ on how what I am calling the Motor Theory of Goal Tracking relates to the Teleological Stance; some regard the Motor Theory as a competitor to the Teleological Stance rather than (as I will suggest in \cref{sec:motor-theory-vs-teleological-stance}) a further specification of it.
}


\addFigureWidth{0.6}{sinigaglia_2015_fig1}{The Motor Theory of Goal Tracking
Source: \citet[][figure 1]{sinigaglia:2015_goal_ascription}}

% %
% \begin{figure}
% \begin{center}
% \includegraphics[width=0.6\textwidth]{fig/sinigaglia_2015_fig1}
% \caption{
% 	\label{fig:sinigaglia_2015_fig1}
% The Motor Theory of Goal Tracking
% Source: \citet[][figure 1]{sinigaglia:2015_goal_ascription}}
% \end{center}
% \end{figure}
% %

The \gls{Motor Theory of Goal Tracking} seems prone to generating confusion. 
To avoid some common causes of confusion, consider three points.
First, the Motor Theory does not postulate that motor processes operate ‘in reverse’: it relies only on the idea that motor processes take representations of outcomes and compute behavioural and sensory outcomes (as can be seen from the directions of the arrows in \cref{fig:sinigaglia_2015_fig1}).
Second, the Motor Theory does not depend on first identifying an outcome to which an observed action might be directed.
This is unnecessary because multiple means--ends computations can occur simultaneously, or at least rapidly enough for action preparation to involve selection on the basis of multiple means-ends computations \citep[for example,][]{wolpert_internal_1998}.
Consequently goal tracking that involves motor processes can begin with a wide range of candidate outcomes.
Third, the Motor Theory does not entail that all goal tracking involves motor processes.
It is consistent with the fact that some goal tracking can be achieved in other ways, for example through deliberation.
The claim is only that motor processes and representations can enable \gls{pure goal tracking}.


\section{The Motor Theory and the Teleological Stance}
\label{sec:motor-theory-vs-teleological-stance}

How is the \gls{Motor Theory of Goal Tracking} related to the \gls{Teleological Stance}?

The \gls{Teleological Stance} provides the basis of a \gls{formally adequate} account of goal tracking in infants and adults: 
disregarding limits on memory, attention or processing speed and the like, someone who took the principles comprising the Teleological Stance to be true and made appropriate inferences from these together with observations of bodily configurations and joint displacements could reliably---although not invariably, of course---reach correct conclusions about the goals of actions (see \cref{sec:teleological-stance}).

The Motor Theory depends on the assumption that the Teleological Stance is indeed formally adequate:
it provides an account of how motor representations and processes could, within limits, implement the computations described by the Teleological Stance.

The Motor Theory also provides a possible link between the Teleological Stance and the mind of an individual.
Suppose we ask,
What links the mind of an individual to the principles comprising the Teleological Stance?
One such account of the link is given by the \gls{Simple View}: the principles comprising the Teleological Stance are known by the adults and they track goals by making inferences from these principles.
Another possible account is suggested by the Motor Theory of Goal Tracking:
the principles comprising the Teleological Stance characterise how motor processes in that individual enable goal tracking.%
\footnote{%
\citet{csibra2008action} and \citet{southgate:2013_infants} offer  alternative accounts of the relation between the Teleological Stance and the role of motor representations in observing actions.
On both their views, motor representations enable predicting joint displacements, bodily configurations and the sensory effects of actions once a goal has been identified (as the Motor Theory also claims); but, contrary to the Motor Theory, they do not matter for identifying goals in the first place.
}


% Accepting this leaves open whether the Teleological stance is \glsadd{descriptively adequate} descriptively or \gls{explanatorily adequate}.
% To address these questions, we seek an account of the link between the principles of the Teleological Stance and the mind of an individual.

%In accepting the Motor Theory, we are therefore committed to the Teleological Stance being approximately correct too.

Can we use the Motor Theory to characterise infants’ earliest ability to track the goals of actions?
Consider what I will call the \emph{\gls{Developmental Motor Conjecture}}:
\begin{quote}
	In the first nine months of life, all \gls{pure goal tracking} is explained by the \gls{Motor Theory of Goal Tracking}. Other goal-tracking processes emerge later in development.%
	\footnote{%
	This conjecture is inspired by \citet{gredeback:2015_eye}, \citet{hunnius:2014_what}
	and \citet{woodward:2014_mirroringa} among others.
	These authors have interestingly different theoretical positions and would be unlikely to endorse the conjecture, for good reasons (see below).
	However, they all provide considerations which motivate considering this conjecture.
	}
\end{quote}
%
This conjecture has a striking virtue.
As we saw, in the first nine months of life, infants’ goal tracking abilities bear an interesting, hard-to-pin-down relation to their abilities to act (see \cref{sec:tracking-is-acting-in-infancy}).
The \gls{Developmental Motor Conjecture} implies that these infants’ goal-tracking abilities should be limited by their abilities to represent actions motorically.
Since abilities to represent actions motorically are loosely related to abilities to perform those actions, the \gls{Developmental Motor Conjecture}  has the potential to explain why infant goal-tracking abilities should be so limited.

There is just one problem.
On the face of it, the \gls{Developmental Motor Conjecture} appears to be completely untenable.
We know that, as a general rule, movements of simple geometric shapes such as those used in \citet{Gergely:1995sq}’s experiment (see \cref{sec:infants-track-goals}) are unlikely to be represented motorically.
And one implication of the conjecture is that infants’ goal-tracking abilities should be limited by their abilities to represent events motorically.
And yet, as we saw, there seems to be abundant evidence that infants can track the goals of actions performed by geometric shapes, cartoon fish, and the like.

This appears to be a compelling reason to reject the \gls{Developmental Motor Conjecture}.%
\footnote{%
Note that this does not entail rejecting the \gls{Motor Theory of Goal Tracking}, which merely states that some \gls{pure goal tracking} involves only motor representations and processes.
}
But appearances can be deceptive.
Despite appearing obviously wrong,  the \gls{Developmental Motor Conjecture} may actually be correct. 
To see why, we first need to distinguish between targets and goals.


\section{Target vs Goal}
\label{sec:goals-target-vs-type} \label{sec:perceptual-animacy}

The  orthodox view we have been uncritically following so far ignores a critical distinction between {goals} and {targets}. 
The \emph{\gls{target}} or \emph{targets} of an action (if any) are the things the towards which it is directed.
If the \gls{goal} of an action is to kick a particular football, this football is the action’s target.
To specify a target of an action is to partially specify one of its goals.
But more is required to fully specify a goal, of course.
A goal typically involves a type of action---kicking rather than smashing, say.
It may also involve one or more manners of action---discretely, firmly, and precisely, for example---and perhaps more besides (see \cref{fig:anatomy_of_a_goal}).



\begin{figure}
	\begin{center}
	\includegraphics[width=0.8\textwidth]{fig/anatomy_of_a_goal}
	\caption{
		\label{fig:anatomy_of_a_goal}
		Fully specifying a goal can involve giving a type of action, a target, some manners of action, and more.
		}
	\end{center}
\end{figure}



Why does the distinction between targets and goals matter?
In adults there is a capacity to track targets only.
This is called \emph{\gls{perceptual animacy}},
the detection by broadly perceptual processes of animate objects and their targets.
To illustrate, consider an experiment by \citet[experiment 1]{gao:2009_psychophysics}.%
\footnote{%
This line of investigation builds on  earlier work by \citet{dittrich:1994_visual}.
Those authors use the term ‘goal’ for target.
% Rochat summary: ‘Dittrich and Lea (1994) further documented the propensity of adults to perceive intentional motion in the animation of nondescript objects. Participants were presented with letters moving on a computer screen. Their task was to detect the one letter that did not move randomly, its motion being oriented toward one of the distractor (ie randomly moving) letters. Results showed, once again, that such detection depended on precise spatial and temporal features of the dynamic display, such as the directness of the movement trajectory or the speed advantage of one dynamic element over another.’ !!!
}
Adults were shown a display which contained some moving circles.
In some cases the circles moved independently of each other, but in other cases there was a ‘wolf’ which chased a ‘sheep’ with varying degrees of subtlety (see \cref{fig:gao_2009_fig2}).
The adults’ task was simply to detect the presence of a wolf.
\citeauthor{gao:2009_psychophysics} established that adults can do this providing the chasing is not too subtle.
In further experiments, they also showed that adults’ abilities to perceptually detect chasing depend on several cues including whether the chaser ‘faces’ its target (‘directionality’) and how directly the chaser approaches its target (‘subtlety’).

%
\begin{figure}
\begin{center}
\includegraphics[width=0.6\textwidth]{fig/gao_2009_fig2}
\caption{
\label{fig:gao_2009_fig2}
Schematic representation of a chasing event in an experiment on
perceptual animacy.
Source: \citet[figure 2]{gao:2009_psychophysics}
}
\end{center}
\end{figure}
%

Perceptual animacy is commonly interpreted as a case of goal tracking.%
\footnote{%
\citet{schlottmann:2010_goal,Scholl:2000eq}  all claim that perceptual animacy is a matter of, or involves, tracking goals.
}
But it is important to make a distinction.
Perceptual animacy involves tracking targets only.
The type and manner of action are unspecified and irrelevant, as are any further features of outcomes (see \cref{fig:anatomy_of_a_goal}).
So perceptual animacy only counts as goal tracking in an attenuated sense.

The detection of animacy appears to be a broadly perceptual phenomena since it depends on areas of the brain associated with vision and influences how perceptual attention is allocated \citep{scholl:2013_perceiving} irrespective of your beliefs and intentions \citep{buren:2016_automaticity}.
Perceptual animacy also appears to depend on simple cues and heuristics involving motion trajectories like directionality and subtlety, as we say.
Applying these heuristics would be detrimental to proper goal tracking.
It is therefore unlikely that perceptual animacy is correctly described by the \gls{Teleological Stance}.
Whatever broadly perceptual abilities underpin perceptual animacy are probably distinct from those which enable goal tracking proper.

To avoid confusion, let us distinguish \emph{merely} tracking targets from \emph{proper} goal tracking, which is goal tracking that is not merely target tracking.
Perceptual animacy involves merely tracking targets.

Are infants in the first nine months of life capable of proper goal tracking?
Nearly all of the studies we considered in \cref{cha:action} do not distinguish whether infants in the first nine months of life are merely tracking targets of actions or whether they are properly tracking goals.
They do not show, for instance, that infants can identify the type of an action---whether it is a grasping or a pushing action, say.
To say that infants can properly track goals and not merely targets implies, minimally, that they can distinguish both the target and the type of an action.

So can infants also distinguish between two actions which are directed to the same target but differ in type?
To answer this question, we would ideally have pairs of scenarios in which the target of an action is kept constant while the type of action varies.
To the extent that subjects respond appropriately to the difference in type of action, we can be confident that they can distinguish actions not just by their targets but also by their types.

\citet{Behne:2005dw} created just such pairs of contrasting scenarios (albeit for a different purpose).
In one of their contrasts, an experimenter holds a ball out for an infant to grasp and then either ‘accidentally’ drops it or teasingly pulls it back.
So in each case there is a goal-directed action involving the ball, but in one case the goal of the action is to pass the ball to the infant whereas in the other case the goal is to tease the infant.
\citet[Study 2]{Behne:2005dw} found that nine-month-olds (but not six-month-olds) consistently and appropriately discriminated between these scenarios by, for example, banging more when the ball was ‘accidentally’ dropped than when it was teasingly retracted.
This and other research
(for example, \citealp{ambrosini:2013_looking} discussed in \cref{sec:tracking-is-acting-in-infancy};
% \citet{ambrosini:2013_looking} is important: shows that type of grasp and not just target is tracked, implying that anticipatory looking is not merely perceptual animacy.
 \citealp{kochukhova:2010_preverbal,green:2016_culture}) suggests that, at least from nine months of age, infants can indeed  distinguish both the type and target of a goal-directed action.

The fact that infants in the first nine months of life are capable of proper goal tracking indicates that their abilities cannot be entirely a consequence of perceptual animacy.
But this leaves open the possibility that some some of the experimental observations standardly considered to support goal tracking in infancy may in fact be explained by the perceptual detection of animacy.
Reflection on this possibility enables us to develop a theory and solve the twin puzzles about infants’ goal tracking.

\section{A Dual Process Theory of Goal Tracking}
\label{sec:theory-of-goal-tracking}

In the first year of life, infants’ abilities concerning physical objects appear to involve at least two distinct kinds of process, one broadly perceptual and the other broadly motoric (see \cref{cha:causation}).
Perhaps something similar is true of their abilities concerning actions.

The \gls{Developmental Motor Conjecture} states that all goal tracking in the first nine months of life is explained by the \gls{Motor Theory of Goal Tracking}.
As we saw (in \cref{sec:motor-theory-vs-teleological-stance}), this Conjecture has the potential to explain how and why infants’ goal-tracking abilities are linked to their abilities to perform actions.
However, it faces an objection: nine month olds seem to exhibit goal tracking when confronted with scenarios involving self-propelled balls \citep{Csibra:1998cx} or cartoon fish \citep{daum:2012_actions}, which are unlikely to trigger motor processes.
But what if the supposed goal tracking in these situations is mere target tracking, and what if it were a consequence of perceptual animacy?

Recall \citet{Gergely:1995sq}’s groundbreaking study with the balls (see \cref{sec:infants-track-goals} and \vref{fig:gergely_1995_fig1,fig:gergely_1995_fig3}).
In principle, this effect is no less likely to be a consequence of perceptual animacy than of goal tracking.
There are hints that infants in the first year of life can perceptually detect animacy \citep{rochat:2004_who}.
% ***TODO : this is 8-10 months and not younger!  So actually would appear to conflict with G+C 1995 (=6 months). However, see Rochat et al 1997 for 3 months+ showing some perceptual animacy involving relations among moving objects \ldots\   Also note that Rochat et al 2004 does not involve cues to perceptual animacy that Schlottmann shows are essential to getting the effect but only bare chasing stimuli. So there are grounds to disagree with their claim, \citet[p.~367]{rochat:2004_who}, that ‘Our results suggest that it is at around the age of 8 ^ 10 months that infants start to show a propensity analogous to what Heider and Simmel (1944), and Michotte (1963) described in the verbal report of adults who saw similar abstract, two-dimensional animated displays.’  Note also that Rochat et al interpret their findings in terms of inference rather than perception (p. 367: ‘These results also corroborate the findings of Gergely et al (1994, 1995) and Csibra et al (1999) demonstrating that 9-month-olds are capable of making teleological inferences on the basis of abstract entities interacting at a distance on a screen.’ and p. 358: ‘by the age of 8^10 months infants do start to infer intentional roles in abstract computer displays, inferring who is doing what to whom. By this age, infants start understanding in addition to perceiving social causality’)
And because \citet{Gergely:1995sq}’s  effect appears to depend on cues to animacy \citep{schlottmann:2010_goal},
% %
% \footnote{%
% Interestingly, effects widely interpreted as goal tracking involving nonbodies appear to depend on cues to animacy. Compare \citet[p.~395]{Biro:2007rz}: ‘Three cues were used that have been widely suggested in the literature as critical to goal-attribution: self-propelledness, equifinal variations of the action, and a salient action-effect. Overall, our findings demonstrate that the presence of these cues can elicit infants’ goal-directed action interpretations, while the lack of these cues can prevent them from interpreting actions as goal-directed.’
% This is a hint that the processes 
% }
an interpretation in terms of perceptual animacy cannot currently be ruled out.

These reflections motivate considering what I will call \emph{\gls{Conjecture MP}}:
\begin{quote}
	In the first nine months of life,
	all proper pure goal tracking is explained by the Motor Theory.
	Other pure goal-tracking processes emerge later in development.
	Further, a mere target-tracking process is also present in these infants.
	This process is identical to \gls{perceptual animacy} in adults.
	And appearances that these infants’ pure goal-tracking abilities are not limited by
		what they can represent motorically are misleading:
		they are due to mistaking mere target tracking for proper goal tracking.%
		\footnote{%
		This conjecture refines and extends \citet{gredeback:2010_infantsa}’s dual process account, which they offer to explain their findings.
		There are some differences:
		they take the \gls{Teleological Stance} to be an alternative to what I label the \gls{Motor Theory of Goal Tracking} rather than something presupposed by the Motor Theory; and they take the Teleological Stance to describe what I conjecture are the effects of perceptual animacy (p.~205).
		}
\end{quote}
%
This conjecture provides a \gls{dual process theory} of goal tracking. 
It invokes distinct kinds of motor and perceptual processes, which are involved in distinct kinds of tracking (namely goal tracking proper and mere target tracking).

The objection to the \gls{Developmental Motor Conjecture} was that nine-month-olds can track the goals of animate balls and cartoon figures, whereas that conjecture implies they could not.
\gls{Conjecture MP} overcomes this objection by allowing that infants‘ goal-tracking involves two distinct kinds of process, one perceptual and the other motoric.

Although \gls{Conjecture MP} is yet to be tested directly, it does generate readily testable predictions.
One source of predictions is the fact that perceptual animacy involves only broadly perceptual processes whereas goal tracking is facilitated by motor processes.
Further, perceptual animacy and goal tracking have distinct \glspl{signature limit}.
Goal tracking can be impaired by tying an observer’s hands or otherwise interfering with her capacities to represent actions motorically (see \vref{sec:motor-theory-goal-tracking}).
By contrast, the perceptual detection of animacy can be impaired by manipulating unrelated factors like directionality and subtlety \citep{gao:2009_psychophysics}.
A final source of predictions is the distinctive effects of perpetual animacy on attention \citep{buren:2016_automaticity}.


\gls{Conjecture MP} is quite likely to be wrong, just like any other bold and as yet untested conjecture is.
But it makes a certain kind of sense.
Given the importance of action, we should expect that even in the first few months of life infants will manifest a variety of abilities in observing actions,
and that these will depend on a mix of perceptual and motor processes which are identical, or closely related, to those found in adult humans.




\section{Puzzles Solved?}
\label{sec:action-puzzles-solved}

In  \cref{cha:action} we encountered two puzzles about goal-tracking in the first nine months of life. \gls{Conjecture MP} offers a possible explanation of both puzzles.

The first puzzle was why infants sometimes respond on the basis of statistical regularities whereas at other times their responses are in line with the \gls{Teleological Stance}.
\gls{Conjecture MP} suggests us a candidate answer, as summarised in \cref{table:action-puzzle-solution}.
This candidate answer depends on three further, plausible assumptions.
The first is that where scenarios involve human agents, either perceptual animacy processes will not occur or else they will tend to be dominated by motor processes in infants’ responses.
The second assumption is that, in infants at least, perceptual animacy processes can drive looking times and pupil dilation but not anticipatory looking.
The third assumption is that infants’ responses will typically only be dominated by statistical regularities when no form of goal tracking is possible.
Given these further assumptions, \gls{Conjecture MP} implies that infants’ responses to action scenarios should manifest reliance on statistical regularities when they are observing actions they cannot perform and responding in a way that manifests anticipation of how the action will unfold.
And, as we saw in \cref{sec:statistical-regularlities}, this is just what \citet{daum:2012_actions} and others have observed.


\begin{table}

  \begin{center}
  \footnotesize	%shrink for better spacing
  
  % add space between rows
  \extrarowsep=7pt
  
  \begin{tabu} to 0.8\linewidth {X[3,l] X[2,c] X[2,c]}
  
  \toprule
  
	& habituation, violation-of-expectation and pupil dilation & anticipatory looking 
	\\
  \cmidrule(r){2-3}
  observing infant- possible actions & motor representations & motor representations
  \\
  observing infant- impossible actions  & perceptual animacy & statistical regularities
  \\
  %
  \bottomrule
  %
  \end{tabu}
  \caption{Which processes will tend to dominate nine-month-olds’  responses to a scenario involving actions? 	According to \gls{Conjecture MP}, the answer depends on both scenario type (rows) and response type (columns).}
  \label{table:action-puzzle-solution}
  \end{center}	%careful -- position of this affects distance between table and caption(!)
  
  \end{table}
  
  \normalsize
  
The second puzzle about action was to understand how infants’ goal tracking is linked to their abilities to act, and why there should be any such link (see \cref{sec:tracking-is-acting-in-infancy}).
According to \gls{Conjecture MP}, the kind of link involved depends on whether the infant is merely tracking targets or engaged in proper goal tracking. 
Because proper goal tracking in the first nine months of life depends exclusively on motor processes, infants’ goal tracking should be limited by their abilities to represent observed actions motorically, much as motor-based goal tracking is limited in adults (see \cref{sec:motor-theory-goal-tracking}).
But because mere target tracking depend on perceptual animacy rather than any kind of motor process, mere target tracking should not be limited in the same way.
Given the assumption that scenarios involving self-propelled balls \citep{Csibra:1998cx} or cartoon fish \citep{daum:2012_actions} involve mere target tracking, it is no surprise that infants’ goal-tracking abilities in these scenarios are not constrained by what they can represent motorically.

We encountered these two puzzles in asking whether the \gls{Teleological Stance}  is \glsadd{descriptively adequate} descriptively or \gls{explanatorily adequate}.
Just here \gls{Conjecture MP}
 has a subtle consequence.
If this conjecture is correct, when infants are merely tracking targets, the Teleological Stance should be neither.
By contrast, when they are engaged in proper goal tracking, it should be both descriptively and explanatorily adequate.






\section{Conclusion}
The \gls{Teleological Stance} demonstrates the theoretical possibility of  \gls{pure goal tracking} (\cref{sec:teleological-stance}).
Humans are capable of \gls{pure goal tracking} from around three months of age or earlier, and they manifest extremely rapid pure goal tracking in anticipatory looking from around six months of age (see \cref{sec:infants-track-goals,sec:tracking-is-acting-in-infancy}).

Do these early goal-tracking abilities involve coming to know simple facts about the goals of particular actions?
According to the \gls{Simple View},
the principles comprising the Teleological Stance are things we know or believe, and we are able to \glspl{track a goal} by making inferences from these principles.
So tracking goals involves whatever kind of inference is involved when a detective figures out who committed a murder, and it does result in knowledge about the goals of particular actions.

The Simple View correctly characterises some goal tracking in human adults (see \cref{sec:simple-view-goal-tracking}), so it is not theoretically incoherent to suppose that it might also characterise younger humans’ goal tracking too.
But there is more to the story about goal tracking in adults.
There is evidence for the view that goal tracking in adults is sometimes a consequence of motor processes and representations only (see \cref{sec:motor-theory-goal-tracking}).
We should therefore at least consider the possibility that motor processes underpin some infant goal tracking.

A first, too simple attempt to articulate a view along these lines is the \gls{Developmental Motor Conjecture}.
This conjecture states that all pure goal tracking in the first nine months of life is explained by the \gls{Motor Theory of Goal Tracking}.
As it stands, this conjecture is untenable because it implies, incorrectly, that infants cannot track goals involving animate balls and cartoon fish(see \cref{sec:motor-theory-vs-teleological-stance}).

To make progress in understanding goal tracking, it may be essential to distinguish goals from targets, proper goal tracking from mere target tracking, and motor processes from perceptual animacy (see \cref{sec:goals-target-vs-type}). 
With these distinctions, we can construct a conjecture that provides natural solutions to two puzzles about action as well as generating some readily testable predictions (see \cref{sec:action-puzzles-solved}).
This is \gls{Conjecture MP}, which holds that goal tracking in the first nine months of life involves a combination of \gls{perceptual animacy} and motor processes.

Whether we stick with the \gls{Simple View} or accept a conjecture along these lines has implications for how the emergence in development of knowledge of action might be explained.
The \gls{Simple View} entails that infants who can track goals know simple facts about the goals of particular actions.
If we accepted the Simple View, we would therefore have to conclude that the emergence of such knowledge depends only on experiences gathered in the first three months of life and on any innate capacities.
And in invoking goal-tracking abilities to explain feats like communication later in infancy, we would be appealing to knowledge states.
By contrast, goal tracking that is correctly characterised by \gls{Conjecture MP} does not necessarily result in knowledge concerning the goals of particular actions.
According to this conjecture, goal tracking in the first nine months of live involves not knowledge states but perceptual and motor representations,  and these are not inferentially integrated with knowledge. \glsadd{inferential integration}
We may therefore suppose that knowledge of the goals of particular actions emerges much later in development, and may depend on rich forms of social interaction and perhaps abilities to communicate with words.






One consequence is that goal tracking can serve as a building block in explanations of how humans develop abilities to engage in rich forms of social interaction (see \cref{cha:joint-action}), and track others’ mental states (see \cref{cha:mind,cha:mind-solution}).
Philosophers tend to focus on these capacities and ignore the pure goal tracking that underpins them.
This is probably a mistake.
In tracking the goals of actions, you are taking the step from mere joint displacements and bodily movements to representations of the outcomes around which these are organized (see \cref{sec:joint-displacements}).
This is arguably the hardest and most significant step in making sense of others, their thoughts and the things they say.
It is like the step from specifying the states of particular electronic circuits in a computer to being able to use machine language, which abstracts from physical considerations like voltage while closely reflecting the architecture of the underlying hardware.
There is much further abstraction to come, of course.
But the transition from electronics to machine language, or from  kinematics to goals, is a foundation for everything else.


% \section{Is There Core Knowledge of Action?}
% If it is true that the Motor Theory of Goal Tracking  characterises some (or all) goal tracking in infancy, it might be reasonable to conclude that there is \gls{core knowledge} of action.
% Certainly the Motor Theory implies that goal tracking has several features associated with core knowledge.
% Since the representations involved are motor representations, they should be subject to limited accessibility and the processes involving them should be informationally encapsulated.
% And there are parallels between goal tracking in humans and other primates \citep{rizzolatti_mirrors_2008,rizzolatti:2016_mirror}, suggesting that goal tracking may arise from systems already present in the evolutionary ancestors of modern humans.
% Apparently, then, combining the Teleological Stance with the Motor Theory of Goal Tracking gives us core knowledge of action.





%*STILL TO DO: Consequence: Motor Theory of Goal Tracking solves the problem of not being too optimal.






%%% Local Variables:
%%% TeX-master: "master"
%%% End:
