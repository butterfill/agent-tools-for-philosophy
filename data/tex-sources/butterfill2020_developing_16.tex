%!TEX root = master.tex

% ∞todo : integrate some of the CEU model, including the dual processes diagrams

% ∞todo : mention Edwards’ new limit! \citep{edwards:2019_level}.

\chapter{Mind: a Solution?}
\label{cha:mind-solution}



Our aim is to understand how knowledge of minds emerges in development.
The obstacle is what I have been calling \gls{the Mindreading Puzzle}.
This puzzle arises from the individual plausibility of three claims which are collectively inconsistent.
As we saw in \cref{sec:mindreading-puzzle},
it seems that for many children, there is an age at which:
%
\begin{enumerate}
\item in performing false belief tasks which are \glspl{A-task},  the child relies on a model of minds and actions not incorporating beliefs;
\item in performing false belief tasks which are \glspl{non-A-task}, such as tasks involving anticipatory looking or violation-of-expectation, the child relies on a model of minds and actions incorporating beliefs;
and
\item the child has a single model of minds and actions.
\end{enumerate}
%
The puzzle is to work out which claim (or claims) to reject.
In \cref{cha:mind} I considered the first two claims.
For each of these claims, the reasons for accepting them outweigh the reasons for rejecting them.
By a process of elimination,
we are all but forced to consider rejecting the third and conjecturing that children (and perhaps adults too) have more than one model of minds and actions.

Rejecting the third claim is progress insofar as it allows us to avoid contradiction.
But merely rejecting the third claim raises many hard questions.
If children have two models of minds and actions, why do they use one in A-tasks and the other in non-A-tasks?
What is the nature of these models?
Why do they have more than one model?
And so on.
To gain insight into the developmental origins of knowledge of minds, we must do more than merely reject the third claim about models on the grounds that the first two claims cannot plausibly be rejected.
We must identify a theoretically coherent and empirically motivated framework for thinking about mindreading, one which will allow us to make sense of the possibility that multiple models of the mental coexist in individual humans.

In this Chapter we will examine such a framework.
We will investigate the idea that there are multiple kinds of mindreading process, and compare this with claims about \gls{core knowledge}.
We will also see how to use signature limits in evaluating theories about mindreading.
All of this will eventually put us in a position to understand how a combination of core knowledge, social interaction and linguistic structures drive the emergence, in development, of knowledge of minds.



\section{Mindreading Is Sometimes Automatic}
\label{sec:automatic-mindreading-sometimes}

% ***MUST MENTION \citep{carruthers:2015_mindreading}

% ***LATER: cannot get from Schneider to the claim that anticipatory looking is automatic in Southgate et al or Low et al procedures; this has to be an inference.

Earlier, when we had to solve a puzzle about infants’ abilities to track briefly unperceived objects (in \cref{cha:causation}), we first considered  the abilities of adults and then related these back to infants. %*TODO check ref: should be to chapter with object indexes
The same strategy is necessary if we are to understand infants’ abilities to track beliefs.

A key question about adults is whether their mindreading is  automatic.
The term ‘automatic’ is used in a variety of ways, but throughout this book a process is \emph{\gls{automatic}} just if whether or not it occurs is to a significant degree independent of your current task, motivations and intentions.
To say that mindreading is automatic is to say that it involves only automatic processes.
Is mindreading in adult humans automatic?

One way to show that mindreading is automatic would be to start with a task that does not require tracking beliefs and then to compare subjects’ performance on a measure of belief tracking in two conditions:
a condition in which they are not given any prior instructions to track beliefs, and a condition in which they are told in advance that they should track beliefs.
Given that the instructions alter  subjects' task, motivations or intentions to track beliefs,
if performance on the belief tracking measure did not vary between conditions, we could infer that mindreading occurs somewhat independently of subjects' tasks, motivations and intentions.
That is, we could infer that mindreading is automatic.


% \citet{Schneider:2011fk} did just this.
% They showed their participants a series of videos and instructed them to detect when a figure waved or, in a second experiment, to discriminate between high and low tones as quickly as possible.
% %They contrived the stimuli in such a way that a protagonist acquired a belief about the location of an object, but this belief was irrelevant to the participants’ task.
% Performing these tasks did not require tracking anyone’s beliefs, and the participants did not report mindreading when asked afterwards.
% % on experiment 1: ‘Participants never reported belief tracking when questioned in an open format after the experiment (“What do you think this experiment was about?”). Furthermore, this verbal debriefing about the experiment’s purpose never triggered participants to indicate that they followed the actor’s belief state’ \citep[p.~2]{Schneider:2011fk}
% Nevertheless, participants’ eye movements indicated that they were tracking the beliefs of a person who happened to be in the videos.
\citet{schneider:2014_task} did just this.
 They showed three groups of adults the same series of videos in which a protagonist happened to acquire a belief about the location of an object.
One group was given no instructions, while a second group was told to track the protagonist's beliefs.
For a third group, \citeauthor{schneider:2014_task}
raised the stakes by instructing participants to keep track of the actual location of a ball in the video, which  would be harder to do if they were tracking another’s beliefs.
So for the third group, tracking another’s beliefs is not merely irrelevant to their task but may actually hinder performance.
Despite this, \citeauthor{schneider:2014_task} found evidence in the adults’ looking times that they were all tracking another’s false beliefs irrespective of variations in instructions.
% See \citet[p.~46]{schneider:2014_task} `An identical 3   3   2 mixed ANOVA this time revealed no significant 3-way interaction (F = 1.05, p = 0.381), however there was a significant interaction between area of interest and belief condition,'
This is evidence for automaticity.%
\footnote{%
Although Schneider is among the authors of one failed attempt to replicate these findings \citep{kulke:2018_implicita},
two earlier reports provide successful conceptual replications \citep{schneider:2012_cognitive,Schneider:2011fk}.
Adding to the puzzle, \citet{naughtin:2017_implicit} use essentially the same paradigm as \citeauthor{schneider:2014_task} and found evidence for belief tracking in brain but not behavioural measures.
}
% \footnote{%
% % quote is necessary to qualify in the light of their interpretation; difference between looking at end (task-dependent) and at an earlier phase (task-independent)?
% %\citet[p.~46]{schneider:2014_task}: ‘we have demonstrated here that subjects implicitly track the mental states of others even when they have instructions to complete a task that is incongruent with this operation. These results provide support for the hypothesis that there exists a ToM mechanism that can operate implicitly to extract belief like states of others (Apperly & Butterfill, 2009) that is immune to top-down task settings.’
% It is hard to completely rule out the possibility that belief tracking is merely spontaneous rather than automatic, of course.
% After all, we can't be certain that manipulating  instructions, as Schneider et al did,
% I take the fact that belief tracking occurs despite plausibly making subjects’ tasks harder to perform to indicate automaticity over spontaneity.
% If non-automatic belief tracking typically involves awareness of belief tracking, then the fact that subjects did not mention belief tracking when asked after the experiment about its purpose and what they were doing in it further supports the claim that belief tracking was automatic.
% }

Further evidence that mindreading can occur in adults even when counterproductive has been provided by \citet{kovacs_social_2010} and \citet{edwards:2019_level}, who showed that another’s irrelevant beliefs  can influence how quickly people press a key,
and by \citet{Wel:2013uq}, who showed that the same can influence the paths people take to reach an object.
Taken together, we have substantial (but not conclusive) evidence that mindreading in adult humans sometimes involves automatic processes only \citep[see][for a review]{schneider:2017_current}.


\section{Mindreading Is Not Always Automatic}
\label{sec:automatic-mindreading-not-always}

Does all mindreading in adult humans involve only processes which are  automatic?
No.  It turns out that verbal responses in false belief tasks that are A-tasks are not typically a consequence of automatic belief tracking.
To show this, \citet{back:2010_apperly} instructed people to watch videos in which someone acquires a belief, either true or false, and then, after the video, asked them an unexpected question about the protagonist’s belief \citep[see also][]{apperly:2006_belief}.
They measured how long people took to answer this question.
Starting with the hypothesis that answering a question about belief involves automatic mindreading only,
they reasoned that the mindreading necessary to answer a question about belief will have occurred before the question is even asked.
Accordingly, there should be no delay in answering an unexpected question about belief---or, at least, no more delay than in answering unexpected questions about any other facts that are automatically tracked.
But they found that people were slower to answer unexpected questions about belief than predicted.
Importantly this was not due to any difficulty with questions about belief as such: when such questions were expected, they were answered just as quickly as other, non-belief questions.
It seems that, when asked an unexpected question about another’s belief, people typically need time to work out what the other believes.
While not decisive, these findings suggest that not all mindreading is automatic.%
\footnote{%
\citet[ms~p.~9]{carruthers:2015_mindreading} objects \citep[following][]{cohen:2009_encoding} that these experiments are ‘not really about encoding belief but recalling it.’
But this objection is already answered by \citet[p.~56]{back:2010_apperly}.
}



Taking these findings together with the research discussed in \cref{sec:automatic-mindreading-sometimes}, we can conclude 
 that mindreading in adults is probably sometimes automatic and sometimes not.
Mindreading that underpins anticipatory looking is often (but not always) automatic.
And mindreading that underpins success on an A-task involves processes which are not automatic.

If so, what determines whether a given response will be underpinned by an automatic belief tracking process?
This cannot be only a matter of which belief is being tracked—after all, in the experiments we have considered all the beliefs being tracked are simple beliefs about the location of an object.
Instead, whether a response is underpinned by an automatic belief tracking process depends in part on the nature of the response.
Anticipatory looking is sometimes based on automatic mindreading (although not always, as we will see below), whereas explicit answers to questions are often based on non-automatic mindreading.

How might these findings about adults help us with the mindreading puzzle, which is about development?
Earlier
(in \cref{sec:infants-track-false}) we noted that there is a discrepancy between two measures of belief tracking:
if you measure anticipatory looking, it appears that two- and three-year-olds can track others’ false beliefs, whereas if you measure verbal responses to a communicative prompt,  young children fail to track others’ false beliefs.
In this section we have considered adults' performance on the same two measures---anticipatory looking and verbal responses.
As we've seen, adults’ anticipatory looking in some false belief tasks is a result of automatic processes whereas their responses to questions are not.
Could these findings be connected?
We shall eventually see that they are.
But first we need a deeper understanding of mindreading in adults.


% will eventually see that these two observations are linked, and that the link is crucial for solving the Mindreading Puzzle.
% But for now let us imagine that the correspondence between the two observations might be a coincidence and continue to investigate mindreading in adults.
%
% The finding that some but not all mindreading is automatic raises a hard issue.
% On standard views about what mindreading involves, it is difficult to see how it could ever be automatic.


\section{A Dual Process Theory of Mindreading}
\label{sec:dual-process}
The discovery that mindreading is sometimes but not always automatic motivates considering a \gls{dual process theory} of mindreading.
 For our purposes, a \emph{\gls{dual process theory of mindreading}} is any theory on which mindreading involves two or more processes which are distinct
 in this sense: the conditions which influence whether one mindreading process occurs differ from the conditions which influence whether another occurs.
Constructing a dual process theory of mindreading is the key to linking findings about adults' performance with the puzzle of mindreading.

The dual process theory I will consider is simple:
\newcommand{\simpleDualProcessTheoryCore}{%
are two (or more) distinct, independent
mindreading processes, one more automatic than the other.
}
\newcommand{\simpleDualProcessTheory}{%
there \simpleDualProcessTheoryCore
}
\newcommand{\SimpleDualProcessTheory}{%
There \simpleDualProcessTheoryCore
}
\begin{quote}
  \SimpleDualProcessTheory
\end{quote}
\glsadd{independent processes}%
Here two kinds of processes are \emph{independent} just if the conditions which  influence whether a process of one kind  will yield an incorrect response differ to some extent from the conditions that influence whether a process of the other kind will yield an incorrect response.

You might say, this is a schematic claim, one lacking any substance.
You’d be almost right---and to lack substance is roughly the point of this dual process theory.
A key feature of this dual process theory is
its \emph{theoretical modesty}:
it involves no a priori
commitments concerning the particular characteristics of the processes.
Identifying characteristics of the process is a matter of discovery.
Further, their characteristics may vary across domains.
The characteristics that distinguish processes involved in goal tracking may not entirely overlap with those that distinguish processes involved in segmenting physical objects and representing them as persisting, for example.
% Of course, identifying one object-tracking process with the operations of a system of object indexes (see \vref{sec:CLSTX}) and another with object-tracking motor processes
% (see \vref{sec:motor}) allows us to offer rich characterisations of the nature of the processes.


There is nothing original about the idea of a dual process theory as such.
We have already encountered dual process theories in reflecting on infants’ abilities concerning physical objects (see \cref{cha:causation}) and actions (see \cref{cha:theory-goal-tracking}).
Indeed, dual process theories have been offered in several areas (see \citealp{evans:2009_two,sherman:2014_dual} for some cases).
Dual process theories sometimes involve controversial claims about features distinguishing the processes, such as consciousness or speed (see \citealp{adolphs:2000_role}).
But for our purposes, no such claims are needed.
Just as we earlier adopted a lighter account of core knowledge (in \cref{sec:core-knowledge-minimal-view}),
so we can avoid building any assumptions about the nature of distinct mindreading processes into the dual process theory itself.
Features that distinguish processes should be discovered, not stipulated.

The dual process theory of mindreading will eventually provide us with a key ingredient we need to solve the mindreading puzzle.
But, first, is this theory correct?
One way to establish the dual process theory would be to show that automatic and nonautomatic mindreading processes are both distinct and independent.

We will identify evidence for independence later, in \cref{sec:signature-limits-mindreading}.
What about distinctness?
Automatic and nonautomatic mindreading processes must be at least somewhat distinct since by definition only automatic processes will be to a significant degree independent of task, motivation and intention.
To justify accepting a dual process theory we need  evidence of further distinctness.
\citet{qureshi:2010_executive} found that  automatic and nonautomatic mindreading processes are differently influenced by cognitive load, and Todd et al (submitted% #TODO:update reference
) provided evidence that adding time pressure affects nonautomatic but not automatic mindreading processes.
As we gain more evidence for distinctness we can be more confident that the simple dual process theory of mindreading is correct.%
\footnote{%
Or it might turn out that there is no good evidence for the simple dual process theory of mindreading after all.
In that case we shall need a different the solution to the mindreading puzzle.
}
% #TODO: return to this point about ‘more evidence’ when have identity false belief tasks.  They show that you can have C without A (whereas Schneider et al, and altercentric intrusion, is A without C)


In addition to finding evidence for it,
there is another, theoretical requirement that must be met before we can accept the dual process theory of mindreading.
Why would there be two (or more) distinct and independent mindreading processes in humans?
Wouldn’t it be redundant---or, worse, chaotic---to have distinct, independent processes for a single purpose?



\section{Speed--Accuracy Trade-Offs}
\label{sec:speed-accuracy}
Any broadly inferential process must make a trade-off between speed and accuracy.
In fact this holds for all animals, not just humans (see \citet{heitz:2014_speed} for a fascinating review).
To illustrate, suppose you were required to judge which  of two only very slightly different lines was longer.
 All other things being equal, making a faster judgement would involve being less accurate, and being more accurate would require making a slower judgement.
(This idea is due to \citet{henmon:1911_relation}, who has been influential although he didn't actually get to manipulate speed experimentally because of ‘a change of work’ (p.~195).)
The same applies to judgements everywhere.
Achieving greater accuracy in identifying another's mental states will require slowing down; identifying mental states more quickly will involve being less accurate.
What is the optimal balance between speed an accuracy for mindreading?
It varies.
In maintaining fluent conversation or rushing to help in an emergency, speed is often paramount.
But you might spend the whole night pondering whether Sal really believed Ayesha was away.
% Such vastly different trade-offs between speed and accuracy probably could not be achieved with a single kind of process.
% True, you might have spent minutes instead of hours pondering whether Sal really believes that Ayesha was away.
% But processes of the kind that lead to decisions in these cases can't reasonably occur in fractions of a second. %why not?!
This is one reason why it  would be (or is) valuable to have distinct, independent mindreading processes.
Different processes can enable different, and complementary, trade-offs between speed and accuracy.

For an analogy, imagine you are washing dishes with a dishcloth.
As the stack of plates grows ever higher, you wash faster and faster.
But there are limits to how fast you can reasonably make this process.
At some point speeding up further  requires a different kind of process.
You could exchange the cloth for a long handled brush, or you could switch from processing  dishes sequentially to  washing them in parallel.
Likewise for mindreading.
Having two (or more) distinct, independent processes could enable vastly different trade-offs between speed and accuracy,  trade-offs that  could not  all be achieved with any single kind of process.


Automatic mindreading processes can be fast enough to enable anticipatory looking \citep[for example,][]{schneider:2014_task,low:2010_preschoolers}
and operate at the speed of action \citep[for example,][]{edwards:2017_reaction}, whereas nonautomatic mindreading is often relatively slow \citep[for example,][]{Wel:2013uq},
% \citep[for example,][]{Wel:2013uq} because they found that people slow down by pausing before making a movement when instructed to track beliefs and there was a discrepancy in beliefs.
and may be error-prone in  its early phases.
% (Maymon et al, in preparation). % #TODO update reference
This motivates considering the possibility that %nonautomatic mindreading should in general be more accurate than automatic mindreading.
automatic mindreading enables trading accuracy for speed.
%Nonautomatic mindreading enables you to make the converse trade-off.

But how could  accuracy be traded to gain speed in mindreading?
In the case of physical cognition, one theory
for which there is some evidence%
\footnote{%
See, for example, \citet[][]{mccloskey:1983_intuitive, kozhevnikov:2001_impetus,oberle:2005_galileo}.
}
holds that different kinds of process achieve complementary speed--accuracy trade-offs by virtue of relying on different models of the physical.
One kind of process is fast by relying on a relatively simple model of the physical.
The simple model enables rapid calculations that are accurate enough in a limited but useful range of situations.
Another kind of process trades speed to gain accuracy by relying on a more complex model of the physical (\citealp{kozhevnikov:2001_impetus}; \citealp[p.~640]{hubbard:2013_launching}).
This makes sense.
When erecting a washing line or fixing a fence, you can rely on a primitive model of the physical that needn't require complex calculations.
But to land a robot on a comet you'll need a more sophisticated model of the physical, and this will mean that even simple predictions can require relatively complex calculations.
Could different mindreading processes likewise make complementary trade-offs between speed and accuracy by relying on different models of the mental?


\section{What Is a Model of Minds and Actions?}
\label{sec:minim-models-ment}
A \gls{model of minds and actions} is a way mental aspects of the world could be (see \cref{sec:models}).
As in the case of the physical, we can use theories of the mental to distinguish models of the mental.

On a widely accepted view, mental states involve  subjects having  attitudes toward contents (see \cref{fig:mental_state}).
Possible attitudes include believing, wanting, intending and knowing.
The content is what distinguishes one belief from all others, or one desire from all others.
The content is also what determines whether a belief is true or false, and whether a desire is satisfied or unsatisfied.
There are two main tasks in constructing a theory of the mental.
The first task is to characterise some attitudes.
This typically involves specifying their distinctive functional and normative roles.%
\footnote{%
For examples, see \citet{Bratman:1987xw} on intention or \citet[][chapter 11]{Velleman:2000fq} on belief.
}
%
The second task is to find a scheme for specifying the contents of mental states.
This typically involves one or another kind of proposition, although some  have suggested other abstract entities including map-like representations.%
\footnote{%
See \citet[p.~163]{Braddon-Mitchell:1996ce}: `what is inside our heads should be thought of as more like maps than sentences.’
}

\begin{figure}
\begin{center}
\includegraphics[width=0.9\textwidth]{fig/theory_of_the_mental.png}
\caption{
\label{fig:mental_state}
	Mental states involve  subjects having  attitudes toward contents.
}
\end{center}
\end{figure}

The \emph{\gls{canonical theory of the mental}} features attitudes like belief, desire, knowledge and intention and relies on a system of propositions to distinguish their contents.
The relation between a protagonist’s mental states and her actions is primarily a matter of what the contents of those mental states provide reasons for her to do.
This theory has been developed by many philosophers over decades, relying on a mixture of commonsense, logic and guesswork.%
\footnote{%
See, for example, \citet{Davidson:1963eq,Davidson:1967oq,Bratman:1987xw}.
Philosophers disagree on many features of the canonical theory of the mental, such as whether it treats mental states as intrinsically normative (see, for example, \citealp{boghossian:2003_normativity,Dretske:2000ky}).
}
The canonical theory of the mental is our best attempt to characterise the model of minds and actions that  characterizes adult humans' most reflective thought and talk about the mind.
Although it is largely untested,
I shall rely on the conjecture that the canonical model characterises adult humans' most reflective thought and talk about minds and actions.%
\footnote{%
Researchers have recently begun to carefully test predictions of this conjecture.
Some of the results are surprising, to philosophers at least---influential counterexamples to the claim that knowledge is justified true belief are not regarded as counterexamples at all \citep[for example,][]{starmans:2012_folk,nagel:2013_authentic,starmans:2013_taking}. 
Perhaps philosophers’ notions of knowledge differ from those implicit in everyday mindreading.
}

Could a mindreading process characterised by the canonical model be as fast as automatic mindreading?
Could such a process be fast enough to support anticipatory looking, for instance?
There are at least two obstacles.
First, on the canonical model, mindreading involves using a system of propositions.
Propositions are abstract objects in the sense that numbers are.
Within limits, you can use sentences to identify propositions in something like the way you can use numerals to identify numbers.
Propositions can be arbitrarily connected and nested one within another, and they can be used to mark arbitrarily fine distinctions between possible states of the world.
So one obstacle to fast mindreading using the canonical model is that, as standardly understood, mindreading requires the use of a system of abstract objects with complex structures.%
\footnote{%
\citeauthor{carruthers:2013_mindreading} (\citeyear[p.~160]{carruthers:2013_mindreading}  and \citeyear[ms.~p.~16]{carruthers:2015_two})
objects that this is no obstacle.
His objection rests on the assumption that a thinker who can have a belief whose content we would specify using a particular proposition has thereby represented that proposition and can perform operations using it.
But this assumption is not obviously correct (see \citealp{matthews:2007_measure} for an overview).
}

A second obstacle to fast mindreading arises from how mental states are linked to actions on the canonical model.
How are predictions about action generated from facts about the agent’s beliefs?
On the canonical model, beliefs do not predict actions in isolation.
Instead they do so as a whole: which actions can be predicted from a given belief can depend, in arbitrarily complex ways, on anything else that the agent believes (and also on anything the agent desires).
In fact, generating predictions about what someone will do from ascriptions of belief involves much the same kind of reasoning that is required to work out what should be done in a counterfactual situation.
This is a second obstacle to speedy mindreading: reasoning about what should be done in a counterfactual situation is among the most demanding things humans do.

These are substantial obstacles.
If anything should demand time and other scarce cognitive resources, it is surely using a system of complex abstract objects to track states, and deriving predictions by reasoning about what should be done.
Given the complexities of the canonical model, it is unsurprising that merely holding simple mental states in mind  and using them to make elementary inferences is typically quite time consuming even for adults  \citep{apperly:2011_developmental,Apperly:2008jv}.
Understanding how a mindreading process characterised by the canonical model  could be as fast as automatic mindreading
is a significant challenge.

None of the arguments offered here show that this challenge cannot be met or that the two obstacles are insurmountable, of course.
But the absence of any proposal for meeting the challenge motivates considering ways to avoid it.
The obvious way to avoid it would be to investigate whether automatic mindreading processes might involve a model of minds and actions other than the canonical model.
After all, there is no obvious reason to assume that automatic and nonautomatic mindreading processes must both be characterised by a \gls{canonical model of minds and actions}.
Maybe the trade-off between speed and accuracy made by automatic mindreading processes depends on their being characterised by a different model of minds and actions.
But what model could this be?

%
% Compare mindreading with using numbers to predict where a flying ball will land.
% In both cases a system of abstract objects is used (propositions or numbers) and both cases require complex reasoning (practical deliberation or solving differential equations).
% It is nevertheless conceivable that, with practice, someone might gain such expertise in performing these tasks that, in a limited range of cases, she was able to do so almost effortlessly.
% So expertise is plausibly one way in which the obstacles to achieving cognitive efficiency in mindreading as standardly understood might be overcome.%
% \footnote{
% In general, transitions from effortful and controlled to automatic performances typically involve changes in the processes underlying performances.
% It would be a mistake to take for granted that mindreading in an expert mindreader will be the same as belief tracking processes in a novice.
% Allowing that expertise can make cognitively efficient mindreading possible in some situations does not require one to accept, further, that mindreading as canonically understood can ever be cognitively efficient.
% }






\section{Minimal Models of the Mental}
\label{sec:minimal-models-of-the-mental}
In asking how different groups—infants, children, adults, nonhuman animals—model minds, researchers have almost universally relied exclusively on the canonical theory of the mental.
They mostly recognise, of course, that humans typically rely on gradually more sophisticated models of the mental as they grow older.
But this is taken to be merely a matter of adding attitudes to the model, so that, say, younger children rely on a model of minds and actions which differs from the adult model in not including attitudes like guessing, supposing and schadenfreude.
What few have yet considered is the possibility that different mindreaders, and perhaps also different kinds of mindreading processes within a single mindreader, might use models of the mental which cannot be specified by anything like the canonical theory.
And it is by exploring just this possibility that we will be able to understand how automatic mindreading processes could trade accuracy to gain speed.

The search for simple models of the physical is made easy by the history of science, which contains many theories that are simple, accurate within limits and wildly wrong.
Take impetus theories, for example.
These theories say that moving objects have something, impetus, that they gradually lose.
When they lose their impetus they stop moving.
If you push them, you impart impetus to them, and that is why they move.
Compared to a Newtonian theory, an impetus theory is  less accurate.
But whereas applying a Newtonian theory involves computing factors like friction and air resistance,
an impetus theory rolls these all together into a single thing, namely impetus.
The simplicity of the impetus theory means there is no obstacle to processes characterised by the model of the physical it specifies being fast \citep{kozhevnikov:2001_impetus}.
%An impetus theory therefore usefully characterises a simple model of the physical \citep{kozhevnikov:2001_impetus}.
We want a counterpart of impetus theories for the mental.
We want, that is, a theory which, compared to the canonical theory, sacrifices some accuracy to gain simplicity.

Unfortunately turning to the history of science yields  nothing useful on the mental.
Fortunately philosophers have provided some simple but wildly wrong theories of the mental.

Inspired by these theories, and in particular by my personal favourite, \citet{Bennett:1976rg},
we can construct a  \gls{minimal theory of mind}.%
\footnote{%
The construction of minimal theory of mind described here in barest outline is due to \citet{butterfill_minimal}.
It is incomplete in many ways and perhaps inadequate \citep{christensen:_twoa},
although  it can be improved by integrating it with theories of goal ascription considered in \cref{cha:action} (as \citealp{butterfill:2016_goal} explain).
}
Start by defining an agent's \emph{field} as a set of objects.
Which objects are in the field changes with time depending on the agent's location and orientation as well on factors such as lighting conditions, the objects' movements and acoustic effects, the presence of barriers.
By carefully specifying such factors we will be able to contrive a definition such that, to a limited but significant extent, the objects in an agent's field are those the agent can perceive.
Equally, by ensuring that the notion of a field is defined in narrowly physical terms, we manifestly avoid  complexities associated with perception such as phenomenology, the possibility of illusion, perceptual modalities, reason-giving and links to other mental states.

Given the notions of a field and of goal-directed action  (see \cref{cha:action} on the latter), we can introduce some mental states, \emph{encountering} and \emph{registration}, by specifying their structures and functional roles.
Structurally, encountering is a relation between an agent and an object and registration is a relation between an agent, an object and a location.
To simplify terminology, let us stipulate that for a registration to be \emph{correct} is for the specified object  to be at the specified location.
Functionally:
%
\begin{enumerate}
  \item \label{itm:field-encounter} All objects in an agent's field are encountered by that agent.
  \item \label{itm:encounter-goal} If an outcome involves a particular object and the agent has not encountered that object, then the outcome cannot be a goal of her actions.
  \item \label{item:encounter-registration} If an agent last encountered an object at a location, she registers it as at that location. (And conversely: if  an agent registers an object at a location, she last encountered it at that location.)
  \item \label{item:registration-success} If an outcome involves a particular object, the agent cannot successfully perform an action directed to that outcome unless she correctly registers that object.
  \item \label{itm:registration-action} If an agent performs an action directed to an outcome involving an object, the agent will act as if the object were in the location she registers it in.
  \end{enumerate}
%
In situations where no assignment of encounterings and registrations can make all of the above principles true, later principles outweigh earlier principles.

Although minimal, using this theory of mind would enable you to pass many false belief tasks.
To illustrate, recall \citet{Wimmer:1983dz} false belief task from \cref{sec:all-about-maxi}:
%
\begin{quote}
‘Maxi puts his chocolate in the BLUE box \ldots\
\end{quote}
%
This tells us that Maxi registers his chocolate in the blue box (via principles \#\ref{itm:field-encounter} and \#\ref{item:encounter-registration}).
%
\begin{quote}
 `\ldots\  and leaves the room to play. While he is away (and cannot see), his mother moves the chocolate from the BLUE box to the GREEN box.
\end{quote}
%
This tells us that Maxi's registration is incorrect.
%
\begin{quote}
`Later Maxi returns. He wants his chocolate.’
\end{quote}
We can now predict that if Maxi attempts to recover his chocolate, he will act as if it were in the blue box (principle \#\ref{itm:registration-action}).
So representing others' registrations enables us to track their beliefs.

As it stands, the minimal theory of mind we have just constructed would not enable you to succeed on tasks involving false beliefs about things like colour, shape, function or taste, of course.
It is also completely silent on motivational factors, so couldn't yet be used to succeed on false belief tasks involving desire or preferences.
But it is possible to extend the fragment above to overcome these limits while retaining two key features that distinguish a minimal theory of mind from a canonical one.
These features are, first, functional roles that can be readily codified; and, second, mental states with  simple structures whose contents can be distinguished by things which, like locations, shapes and colours, can be held in mind using some kind of quality space or feature map (using propositions or other complex abstract objects for distinguishing the contents of mental states is not allowed).

Minimal and canonical theories of mind are also markedly different in the range of uses they can be put to.
A canonical theory of the mental supports
explanation, regulation (of self and others) and story telling.%
\footnote{
See, for example, \citet{Heal:2002ew,McGeer:1996ue}.
}
In many ways it is more like a myth-making framework than a scientific theory.
A \gls{minimal theory of mind}, by contrast, is not designed to be used in any of these ways.
It merely  enables predictions.

The construction of minimal theory of mind is an attempt to provide a mental counterpart of an impetus theory of the physical.
This provides us with a possible explanation of how a mindreading process could trade accuracy to gain speed: it could rely on a minimal, rather than a canonical, model of minds and actions.
It will also eventually enable us to remove theoretical obstacles which stand in the way of solving the mindreading puzzle.
But before we get to that, we face a more pressing question.
How can we discover whether a particular mindreading process uses a canonical or a minimal model of minds and actions?
Without an answer to this question, contrasting minimal with canonical models would merely add a theoretical complication.

\section{Signature Limits in Mindreading}
\label{sec:signature-limits-mindreading}
A \emph{\gls{signature limit}} of a model is a set of predictions derivable from the model which
are incorrect, and which are not predictions of other models under consideration.
We can use signature limits to distinguish competing hypotheses about which model characterises a process.
This approach is well established in the case of physical cognition (as we saw earlier, in \cref{sec:signature-limits}), and extends straightforwardly to mindreading.
Consider two conflicting hypothesis:
%
\begin{enumerate}
  \item Automatic mindreading processes are characterised by a \gls{canonical model of minds and actions}.
  \item Automatic mindreading processes are characterised by a \gls{minimal model of minds and actions}.
\end{enumerate}
In order to distinguish these hypothesis using the method of signature limits, we first need to identify a prediction generated by minimal models which is incorrect and not a prediction of a canonical theory of mind.

One signature limit on minimal models of the mental concerns false beliefs about numerical identity.
These are the kind of false belief Lois Lane has when she falsely believes that Superman and Clark Kent are different people.
For the world to be as Lois Lane  believes it to be, there would have to be  two objects rather than one; her beliefs expand the world.
Consider  Lois Lane  at a time when  Clark Kent has disappeared and she is observing Superman performing a daring rescue.
Does she know where Clark is?
Clearly not.
But it is impossible to track her ignorance using a  minimal model of minds and actions.
This is because where  mindreading is  characterised by a minimal model, tracking others' mental states is done using objects themselves rather than senses or concepts or any other kind of proxy.
So a minimal mindreader trying to track Lois Lane's beliefs about Clark Kent does so by representing states, such as encountering and registration,%
\footnote{%
Encountering and registration were defined in \cref{sec:minimal-models-of-the-mental}.
}
which are relations between Lois Lane and Clark Kent.
But since Clark Kent is Superman, to encounter Clark Kent is the same thing  as to encounter  Superman.
That is, encountering (or registering) something is like being left of it.
If you are left of Clark Kent then you are also left of Superman (since they are one and the same); likewise for encountering and registration.
So when Lois Lane  is observing Superman, a minimal mindreader represents Lois as encountering Superman, which is the same thing as representing Clark Kent.
A minimal model of minds and actions makes systematically incorrect predictions about anyone who, like Lois, has a false belief about numerical identity.
This is one signature limit of minimal models.

The hypothesis that automatic mindreading processes are characterised by a minimal model of minds and actions generates the distinctive prediction that automatic mindreading processes are subject to the signature limits of those models, including the one concerning false beliefs about numerical identity.

Is this prediction correct?
Low and colleagues set out to test it \citep{Low:2012_identity,wang:2015_limits}.
 They created and filmed simplified versions of the {Superman} story.
Their films  starred a robot which, like Superman/Clark Kent, looked unexpectedly different on different occasions.
The films also featured a protagonist, Lois Lane's counterpart, who manifestly believed, incorrectly, that this robot was not one thing but two.
And much as in the original film the audience (but not Lois Lane) gets to see Clark Kent's transformation into Superman, so also Low et al's films allowed the audience  (but not the protagonist) to witness the robot's transformation.
In order to use these films to measure belief-tracking abilities, there was a key moment in the films when it was clear that the protagonist would reach into a box to retrieve a particular robot.
At this key moment there were two boxes into which the protagonist might reach.
One box actually contained the robot he sought (call this the \emph{actual location}), whereas the other box was where, given the events of the scenario, the protagonist believed the robot to be (call this the \emph{false location}).
Where would viewers expect the protagonist to reach?
If their predictions were based on a \gls{canonical model of minds and actions}, they should predict that the protagonist will reach for the false location.
But if their predictions were based on a minimal model of minds and actions they should predict that the protagonist will reach for the actual location.

Importantly, Low and colleagues measured subjects'  predictions in two ways, using both anticipatory looking and explicit verbal predictions.
Anticipatory looking is  likely to be a consequence of automatic mindreading \citep{schneider:2014_task}, whereas explicit verbal prediction are likely to be a consequence of nonautomatic mindreading \citep{back:2010_apperly}.
As expected, adult viewers' verbal predictions were nearly all correct.
(That is, they predicted that the protagonist would reach to the false location.)
By contrast, adults' anticipatory looking implied the opposite prediction: at the key moment, most fixated on the actual location.
 This is a reason to prefer the hypothesis that automatic mindreading processes are characterised by a minimal model of minds and actions.

We should be cautious, of course, in putting too much weight on evidence from a single paradigm.
(Indeed, \citet{kulke:2018_implicita} have since identified likely confounds which cast doubt on the interpretation of \citet{Low:2012_identity}’s findings.)
That said, \citet{low:2014_quack} found the same pattern of results using a different scenario, and
\citet{edwards:2017_reaction} found evidence for the same signature limit using not only a different scenario but also a different kind of response (reaction times).
The variety of evidence significantly strengthens the case.
In weighing the evidence, it is also important to note the predictions were formulated before the experiments to test them were done,%
\footnote{%
It is important to distinguish between testing a hypothesis (first make the prediction, then do the experiment) and fitting a hypothesis to existing results (first study the findings, then formulate a hypothesis).
There will always be alternative post hoc explanations for evidence that confirms a prediction.
}
 and that the experimenters were independent of the theorists who formulated the predictions.
 Even so,  all the evidence concerning signature limits on automatic mindreading in adults currently comes from a single lab (Jason Low's) and it is also all relatively recent.
As a general rule, it is prudent to base conclusions on evidence from multiple paradigms and from several labs that has been around for a while.
Evidence from other labs, perhaps testing other signature limits of minimal models of the mental, will be informative.
Even so, I propose we provisionally accept, on the basis of Low et al's various experiments, that automatic mindreading processes are characterised by minimal models of the mental.

This is the key to constructing a theory about mindreading that will enable us to solve the mindreading puzzle and bring us closer to understanding  how knowledge of minds emerges in development.


\section{A Developmental Theory of Mindreading}
\label{sec:developmental-theory-mindreading}

In constructing a developmental theory of mindreading, the \gls{dual process theory of mindreading} is good starting point because it is relatively modest and supported by evidence (as we saw in \cref{sec:automatic-mindreading-sometimes,sec:automatic-mindreading-not-always,sec:dual-process}).
According to the dual process theory, \simpleDualProcessTheory

Until now there was a gap in the evidence for the dual process theory: we had not identified evidence for independence.
But we have just seen findings indicating that automatic and nonautomatic mindreading processes are not merely distinct (in the above sense) but also {independent}. \glsadd{independent processes}%
That is, the conditions which influence whether an automatic mindreading process will yield an incorrect response are  to some extent distinct from the conditions that influence whether a nonautomatic mindreading process will yield an incorrect response.
How do we know?
Recall the findings about the signature limits of automatic but nonautomatic mindreading (from \cref{sec:signature-limits-mindreading}).
 These indicate that, for a single subject responding to a single scenario, automatic mindreading processes can generate an incorrect prediction (or do not generate a prediction) even while nonautomatic processes generate a correct prediction.
 And, conversely, studies with infants in the first three or four years of life indicate the converse: nonautomatic mindreading processes can generate an incorrect prediction even while automatic processes generate a correct prediction (see \cref{sec:infants-track-false}).
There seem to be at least two distinct mindreading processes, and these are independent.


But why should there be two (or more) distinct, independent mindreading processes?
(This question came up at the end of \cref{sec:dual-process}.)
Perhaps it is because they enable complementary trade-offs between speed and accuracy (see \cref{sec:speed-accuracy}).
Compared to nonautomatic mindreading processes, automatic processes might sacrifice accuracy in order to gain speed.
But how could such trade-offs be achieved?
In principle, complementary speed--accuracy trade-offs might be achieved by having different kinds of mindreading process rely on different models of the mental (see \cref{sec:minim-models-ment}).
Given that nonautomatic mindreading processes rely on a \gls{canonical model of minds and actions}, automatic processes could trade accuracy for speed by relying on a minimal model of minds and actions (see \cref{sec:minimal-models-of-the-mental}).
And the hypothesis that automatic processes rely on minimal models of the mental combined with facts about the signature limits of minimal models generates testable predictions.
So far, attempts to test these predictions on adults have supported the hypothesis (see \cref{sec:signature-limits-mindreading}).

In short, speed--accuracy trade-offs motivate considering a hypothesis about minimal models of the mental, and this in turn generates testable predictions thanks to signature limits.
To the extent that those predictions are confirmed, we can accept the hypothesis about minimal models.%
\footnote{%
Note that speed--accuracy trade-offs carry little argumentative weight.
They play no essential role in characterising mindreading processes, nor in justifying the acceptance of hypothesis.
They serve merely to motivate testing hypotheses.
}

This hypothesis allows us to extend the dual process theory.
There are two (or more) distinct, independent mindreading processes, one or more automatic than the other.
And the two processes differ in the kinds of model of minds and actions they rely on:
whereas nonautomatic mindreading relies on a \gls{canonical model of minds and actions}, automatic mindreading relies on a minimal model.
\glsadd{minimal model of minds and actions}%

Note that even when extended in this way, the theory we are considering remains modest in its ambitions.
In terms of \gls{Marr's three levels}, the canonical and minimal theories of mind provide computational descriptions of automatic and nonautomatic mindreading processes.
But the theory offered here is silent on algorithms and representations, and on the hardware implementation.
A deeper theory with commitments at these levels could generate many additional predictions.
But we would need new experimental findings to support (or refute) any such theory.
And the theory as it stands is a good starting point for thinking about development.

How can we connect the dual process theory of mindreading with development?
Consider a conjecture about development:
%
\newcommand{\conjectureDevelopmentMindreading}{%
In the first three or four years of life, nonautomatic mindreading processes do not typically enable belief tracking.
What changes over development is typically just that nonautomatic mindreading comes to enable belief tracking.
}
\begin{quote}
  \conjectureDevelopmentMindreading
  \end{quote}
%
This conjecture is consistent with a limited variety of explanations for infants’ failure on \glspl{A-task}.
Failure may arise because an A-task measures a response that is not driven by an automatic mindreading process.
Alternatively, the response may be a consequence of some combination of automatic mindreading and nonautomatic  processes, but the nonautomatic responses dominate.
The conjecture is also consistent with a limited range of different possibilities concerning what changes over development.
It may be that nonautomatic mindreading processes are initially error-prone in the sense that they yield responses that do not take into account particular mental states, and that they become less error-prone over development.
 It may also be that the probability that a nonautomatic mindreading process occurs increases over development.
 The conjecture is neutral between these possibilities (or combinations of them).

 The conjecture about development links discoveries about mindreading in adults with the mindreading puzzle about children's development.
And it leads to many testable predictions about infants' mindreading abilities.

One set of predictions arises from the implication that \glspl{non-A-task} are tasks on which automatic mindreading processes dominate.%
\footnote{%
 For a particular response, let $A$ be the probability that automatic mindreading influences this response and $C$ the probability that nonautomatic mindreading influences it.
To say that automatic mindreading processes \emph{dominate} on a particular task is to say that the task involves measuring a response for which the ratio of $A(1-C)$ to $C$ is large enough to enable us to measure the effects of automatic mindreading.
}
This is a bold and refutable claim because we defined non-A-tasks as those  false belief tasks that children tend to pass until around three to  five years of age (see \cref{sec:task-analysis}).
The claim thus provides a link between infants' and adults' performance on false belief tasks.
Where a task involves conditions that tend to promote automatic mindreading or to suppress nonautomatic mindreading, it should be a non-A-task.
And, conversely, any non-A-task should involve conditions that tend to lead to performance in adults being dominated by automatic mindreading processes.

A further set of predictions arises from signature limits on minimal models of the mental.
If the conjecture is true,
infants' mindreading involves automatic processes only.
Since these involve minimal models of the mental only, infants' mindreading should be subject to signature limits.
In particular, as with automatic mindreading processes in adults, infants' mindreading should not enable them to track false beliefs about numerical identity.

This prediction has been tested using the same scenarios and measures that, as we saw in \cref{sec:signature-limits-mindreading}, have also been used with human adults.
And indeed infants' performance does appear to resemble automatic mindreading in adults insofar as it is apparently subject to signature limits \citep{Low:2012_identity,low:2014_quack,wang:2012_chinese}.
This prediction about signature limits has also been tested in further experiments using a variety of scenarios and measures in which only children (no adults) participated.
The results so far are mixed.
Two studies have yielded what appears to be evidence that infants can track false beliefs involving numerical identity (\citealp{scott:2015_infants,kamps:2016_conf}; but see \citealp{low:2016_cognitive} for objections to Scott et al).%
\footnote{%
Some researchers have used the term ‘identity’ not for numerical identity but in referring to the category of an object (for example, whether it is a fish or a sponge: \citealp{buttelmann:_14montholds,buttelmann:2015_what}).
As the signature limit under discussion concerns numerical identity, not the category or function of an object, these studies are not directly relevant here (although the findings are fascinating in their own right).
}
And two studies have yielded contrary evidence in support of the signature limit \citep{Fizke:2014_fbidentity,oktay-gur:2018_children}.

These apparently conflicting findings, together with limited evidence concerning other predictions, indicate that we cannot yet confidently accept or reject the conjecture about development.
It may be entirely incorrect, partially correct (perhaps infants’ abilities to track mental states are underpinned by a number of different  kinds of processes), or it may even turn out to be correct.
Since the balance of evidence is in its favour, I propose we tentatively accept the conjecture about development and rely on it in attempting to solve the mindreading puzzle.
Even if this means we will not know that the solution developed here is correct, the solution will still be an improvement on the alternatives as it is readily testable, already supported by some evidence and not yet known to be incorrect.

Considered alone, there is a theoretical obstacle to accepting the conjecture about development.
Why should one mindreading process change during a period of development in which the other is (mostly, at least) unchanging?
Combining the conjecture about development with the hypothesis that automatic mindreading relies on a minimal model of minds and actions whereas nonautomatic mindreading relies on a \gls{canonical model of minds and actions} enables us to answer this question.
A minimal model of minds and actions---even one rich enough to enable tracking others’ beliefs---can be acquired with relatively little social input.
By contrast, the vastly greater sophistication of a canonical model of minds and actions featuring beliefs suggests that acquiring facility with one as a child could require social support for much the reasons that acquiring other sophisticated capacities, such as reading does \citep[compare][]{heyes:2014_cultural}.
Given that automatic and nonautomatic mindreading processes rely on different models of the mental, minimal and canonical respectively, we can make sense of the two kinds of mindreading process having quite different developmental trajectories.

We now have a developmental theory of mindreading.
There are two (or more) distinct, independent mindreading processes, one more automatic than the other.
The relatively automatic mindreading process involves a \gls{minimal model of minds and actions}, whereas the nonautomatic process involves a \gls{canonical model of minds and actions}.
\conjectureDevelopmentMindreading
This theory allows us, finally, to solve the mindreading puzzle.



\section{How to Solve the Mindreading Puzzle}
\label{sec:How to Solve the Mindreading Puzzle}
Recall that \gls{the Mindreading Puzzle} rests on three claims which are collectively inconsistent.
It seems that for many children, there is an age at which:
%
\begin{enumerate}
\item in performing false belief tasks which are A-tasks,  the child relies on a model of minds and actions not incorporating beliefs;
\item in performing false belief tasks which are not A-tasks, such as tasks involving anticipatory looking or violation-of-expectation, the child relies on a model of minds and actions incorporating beliefs;
and
% #∞done: should it be ‘violation-of-expectation’ or ‘violation-of-expectations’? Answer: \gls{violation-of-expectation}
\item the child has a single model of minds and actions.
\end{enumerate}
%
The puzzle is to work out which claim (or claims) to reject.

Accepting the developmental theory of mindreading just introduced (see \cref{sec:developmental-theory-mindreading}) would entail rejecting both claims \#2 and \#3.

Claim \#2 turns out to false because, according to the theory at least, the automatic mindreading in the child relies on a \gls{minimal model of minds and actions}.
This does enables the child to track beliefs in non-A-tasks.
But the model features registration, a belief-like state, rather than belief.

Claim \#3 turns out to be false because both automatic and nonautomatic mindreading processes occur in the child, and these involve different models of minds and actions, minimal and canonical.
The child passes non-A-tasks because performance on these is dominated by automatic mindreading processes and because automatic mindreading involves a minimal model of minds and actions which enables the child to track others’ beliefs.
The child fails A-tasks because performance on these is dominated by nonautomatic processes, because nonautomatic mindreading involves a canonical model of minds and actions, and because the child’s
current canonical model does not feature belief and does not enable her to track others’ beliefs.
Instead, it is a model on which what determines how people act is not what they believe to be the case but rather what is the case (see \cref{sec:truly-contr-resp}).

As these claims require,
the development of nonautomatic mindreading appears to involve acquiring facility with a sequence of canonical models of the mental.
These models gradually incorporate a wider range of mental states and a more sophisticated understanding of them.
In this way, the canonical model underpinning the child's nonautomatic mindreading
 gradually comes to approximate more closely a full canonical model (\citealp[compare][]{wellman:2011_sequentiala}; see also \citealp{roessler:2013_teleology} for a contrasting but related view).
From an adult point of view, over development the child’s mindreading becomes gradually less error-prone.
It is also possible that in parallel with acquiring a gradually more sophisticated canonical model of minds and actions, children become more likely to engage in nonautomatic mindreading.
The acquisition of an increasingly sophisticated canonical model of minds and actions over development is plausibly facilitated by developments in linguistic abilities, executive function and social interactions (\citealp{low:2010_preschoolers,sanjuan:2012_bridging,devine:2014_relations}; see \cref{sec:all-about-maxi}).

We have now identified a candidate solution to the mindreading puzzle.
But what have we learnt about the emergence in development of knowledge of minds?
Before facing this question, let us first tidy up a loose end.


\section{Task Analysis Revisited}
\label{sec:Task Analysis Revisited}
The mindreading puzzle hinged on an unsatisfactory distinction between two kinds of false belief task.
An \gls{A-task} is one that typically developing children tend to fail until around three to five years of age,
whereas a \gls{non-A-task} is one that typically developing children tend to pass in their first or second year of life.
Earlier (in section \cref{sec:task-analysis}) we asked, What determines whether a given false belief task in an A-task, a non-A-task, or neither?
An adequate answer to this question should enable us to predict, for a completely new false belief task, which category it falls in.
But, as we saw,
various existing attempts to distinguish non-A-tasks from A-tasks have all failed.
Why is task analysis so hard?

The developmental theory of mindreading we have been considering provides an answer.
On any false belief task, performance is likely to reflect some combination of automatic and nonautomatic mindreading processes.
And there are multiple ways to turn a non-A-task into an A-task.
You can change features of the task so as to increase the probability that nonautomatic mindreading will influence performance (call this probability $C$); for example, you might tell subjects to pay attention to beliefs in advance.
Or you can change features of the task so as to decrease the probability that automatic mindreading will influence performance (call this $A$), perhaps by changing the timing or task demands.%
\footnote{%
For example, \citet[p.~46]{schneider:2014_task}'s findings indicate that anticipatory looking during one interval  reflects automatic mindreading, whereas anticipatory looking during another, later one interval reflects nonautomatic mindreading.
}
% ∞TODO : why is next sentence true? seems wrong unless fast process dominates:
A further way to turn a non-A-task into an A-task is to change features of the task that influence the probability that automatic mindreading will yield a correct ascription or prediction (call this $E_A$).
You could do some combination of these, of course.
Importantly, the effect any given feature has on $C$,  $A$ and $E_A$ is likely to depend on which other features are present.

Existing attempts at task analysis all focus on one or two factors, such as whether a task involves a response elicited by communicative actions directed to the subject.
According to the developmental theory of mindreading, the problem of task analysis is the problem of identifying which conditions affect the probability of errors in mindreading, or the probability that automatic or nonautomatic mindreading occurs, and of understanding how changes in these conditions interact.
If this is right, the problem of task analysis has been misunderstood.
It is no surprising that prior attempts at task analysis have failed.

There is, of course, a good chance that the developmental theory of mindreading we are considering (see \cref{sec:developmental-theory-mindreading}) is incorrect.
After all, it makes some readily testable predictions and relatively few of these have so far been tested.
It could turn out, for example, that whether a false belief task is an A-task has nothing to do with whether subjects’ performance on it is dominated by automatic mindreading processes.
In that case, we will need a new theory.
But since it has not yet been refuted, we should consider how, other than providing a solution to the mindreading puzzle, the theory might help with understanding the emergence in development of knowledge of minds.

% \section{Formal Statement of the Theory}
% \label{sec:Formal Statement of the Theory}





% \section{*OUTLINE*}
%
% \begin{enumerate}
%
%   \item Conjecture about models: automatic mindreading (but not nonautomatic mindreading) is characterised by a minimal model of minds and actions.  (Note that this conjecture implies the simple dual process theory of mindreading.  It does not imply the dual process conjecture about development.)
%
%   \item Prediction of the conjecture about models: identity vs location in automatic vs nonautomatic processes.
%
%   \item Evidence for the prediction (adults)
%
%   \item The evidence about identity is also evidence for \emph{independence}.
%
%   \item So far we have treated the conjecture about models and the dual process conjecture about  development as independent.
%    If we combine the dual process conjecture about  development with the conjecture about development, we have a theory, the \emph{the dual process minimal theory of mindreading}.
%    To do this we should update our formalism formulation (see the final version in \cref{sec:formal-model}).
%
%    \item With the dual process minimal theory of mindreading we can answer the two outstanding theoretical questions from the end of \vref{sec:dual-process}.
%   \begin{enumerate}
%     \item Why are there two (or more) mindreading processes in humans?  To enable complementary speed--accuracy trade-offs.
%     \item Why should the probability of error in one mindreading process but not the other decrease with age in typically developing children?  Because acquiring a canonical model of minds and actions is a gradual process involving a sequence of increasingly sophisticated models of the mental, and which requires cultural learning \citep{heyes:2014_cultural,roessler:2013_teleology}.  See \vref{sec:orig-knowl-mind}.
%   \end{enumerate}
%
% \item As yet we have no evidence for the dual process conjecture or the dual process minimal theory of mindreading of which it is part.
% But it generates predictions about the signature limits of infant mindreading \ldots\  and there is some evidence for these predictions (and some apparently conflicting evidence too).
%
% \item We should provisionally accept the dual process minimal model theory.  (Only provisionally because many of its predictions remain to be tested.) This allows us to solve the mindreading puzzle (see \vref{sec:solut-mindr-puzzle})
%
% \item Outstanding challenge: developmental relation between automatic and nonautomatic mindreading processes.  The evidence about non-A-tasks and profoundly deaf children born to hearing parents (see the end of \cref{sec:orig-knowl-mind}) is a hint that there may be connections, although \citet{grossewiesmann:2016_implicit} suggests they could be independent.
%
% \end{enumerate}
%
%
%





 \section{Is There Core Knowledge of Minds?}
 \label{sec:core-system-mindreading}

We have seen that there are (at least) two distinct, independent kinds of mindreading processes in humans, one more automatic than the other.
The relatively automatic process enables belief-tracking from early infancy, certainly from early in the second year of life, perhaps  from six-months of age \citep{southgate:2014_belief-based}, and possibly even earlier.
Automatic mindreading in infants also exhibits some features associated with the operations of \glspl{core system}.
As far as we know, it is largely unchanging over the course of development (see \citealp{Low:2012_identity,low:2014_quack,edwards:2017_reaction} for evidence that a signature limit persists over development).
Given that automatic and nonautomatic mindreading are \gls{independent processes}, automatic mindreading it is also likely to be informationally encapsulated to some degree.
These considerations justify provisionally concluding that infants’ earliest mindreading abilities do not rest on knowledge of others’ mental states.
Given the lighter account of \gls{core knowledge} (introduced in \cref{sec:core-knowledge-minimal-view}), we may conclude that infants have core knowledge of minds.%
\footnote{%
\citet[p.~613]{low:2010_preschoolers} appears to endorse this conclusion.
% \citet[p.~613]{low:2010_preschoolers}: ‘representational changes in mental state understanding develop as a function of intersections between language, cognitive control, and early core knowledge.’
}

Concluding that infants have core knowledge of minds raises more questions than it answers.
In the case of physical objects, we were able to identify constituents of core knowledge: object indexes, motor representations and metacognitive feelings (see \cref{cha:causation,cha:metacognitive-feelings}).
For each of these constituents, there are reasons to believe it exists independently of any theory invoking core knowledge.
A future challenge for proponents of core knowledge of minds is to identify its constituents.

% This conclusion is not so much a substantive discovery about infant mindreading as a constraint on theorising about core knowledge and core systems.
% If there is a unified notion of ‘core system’ capable of explaining the developmental emergence of knowledge across several domains, these domains should include not only physical objects, number and geometry but also mind.

%
%
%
% \section{An Anomaly}
% \label{sec:an-anomaly}
% Why do people perform at ceiling?
% They can’t be making predictions.
% Refer back to the third-person/second-person point made at the beginning.
% Adults and older children approach the task as one that involves constructing a narrative.




\section{Origins of Knowledge of Mind: Rediscovery}
\label{sec:orig-knowl-mind}

How does knowledge of minds emerge in development?
On the view considered and partially defended in this chapter, there are two (at least) distinct, independent mindreading processes, one emerging early in infancy and another first enabling humans to track false beliefs from around four years of age.
The former, early-developing kind of mindreading is relatively automatic and appears to involve core knowledge of mind.
By contrast the second, nonautomatic kind of mindreading is necessary for knowledge of others' minds.
So, as in the case of physical objects and colour, to understand how knowledge of minds emerges in development we need to ask what the role of core knowledge is.
What is the role of early-developing, relatively automatic mindreading in explaining the emergence in development of knowledge of minds.

Some studies have begun to address this question by looking for correlations between infants' success on early mindreading tasks and the subsequent emergence of abilities to succeed on \glspl{A-task}.
One study found no correlation between these \citep{grossewiesmann:2016_implicit},
while another study did find a correlation using slightly different measures %\citep{low:2010_preschoolers} used low verbal tasks
 \citep{low:2010_preschoolers}.
Importantly, both studies found that performance on A-tasks, which measure knowledge of mind, was correlated with factors that do not appear to influence infants' success on early mindreading tasks.
These include components of linguistic mastery and inhibitory control.

The emergence of knowledge of minds in development appears to be a consequence not only of early-developing mindreading capacities but also of  developments in linguistic abilities, executive function and social interactions (see \cref{sec:all-about-maxi} and especially \citealp{sanjuan:2012_bridging,devine:2014_relations}).
The importance of language and social interaction in acquiring a full \gls{canonical model of minds and actions} is illustrated by cases in which these are deficient.
Some adults who lack a syntactically sophisticated language may never acquire a full canonical model of minds and actions.%
\footnote{%
According to \citet[p.~810]{pyers:2009_language}, ‘language is indeed a necessary prerequisite, one that cannot be replaced by even 25 years of social experience. Adults who had no congenital cognitive deficits, but whose language was incomplete, failed to fully understand the beliefs of others.’
See also \citet[p.~1166]{meristo:2007_language}: ‘the expression of [mindreading] in native-signing deaf children may also depend on children's continuing exposure to opportunities for monitoring the nature of conversational input about mental states and its implications for evaluating beliefs and other mental states as true or false.’
}
And individuals born deaf into hearing families do not pass A-tasks until years later than hearing individuals \citep{peterson:2000_insights,peterson:2009_development}, suggesting that their more limited opportunities for rich social interactions may make acquiring a full canonical model of minds and actions harder.%
\footnote{%
Intriguingly, there is also evidence that infants born deaf into hearing families do not pass non-A-tasks (\citealp{meristo:2012_belief}; see also \citealp{morgan:2014_mental}).
% \citet{morgan:2014_mental}: ‘Mothers of hearing children used far more cognitive mental state language with their infants and their conversations were characterized by more communicatively effective turn-taking than mothers of deaf children.’
}


Acquiring knowledge of minds appears to be another case in which development is rediscovery.
There is an early-developing capacity to represent mental states, but this does not appear to be connected in any straightforward way to later-developing knowledge of mental states.
Instead the emergence of this knowledge hinges on social (and cognitive) skills.

This picture is complicated by the fact that an early-developing capacity to represent mental states plausibly enhances a child’s social skills, and may also influence her exposure to situations that promote the development of linguistic skills.
Perhaps an early-developing form of mindreading makes possible social interactions which, together with linguistic and cognitive developments, eventually enable us to know facts about others’ minds.



% #TODO: tell Jason or someone we should do anticipatory looking time versions of the \citet{apperly:2011_developmental}.  Questions: (i) Are automatic responses possible in all cases (for example, B-D-)? (ii) If so, do the conditions differ w.r.t. response time and error, and do they do so in the same ways that nonautomatic responses do [response time = time to first fixation, as used in the Low et al helping study]; (iii) Whether or not, do response times or error rates change with age in the way that they do for nonautomatic processes?


% \subsection{*notes}
% What is the significance of the fact that individuals born deaf in hearing families do not pass A-tasks until years later than hearing individuals \citep{peterson:2000_insights,peterson:2009_development}?
% \begin{quote}
% ‘for most severe or profoundly deaf children of hearing parents, the development of the skills needed to pass false belief tests is likely to be an unusually protracted process extending well into, and perhaps beyond, middle childhood …  Fifth and finally, there are tentative hints, especially from longitudinal evidence, that a continuing though very gradual upward progression of ToM development may be more possible and more widespread for severely and profoundly deaf children than some previous cross-sectional evidence would have suggested.’
% \end{quote}
%
% \citep{meristo:2007_language}: ‘the expression of ToM in native-signing deaf children may also depend on children's continuing exposure to opportunities for monitoring the nature of conversational input about mental states and its implications for evaluating beliefs and other mental states as true or false.’
%
%
% \citep{sanjuan:2012_bridging} is full of useful suggestions on how language might be involved in the development of mindreading.
%
% What I’d really like to know: how do deaf children do on non-A-tasks?  Not so well \citep{meristo:2012_belief}!  (see also \citep{morgan:2014_mental}).







%
% \section{A Dual Process Hypothesis about Development}
% \label{sec:formal-model}
%
% Recall our earlier, not entirely successful attempts at task analysis.
% The dual process, dual representation theory we are currently considering suggests that the approach we took to task analysis is misguided.
% Performance on any task will always be affected by both automatic and nonautomatic processes \citep{jacoby:1991_process}.
% What we seek instead is a way of disentangling the contributions of these two.
% Consider two-year-olds.
% According to the present theory, automatic processes in these subjects can track false beliefs about location and other properties.
% But nonautomatic processes cannot.
% Further, we make two simplifying (but perhaps incorrect) assumptions.
% First, a response will be dictated by a nonautomatic  process if one occurs; and, second, whether an automatic process occurs and whether a nonautomatic process occurs are independent.
% We also ignore the possibility that no perspective taking process will occur. % even though our independence assumption implies that the probability that no perspective taking process will occur is $(1-A)(1-C)$.
% Accordingly we can formally model two-year-olds’ performance like this, where $A$ is the probability that automatic perspective taking occurs and $C$ is the probability that nonautomatic perspective taking occurs:
% %
% \begin{quote}
%   P(correct performance) = $A(1-C)$
%
%   P(incorrect performance) = $C$ %$C + (1-A)(1-C)$
% \end{quote}
% %
% The values of $A$ and $C$ will vary between tasks, and are likely affected by many aspects of a task such as how quickly a response is required or what kind of response is required (for example, communicative or noncommunicative).
% Properly understood, an A-task is a task where $C /(A(1-C))$ is large enough that subjects appear to be giving systematically correct answers.
%
% This implies that there are multiple ways to turn a non-A-task into an A-task.
% You can change features of the task so as to increase $C$, or you can change features of the task so as to decrease $A$, or you can do both.
% If this is right, it is not surprising that attempts at task analysis which focus on a single factor (such as whether a task involves an elicited response) have failed.
%
% Thinking about two-year-olds is particularly simple because  the theory we are considering implies that nonautomatic belief-tracking processes invariably provide an incorrect response.
% But the theory also says that, as typically developing children age, the probability that belief-tracking processes provide an incorrect response decreases.
% Consider a further formal model of performance, where $E_C$ is the probability that a nonautomatic process will yield an incorrect response:
% %
% \begin{quote}
%     P(correct performance) = $A(1-C) + (1-E_C)C$
%
%     P(incorrect performance) = $E_C \cdot C$
% \end{quote}
%
% So far we have assumed that automatic belief-tracking process invariably provide correct responses.
% But, as we saw, there is evidence which speaks against this.
% Instead it appears that in tasks involving identity rather than location, or otherwise requiring belief-tracking beyond the limits of minimal theory of mind, automatic perspective taking processes will yield incorrect responses.
% We can extend our formal model accordingly, where $E_A$ is the probability that an automatic process will yield an incorrect response:
% %
% \begin{quote}
%     P(correct performance) = $(1-E_A)A(1-C) + (1-E_C)C$
%
%     P(incorrect performance) = $E_C \cdot C + E_A \cdot A(1-C)$
% \end{quote}
%
% *Illustrate with figure showing the different routes to correct and incorrect answers.
%
% The final formal model has four parameters.
% But we can impose constraints by making several assumptions:
% %
% \begin{enumerate}
% \item $E_C$ decreases with age in childhood.
% \item $A$ and $E_A$ do not change with age. [?!]
% \item Where a task involves making a choice, $C$ increases with time.
% \end{enumerate}
% %
% Although each of these assumptions is questionable, they are theoretically motivated and provide useful constraints on the formal model.
% [*Are these assumptions or predictions motivated by the theory?]
%
%
% The formal model guides us in answering questions.
% What changes over development?
% $E_C$ decreases.
% Why do infants fail A-tasks?
% Because on A-tasks, $C /(A(1-C))$ is large enough that their performance is dominated by nonautomatic processes, and $E_C$ is large at this age.
% Why do infants pass some non-A-tasks?
% Because on non-A-tasks, $C /(A(1-C))$ is small enough that their performance is dominated by automatic processes, and $E_A$ is small on these tasks.
% Why does infants’ performance differ between false belief tasks involving mistakes about location and those involving mistakes about identity?  Because $E_A$ is higher in tasks involving mistakes about identity.
% On false belief tasks involving mistakes about identity, why does adults’ performance differ depending on which responses (for example,~first looks, looking times or verbal responses) are measured?
% Because changing the response measured alters the value of $A$ or $C$ (or both).
%






%
% \section{How Could Mindreading Ever Be Automatic?}
% \label{sec:mindreading-efficiency}
%
% Even adult humans have limited cognitive resources.
% There are limits on how much information they can hold in working memory (imagine hearing an arbitrary large number and then having to repeat it), and on their capacity to inhibit tendencies to think or act (try counting to 10 alternating between two languages---eins, two, drei, four, …).
% An automatic process which placed significant demands on working memory, inhibitory control or other scarce cognitive resources would be a significant impairment.
% After all, automatic processes are, by definition, those whose occurrence is to a significant degree independent of your current tasks and motivations.
% And, generally speaking, for things to go well for you it is necessary that scarce cognitive resources are allocated in ways appropriate to your current tasks and motivations.
% It would not be good if you were built in such a way that the occurrence of an automatic process caused you to forget a name you were trying to hold in mind or interfered with your ability to suppress antisocial behaviours in public.
% These general considerations apply to automatic mindreading in particular.
% When mindreading is automatic, it cannot consume too much in the way of scarce cognitive resources—it must be cognitively efficient.
%
%
%
% Are there other ways in which mindreading could be cognitively efficient?
% If it turned out that adults’ mindreading were only ever cognitively efficient as a consequence of expertise effects,
% the existence of automatic mindreading in adults would be barely relevant to understanding development.
% After all, expertise cannot plausibly explain the patterns in the development of mindreading which we identified in \cref{cha:mind}.
% But there is an alternative was in which mindreading could be cognitively efficient, whether it occurs in infants or adults.
%




% But how could belief tracking ever be cognitively efficient?
% To appreciate how difficult it is to answer this question, we need to understand how belief tracking is standardly understood to work.
% As standardly understood,
% belief tracking in human adults involves mindreading, that is, a process of representing mental states as the mental states of particular individuals.
% In particular, it involves representing beliefs as the beliefs of particular individuals, as well as using beliefs in explaining and predicting actions.

% Step back for a moment and think about what mental states, including belief, are.
% Mental states such as belief have at least three distinguishing features: subject (whether it is your mental state or mine), attitude (whether it is a belief or a desire), and content (whether it concerns, say, there presence of mice in the cellar or bats in the loft).
% Accordingly, to represent mental states such as belief you will need a way of distinguishing between different potential subjects of belief, a way of locking onto different attitudes, and a way of distinguishing different contents beliefs can have.






%%% Local Variables:
%%% TeX-master: "master"
%%% End:
