%!TEX root = master.tex

% ∞TODO

% Split of a bit of thsi chapter into conclusion to Part I?

% Make sure to use the KCL/Bochum draft paper for this chapter and conclusion to Part I

\chapter{Metacognitive Feelings} % was Development Is Rediscovery
\label{cha:metacognitive-feelings}

How do humans first come to know simple facts about particular physical objects?
It seems we are still quite far from an answer to this question, the one from the very start of \cref{cha:principles-object-perception}.
Evidence incompatible with the \gls{Simple View} suggests that infants probably do not already know any simple facts about physical objects at four- to six months of age (see \cref{cha:linking-problem}).
And considerations in favour of \gls{Conjecture O} suggest that motor representations of objects and a system of object indexes may both somehow play a role in explaining the developmental emergence  of knowledge (see \cref{cha:causation}).
But how could they do this?

In the present chapter we will approach this question in characteristically destructive philosophical mode:
the first step is to see why \gls{Conjecture O} does not quite explain what it is supposed to explain.
We will then explore one way to fill the resulting explanatory gap by invoking something called \glspl{metacognitive feeling}.
By the end you should understand what \glspl{metacognitive feeling} are, how they might be linked to the operations of  \glspl{object index}, and why \glspl{metacognitive feeling} might be essential for solving the \gls{Linking Problem}.


% OLD [changed because scope of chapter reduced and some material move to Conclusion of Part I chapter]:  \glspl{metacognitive feeling} might, in combination with other factors like joint action and referential communication, explain how knowledge of simple facts about physical objects emerges in development.



\section{Objection to Conjecture O}
\label{sec:objection-to-conjecture-o}
We have been working on the assumption that either the \gls{CLSTX Conjecture} or, more likely, \gls{Conjecture O} can explain four-month-olds’ performance on \gls{violation-of-expectation} tasks involving occlusion.
According to either conjecture, their performance on these tasks is a consequence the operations of \glspl{object index}.
But what can object indexes explain?
The primary functions of object indexes include influencing the allocation of attention and perhaps guiding ongoing action (see \cref{sec:object-indexes-adults}).
If this is right, it may be possible to explain anticipatory looking directly by appeal to the operations of  object indexes.
But operations involving object indexes cannot directly explain differences in how novel things are to an infant.
And nor can operations involving object indexes directly explain why infants look longer at a physically possible than a physically impossible event.



% \begin{enumerate}
% \item Infants can represent objects as persisting, and track their causal interactions because v-of-e anticipatory looking habituation
% \item infants cannot represent objects as persisting, nor track their causal interactions because of failures to search
% \item solution is to distinguish kinds of representation, of course.  The challenge is how to do so in a way that might be explanatory.

% \item To clarify, distinguish between (a) avoiding contradiction, and (b) explaining the findings, i.e. (i) why infants do not search in certain cases; and (ii) why infants do look longer, do anticipatory looking etc.  (In this section, the distinction between object indexes and belief gives will us the beginnings of an explanation of (i).  Explaining (ii) is a task for the next section and its phenomenal expectations)

% \item I think (2) shows infants representations are not knowledge
% \item And the \gls{CLSTX Conjecture} suggests that infants representations are object indexes (or object files---which are not yet explained!)
% \item Is the \gls{CLSTX Conjecture} consistent with the pattern of findings about what infants can and can’t do?
% \item Level 1: object indexes are distinct from beliefs (return to the feature change experiments; also mention \citet{Mitroff:2004pc})
% \item Level 2: \citep{charles:2009_object} have a nice darkness / occlusion vs reaching / looking contrast.
% Could we explain this using object indexes?
% Yes, if object indexes are lost in the darkening condition (which I guess \citep{moore:2010_features} hints they might be).
% \item level 3: object indexes can guide actions that have already been initiated but do not play a role in planning
% \item *todo: check the experiment in which, apparently, infants can search.  Might also consider \citet{moore:2008_factors}.

% \item conclusion of this section: the \gls{CLSTX Conjecture} is consistent with the pattern of findings concerning infants’ abilities to represent unperceived objects.
% It even stands a chance of explaining the pattern.
% To explain the pattern we need to explain (i) why infants do not search in certain cases; and (ii) why infants do look longer, do anticipatory looking etc.  % The distinction between object indexes and belief gives us the beginnings of an explanation of (i).  But what about (ii)?  To explain (ii) we need a further claim: object indexes can (but do not always) generate phenomenal expectations …
% % %But the \gls{CLSTX Conjecture} does not explain these findings, as it stands.  For an explanation we would need to add the assumptions that object indexes influence looking times and anticipatory looking but do not typically influence searching (although they do sometimes).

% % % \item The \gls{CLSTX Conjecture}’s consistency the findings removes an obstacle to understanding how humans first come to know simple facts about particular physical objects.

% % \item The \gls{CLSTX Conjecture} also suggests the account about emergence of knowledge will need to take a new direction.  Before this comes into view, we need to fill in some more details …

% \item If object indexes don’t lead to knowledge, what do they lead to that makes a difference to the subject? … [leads to section on phenomenal expectations]
% \end{enumerate}

In short, it is unclear how the operations involving object indexes could explain the looking behaviours they are supposed to explain according to both the \gls{CLSTX Conjecture} and \gls{Conjecture O}.

% Facts about \glspl{object index} cannot explain phenomena such as looking times all by themselves.
To illustrate, recall  \citeauthor*{kellman:1983_perception}’s experiments depicted in \vrefrange{fig:kellman_1983_fig3part1}{fig:kellman_1983_fig3part2}.
In part of the experiment, infants are \glslink{habituation}{habituated} to two stick ends, partially concealed by a screen, moving together.
When the habituation phase is over, the screen is removed and infants are either shown two unconnected short sticks or they are shown a single connected stick.
Infants look longer at the former.
In cases such as this, infants (and adults) assign a single object index to the stick ends moving together (see \cref{sec:object-indexs-princ}).
Consequently, when the screen is removed and two unconnected sticks are shown, two object indexes will be assigned where before a single object index was assigned.
This is true (or at least it must be if the \gls{Conjecture O} is right).
But it cannot explain why infants look longer all by itself.
How does a difference in operations involving object indexes result in a difference in looking times?

The difficulty this question poses becomes even greater if we consider the timings involved.
Mean looking times in the first test trials are around 10 seconds in one condition versus around 40 seconds in the other (see \vref{fig:kellman_1983_fig4a}).
Given that, even in adults, object indexes do not survive occlusion for more than around eight seconds \citep[compare][]{noles:2005_persistence}, how could they explain such large differences in extended looking times?

% This question does not only arise when \gls{habituation} is used.
The same question arises for looking times in \gls{violation-of-expectation} experiments, which also involve measuring differences in looking times.
It is a mystery how differences in operations involving object indexes could explain such differences in looking time.

Can we disregard these problems as merely the kind of outstanding detail likely to be associated with any bold conjecture?
No.
The only known way of supporting \gls{Conjecture O} (and the \gls{CLSTX Conjecture}) involves relying on the method of \glspl{signature limit}.
This  cuts both ways.
If it would be right to regard \gls{Conjecture O} as supported by limits on the nature and timing of infants’ responses (as argued in \cref{sec:signature-limits}),
it must also be right to reject this conjecture when infants’ responses do not appear to confirm to signature limits.

% In any such case, an object disappears and an action related to that object is prepared or initiated only after its disappearance and concludes before, or concurrently with, its reappearance.
% This is not a matter of object indexes influencing ongoing actions, but of their somehow influencing spontaneous responses (in the case of violation-of-expectation).
% How are such influences possible?

At this point it may be tempting to suppose that the operations of object indexes give rise to beliefs or to knowledge states, and that these in turn explain the looking behaviours.
We should resist this temptation.
Not only because the supposition is false (see \cref{sec:paradox-lost}).
Worse,
accepting it would mean we had sacrificed simplicity only to end up  generating all of the incorrect predictions associated with the \gls{Simple View} (see \cref{sec:against-simple-view,sec:even-worse-for-the-simple-view,sec:furth-evid-against}).
We need an alternative.

None of this means we should give up on \gls{Conjecture O} yet. 
Perhaps there is something that connects operations involving object indexes with on looking behaviours in \gls{habituation} and \gls{violation-of-expectation} experiments.
And perhaps finding this connection will give us a clearer understanding of the cognitive systems that comprise \gls{core knowledge} of objects.

% My guess is that the influences of \glspl{object index} on looking behaviours in \gls{habituation} and \gls{violation-of-expectation} experiments involves metacognitive feelings.



\section{Metacognitive Feelings: A First Example}
\label{sec:metacognitive-feelings-ex1}

% ∞TODO this is a bit hard to follow probably; maybe order is not optimal

What connects operations involving object indexes to
patterns in looking duration?
It cannot be beliefs or knowledge states (see \cref{sec:objection-to-conjecture-o}).
Instead I will suggest that it is something called a metacognitive feeling.

According to \citet[p.~150]{koriat:2000_feeling}, ‘metacognitive feelings ... allow a transition from the implicit-automatic mode to the explicit-controlled mode of operation.’
We might guess that the operations involving indexes involved an ‘implicit-automatic mode’ (whatever exactly that means) and that patterns of looking involve an ‘explicit-controlled mode’. 
If that guess is right, and if Koriat is right about the role of metacognitive feelings, then they are just what we need to overcome the objection to \gls{Conjecture O}.

But what are metacognitive feelings?
Rather than giving a definition, I think it is helpful to start with particular cases which have been successfully investigated.
The successes will give us confidence that there is something worth defining and constrain what a definition should say.



As a first example of a metacognitive feeling, take the sense of agency.
As illustrated in \vref{fig:sense_of_agency3}, it’s quite well established that there are feelings of agency. 
These feelings seem to arise
from a number of cues including   comparison between outcomes represented motorically and outcomes detected sensorily, and  the fluency of an action selection process (that is, the ease or difficulty involved in 
selecting one among several possible actions to perform motorically).
The latter can be manipulated by,
for example, providing helpful or misleading cues to action
\citep{wenke:2010_subliminal,sidarus:2013_priming,sidarus:2017_how}.


\addFigure{sense_of_agency3}{Cues giving rise to a sense of agency. Source: adapted from \citet[figure 5]{sidarus:2016_difficult}}

The sense of agency is relevant to us because it serves to link two 
largely independent processes concerned with evaluating whether you are the agent of an event.
One involves detecting the  cues just mentioned;
the other involves thinking about how likely it is that you are the agent of an event, 
perhaps in the light 
of your background knowledge.
 
Suppose you are a subject in an experiment and the experimenter asks you whether you felt you were in control of an event.
You do not need to go with your
feelings. 
You could think about how likely it is that you are the agent of an event. The
right answer may well be to say
‘I don’t know, this is a psychological experiment so there’s a good chance you were tricking me.’
But despite all of the possible ways in which reflection on the question might
lead to this or another answer, adults systematically give answers which seem to reflect the fluency of action selection. 
Why?

It seems that action selection fluency modulates the feeling of agency,
and that feeling is associated by adults with being the agent of an event.
So the feeling plus association can bias adults’ answers to questions about agency.
(Of course there may be cases in which adults’ answers do not reflect their 
feelings of agency.)
 
So what is this feeling (or ‘sense’) of agency? 
First, it is phenomenal rather than epistemic.
It is an aspect of the phenomenal character of some experience associated with acting.
So we can call it a \emph{feeling}.
 
Second, the feeling of agency is metacognitive in the sense that it’s normal causes include processes which 
monitor action selection and production.
So we can call it a \emph{metacognitive} feeling.%
\footnote{%
Compare \citet[p.~310]{dokic:2012_seeds}: ‘the causal antecedents of noetic feelings
can be said to be metacognitive insofar as they involve implicit monitoring mechanisms that
are sensitive to non-intentional properties of first-order cognitive processes.’
}

\section{More Metacognitive Feelings}
\label{sec:metacognitive-feelings-familiarity}
The feeling of agency is far from the only metacognitive feeling.
As a second illustration, consider the feeling of familiarity.
If you look at the face depicted in \vref{fig:obama_clinton},  you will hopefully have a feeling of familiarity on seeing this face.
Having a feeling of familiarity need not involve believing that the thing encountered is familiar.
Even if you know for sure that you have never encountered the person depicted there (and trust me, you have not), the feeling of familiarity will persist.
% compare \citealp[pp.~301ff]{koriat:2007_metacognition} on ‘Dissociations between Knowing and the Feeling of Knowing’
Nor is the feeling merely a perceptual experience.
After all, the familiarity of something to you depends on arbitrary past events.
You cannot perceptually experience familiarity any more than you can perceptually experience events from your childhood.
So the feeling of familiarity is something different from both belief and perceptual experience.

\addFigure{obama_clinton}{The feeling of familiarity is not always caused by the fact of familiarity.
source: politicsdaily.com}
% \url{www.politicsdaily.com/media/2008/03/hil_obama_finalv2.jp}

The face  in \cref{fig:obama_clinton} is a composite of Obama and Clinton.
I chose it to illustrate that the feeling of familiarity is not a consequence of how familiar things actually are.
Instead it appears to be a consequence of the degree of fluency with which unconscious processes can identify
perceived items \citep{Whittlesea:1993xk,Whittlesea:1998qj}.
Learning a grammar can also generate feelings of familiarity.
Subjects who have implicitly learned an artificial grammar report feelings of familiarity when they encounter novel stimuli that are part of the learnt grammar \citep{scott:2008_familiarity}.

Adults are not compelled to treat feelings of familiarity as actually being about familiarity.
It is possible, for example, to exploit the feeling of familiarity in deciding whether a string can be generated by a certain set of rules \citep[for example,][]{Wan:2008_familiarity}.
In doing this you are treating the ‘feeling of familiarity’ as having nothing to do with familiarity as such.

The feelings of agency and familiarity is not isolated cases.
There is also the feeling you have when someone’s eyes are
boring into your back, the feeling associated with having a name on the tip of your tongue, the feeing of déjà vu \citep{brown:2003_review}, the feeling of knowing \citep{koriat:1993_how}, and more besides.
These are all candidate \glspl{metacognitive feeling}.

\section{What Is a Metacognitive Feeling?}
\label{sec:metacognitive-feeling-df}
It might be useful to compare metacognitive feelings with the feeling of electricity.
Contrast two sensory encounters with a wire. In the first you visually experience the wire as having a certain shape. In the second you receive a mild electric shock from the wire without seeing or touching it.%
\footnote{%
This illustration is borrowed from \citet[pp. 133–4]{Campbell:2002ge}; I use it to support a claim weaker than his.}
%
The first sensory encounter involves perceptually experiencing a property of the wire whereas the second does not.
(If anything is perceptually experienced in receiving the mild electric shock, it is probably your own body.)
Yet the electric shock involves rich phenomenology, and its particular phenomenal character depends in part
on properties of its cause.
Changes in current result in encounters with different phenomenal characters, so that in principle---if contact with live wires were not so often fatal---you could learn to estimate how much electricity is flowing through the wire.

Neither the feeling of familiarity nor the sensations associated with electric current involve standing in any intentional relation to the properties of familiarity or electricity.
But they are things that adults can, and often do, interpret as being informative about familiarity and electricity.
I suggest that this is true of metacognitive feelings generally:
%
\begin{quote}
\emph{\Glspl{metacognitive feeling}} are, or involve, aspects of the overall phenomenal character of experiences which their subjects take to be informative about things that are only distantly related (if at all) to the things that those experiences intentionally relate the subject to.
And there is no further phenomenal feature of metacognitive feelings.
\end{quote}
%
It follows that metacognitive feelings can lead to beliefs only via associations or further beliefs.
They are signs which are open to interpretation by their subjects.
Just as coming to associate feeling of electricity with electric current depends on learning (and may never happen), so connecting metacognitive feelings to familiarity, confidence or anything else can only be a consequence of learning.

As the scientist, you can pick out the feeling of familiarity as (very approximately) that metacognitive feeling which is normally caused by the degree to which certain processes are fluent.
But as the subject of who has that metacognitive feeling, you do not necessarily know what its typical causes are.
This is something you have to work out in whatever ways you work out the causes of any other type of event.
In this way, metacognitive feelings may contrast with perceptual experiences.
Several philosophers hold that perceptual experiences make corresponding beliefs available.%
\footnote{%
For instance, compare \citet[p.~222]{Johnston:1992zb}:
‘[j]ustified belief … is available simply on the basis of visual perception’;
\citet[p.~143–4]{Tye:1995oa}:
‘Phenomenal character ...\ stands ready … to make a direct impact on beliefs’;
and
\citet[p.~291]{Smith:2001iz}:
‘[p]erceptual experiences are … intrinsically … belief-inducing.’
}
This is untrue of metacognitive feelings.
Instead, metacognitive feelings are not far from being sensations in approximately Reid’s sense (\citeyear{Reid:1785cj,Reid:1785nz}).
Reid’s sensations are monadic properties of events, specifically perceptual events,
individuated by their normal causes% %{Tye, 1984 #1744@204}
% ---in the case of feelings of familiarity, its normal cause is ease of processing
which alter the overall phenomenal character of those experiences
 in ways not determined by the experiences’ contents
(so two perceptual experiences can have the same content while one has a sensational property which the other lacks).%
\footnote{%
Those who, like \citet{byrne:2001_intentionalism}, hold that phenomenal character is determined by intentional properties would of course reject the existence of sensations in Reid’s sense.
They might hold instead that metacognitive feelings are
perceptual experiences of the body or of bodily reactions,
or that they involve some kind of cognitive intentional object.
}



% ∞acknowledgement: thanks to Nick Shea:
Much of the work metacognitive feelings do is surely independent of what you believe.
Irrespective of which if any beliefs they lead to, feelings of agency, familiarity, déjà vu and the rest can arouse interest and modulate effort, causing you to slow down and look longer at events that would otherwise seem unremarkable.
They may even lead you to explore ways to recreate that feeling.
In all of these ways, metacognitive feelings can cause you to focus on events and tasks in contexts that provide opportunities for learning.
And they do this regardless of how, if at all, you interpret them.
Whatever your research on the feeling of familiarity leads you to conclude about its causes, its occurrence in you will  probably still influence how you look at, and attend to, things.
Just as the feeling of electricity does.

Note that the partial characterisation of metacognitive feelings I am offering is unlikely to be accepted by all researchers.
% %
% \footnote{%
% One line of objection would be to question in what sense metacognitive feelings as characterised here are genuinely metacognitive.
% The answer is that their causal antecedents are ‘metacognitive insofar as they involve implicit monitoring mechanisms that are sensitive to non-intentional properties of first-order cognitive processes’ \citep[p.~310]{dokic:2012_seeds}.
% }
Are metacognitive feelings really like the feeling of electricity, as I am suggesting?
% There is currently disagreement among researchers on how to characterise metacognitive feelings.
Koriat, who is surely the leading researcher on this topic, proposes that:
%
\begin{quote}
  ‘metacognitive feelings are mediated by the implicit application of nonanalytic heuristics \ldots\ [which] operate below full consciousness, relying on a variety of cues \ldots\ [and] affect metacognitive judgments by influencing subjective experience itself’ (\citealp[p.~158]{koriat:2000_feeling}; see also \citealp[pp.~313--5]{koriat:2007_metacognition}).
\end{quote}
%
Koriat’s deep and carefully developed theory informs the present discussion of metacognitive feelings.
His theory is consistent with the characterisation of metacognitive feelings I have offered.
Indeed, his suggestion that  the ‘processes that take off from subjective experience generally have no access to the processes that have produced that experience in the first place’ \citep[p.~314]{koriat:2007_metacognition} is in line with my partial characterisation of metacognitive feelings.
But many other approaches to characterising metacognitive feelings are incompatible, at least superficially, with thinking of them as like the feeling of electricity.%
\footnote{%
See \citet{dokic:2012_seeds} for an analysis of three approaches to metacognitive feelings (which he calls ‘noetic feelings’).
My partial characterisation of metacognitive feelings is closest to what he calls the ‘Water Diviner Model’.
% The difference between Dokic’s carefully developed Model and the less detailed, partial characterisation offered in this book concerns the intentional contents of metacognitive feelings.
% (p. 310:‘noetic feelings can be said to be metacognitive insofar as their intentional contents yield information (or misinformation) concerning one’s own epistemic states, processes, and abilities.’)
But I depart from this Model in denying that there is any need to posit intentional contents for metacognitive feelings.
This makes my partial characterisation of metacognitive feelings closer to Dokic’s view of aesthetic experiences---he argues that aesthetic experiences ‘are non-intentional’ and should be characterised by an adverbial theory \citep[p.~85]{dokic:2016_aesthetic}.
I suggest that this is true of metacognitive feelings.
}




Metacognitive feelings have been quite widely neglected in philosophy and developmental psychology.
They are a means by which cognitive processes enable perceivers to acquire dispositions to form beliefs about objects’ properties which are reliably true.
Metacognitive feelings provide a low-cost but efficient bridge between non-conscious cognitive processes and conscious reasoning \citep{koriat:2000_feeling}.


\section{A Metacognitive Feeling of Surprise?}
\label{sec:metacognitive-surprise}

As a step towards using metacognitive feelings to resolve the objection to \gls{Conjecture O} (see \cref{sec:objection-to-conjecture-o}),
 we need to postulate a novel metacognitive feeling, one not included on standard lists of metacognitive feelings.

One kind of surprise has features characteristic of a metacognitive feeling.
According to \citet[p.~271]{reisenzein2000subjective},
‘the intensity of felt surprise is not only influenced by the unexpectedness of the
surprising event, but also by the degree of the event’s interference with ongoing mental
activity.’
On this view, unexpectedness is not, or not only what generates the feeling of surprise.
Rather, it is metacognitive monitoring of mental activity: 
the less fluently an event is processed, the more surprising it feels.
This makes sense given that, within limits, unexpectedness is reliably correlated with the fluency with which an event can be processes, and given that unexpectedness is in general harder to monitor than processing fluency.% 
\footnote{%
An alternative is proposed by \citet[p.~79]{foster:2015_whya}:
‘the MEB theory of surprise posits that: Experienced surprise is a metacognitive assessment
of the cognitive work carried out to explain an outcome. Very surprising events are those
that are difficult to explain, while less surprising events are those which are easier to
explain.’
\citet{foster:2015_whya} is about reactions to reading about something unexpected, whereas
\citet{reisenzein2000subjective} measures how people experience unexpected events (changes to
stimuli while solving a problem). 
\citeauthor{reisenzein2000subjective}’s study is closer to our interest in infants’ performance on \gls{habituation} and \gls{violation-of-expectation} experiments.
The truth of either account of surprise, or of an account combining the two insights, would indicate that there is a metacognitive feeling of surprise.
}

If, as \citeauthor{reisenzein2000subjective} proposes, there is a phenomenal consequence of mental interference, then it is a metacognitive feeling (see \cref{sec:metacognitive-feeling-df}).
Let us call it the metacognitive feeling of surprise.

The metacognitive feeling of surprise contrasts with the feelings of agency and familiarity in one interesting respect.
Agency and familiarity are stronger when action selection (see \cref{sec:metacognitive-feelings-ex1}) or identification (see \cref{sec:metacognitive-feelings-familiarity}) is more fluent.
By contrast, the metacognitive feeling of surprise is stronger when the processing of an event is less fluent.
But  all three do have in common the key characteristic of a metacognitive feeling: they are all feelings that arise from mental processes monitoring the fluency of mental processes.

The existence of a metacognitive feeling of surprise can explain a mundane observation about magic tricks.
Even if you have seen the trick before, 
and even if you know how it is done, 
so that what you see is exactly what you expect,
the trick may nevertheless retain its magic.
How does this ever happen?
The retained magic is the metacognitive feeling of surprise, which occurs because the trick interferes with the smooth operation of broadly perceptual processes in you. 
This is why familiarity, and even insight, do not always prevent feeling the magic.




\section{Conjecture O\textsuperscript{m}}
\label{sec:metacognitive-feelings-object-indexes}
I introduced metacognitive feelings on the basis that they would enable us to overcome the objection to \gls{Conjecture O} from \cref{sec:objection-to-conjecture-o}.
But how are metacognitive feelings relevant to understanding four-month-old infants’ abilities concerning physical objects?

I conjecture that violation-of-expectation experiments are not far from magic tricks. 
Operations involving object indexes can give rise to metacognitive feelings of surprise, and these metacognitive feelings can in turn influence looking behaviours in \gls{habituation} and \gls{violation-of-expectation} experiments. 

To illustrate this conjecture,
recall  once more the one stick/two sticks experiment depicted in \vrefrange{fig:kellman_1983_fig3part1}{fig:kellman_1983_fig3part2}.
Imagine it is you, rather than an infant seeing the stimuli.
You might have seen them many times before, so that you know just what to expect when that middle box is removed.
Still, the event in which the two sticks is revealed is, like a magic trick, liable to feel surprising in some small way.
This is characteristic of the metacognitive feeling of surprise: its normal cause is not, or not only, the unexpectedness of the event but rather the extent to which it interferes with ongoing mental activity.
But what mental activity does the event of revealing the two sticks  interfere with?
During the habituation phase, you assign a single object to the stick behind the box.
Then, when the habituation phase is over and the box is removed, you are shown two unconnected short sticks.
So whereas you assigned a single object index to the stick ends moving together, when the box is removed two object indexes are needed.
There is an error in your system of object indexes, which is an interference with ongoing mental activity.
This interference can give rise to a metacognitive feeling of surprise, which could in turn influence your looking behaviour.

We are now in a position to overcome the objection to \gls{Conjecture O} introduced in \cref{sec:objection-to-conjecture-o}.
Recall that, according to Conjecture O, four-month-olds’ abilities to segment objects, to represent them as persisting and to track their causal interactions variously involve object indexes and motor representations.
In particular, object indexes (and not motor representations) are held to explain why four-month-olds succeed on \gls{habituation} and \gls{violation-of-expectation} tasks involving objects occluded by an impenetrable screen (see \cref{sec:CLSTX-conjecture,sec:motor-representation-objects}).
The objection to \gls{Conjecture O} is that object indexes are fundamentally unsuited to explaining success on these tasks, both because they involve voluntary looking behaviours which object indexes are not thought to explain and also because the timings involved are not on the scale on which object indexes are thought to operate (see \cref{sec:objection-to-conjecture-o}).

This is a good objection: \gls{Conjecture O} really cannot explain the looking behaviours that it needs to explain.
But the objection can be overcome by switching to what I will call \emph{\gls{Conjecture Om}} (‘m’ for metacognitive).
This is just Conjecture O together with the further conjecture that
errors in operations on object indexes (and motor representations) can give rise to metacognitive feelings of surprise.
If \gls{Conjecture Om} is correct, it is metacognitive feelings of surprise that connect operations on object indexes to looking behaviours.

% What explains the looking behaviours is the metacognitive feeling of surprise triggered by operations involving object indexes.

% The metacognitive feeling of surprise is how a system of object indexes informs you about irresolvable problems.

Note that this is just a guess---and not one that, as far as I know, anyone else would yet endorse.
We have moved well beyond the evidence into the realm of theoretical speculation (although perhaps not as far beyond the evidence as proponents of core knowledge, whose theoretical commitments are considerably bolder; see \cref{sec:core-knowledge,sec:against-core-systems}).
% But it is at least worth refuting.
But what is the alternative?
If you do not accept that metacognitive feelings can be triggered by operations on object indexes, then you face a problem.
 You will either need to scrap  \gls{Conjecture O} altogether and find an alternative solution to the \gls{Linking Problem}.
Or else you will have to provide an alternative account of how operations on object indexes influence looking times in \gls{habituation} and \gls{violation-of-expectation} experiments.%
\footnote{%
An alternative account of how operations involving object indexes influence looking times might start from Smith’s discussion of perceptual anticipations \citep[pp.~736--9]{smith:2010_seeing}.
\citet[§4]{butterfill:2015_perceiving} proposed a way of elaborating on Smith’s notion, relabelling it a ‘phenomenal expectation’.
In earlier presentations of this work, I suggested that the objection to Conjecture O could be solved by invoking these phenomenal expectations.
I now think that was probably a mistake and that metacognitive feelings provide a better explanation.
}

% This is partly because, by contrast with notions like representation, there has been strikingly little progress in understanding how bodies and brains yield subjects of experiences.
% To talk about experience is to make use of a commonsense model of the mind rather than a scientific model.
% And we know that commonsense models, unlike scientific ones, are not supposed to be accurate at any cost: instead they balance accuracy against practical and normative benefits.
% But if we want to understand the origins of knowledge we will have to understand how things appear from the point of view of a four-month-old. We can hardly hope to do this without formulating and testing conjectures about her experiences.

% One non-evidential attraction of my guess is that, if true, adults can experience roughly what it is like to be an infant in a \gls{violation-of-expectation} experiment.
% What you experience when you see the thumb lifted to reveal a broken triangle for the tenth time is roughly what infants experience when their expectations are violated.
% This is what it is like to be an infant.
% Their expectations are not beliefs but metacognitive feelings caused by the operations of a system of object indexes.

% The guess about object indexes and metacognitive feelings allows us to make sense of some puzzling patterns in four-month-olds’ abilities to represent objects as persisting.
% Recall the contrast (from \cref{sec:against-simple-view}) between Baillargeon’s drawbridge experiment and Shinsky and Munakata’s similarly designed experiment in which infants could remove a screen to see an object.
% Suppose an object index remains attached to an object which disappears behind a screen,
% and that this gives rise to a phenomenal expectation concerning what would happen if the screen were moved in much the way that seeing the right panel of \cref{fig:kellman_1983_fig2} gives rise to a phenomenal expectation.
% When the screen does move and the phenomenal expectation is not met, infants are briefly captivated just as adults would be.
% This may be why they manifest an ability to represent objects as persisting in many \gls{violation-of-expectation} tasks.
% But why don’t they also manifest this ability when, as in Shinsky and Munakata’s experiment, they are given an opportunity to remove the screen for themselves?
% As phenomenal expectations do not directly feed into preparation for action, supposing that infants have a phenomenal expectation does not generate the incorrect prediction that they will move the screen or take other steps in order to retrieve the object.
% This would require a belief or some other action-guiding state about the object’s location,
% and phenomenal expectations are neither beliefs nor invariably give rise to beliefs.


While much remains uncertain, three things seem clear.
First, the existence in four-month-old infants of a system of object indexes is likely to explain, in part, many of their abilities concerning physical objects (see \cref{sec:signature-limits}).
Second, operations involving object indexes cannot by themselves explain patterns of looking duration in \gls{habituation} and
\gls{violation-of-expectation} experiments (see \cref{sec:objection-to-conjecture-o}).
Third, it might just be that some operations involving object indexes give rise to metacognitive feelings, which in turn influence looking durations.

If this is right, we have solved the \gls{Linking Problem}.
The problem was to identify what links the \gls{Principles of Object Perception} to the mind of an individual (see \cref{sec:the-challenge}).
The solution hinges on \gls{Conjecture Om}.
In the individual mind there are object indexes and motor representations; and the Principles characterise the individual’s abilities concerning physical objects insofar as these principles constrain operations on  object indexes and motor representations.
This solution to the Linking Problem has a variety of consequences for understanding the developmental emergence of knowledge, as we will see in the conclusion to \cref{part:physical-objects}.

% To see that there is a question about how object indexes might influence actions, let us distinguish two kinds of influence they might have: influence on ongoing actions, and influence on preparation for action.
% Take ongoing actions first.
% Suppose an infant prepares to reach for an object before it disappears, or is visually tracking an object which disappears and then shows anticipatory looking in advance of its reappearance.
% In such cases, an action directed to a perceived object is initiated; the object momentarily becomes unperceived; and the action concludes before, or unfolds in anticipation of, the object’s reappearance.
% Here there does not seem to be a need to introduce belief or anything other than object indexes.
% The very possibility of performing and understanding object-directed actions depends on an object index of some kind influencing how actions unfold.

% But things are quite different when we consider preparation for action.
% Consider search tasks where the actions are not prepared in advance of an object’s disappearance (perhaps because the things on which a subject would act are out of reach until after the object has disappeared).
% Or consider looking actions in violation-of-expectation experiments, where much of the looking is initiated after an object is no longer visible.
% We cannot explain why such actions occur merely by appeal to object indexes.

% **



% \begin{enumerate}
% \item if object indexes do not directly influence beliefs or planning for action, how do they influence looking times, anticipatory looking or any other kind of behaviour?
% \item Claim: they influence ongoing actions directly.  This explains some searching and anticipatory looking.  What it can’t explain is violation-of-expectation.
% \item two ways something can be perceptually present: the Michotte et al triangle thumb trick
% \item object indexes create phenomenal expectations, at least sometimes (\citet{Mitroff:2004pc} might require us to be cautious in thinking that there is a direct relation)
% \item conjecture (level 1): object indexes mostly influence planning for action, verbal responses and belief or knowledge states indirectly, via phenomenal expectations
% \item level 2: object indexes can probably also influence how actions unfold, although probably not preparation for action, in ways that do not involve phenomenal expectations

% \item let’s apply this idea to the discrepancy between looking times and reaching (Baillargeon’s drawbridge vs Shinsky and Munakata’s pull-barrier): there is an object index; the object index creates a phenomenal expectation; the phenomenal expectation is violated when the drawbridge moves through the object; but the phenomenal expectation does not create a belief that the object is there and so does not influence action.

% \item but why doesn’t the phenomenal expectation create a belief?  One possibility is that four-month-olds never make the transition from phenomenal expectation to belief.  But then there is the problem with reaching into milk but not behind barriers.  Spelke’s suggestion about multiple objects can be recycled here: whereas she thinks *xyz, it may be that the transition from phenomenal expectation to belief is harder to make when scenes contain multiple objects and one occludes another.  (But then why does Baillargeon’s drawbridge work?  Here there are multiple objects and one occludes another!  Perhaps the problem is specifically with beliefs when multiple objects are involved?)

% \item Here’s another possibility on why doesn’t the phenomenal expectation create a belief?  The phenomenal expectation is formed relatively late, only when there is a violation of a physical constraint on the movements of objects.

% \item So adding to the \gls{CLSTX Conjecture}: object indexes give rise to phenomenal expectations which are not beliefs. (They are judgement-independent.)  This is an attempt to explain why object indexes influence behaviour, and why they do so only in limited ways.

% \item on the Simple View there are two kinds of belief: belief in the principles, and beliefs in facts about particular objects’ locations, shapes etc.  On the \gls{CLSTX Conjecture}, there are neither.  Beliefs in the principles becomes facts about how object indexes operate (which are probably not represented; cf logistician whose pins are on rails), and beliefs about particular objects become phenomenal expectations.

% \item Relation to core knowledge?  Object indexes (or object files) are the core system; principles still count as core knowledge by our definition.  But what does the explaining is not the general notion of core knowledge but rather the identification of a particular system, the system of object indexes.
% \end{enumerate}





% \section{How Do Infants Track Causal Interactions?}
% \label{sec:do-infants-detect}

% [*Excellent review of perceptual causation developmental findings: §2.2.5 of \citep{hubbard:2012_phenomenala} pp.~27ff]




% \begin{enumerate}

% \item We have been postponing the question of whether infants’ abilities to track causal interactions might be a consequence of their having a system of object indexes.

% \item Recall, for example, that infants expect a barrier to stop a ball (Spelke ball drop), and they expect a rotating drawbridge to be stopped by a solid object behind it.

% \item There is a quick argument against the claim that this could be a consequence of operations on object indexes: (i) object indexes are insensitive to solidity, mass and other causal features; they are concerned exclusively with velocity and the like; (ii) therefore it’s impossible that infants’ abilities to track causal interactions could be a consequence of a system of object indexes.

% \item Strategy for replying to this argument is two-fold.
%   \begin{enumerate}
%   \item weak strategy: show that there is some processes that doesn’t involve belief and inference which could in principle underpin infants’ abilities to track causal interactions.
% Here the claim is that there are phenomenal expectations; whether these phenomenal expectations are consequences of object indexes or something else is an open question.
% First step to the weaker strategy is Representational Momentum; second step is Michotte on launching.  If either step is right, we are ok.
%   \item stronger strategy: object-indexes underpin representational momentum or launching; in that case all we need is the Brilliant Conjecture

%   \end{enumerate}

% \item Even the weaker strategy is controversial; the stronger strategy involves ideas so wildly controversial that several states have now banned those under 21 from entertaining them.

% \end{enumerate}




% \section{Representational Momentum}
% \label{sec:representational-momentum}

% \begin{enumerate}

% \item Representational momentum sometimes occurs when adult humans observe a moving object that disappears and are asked about its final location.
% Sometimes they will misremember the location of its disappearance in way that reflects its momentum; this effect is called \emph{representational momentum} \citep{freyd:1984_representational,hubbard:2010_rm}.

% \item RM strictly speaking shows an effect of physical properties (momentum) rather than merely velocity
% The trajectories implied by representational momentum reflect mechanical principles \citep{freyd:1994_representational,kozhevnikov:2001_impetus,hubbard:2001_representational,hubbard:2013_launching}.
% \footnote{%
% Note that momentum is only one of several factors which may influence mistakes about the location at which a moving object disappears \citep[p.~842]{hubbard:2005_representational}.
% %:
% %\begin{quote}
% %`The empirical evidence is clear that (1) displacement does not always correspond to predictions based on physical principles and (2) variables unrelated to physical principles (for example, the presence of landmarks, target identity, or expectations regarding a change in target direction) can influence displacement.'
% %
% %\ldots\
% %
% %`information based on a naive understanding of physical principles or on subjective consequences of physical principles appears to be just one of many types of information that could potentially contribute to the displacement of any given target'
% %\end{quote}
% }

% \item RM is belief-independent.
% \citep{freyd:1994_representational,kozhevnikov:2001_impetus}.
% Representational momentum therefore reflects judgement-independent expectations about objects’ movements which track their causal features.%



% %
% \begin{figure}
% \begin{center}
% \includegraphics[width=0.9\textwidth]{fig/freyd_1994_fig1.png}
% \caption{
%   \label{fig:freyd_1994_fig1}
%   Three trajectories.
%   An object exiting a spiral tube will actually follow the straight path (A).
% %Impetus mechanics, a theory of the physical, incorrectly predicts that an object existing a spiral tube will follow a spiral path (C).
% But even in people who know this to be true, representation momentum shows that in some sense they expect an object to follow the spiral path (B).
%   Source: \citet[][figure 1]{freyd:1994_representational}.
% }
% \end{center}
% \end{figure}
% %





% \item I claim that RM give rise to phenomenal expectations.  This is why it can affect judgements about location while being belief independent.  (Other accounts are possible).

% \item Is there RM in infants?  One study with two-year-olds suggests that there is \citep{perry:2008_representational}, but this is defective (object does not disappear but moves behind a barrier, and the feature measured is not momentum but speed)

% \item RM gets us a weak conclusion: there are processes for tracking objects which take into account causal features like momentum and are to a significant degree belief-independent.

% \end{enumerate}



% \section{Perceiving Causes?}
% \label{sec:perceiving-causes}




% \begin{enumerate}

% \item There is a launching effect.

% \item The launching effect occurs in infants

% \item But is it perceptual or modular or is it simply a matter of belief and knowledge?

% \item dissociations suggest it is; context effects suggest it may not be.
% Here I’m following \citep{rips:2011_causation}.
% [*On claim that causal is perception, see Rips on dissociations; he also offers a very balanced view of the evidence (see p.~92 on dissociations vs context effects and individual differences; it all hangs on whether context effects ‘truly alter the impression of causality, or do they only exert bias on people’s decisions about whether a causal event has occurred after the fact’]

% \item There are causal constrains on perceptual processes.  To see why, we need to ask a question.
% Why does the illusory causal crescent appear?  Scholl and Nakayama suggest a
% ‘a simple categorical explanation for the Causal Crescents illusion: the visual system,
% when led by other means to perceive an event as a causal collision, effectively
% ‘refuses’ to see the two objects as fully overlapped, because of an internalized
% constraint to the effect that such a spatial arrangement is not physically possible.
% As a result, a thin crescent of one object remains uncovered by the other one-as
% would in fact be the case in a straight-on billiard-ball collision where the motion
% occurs at an angle close to the line of sight.’
% \citep[p.~466]{Scholl:2004dx}

% \end{enumerate}




% \section{Object Indexes, Representational Momentum, and Launching}
% \label{sec:object-indexes-rm-launching}


% \begin{enumerate}

% \item RM to launching: \citep{hubbard:2001_representational} \citep{hubbard:2013_launching}

% \item Launching to object indexes \citep{Krushke:1996ge}

% \end{enumerate}


\section{Metacognitive Feelings are Intentional Isolators}
\label{sec:intentional-isolation}

We have revised \gls{Conjecture O}, adding the further conjecture that  operations on object indexes and motor representations can give rise to metacognitive feelings of surprise, which in turn can cause looking behaviours such as those observed in \gls{habituation} and \gls{violation-of-expectation} experiments.
You might object that this combination of conjectures, \gls{Conjecture Om}, generates some of the same incorrect predictions that were fatally generated by the \gls{Simple View}. (These predictions and the evidence disconfirming them were identified in \cref{sec:against-simple-view,sec:furth-evid-against,sec:even-worse-for-the-simple-view}.)

To see how the objection might arise, consider another application of \gls{Conjecture Om}.
What happens when an infant (or adult) observes an object being occluded by an impenetrable barrier?
According to \gls{Conjecture Om},
an object index attached to the object is maintained at a location behind the barrier.
Further, if the barrier is removed and the object does not appear, the system of object indexes will encounter an error condition:
an object index is assigned to nothing.
This interference in processing the scene can give rise to a metacognitive feeling of surprise, which could cause the infant to look longer and be more interested in this event than she would be in another, less magical event.
So far this is exactly what we want.
But the objection is that the metacognitive feeling involves or entials knowledge of, or belief about, the object’s location.
In that case, our view would imply that infants know or believe something about the object’s location.
So \gls{Conjecture Om} appears to generate the incorrect prediction that even four-month-olds will manually search behind the impenetrable barrier (see \cref{sec:against-simple-view})

This objection is based on a mistake about metacognitive feelings, at least as characterised here (see \cref{sec:metacognitive-feeling-df}).
It is true, of course, that metacognitive feelings enable adults to transition from perceptual or motor processes to beliefs.
When encountering an event or viewing a face, for example, metacognitive feelings of agency or of familiarity may enable you to acquire the belief that the event is an action of yours, or that the face is familiar to you.
But there is an important detail in how metacognitive feelings enable you (as an adult) to acquire beliefs.

Metacognitive feelings have no intentional objects, or none that are related to any beliefs that you might ordinarily acquire on the basis of them (see \cref{sec:metacognitive-feeling-df}).
They therefore serve as \emph{\glspl{intentional isolator}}.
That is, they provide a nonintentional link between two intentional states.
The processes involved in action selection (or face processing) involve representations of an action (or face), and the beliefs you acquire also intentionally specify an action (or face). 
But the metacognitive feeling of agency (or of familiarity) that connects these two has no such intentional object.
In acquiring the belief, you have to form a view about what caused the feeling.

Compare being stung by a subtle nettle, where the sting can only be felt some time after brushing against plant.
Feeling the stinging sensation, you might look around to see what caused it and identify the likely plant behind you.
The stinging sensation is like a metacognitive feeling.
Neither has an intentional object that is relevant to the belief you eventually acquire.
Instead, acquiring the belief involves identifying a likely cause of the sensation or metacognitive feeling.

The objection to \gls{Conjecture Om} arises because it is often so effortless for adults to acquire beliefs on the basis of \glspl{metacognitive feeling} that they rarely notice any inference is needed at all.
But once we recognise that metacognitive feelings lead to beliefs only via a process of inference which involves identifying a likely cause 
%(and the probable significance) 
of the metacognitive feeling,
we can see that the objection rests on a mistake.
We know four- and five-month-olds are not in a position to make such inferences: if they were, they would have beliefs about, or knowledge of, the locations of briefly occluded physical objects; which they do not (see \cref{sec:against-simple-view,sec:furth-evid-against}).
So  \gls{Conjecture Om}  does not generate incorrect predictions about infants’ beliefs about, or knowledge of, particular physical objects.

When adults have a metacognitive feeling of surprise, they are likely to have an idea of why they are surprised.
Four- and five-month-olds also have metacognitive feelings of surprise, but they are unlikely to have any idea why they are surprised.
This will eventually be critical for understanding how humans first come to acquire knowledge about particular physical objects in development, because it tells us something about how the early developing capacities concerning physical objects are isolated from epistemic capacities.

% After all, wouldn’t having a metacognitive feeling of surprise involve having beliefs about the surprising event?



\section{Conclusion}

Our overall aim is to understand how humans first acquire knowledge of simple facts about particular physical objects in their development.
A key discovery is that infants at around four months of age already manfiest abilities that seem likely to support, somehow, the emergence of such knowledge (see \cref{cha:principles-object-perception,cha:simple-view}).
The problem we encountered was to understand the kind of cognitive states and processes involved in these abilities.
After ruling out belief and knowledge states (in \cref{cha:linking-problem})
and finding that merely invoking core knowledge is  at best insufficient (in \cref{cha:core-knowledge}),
 we hit on \gls{Conjecture O} according to which infants’ abilities are underpinned by a combination of object indexes and motor representations of objects (see \cref{cha:causation}).
This conjecture is most promising insofar as it does not generate the incorrect projections that other views do, but does generate many readily testable predictions, some of which have already been tested.
Yet in this chapter, we saw that there is an objection to the conjecture (in \cref{sec:objection-to-conjecture-o}).
The objection can be overcome by invoking \glspl{metacognitive feeling}.

This objection and its resolution matters in two ways.
On the one hand, we cannot claim to have characterised the states and processes involved in infants’ earliest abilities concerning physical objects without overcoming it.
On the other hand, overcoming the objection indicates that infants’ earliest abilities concerning physical objects involve a further ingredient, namely \glspl{metacognitive feeling}.
This last ingredient will shortly turn out to be essential for understanding something about how humans first acquire knowledge of simple facts about particular physical objects.







%%% Local Variables:
%%% TeX-master: "master"
%%% End:
