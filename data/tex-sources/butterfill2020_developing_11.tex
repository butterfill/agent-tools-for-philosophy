%!TEX root = master.tex

\chapter{Innateness}
\label{cha:innateness}

% Are syntactic abilities \gls{innate}?
If you are going to discuss whether something is \gls{innate}, you had better fix on a particular notion of innateness.
There are plenty to choose from, although no one notion seems to be entirely satisfactory \citep[as][argue]{mameli:2011_evaluation}.
I will simply stipulate that for a cognitive ability to be \gls{innate} is for its developmental emergence not to be a direct consequence of data-driven learning.
In short, innate = not learned.

This way of characterising innateness fits perfectly with the most compelling arguments for innateness, which take the form of \glspl{poverty of stimulus argument}.
Such arguments aim to show that some aspect of humans’ syntactic abilities, say, are not entirely  a consequence of data-driven learning (see \citealp{pullum:2002_empirical} for details).
 As we will see in a moment,
 there are some compelling poverty of stimulus arguments.
So any acceptable characterisation of innateness must respect the fact that poverty of stimulus arguments can establish innateness.
And equating innate with not learned, as I propose, is the simplest way to achieve this.

Samuels objects to equating innate  with not learned (see \citealp[p.~139]{Samuels:2004ho}).
He notes that it is unilluminating on the grounds that the notion of  learning is difficult to characterise for much the same reasons that the notion of innateness is.
Samuels aims for a deeper understanding of innateness.
I think Samuels is right that equating innate with not learned is barely informative, but I see this as a virtue rather than a deficit.
If the last couple of millennia or so are any guide, philosophical methods do not enable us to give substantial characterisations of nonlogical phenomena.
With this in mind, modesty in philosophical characterisation can be a virtue---we aim merely for theoretical coherence, recognising that discovering substantial truths usually requires doing more than just thinking about them.
This is why I characterise innateness simply as a matter of not being a direct consequence of data-driven learning.


\section{Syntax}
\label{sec:syntax}
The domain in which innateness has been most extensively and carefully researched is probably syntax.
What is syntax?
Contrast these two sequences (to adapt a famous example from Chomsky):
%
\begin{enumerate}
  \item The turnip of shapely knowing isn't yet buttressed by death.
\item  *The buttressed turnip shapely knowing yet isn't of by death.
\end{enumerate}
%
Whereas the second sequence of words is not a sentence, the first is widely recognised as a sentence---although not one you are likely to make much sense of.
Now take a second contrast:
\begin{enumerate}[resume]
  \item Ayesha ate Ben.
  \item Ben ate Ayesha.
\end{enumerate}
Again, the two sentences involve the same words, but they have quite different implications.
The difference is syntax.

Although capable of tracking syntactic differences and exploiting them in communicating with words, humans are often unaware of syntax.
Consider a simple phrase, ‘the red ball.’
In principle, this phrase could have two different syntactic structures, as illustrated in \vref{fig:lidz_2003_fig0}.
The difference in structure may not appear very significant at first.
But consider what would happen if someone said, ‘This red ball is broken, please bring me another one!’
Can you comply by bringing a blue ball, or does the ball have to be red?
The answer turns out to depend on the structure of the phrase ‘this red ball’.
If you think only a red ball will do, you are treating the phrase as having nested structure (compare \vref{fig:lidz_2003_fig0}).
This is because the ‘one’  in ‘please bring me another one’ can only refer to something that a constituent of an earlier part of the sentence already introduced, and if the ‘this red ball’ has a flat structure, ‘red ball’ is not a constituent.
Many people, including me, would have no idea which structure, flat or nested, they treat phrases like ‘the red ball’ as having in advance of having this explained.


\addFigure{lidz_2003_fig0}{%
Two possible structures for ‘the red ball’.
Source: \citet{lidz:2003_what}}


Abilities to track syntactic properties and exploit them in communicating with words cannot be based merely on past exposure to particular sentences. 
After all, you can often distinguish entirely novel sentences from nonsentences.
To describe the ability, it seems that we will need principles of syntax,
much as we needed principles to characterise abilities to segment objects, represent them as persisting and track their causal interactions (see \cref{cha:principles-object-perception}).
We can think of the syntactic principles as entailing, for any arbitrary sequence of words, whether or not it is a sentence.



\section{A Poverty of Stimulus Argument}
\label{sec:pos-argument}
How can we discover whether syntactic abilities are innate?
\citet{lidz:2003_what} set out to answer this question using phrases like ‘the red ball’.
As we saw in \cref{sec:syntax}, such phrases can in principle be assigned either of two syntactic structures, one flat and the other nested.
(These were illustrated in \vref{fig:lidz_2003_fig0}.)
\citeauthor{lidz:2003_what} aimed to show that
18-month-old infants already interpret this sort of phrase as having nested structure, and that they could not have learnt to do this just on the basis of their experiences of language.
This would imply that at least some aspects of humans’ syntactic abilities are a consequence of things that are innate.


The first task is to determine how 18-month-old infants interpret phrases like ‘the red ball.’
Suppose someone says, ‘Look, a red ball!  Do you see another one?’
What would you be looking for?
If you were to look for any ball, whether red or not, this would indicate that you had interpreted the phrase as having flat structure.
But if you are like me, you will be looking for another red ball.
This indicates that you interpret the phrase ‘the red ball’ as having nested structure.
We can therefore find out whether some people interpret the phrase as having nested or flat structure by measuring whether they would look for a red ball or just any ball.
This is the perfect approach for 18-month-old infants, since it does not require them to make any verbal responses.
Accordingly, \citeauthor{lidz:2003_what} played infants a recording of sentences like ‘Look, a red ball!  Do you see another one?’
While they heard this recording, they could see a red ball (or other object corresponding to the sentence they heard).
After the recording, the infants were shown two things simultaneously, such a red ball and a blue ball.
It turned out that infants looked significantly longer at the red ball than the blue one.

You might object that this finding is not revealing.
After all, infants’ tendency to look longer at the red ball than at the blue ball might be due to the fact that they have just seen a red ball.
To rule this out, \citeauthor{lidz:2003_what} had a control condition.
This was exactly like the main condition except that the sentence infants heard was like  ‘Look, a red ball!  What do you see now?’.
That is, instead of asking ‘Do you see another one?’, they asked ‘What do you see now?’.
In this control condition, infants showed the opposite pattern in their looking times: they looked significantly longer at the blue ball or other novel object (see \vref{fig:lidz_2003_fig1}).
So infants’ tendency to look longer at the red ball really does indicate that they interpret this phrase as having nested structure.

\addFigure{lidz_2003_fig1}{%
In response to ‘Look, a red ball!  What do you see now?’  (‘Control’), infants look longer at the blue (‘novel’) ball.  But in response to  ‘Look, a red ball!  Do you see another one?’ (‘Anaphoric’), infants look longer at the red (‘familiar’) ball.  Source: \citet[figure~1]{lidz:2004_reaffirming}}

The fact that 18-month-olds tend to interpret phrases like ‘the red ball’ as having nested structure is not by itself evidence for innateness.
We also need evidence that infants could not have learnt the nested structure on the basis of experiences of language.
\citet{lidz:2003_what} aimed to provide this by analysing a large collection of around 45000 utterances directed to children infants.
In all of these utterances, they found only two which could have indicated nested structure to infants.
To put this into perspective, there were four ungrammatical uses of ‘one’.
This survey suggests that 18-month-olds’ experiences of language do not provide them with a basis for learning that phrases like ‘the red ball’ have nested structure.

Putting \citeauthor{lidz:2003_what}’s  two findings together gives us a \gls{poverty of stimulus argument}.
The two findings are that 18-month-olds can interpret phrases like ‘the red ball’ as having nested structure, although their experiences of language do not provide them with a basis for learning this.
The \gls{poverty of stimulus argument} goes like this:
%
\begin{enumerate}
  \item 18-month-olds can interpret phrases like ‘the red ball’ as having nested structure.
  \item To acquire this ability by data-driven learning would require experiences of utterances in which such phrases had to be understood as having nested structure.
  \item But extremely few such utterances are directed to 18-month-olds.
  \item So 18-month-olds do not acquire the ability to  interpret nested structure by data-driven learning.
  \item But all acquisition is either data-driven learning or innately-primed.
  \item So 18-month-olds’ acquisition of the ability to  interpret nested structure is innately-primed.%
  \footnote{%
  This is adapted from \citet[][]{pullum:2002_empirical}, who provide a detailed discussion of poverty of stimulus arguments.
  }
\end{enumerate}
%
Note that this argument establishes that something is innate but does not tell us what is innate.
The conclusion is not that the ability to interpret phrases like ‘the red ball’ as having nested structure is innate.
It is that acquiring this ability depends on something that is innate.
Establishing what that innate thing is would require further work.


Some linguists assume that some specifically syntactic abilities must be innate.
For instance, \citet[p.~25]{chomsky:1965_aspects} describes the linguists’ task as that of characterising the ‘innate linguistic theory that provides the basis for language learning.’
% ‘Thus what is maintained, presumably, is that the child has an innate theory of potential structural descriptions that is sufficiently rich and fully developed so that he is able to determine, from a real situation in which a signal occurs, which structural descriptions may be appropriate to this signal, and also that he is able to do this in part in advance of any assumption as to the linguistic structure of this signal.’
\Glspl{poverty of stimulus argument} provide no justification for this assumption.
Further evidence would be needed to support a conclusion about what in particular is innate.


How strong is \citeauthor{lidz:2003_what}’s argument for innateness?
The first premise, 1, is that 18-month-olds can interpret phrases like ‘the red ball’ as having nested structure.
To date this depends on evidence from a single lab, so should be regarded with caution.
The third premise is that few utterances in which phrases like ‘the red ball’ have to be interpreted as having nested structure are directed to infants.
Confidence in this premise should be high given that it is based on a large number of recorded utterances.
However, there is a possible line of objection linked to premise 2, which is about the need for such utterances.
\citeauthor{lidz:2003_what} considered utterances directed to infants where the nested interpretation of phrases like ‘the red ball’ is required to understand the sentence.
But it is possible in principle that infants may be exposed to some combination of linguistic and nonlinguistic evidence that enables them to arrive at the nested structure interpretation.
Attempts to develop a concrete objection along these lines (see \citealp{akhtar:2004_learning}) are not convincing \citep[pp.~161--2]{lidz:2004_reaffirming}.
While new discoveries about evidence or learning mechanisms are always possible, as things stand the balance of evidence appears to favour the view that some syntactic abilities depend on something innate.

The discovery that something innate underpins some of infants’ syntactic abilities is a major breakthrough.
It shows that there is no general reason to hold that development in other domains could not depend on things that are innate.
And reflection on the case of syntax supports the view that  early-developing abilities and states, perhaps including core knowledge, play an important role in development.
This is a major challenge to theories about the developmental emergence of knowledge which exclusively invoke social interaction.



\section{The Poverty of Poverty of Stimulus Arguments}
\label{sec:poverty-of-poverty-of-stimulus}
Innateness is an exciting, attention grabbing topic.
Why risk destroying interest in it by focussing so narrowly on the case of ‘the red ball’?
% (If you have been reading from the start of the chapter, I am quite sure you never want to hear that phrase again.)
Consider an example of how poverty of stimulus arguments have been wielded in philosophy:
%
\begin{quote}
‘There would seem not to be enough ambient information available to account for the functional architecture that minds are found to have’ \citep[p.~35]{Fodor:1983dg}.
\end{quote}
%
It is hard to detect an argument here.
At least, if there is an argument, an equally compelling argument can be obtained by deleting the word ‘not’ from this sentence.

Things are not very different in linguistics.
In a recent defence of poverty of stimulus arguments for a conclusion about an innate basis for syntactic abilities, \citet{berwick:2011_poverty} cite no evidence at all concerning the experiences available in development.
They also cite no evidence at all concerning the development of syntactic abilities.
When making this sort of observation, I am usually told that the evidence is of a general nature and too familiar to need explicit mention.
But this is not quite right.
Instead \citeauthor{berwick:2011_poverty}’s argument has a familiar form, which I propose to label the \emph{\gls{poverty of theory argument}}:
%
\begin{enumerate}
  \item Current theories about how certain syntactic abilities are acquired are inadequate.
  \item So the acquisition of such abilities depends on innate representations of syntactic structure.
\end{enumerate}
%
It is important to distinguish poverty of theory arguments from poverty of stimulus arguments.
Poverty of theory arguments could be used to establish that almost everything is innate.
After all,  fully adequate developmental theories are scarce.
The problem, of course, is that poverty of theory arguments assume that the inadequacy of theories is not due to the inadequacy of theorists.
By contrast, poverty of stimulus arguments do not require this assumption.
Poverty of stimulus arguments therefore provide potentially more convincing grounds for accepting conclusions about innateness.
%Whatever their merits (you would need to be licensed to drive a bus in order to navigate the holes in poverty of theory arguments), it is important to distinguish poverty of theory arguments from poverty of stimulus arguments.
% Only the latter involve evidence concerning development and the availability of information.




You might think that poverty of stimulus arguments have been around for a long time and are widely used to establish innateness, especially concerning syntax.
But it turns out that this is a myth.
In a thorough review,
\citeauthor{pullum:2002_empirical} observe that
%
\begin{quote} ‘the APS [poverty of stimulus argument] still awaits even a single good supporting example’
  (\citeyear[p.~47]{pullum:2002_empirical})
\end{quote}
%
Shortly after they wrote this, \citet{lidz:2003_what} published their example involving ‘the red ball’.
And that is the best, most careful attempt to provide a poverty of stimulus argument for human syntactic abilities to date.



\section{Is Core Knowledge Innate?}

\Gls{core knowledge} is often characterised by a list of properties including innateness \citep[for example,][p.~520]{Carey:1996hl}.
Why accept that all, or even any, core knowledge is \gls{innate}?

\Glspl{poverty of stimulus argument} reveal that some abilities are innate in nonhumans \citep[for example,][]{chiandetti:2011_chicks_op}, and perhaps that something underpinning the acquisition of some syntactic abilities is innate (as we have just seen).
So there is no general reason to oppose claims about innateness.
On the other hand, in no domain other than syntax has a poverty of stimulus argument been provided for humans.
(And even in the case of syntax, the critical evidence is not overwhelming.)

Why are even some of the most careful researchers willing to assert that core knowledge is innate?
Sometimes the arguments appear to be variants on ‘poverty of theory’ arguments. For example:
\begin{quote}
  ‘It is far from clear how children could learn anything about the entities in a domain \ldots\ if they could not single out those entities in their surroundings’ \citep[p.~439]{Spelke:1994oz}.
\end{quote}
But there are also more interesting lines of argument:
\begin{quote}
  ‘if early knowledge encompasses environmental constraints that are not obvious in the child’s perceptual and motor experience while failing to encompass more obvious constraints, then this knowledge is not likely to have been shaped by the child’s perceptual and motor experience’
  \citep[p.~438]{Spelke:1994oz}.
\end{quote}
We might call this an \emph{irrelevance of the stimulus} argument. 
If the stimuli (or data) available to infants are unconnected to an ability they are acquiring, we might suppose that the ability is not entirely a consequence of data-driven learning.
While interesting idea, this idea seems to have limited force.
One problem is that far more information is processed perceptually and motorically than is experienced; and experience is not surely necessary for learning to take place.
Another problem is a weaknesses shared with informal poverty of stimulus arguments:
in both cases, the informal arguments rest on unargued claims about the availability of information and what learning processes might extract from it.




In the absence of  evidence for the innateness of core knowledge, and given how little is currently known about developing minds, it seems to me that we should be agnostic.
We do not currently know whether any core knowledge is innate.%
\footnote{%
Compare \citet[p.~193]{Spelke:1998im}: ‘If one bases conclusions only on evidence, then I believe that studies of infants suggest that development is not strongly skewed toward either pole of the nativist-empiricist dialogue.’}



\section{Syntax and Rediscovery}

This is really a chapter about innateness;
but discussing innateness requires reference to syntax, which provides a perfect opportunity to highlight the idea that the developmental emergence of knowledge is a process of resdiscovery.

To make the connection, we should first return to 
the three-fold distinction between \glslink{formally adequate}{formally}, \glslink{descriptively adequate}{descriptively} and \gls{explanatorily adequate} (see \vref{table:levels-claims-about-the-principles}).
Consider formal adequacy.
Imagine someone who is omniscient except concerning which sequences of words are sentences, and who has unlimited cognitive resources.
Suppose this person took certain principles of syntax to be true concerning a particular language.
Could she now say, for any sequence of words, whether it is a sentence of that language?
If she could, these principles are formally adequate for the language.
Although constructing formally adequate principles of syntax for a language turns out to be a difficult problem \citep[for example,][pp.13--25]{chomsky:1957_syntactic},
it seems clear that the problem can be solved (see \citealp[Chapter 3]{jackendoff:2003_foundations} for an introduction to one approach).
It is also plausible that we could identify principles of syntax which are \gls{descriptively adequate} for a particular group of speakers.
These principles would enable us to predict which sequences of words this group of speakers will identify as sentences.
The controversial issue is \glslink{explanatorily adequate}{explanatory adequacy}.
Just here we run up against a counterpart of the Linking Problem (from \cref{sec:the-challenge}).
What is the relation between principles of syntax that are descriptively accurate for a particular language user and the mechanisms which underpin her syntactic abilities?

Inspired by the \gls{Simple View} (from \cref{sec:simple-view}), we might start with the idea that the language user knows principles of syntax.
What links the principles to an individual thinker is the thinker’s knowledge of the principles.
This  enables her to apply the principles in just the way a scientist would: she can make inferences from them to determine whether a given sequence of words is a sentence or not.
The problem is that adults are typically ignorant of syntax.
Worse, they are susceptible to false beliefs which do not directly impair their performance.
So the counterpart of the Simple View for syntax is uncontroversially false.%
\footnote{%
Theorists who assume language users have knowledge of syntax are clearly not referring to the state which is constitutively linked to practical reasoning and to inference, and which is inferentially integrated with other attitudes including belief, desire and intention (see \cref{sec:knowledge}).
\citet[p.~8]{chomsky:1965_aspects} writes that ‘a generative grammar attempts to specify what the speaker actually knows’, although the context makes it clear that he cannot be writing about the kind of state identified in \cref{sec:knowledge}.
In later work, the linking problem for syntax is dodged by dropping talk of knowledge in favour of an ‘I-language’ \citep[p.~14]{Chomsky:1995yg}.
}


Since we do not know what states might link descriptively adequate principles of syntax to an individual thinker, we might as well call the things that provide this link, whatever they turn out to be, her ‘core knowledge’ of syntax.%
\footnote{%
This is 
Jackendoff’s idea too, although he uses a different term, ‘f-knowledge’,
‘to describe whatever is in speakers’ heads that enables them to speak and understand their native language(s)’ (\citeyear[p.~652]{jackendoff:2003_precis}; see \citealp[p.~29ff]{jackendoff:2003_foundations}).
}

How does knowledge of the syntactic aspects of utterances emerge in development?
It is usually not until relatively late in life, and often as a consequence of formal education, that humans ever come to know facts about syntax, if they do at all.
As Fodor notes, in coming to know these facts we are rediscovering something that is in some sense already implicit in core knowledge of syntax:
%
\begin{quote}
‘when you learn about English syntax (e.g., in a linguistics course), what you are learning is something that, in some sense, you already knew’ \citep[p.~134, foonote~23]{Fodor:1983dg}.
\end{quote}
%
Further, the process of coming to know facts about syntax is clearly not a matter of transforming the contents of core knowledge states (if indeed they have contents) into the contents of knowledge states.
That is, the \gls{Assumption of Representational Connections} would not be correct in the case of syntax.
Core knowledge does not later become knowledge proper.
Instead, gaining knowledge of syntax is a paradigm case of development as \gls{rediscovery}.

What does rediscovery mean?
The things to be known are in some sense already implicit in your core knowledge, or in  early-developing capacities in you, such as the capacities which enable you to learn about the nested structure of phrases like ‘the red ball’.
Yet when you come to acquire the knowledge, you do so in roughly the way a linguist does.
The basis for learning is not the early-developing syntactic abilities but the things they enable you to do:
 to produce utterances which in fact have certain syntactic features, and to discriminate among utterances according to what are in fact their syntactic features.
Reflection on the behaviours and patterns of discrimination on your part, together with experience of these and social interaction with others typically involving instruction, eventually enables you to acquire knowledge about syntax (see \vref{fig:rediscovery_syntax.svg}).

\addFigure{rediscovery_syntax.svg}{The developmental emergence of knowledge of syntax involves \gls{rediscovery}.}

As in other cases, so for syntax: core knowledge or other early-developing abilities may play a key role in the emergence of knowledge not by providing building blocks for the contents of knowledge states but by enabling you to do, feel and experience things reflection on which together with social interaction plays a role in facilitating the emergence in development of knowledge.



\section{Conclusion}
To be innate is to emerge in development otherwise than as a direct consequence of data-driven learning.
The only known way to establish that something is innate is by way of a poverty of stimulus argument.
There has been striking success in identifying evidence for innateness in a single case, ‘the red ball’ (see \cref{sec:pos-argument}),
and in comparative research \citep[for example,][]{chiandetti:2011_chicks_op}.
These successes allow us to conclude that early-developing abilities can depend on things which are innate.

But contrary to any impression that developmental scientists are deeply engaged in battles between rationalism and empiricism, 
progress in providing genuine poverty of stimulus arguments concerning humans has been limited to one case.
Given the dramatic progress that has been made in understanding other aspects of development, 
this indicates that debates about innateness might not be very important after all.%
\footnote{%
Few would agree;
see \citet{scholl:2005_innateness} for one opposing perspective.
}
Certainly we should be agnostic on whether core knowledge in domains other than syntax is innate; and even in the case of syntax we should recognise that very little is known about what is innate.



% % How do humans first acquire knowledge of syntax in development?
% % Uniquely among the various domains of knowledge we have considered, physical objects, colours, actions and the rest, in the case of syntax there is evidence for innateness.
% % This evidence supports a \gls{poverty of stimulus argument}.
% % Utterances directed to children do not appear sufficient to specify some of the syntactic features which even 18-month-old infants take utterances to have.
% % This is a reason, not decisive but compelling, to hold that infants’ syntactic abilities depend on, or involve, something innate (see \cref{sec:pos-argument}).


% We do not currently know much about what is innate, however. **
% Formally and descriptively adequate theories of syntactic abilities can be constructed using one or another set of principles.
% This motivates considering a counterpart of the \gls{Simple View} as a starting point: infants’ and adults’ syntactic abilities are based on knowledge of principles of syntax.
% But in the case of syntax, the Simple View manifestly generates systematically incorrect predictions (see \cref{sec:syntax-adults}).
% It must therefore be rejected.
% Developing an \gls{explanatorily adequate} theory for syntactic abilities remains a major challenge.


% In opening this chapter, I mentioned that knowledge of language most clearly demonstrates the need for at least two ingredients in a theory of development, namely core knowledge and joint action.
% Syntactic abilities are essential if referential communication is to become communication by language, of course.
% But, conversely, there would be little point in having syntactic abilities without also being able to communicate.
% Early in development, purposively communicative, referential actions are essentially forms of joint action, at least from an infant’s point of view (see \cref{cha:communication} and \cref{cha:words}).
% What enables us to communicate by language is the combination of core knowledge of syntax and abilities for joint action.



%%% Local Variables:
%%% TeX-master: "master"
%%% End:
