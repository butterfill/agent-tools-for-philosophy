%!TEX root = master.tex

\chapter{Action}
\label{cha:action}

There are two fundamental ways of specifying an action.
Suppose you reach out and grasp something nearby, a mug say.
Your action can be specified in terms of the joint displacements and bodily configurations involved.
For example, your elbow joint becomes straighter as you reach out, your fingers preshape the mug or its handle and your wrist rotates.
Alternatively, your action can be specified in terms of its goal, namely to reach out and grasp the mug.
Importantly, actions involving quite different patterns of joint displacements and bodily configurations can all be actions directed to this goal.
Change even a tiny detail about the mug you were grasping by, for example, rotating the handle away from you, and you will systematically alter the joint displacements and bodily configurations involved without changing the goal.

Knowledge of goals is essential for understanding others’ thoughts and actions.
I seize little Isabel by the wrists and swing her around, thereby making her laugh and breaking a vase.
Looking on, you might wonder what the goal of my action was.
Did I act in order to break the vase or just to make Isabel laugh? Or was my action perhaps directed to some other goal, one not realised because my action failed?
% In this case, you could just ask me and it probably doesn’t matter very much what the goal of my action was anyway.
Knowing facts about the goals of others’ current and imminent actions is every bit as essential for social animals like humans as knowing facts about the movements and interactions of merely physical objects.
But how do humans first come to have such knowledge?


\section{Tracking vs Knowing}
\label{sec:tracking-vs-knowing}
Start by fixing terminology.
Among all of the actual and possible outcomes of a purposive action, some are outcomes to which the action is directed.
\glsadd{goal}
These outcomes are the \emph{goals} of that action.
Note that goals are not intentions, nor mental states of any kind.
Take someone who tells you that the goal of her actions is the release of a political prisoner.
She is not talking about her intentions: she’s talking about a (currently nonactual, but possible) state of the world.
This is potentially confusing because others sometimes use the term ‘goal’  to refer to mental states in virtue of which actions are directed to outcomes.
But I will always use ‘goal’ to refer outcomes to which actions are directed.

To know simple facts about the goals of an action is to know something about to which outcomes that action is directed.
\glsadd{track a goal}
By contrast, to \emph{track the goals of an action} is merely for there to be a process in you whose unfolding nonaccidentally depends in some way on which outcomes are the goals of that action.
It is possible, in theory at least, to track the goals of actions without knowing or representing any facts about goals at all.


To illustrate, consider an analogy.
An ant that has exploited a food source will lay a pheromone trail on its way back to the colony.
Other ants may then detect this pheromone trail, follow it and exploit the food source for themselves
 \citep{sumpter:2003_nonlinearity}.
%  \citet[p.~273]{sumpter:2003_nonlinearity}. ‘For the foraging of many species of ants, feedback mechanisms are in the form of pheromone trails, chemicals that are deposited by ants that have found a profitable food source and connect the nest with the food source (Hölldobler & Wilson 1990). These pheromone trails guide nestmates to discovered food sources and, upon finding the food, these recruited ants leave their own pheromone trail during their return to the nest. The trail is thus reinforced and the probability that other ants will follow the trail is further increased. The formation of such foraging trails allows nestmates to locate and exploit the source.’
These ants are tracking the locations of food in their environment.
But their doing so does not involve knowing about or representing locations, nor does it require a model of space.
Instead what the ants identify are the pheromone trails.
So identifying pheromone trails enables ants to track the locations of food sources without knowledge of locations.
As this illustrates, tracking something does not always involve knowing about it, nor even representing it.%
\footnote{%
There may be very liberal notions of mental representation on which tracking something is sufficient for representing it.
Cases like that of the ants to demonstrate the value of distinguishing tracking from representing.
}

In this chapter we will first consider research showing that infants, from three months of age or earlier,  can track the goals of actions.
Whether this tracking involves knowledge of goals is a further question we will come to later.



% ∞TODO more on the tracking/knowing difference?

 \section{Three-month-olds Track the Goals of Actions}
\label{sec:infants-track-goals}

\citet{Gergely:1995sq} asked what happens when twelve-month-olds  see an action.
Do they represent the movements involved in the action only, or are they also sensitive to the goals of the action?
To answer this question, \citeauthor{Gergely:1995sq} compared two changes: in one, there is a change in movement trajectory but no change in goal; in the other there is a change in goal but no change in movement trajectory.
If infants  track trajectories only, they should be more interested in  the former change.
But what  \citeauthor{Gergely:1995sq} actually found was that their twelve-month-olds  were more interested when the goal changed even though this involved no change in movement trajectory.

How did their experiment work?
 \citeauthor{Gergely:1995sq} habituated all the infants to the animation represented in \cref{fig:gergely_1995_fig1}.
There are two balls in this animation, a small one and a large one.
The animation starts with cues  suggesting affinity between the two balls and indicating that the small ball's movements are self-propelled.
The small ball then moves over the barrier and stops by the large ball.
Once infants were habituated to this animation, they were shown one of the two new animations  represented in \cref{fig:gergely_1995_fig3}.
In the ‘New Action’ condition, infants saw the small ball move directly to the larger ball, whereas in the ‘Old Action’ condition the small ball follows the same trajectory it moved along when there was a barrier between it and the large ball.
Thinking purely in terms of objects’ trajectories, the animation shown in the New Action condition is most different from the animation to which infants were habituated.
So if the infants track trajectories only,  they should show greater dishabituation in the New Action condition than in the Old Action condition.
By contrast, the hypothesis that the infants can track goals to which actions are directed is consistent the opposite pattern of \gls{dishabituation}.
To see why, imagine you had seen the initial animations and concluded that the small ball’s actions are directed to the goal of reaching the large ball.
Then you are in the Old Action condition and see the small ball following the old trajectory.
Seeing this should suggest that you have missed something---if the goal of the small ball’s actions is merely to reach the large ball, it is unclear why it is taking such an indirect route to the large ball.
And \citeauthor{Gergely:1995sq} did indeed find that the Old Action condition produced greater dishabituation in their infants.
They concluded that ‘by the end of the first year infants are \ldots\  capable of \ldots\  interpreting the goal-directed behavior of rational \glspl{agent}’ \citep[p.~184]{Gergely:1995sq}.

%
\begin{figure}
\begin{center}
\includegraphics[width=0.6\textwidth]{fig/gergely_1995_fig1}
\caption{
	\label{fig:gergely_1995_fig1}
The small ball moves over the barrier and stops by the larger ball. This is an illustration of part of a movie used in an experiment on infants' abilities to track the goals of actions.
  Source: \citet[figure 1(b)]{Gergely:1995sq}.}
\end{center}
\end{figure}
%

%
\begin{figure}
\begin{center}
\includegraphics[width=0.6\textwidth]{fig/gergely_1995_fig3}
\caption{
	\label{fig:gergely_1995_fig3}
Following habituation, infants were shown one of the two movies represented above.
In (a), the small ball moves directly to the large ball; in (b) the small ball takes the same trajectory taken when there was a barrier between it and the larger ball.
Source: \citet[figure 3]{Gergely:1995sq}.}
\end{center}
\end{figure}
%

You might object that it is bizarre to use balls in a study about actions.
After all, it seems unlikely that many twelve-month-olds really think that mere balls perform goal-directed actions.
But it turns out that human adults are surprisingly willing to detect goals (and even motives and needs) in patterns of movement involving geometric shapes \citep{Heider:1944ts}.
It is a reasonable guess that infants, like adults, apply whatever capacities they have for tracking goals to mere balls
(although there may be grounds to doubt this guess, as we will see in \cref{sec:perceptual-animacy}).
And of course there is an excellent reason for \citeauthor{Gergely:1995sq}  to use balls rather than more realistic depictions of agents.
The simplicity of their animations enabled them to create carefully matched control conditions, and so to exclude with more confidence the possibility that their results were an artefact of some irrelevant feature of the animations.
But is such simplicity necessary or are infants in the first year of life also sensitive to the goals of actions performed by ordinary humans?

Evidence that they are comes from an elegant and much-replicated study by  \citet[][study 3]{Woodward:1998dm}.
In Woodward’s paradigm,
infants are presented with a scenario involving two objects, a ball and a teddy, say.
In the habituation phase, a hand enters the scene and grasps one of the two objects, as depicted in \cref{fig:woodward_2001_fig1}.
In the main phase, infants are then shown one of two new events.
As in \citet{Gergely:1995sq}’s experiment, the idea is to pit sameness of movement trajectory against sameness of goal.
In both new events, the locations of the ball and teddy are reversed.
In one event, the hand follows the same trajectory as before and so grasps a new object.
In the other event, the hand grasps the same object as before, which requires following a new trajectory.
If infants ignore goals and track trajectories only, then they should find the event with the new trajectory more interesting.
But if infants track the goals of actions, then they may well be more interested in changes in the goals of actions than in changes in movement trajectories.
In that case they should find the event with the new goal more interesting and so dishabituate more strongly to it.
And this is just what Woodward observed in both nine- and six-month-olds.
She concluded that
‘early in life, infants begin to set up a system of knowledge of human action that has features in common with more mature understandings, and that is distinct from their knowledge of inanimate object motion’ \citep[p.~31]{Woodward:1998dm}.

%
\begin{figure}
\begin{center}
\includegraphics[width=0.6\textwidth]{fig/woodward_2001_fig1}
\caption{
	\label{fig:woodward_2001_fig1}
Habituation and test stimuli from an experiment by \citet{Woodward:1998dm} on six-month-olds’ abilities to track the goals of actions.
Source: \citet[figure 1]{woodward:2001_making}.}
\end{center}
\end{figure}
%

When can humans first track the goals of purposive actions?
Using an ingenious manipulation that we shall discuss later (‘sticky mittens’; see \cref{sec:tracking-is-acting-in-infancy}),
\citet{sommerville:2005_action} created a variation on Woodward’s paradigm to show that even
three-month-olds can form expectations based on the goal of an action
(for another study with three-month-olds, see  \citealp{luo:2011_threemonthold}; and \citealp{gredeback:2011_teleological} on four-month-olds).
This makes sense.
Humans are social animals and tracking the goals of actions is necessary for almost any kind of social cognition.
So we might expect that they can track the goals of actions as early as they can track the merely physical behaviours of  objects.





% You need to move sections 9.5 and 9.6 here, ahead of the next section.

% Otherwise \citet{paulus:2011_role}’s challenge doesn’t make sense.

% But if you move these sections first, you can set up \citet{paulus:2011_role} as a challenge to the idea that infants are using the teleological stance. (Who knows whether they are tracking outcomes.)

% Can then set up Paulus/Daum (plus the spoon feeding study) as a first puzzle about infants’ goal tracking. (If we are to use the method of signature limits, we do not want to be in a position where we say that infants are sensitive to goals when they perform one way and sensitive to statistical regularities when they perform another way.)




\section{Pure Goal Tracking}
\label{sec:joint-displacements}
\label{sec:pure-goal-tracking}

We have seen that infants, even at three months of age, can track not only movements but goals to which actions are directed.
How do they do this?
Tracking goals is an astonishingly complex problem unless you can communicate with others or already know something about their intentions.
You start with information about joint displacements and bodily configurations, and, if things go well, you somehow end up with an outcome to which those all are directed.
How is this possible?
% Csibra and Gergely’s Teleological Stance is the only available account of how such a transition might be made.
% As things stand, then, only one hypothesis about the most basic form of goal tracking in infancy is available:
% infants use the Teleological Stance.


In tracking the goals of some actions, it may be that adults are ascribing intentions and other mental states to the \gls{agent} of those actions.
Someone watching you feed the infant may be thinking that you are acting on an intention to feed it, for example.
Are infants also ascribing intentions  when tracking goals?

Many suggest that they are.
For instance, \citet[p.~53]{woodward:2009_infants} suggests that
‘infants understand intentions as existing independently of particular concrete actions and as residing within the individual,’ and she remarks that this ‘is essential to recovering intentions from observed actions’ (see further \citealp[p.~168]{woodward:2001_making}; and \citealp[p.~14]{Premack:1990jl}).
%\citet[p.~168]{woodward:2001_making}: ‘to the extent that young infants are limited \ldots\ , their understanding of intentions would be quite different from the mature concept of intentions’
% \citet[p.~14]{Premack:1990jl}: ‘in perceiving one object as having the intention of affecting another, the infant attributes to the object \ldots\  intentions’

In considering these claims it is vital to separate two issues.
One issue is whether infants in the first year of life ever track agents’ intentions as well as the goals of their actions.
The other issue is whether infants (and adults) can track goals independently of any knowledge of, or information about, mental states.

On the first issue, as far as I know, there are as yet no experiments which directly address this issue (which means not confounding tracking intentions  with tracking goals).%
\footnote{%
For an attempt to argue that infants do track intentions, see \citet[pp.54--5]{woodward:2009_infants}.
According to \citet[p.~618]{uithol:2014_what}, ‘intention attribution seems to be dependent on language capacities [and] does not emerge until language is sufficiently developed.’
However they support this claim by mentioning just one paper which does not appear to be about intention at all.
}
%Certainly the evidence for goal tracking in the first year of life is not obviously inconsistent with the conjecture that such young infants never track agent’s intentions.
Research on infants’ understanding of other mental states  has uncovered plenty of evidence that infants in the first year of life can track at least some mental states (see \cref{cha:mind}).
This should make us at least open to the as yet untested possibility that these infants can track intentions too.
But this is not an issue that will concern us in this chapter, as our aim to to gain a more fundamental understanding of infants’ abilities to track the goals of actions.

% SMALL SCALE (next two paragraphs ; move to after mental state issue)

The second issue is whether infants and adults can track goals without relying on any information about mental states.
To see why this issue matters, note that even simple actions typically involve a hierarchy of goals.
To illustrate, consider feeding an infant.
You reach for, grasp and pick up the spoon, scoop up some goo and deftly pilot the spoon around the infant’s flailing arms and  towards her opening mouth, spontaneously opening your own mouth at the same time as if to show the infant what to do.
Considered as a whole, the goal of your action might have been to feed the infant.
But many components of your action also have goals in their own right.
Reaching for the spoon, grasping it and picking it up are all goals relative to which you succeeded but could have failed.
These smaller goals are related to the larger goal of feeding the infant as means to that end.

Now imagine you are an infant being spoon fed your first solids by an adult.
To track the goal of the adult’s action when feeing you, you need to identify the smaller goals and not only the larger goal.
After all, achieving any of these smaller goals, such as reaching for the spoon, can involve indefinitely many different patterns of joint displacements and bodily configurations.
As you repeat the larger action, putting one spoon after another into the infant’s mouth, the joint displacements and bodily configurations vary enormously while the goals are relatively invariant.
It would be inefficient and probably infeasible to move directly from patterns of joint displacements and bodily configurations directly to the goal of feeding the infant.
Instead, goal tracking starts with very small bits of purposive action like reaching, grasping and transporting.

Assume (for the sake of argument)  that you, as the infant being fed solids for the first time, know what intentions are and can ascribe them.
Still, on what basis can you determine the intentions behind the adult’s actions?
You can’t communicate linguistically with them.
In fact it seems that the only access you have to another’s intentions is via the actions they perform.
Suppose that to track the goals of actions, you have to identify the intentions with which those actions are performed.
This is a significant burden.
What are the intentions of the adult who has just grasped the spoon?
If spoon feeding is completely new to you, you are in no position to detect that she intends to feed you.
You will therefore have to identify intentions with which smaller actions are performed, like the intention of grasping the spoon.
But consider what this involves.
To be tracking intentions and not merely goals, you need to be sensitive to the possibility that intentions and goals can come apart.
(Otherwise there would be insufficient reason to suppose that you are tracking intentions.)
% Woodward argues track intentions because distinguish people.
You need to be sensitive to the possibility that the action she performs in grasping the spoon is directed to the goal of grasping the spoon although she had no intention to grasp the spoon at all---perhaps, for example, her intention was to grasp the fork, but she got distracted and acted counter to her own intentions.
Detecting that intentions and goals have come apart requires relating communicative error signals to particular actions (‘oops’), or tracking relations between the goals of very small actions and the goals of larger sequences of actions.
So if all goal tracking requires information about mental states, it depends on a relatively rich appreciation of the structures of sequences of actions.

This is why the second issue matters: if all goal tracking requires information about mental states, the complexity of the structures infants are tracking means that it will be very difficult to understand how this is possible, even in principle.
By contrast, suppose some goal tracking can be done independently of any information about mental states.
Then not only is the task of constructing a theory of goal tracking significantly simpler.
Also, it is theoretically coherent to conjecture that infants (and adults)  would be able to first track the goals of very small actions and learn about the structure of larger sequences of actions so as eventually to use information in  ascribing intentions to \glspl{agent}.
Indeed, in this case goal tracking could be a foundation for the ascription of mental states and the identification of expressions of emotion, for social interactions and for communication.

So can goal tracking in infants (or adults) occur independently of tracking intentions and other mental states?
Let us say that \emph{\gls{pure goal tracking}} is goal tracking which does not involve ascribing intentions or any other mental states.
Woodward and others who assert that infants understand intentions may do so because they assume that ascribing intentions is an essential part of goal tracking, which would imply that pure goal tracking is impossible even in theory.
But is pure goal tracking theoretically impossible?
And if not, does it actually occur in infants or in adults?

\section{The Teleological Stance}
\label{sec:teleological-stance}
The fact that infants can track the goals of actions from three months of age or earlier (see \cref{sec:infants-track-goals}) motivates us to ask a question. How do they do this? 
The previous section argued for the value of considering the possibility of goal tracking that is independent of any information about mental states, that is, of \gls{pure goal tracking}.
Now we would like to know how in principle an infant (or anyone) might track the goal of an action.

\citet{Csibra:1998cx} have argued that pure goal tracking is possible and can be achieved by assuming what they call the ‘principle of rational action’:
%
\nopagebreak
\begin{quote}
   For an outcome to be among the goals of an action is for
   the joint displacements and bodily configurations which realise
   that action to be the best way to achieve that outcome
   that is  available to the \gls{agent}
   given current constraints on her possibilities for action.%
   \footnote{%
   Or, as they phrase it, ‘an action can be explained by a goal state if, and only if, it is seen as the most justifiable action towards that goal state that is available within the constraints of reality’
   (\citealp[p.~255]{Csibra:1998cx}; \citealp{Csibra:2003kp}).
   }
\end{quote}
%
How does this work?
To illustrate, suppose you are observing someone who is moving in the way represented in \cref{fig:teleological_stance_diy}.
You might wonder whether the goal of her action is to reach the blue house or whether it is to reach the red house.
Since her movements are not  the best available  way to reach the blue house, we can use the above principle to exclude the possibility that her goal is to reach the blue house.
In general, the strategy is this.
Start with a set of outcomes that are candidate goals and, for each outcome, ask yourself whether the observed joint displacements and bodily configurations are the best way to achieve that outcome  available to the \gls{agent} given the current constraints.
Whenever the answer is no, exclude the outcome.
Any remaining outcomes can be regarded, at least provisionally, as among the goals of the agent’s actions.



%
\begin{figure}
\begin{center}
\includegraphics[width=0.6\textwidth]{fig/teleological_stance_diy}
\caption{
	\label{fig:teleological_stance_diy}
Is the goal of these movements to reach the blue house or to reach the red house?}
\end{center}
\end{figure}
%


In applying this principle we are adopting what Csibra and Gergely call the \emph{Teleological Stance}.
\glsadd{Teleological Stance}
They do not claim that the principle is true (which is good as the principle is clearly false).
Instead their claim is that assuming the principle would enable us, in a limited but useful range of situations, to accurately track the goals of actions.
If this is correct, they have demonstrated that pure goal ascription is possible.

One limit on the Teleological Stance as so far characterised is that it could not underpin abilities to track the goals of actions that fail.
To illustrate, consider someone who, intending to chop zucchini, accidentally  slices a sliver of skin off the end of her own finger.
Of course the goal of her action was to chop the zucchini rather than her finger.
But as so far characterised, the Teleological Stance implies that the opposite was true.
For since a better way  to chop the zucchini was available to her given the current constraints, the above principle implies that chopping the zucchini could not have been a goal of her actions (which, by stipulation, it was).
Further, since no better way of taking off the end of her finger was available to her, the above principle will never enable us to exclude the hypothesis that doing this was among the goals of her action (which, by stipulation, it was not).
As this illustrates, the Teleological Stance seems to be limited to actions that succeed.%
\footnote{%
Except possibly where no means of acting successfully are available to the agent given the current constraints on her possibilities for action.
}

The existence of this limit is not in itself an objection to any claim we have yet considered.
But abilities to track the goals of actions which fail are important, and clearly present in infancy at least by nine months of age \citep{Behne:2005dw}.
Can we refine the Teleological Stance to overcome this limit?
As a first step, we might add a second principle to Csibra and Gergely’s ‘principle of rationality’, one which must also be satisfied:
%
\begin{quote}
    Outcomes which are among the goals of an action are outcomes bringing about which would typically be desirable to animals like the \gls{agent} of the action.
\end{quote}
%
This excludes taking off the end of her finger as a goal of the failed zucchini-slicer’s actions, as bringing about this outcome is not typically desirable.
But can we invoke desirability in elaborating a theory of what is supposed be pure goal tracking?
Wouldn’t doing so violate the requirement that pure goal tracking cannot involve ascribing mental states?
In fact it would not.
To see why, note that there are objective notions of desirability.
Ayesha believes herself stuck in London for the night and has no inkling that she could still catch the last train because its departure has been greatly delayed.
Although she therefore does not want or intend to go to the station, we as outside observers might comment that it would be desirable for Ayesha to go to the station.
In characterising pure goal tracking, we can safely appeal to a non-mentalistic notion of desirability such as this in adding the above principle.

Appealing to desirability is not quite sufficient to overcome the limit on tracking the goals of actions which fail.
Considering desirability will often be sufficient to exclude the actual outcome of a failed action as a goal of that action.
But  actions which fail are not usually the best available ways to achieve the outcomes which are actually their goals.
In the zucchini case, for example, moving the finger further from the blade’s trajectory would be a better way to chop the zucchini.
For this reason, it appears that the Teleological Stance as so far characterised could not underpin successful pure goal tracking when actions fail.

To overcome this limit, we do not need to add or refine the two principles already suggested.
Instead we need to consider the relation between tracker and trackee, that is, between the individual who is tracking goals and the \gls{agent} of the actions being tracked.
The trackee’s task is to select the best available way to achieve the outcomes which are in fact the goals of her action.
In effect, she is computing joint displacements and bodily configurations given goals and constraints on her possibilities for action.
The tracker’s task is to discern which outcomes the observed joint displacements and bodily configurations are the best way of achieving given the current constraints on the agent’s actions.
In effect, she is computing goals given joint displacements and bodily configurations and constraints on action.
For the purposes of goal tracking, it is not important that the tracker be especially good at identifying the best available ways to bring outcomes about.
To maximise the chances of successful goal tracking, what matters is that the tracker and trackee are as similar as possible.
When the trackee misses the zucchini, the tracker can still successfully track the goals of her actions providing she relies on similarly flawed processes to compute the best available ways of achieving outcomes.
As we will see (in \vref{sec:motor-theory-goal-tracking}), there is a way of implementing the Teleological Stance which is well suited to exploiting similarities between the trackee and the tracker.

It may be possible to further refine the Teleological Stance.
However that is done, some limits on the range of situations in which it enables successful goal tracking will surely remain.
Whatever its limits eventually turn out to be, Csibra and Gergely’s discovery of the Teleological Stance is exciting because it demonstrates the theoretical possibility of pure goal tracking.

But what might the \gls{Teleological Stance} tell us about infants’ goal tracking?
The Teleological Stance can be thought of as a principle characterising action, much as the \gls{Principles of Object Perception} characterise objects and their interactions (see \cref{sec:principles-object-perception}).
In both cases, we can distinguish three questions (see \cref{table:levels-claims-about-the-teleologica-stance}).
First, this section has been argued that the Teleological Stance is \gls{formally adequate}: it explains the possibility in principle of pure goal tracking, or could be made to do so with further refinements.
The next questions we should consider are whether the Teleological Stance is \gls{descriptively adequate} and, if it is, whether it is also \gls{explanatorily adequate}.
Given how things went in \cref{part:physical-objects}, you might be anticipating that our focus will be on the question of explanatory adequacy.
This is correct, although in the next section we will see that the question of descriptive adequacy is not entirely straightforward.


\begin{table}

	\begin{center}
	\footnotesize	%shrink for better spacing
	
	% add space between rows
	\extrarowsep=7pt
	
	\begin{tabu} to 0.8\linewidth {X[1,l] X[3,l]}
	
	\toprule
	
	Formal Adequacy & If someone took the Teleological Stance to be true, was omniscient about joint displacements and bodily configurations, and had unlimited cognitive resources, to what extent would she be able to track the goals of actions?
	\\
	Descriptive Adequacy & Does the Teleological Stance enable us to generate correct predictions about infants’  and others’ abilities to track the goals of actions?
	\\
	Explanatory Adequacy &  Is there a link between the Teleological Stance and infants’ or others’ minds, and does this link partly explain how it is they are able to track the goals of actions?
	
	%Do the principles play a role in characterising processes, representations or systems underlying infants’ and others’ abilities to segment physical objects, represent them as persisting and track their causal interactions?
	\\
	%
	\bottomrule
	%
	\end{tabu}
	\caption{Three questions about the Teleological Stance.}
	\label{table:levels-claims-about-the-teleologica-stance}
	\end{center}	%careful -- position of this affects distance between table and caption(!)
	
	
	\end{table}
	
	\normalsize


% 2012-12-23 = p. 166 (half)

% --- How could infants’ be tracking goals? The Teleological Stance is the only candidate explanation. (Statistical regularities cannot provide a full explanation, although they may explain part of the phenomena.)
%
% --- How is the teleological stance realised in terms of algorithms and representations?  Csibra & Gergely: by reasoning.  Is there an alternative? Consider adults \ldots\









\section{Statistical Regularities}
\label{sec:statistical-regularlities}

Is the \gls{Teleological Stance} \gls{descriptively adequate}?
\citet{Gergely:1995sq} shows that there are cases in which infants’ goal tracking does indeed conform to the Teleological Stance (see \cref{sec:infants-track-goals}), and there are many further studies extending this one (see \citealp{Csibra:2003kp} for a review).
But, confusingly, there are also cases which indicate that the Teleological Stance is not descriptively adequate.

To see why, step back and think about a burglar watching your house for an opportunity to break in and steal your things.
In predicting your actions, 
she is likely to be interested not only in the goals of your actions but also in your routines.
She may know that you habitually leave home at 11:30 without knowing what your further goals in doing so are.
As this illustrates, statistical regularities can be useful in predicting actions even without any deep insight into the goals of those actions.


We might expect that infants, like burglars, will sometimes make use of statistical regularities in their attempts to understand and predict actions.
Indeed, there is evidence that they do.
\citet{paulus:2011_role} showed nine-month-olds and adults a sequence involving a cow who comes to a fork in the road.
The cow takes the upper road at the fork.
This is the longer route to its destination but also apparently the best choice, as the lower road is blocked
(see \vref{fig:paulus_2011_fig1bi}).
Infants watched this sequence repeatedly until habituated to it, and adults saw it eight times.
They were then shown a new scene in which the lower, shorter road is no longer blocked (see \vref{fig:paulus_2011_fig1ci}).
\citeauthor{paulus:2011_role} wanted to know which road the infants and adults thought the cow would take.
If their predictions were based only on information about the goal of the cow’s actions, they might well anticipate that the cow would take the newly open shorter road.
But if their predictions were based on statistical regularities, then they should expect the cow to take the longer road since  it had repeatedly done so in the past.
And this is what both infants and adults in fact expected on first seeing the new sequence, indicating that their predictions were indeed based on statistical regularities (for further evidence, see \citealp{gredeback:2010_infantsa,green:2016_culture}).

%
\begin{figure}
\begin{center}
\includegraphics[width=0.6\textwidth]{fig/paulus_2011_fig1bi}
\caption{
	\label{fig:paulus_2011_fig1bi}
The cow crosses the scene using the top path and the bottom path is broken.
Source: \citet[figure 1B (part)]{paulus:2011_role}}
\end{center}
\end{figure}
%

%
\begin{figure}
\begin{center}
\includegraphics[width=0.6\textwidth]{fig/paulus_2011_fig1ci}
\caption{
	\label{fig:paulus_2011_fig1ci}
The cow has entered the scene from the left and is now behind the oval occluder.
Both the longer and the shorter path are available.
Where do subjects anticipate the cow will emerge?
Black rectangles show regions of interest for anticipatory looking.
Source: \citet[figure 1C (part)]{paulus:2011_role}}
\end{center}
\end{figure}
%

But how did \citeauthor{paulus:2011_role} measure expectations?
They used anticipatory looking.
The fork in the cow’s road was covered by an oval occluder (as shown in \vref{fig:paulus_2011_fig1ci}).
We know that when a moving object disappears behind an occluder, infants and adults alike will often proactively gaze to the place where they expect it to reappear.
Accordingly we can detect which road an infant or adult watching the sequence expects the cow to take by measuring where they are looking in anticipation of the cow’s emergence from the oval occluder.
\citeauthor{paulus:2011_role} observed that nearly all infants and adults who looked in anticipation of the cow’s emergence from the oval occluder looked towards the upper road that it had always previously travelled along.
This suggests that nine-month-olds sometimes use statistical regularities in anticipating actions.

Are these findings evidence that the \gls{Teleological Stance} is not \gls{descriptively adequate}?
\citeauthor{paulus:2011_role} appear to draw this conclusion.
They note that the nine-month-olds, but not the adults, continue to anticipate that the cow will take the longer path even on seeing the new sequence for the fourth time.
This, they suggest, contradicts the view that nine-month-olds track the goals  of purposive actions in accordance with the Teleological Stance.%
\footnote{%
See \citet[p.~981]{paulus:2011_role}: ‘our results provide evidence that infants do not yet predict actions of others based on the principle of rational action but rather rely on frequency information in forming action predictions.’
}

But drawing this conclusion is not obviously quite correct.
To see why, recall the burglar watching your house for an opportunity to break in.
She has partial information about your goals and partial information about regularities in your behaviour.
When the two kinds of information point in different directions, there isn’t obviously anything wrong in prioritizing information about regularities over information about goals.
Nor would her doing so reveal that she cannot track goals using the Teleological Stance.
Similarly, findings that infants, like burglars, will sometimes make use of statistical regularities in their attempts to understand and predict actions are not evidence that they lack a goal-tracking ability for which the \gls{Teleological Stance} is descriptively adequate.

In fact, using statistical regularities as the burglar does depends on goal tracking.
To see this, note that the regularities in question do not concern patterns of joint displacements and bodily configurations but rather goal-directed actions.
In leaving the house at 11:30, one day you wearing heels and the next day you are limping along in flats (following an unfortunate accident).
The joint displacements and bodily configurations are markedly different: but what the burglar cares about is a regularity with respect to a goal, namely the goal of leaving the house.
Similarly, Woodward’s experiment with the grasping hand (\citet{Woodward:1998dm} from \cref{sec:infants-track-goals}) tacitly relies on the assumption that infants use statistical regularities.
To generate the prediction that infants will find an action with a new goal more novel than an action with the same goal,
it is not enough to suppose that infants can track the goals of actions:
you must also suppose, further, that they can detect whether people do in future what they did in the past.
But the regularity in question is specified in terms of a goal: grasping (or touching, or reaching).
So here, again, exploiting statistical regularities is not an alternative to goal tracking but something that builds on it.


% So the fact that infants sometimes rely on statistical regularities does not show that they never use information about goals to predict actions.

So \citeauthor{paulus:2011_role}’s findings should not convince us that the Teleological Stance is not \gls{descriptively adequate}.
But they do create an interesting problem.
We cannot conclude that the Teleological Stance is \gls{descriptively adequate} on the basis of those experiments in which infants’ responses appear to fit this conclusion and also conclude that infants can make use of statistical regularities in goal tracking on the basis of other experiments.
After all, this way of sorting the experiments all but guarantees that, whatever infants are actually doing, we will reach these conclusions.
To avoid these conclusions being practically irrefutable, we need a richer theoretical framework,
one that will enable us to understand why infants’ sometimes respond on the basis of statistical regularities (as in \citealp{paulus:2011_role}) and why they sometimes respond in line with the \gls{Teleological Stance} (as in \citealp{Gergely:1995sq}).


\section{A Methodological Explanation?}
We have just encountered a first puzzle about goal tracking in development (in \cref{sec:statistical-regularlities}).
Why do infants sometimes respond on the basis of statistical regularities and sometimes respond in line with the \gls{Teleological Stance}?


\citet{daum:2012_actions} took a step towards addressing this question by adapting Woodward’s paradigm with the hand reaching for and grasping a teddy or a ball (see \cref{sec:infants-track-goals} and \vref{fig:woodward_2001_fig1}).
The changes meant that \citeauthor{daum:2012_actions}
 could measure both anticipatory looking and \gls{dishabituation} in a single individual observing a single scenario.
Replicating Woodward’s findings (see \vref{sec:infants-track-goals}), patterns of dishabituation in nine-month-olds indicated that they use information about goals in forming expectations about actions.
However the nine-month-olds’ anticipatory looking indicated that they were relying more on statistical regularities in forming expectations about actions.
Relatedly,  \citet{gredeback:2010_infantsa} report a dissociation between failure to provide evidence of goal tracking in anticipatory looking but success in doing so when the measure was pupil dilation.
(Pupil dilation indicates arousal, and so hints that an infant finds something unusual).
Why did patterns of dishabituation and pupil dilation indicate goal-based predictions whereas anticipatory looking indicated regularity-based predictions?

One possibility considered by \citet{daum:2012_actions} is that anticipatory looking requires rapid computation of the goal and its consequences for movement.
It may be that nine-month-olds simply cannot compute the goal in the few hundred milliseconds available for anticipatory looking.
This would fit with \citeauthor{daum:2012_actions}’s finding that between the ages of nine months and three years, infants’ anticipatory looking gradually becomes more adult-like in showing increasingly strong evidence of goal-based action predictions.
It would also explain why infants in the first year of life rely on statistical information in anticipating the cow’s behaviour in \citet{paulus:2011_role}, which measured anticipatory looking, but respond in line with the \gls{Teleological Stance} in \citet{Gergely:1995sq} which used habituation: as in  \citeauthor{daum:2012_actions}’s study,  infants in \citeauthor{paulus:2011_role}’s study may simply not have had enough time to incorporate information about goals into action predictions.

But things are not so straightforward.
Contradicting the view that  nine-month-olds simply cannot compute the goal in the few hundred milliseconds available for anticipatory looking,
some researchers have argued that nine- and even six-month-olds can show evidence of goal-based action predictions in anticipatory looking \citep[for example,][]{kochukhova:2010_preverbal,green:2016_culture,cannon:2012_infants}.%
\footnote{%
At this point, however, it might be objected that the anticipatory looking in question is entirely driven by statistical regularities in the ways objects are used and never by information about the goals of actions (compare \citealp{hunnius:2010_early}).
Later we will encounter an argument against this objection and a defence of the view that anticipatory looking in nine-month-olds can indeed be driven by information about goals (see \vref{sec:motor-theory-goal-tracking}).
}
If this is right, methodological considerations cannot, or cannot fully, explain why infants in the first year of life rely on statistical information in anticipating the cow’s behaviour in \citet{paulus:2011_role}.

An alternative explanation is needed.
 It seems that we do not yet have a rich enough theoretical framework to fully make sense of how infants’ use of different kinds of information in anticipating actions differs from adults’.%
 \footnote{%
Recent attempts to address this issue (aside from \cref{sec:perceptual-animacy} below) include
 \citet{uithol:2014_what} and \citet{gredeback:2015_microstructure}.
 }
The first puzzle about goal tracking remains a puzzle.

 % tempting idea: dual process; reasoning is involved in the Cow and Fish (Daum et al) cases, where agency cues are quite limited; reasoning takes time, especially in younger children, so goal-based predictions are observed for anticipatory looking only from around three years of age (as in Daum et al, 2012). But where actual bodies in action are shown, there is motor representation and so goal-based anticipation of action.


\section{A Second Puzzle: Acting and Tracking}
\label{sec:tracking-is-acting-in-infancy}

Infants’ abilities to track the goals of others’ actions appear to be related to their own abilities to act, at least in the first nine months of life.

How do we know?
Part of the evidence for this claim comes from studies in which infants’ abilities to act are enhanced.
Three-month-olds cannot typically grasp objects or manipulate them with their hands.
\citet{needham:2002_pickmeup} put ‘sticky mittens’ on three-month-old infants and let them play with blocks and other toys.
By wearing the mittens, these infants could pick up and manipulate the toys in ways normally impossible for three-month-olds (see \vref{fig:needham_sticky_mittens}).
The next step was to ask whether extending  infants’ capacities to act in this way might also enhance their abilities  to track the goals of observed actions \citep{sommerville:2005_action}.
To measure goal-tracking abilities, these researchers used Woodward’s test with the hand reaching for and grasping a teddy or a ball (see \cref{sec:infants-track-goals} and \vref{fig:woodward_2001_fig1}).
One group of infants first played with objects while wearing the ‘sticky mittens’ and then took part in the goal tracking test, whereas another group were tested on goal tracking  before getting to play while wearing the mittens.
\citeauthor{sommerville:2005_action} reasoned that since playing while wearing the mittens enhances action abilities, infants who played first should be better at goal tracking.
And this is just what they found.
Three-month-olds who had not yet played while wearing the ‘sticky mittens’ gave no sign that they could track the goals of observed actions, whereas those who played first did provide evidence of goal tracking.
% helpful review:  \citep{woodward:2009_infants}

%
\begin{figure}
\begin{center}
\includegraphics[width=0.6\textwidth]{fig/needham_sticky_mittens}
\caption{
\label{fig:needham_sticky_mittens}
How to extend a three-month-olds’ possibilities for action.
Wearing ‘sticky mittens’ enables her to grasp an object covered in velcro strips \citep{needham:2002_pickmeup}.
Source: https://news.vanderbilt.edu/files/sticky-mittens.jpg
}
\end{center}
\end{figure}
%

Even at three months of age, some or all of infants’ abilities to track the goals of actions they observe are linked to their abilities to perform actions.

Or are they?
A potential objection to \citeauthor{sommerville:2005_action}’s study is that the infants who played while wearing ‘sticky mittens’ first had spent longer observing actions by the time they took part in the goal tracking experiment than the infants who did the goal tracking experiment first.
It is conceivable that the difference in their goal tracking is not due to differences in the infants’ abilities to act but to differences in the actions they had recently observed.
To address this issue, \citet{sommerville:2008_experience} conducted a further study.
In this study, ten-month-olds were introduced to a novel tool.
Infants in one group were given chance to use the tool themselves,  thereby perhaps extending their action possibilities \citep[compare][]{costantini:2011_tool}.
 Meanwhile, infants in another group were merely allowed to observe the tool being used without holding it themselves.
 Both groups of infants were then given a goal tracking test similar to Woodward’s test (see \cref{sec:infants-track-goals} and \vref{fig:woodward_2001_fig1}) but in which objects were not grasped by hand but with the novel tool.
 Would the infants track the goals of actions involving the tool?
As the \gls{Motor Theory of Goal Tracking} predicts,
only infants who had learnt to use the tool themselves did so.

Further support for a relation between infants’ goal-tracking abilities and their abilities to act is provided by studies of anticipatory looking in infancy.
When observing a hand that is approaching some objects and about to grasp one of them, infants will, like adults,  often look to the target of the action in advance on the hand arriving there \citep{Falck-Ytter:2006dg}.
This proactive gaze demonstrates rapid goal tracking.
Critically, though, infants only do this when they themselves can perform reaching actions.
The eyes of infants not yet able to reach do not arrive on an object to be grasped in advance of the hand grasping it
 \citep{kanakogi:2011_developmental}.

 Even more strikingly,
consider what happens when a hand is approaching two objects of different sizes, as in \vref{fig:ambrosini_2013_fig1ii}.
Because a reaching hand will be shaped differently depending on whether a large or a small object is about to be grasped, it is possible in principle to detect which object will be grasped.
We know that adults can do this because they proactively gaze to the target of action in advance of the hand reaching it (see \cref{sec:motor-theory-goal-tracking}).
What about infants?
Their abilities to grasp objects develops slowly.
At some point they start to grasp with the whole hand, and then gradually learn to grasp with fewer and fewer fingers.
\citet{ambrosini:2013_looking} studied infants with different levels of  ability to grasp.
They showed that infants’ proactive gaze closely matched their grasping abilities.
How well infants could grasp with a whole hand was related to how far in advance (or not) their eyes moved to the target of an observed whole-hand grasping action.
And infants’ abilities to grasp  objects precisely were likewise related to their proactive gazes to the targets of observed actions involving a precision grip.

Overall, much evidence links  infants’ abilities to perform certain actions to their abilities to track the goals of those actions \citep[see][pp.~593--5 for a review]{gredeback:2015_eye}.
Of course, the two do not correspond perfectly.
There are cases where researchers could find no link.%
\footnote{%
\citet{melzer:2012_production} argue that, for contralateral grasping, action performance and goal tracking are unrelated at six months of age and only become correlated later in the first year of life.
Note also that anticipatory looking to the targets of action is not a pure indicator of goal tracking but is clearly also driven by statistical regularities  (as \citet{eshuis:2009_predictivea,green:2016_culture} among others argue; see also \cref{sec:statistical-regularlities}).
}
Further,
there is evidence that three-month-olds can track the goals of reaching actions although they are not capable of reaching (but only of pre-reaching), and that they can do so even for actions like reaching over a high barrier which they could not do at all \citep{skerry:2013_firstperson}.
And six-month-olds can track the goals of  actions (specifically, phonetic gestures) they are wholly unable to perform \citep{bruderer:2015_sensorimotor}.
So we cannot say that infants’ abilities to track the goals of actions are limited by their abilities to perform corresponding actions:
and yet there is clearly some link between abilities to perform actions and abilities to track the goals of those actions.

This makes things more puzzling, not less.
Why should there be any relation at all between infants’ ability to perform an action directed to a type of goal and their ability to track goals of that type?


An initially tempting idea is that performing an action provides new knowledge of means-ends relations, which in turn enhances goal-tracking abilities  \citep[compare][p.~18732]{skerry:2013_firstperson}.
Despite promising a straightforward explanation of the puzzle, there are three obstacles to accepting this idea.
We would also need to explain why infants’ goal tracking is sometimes but not always linked to their abilities to act.
Second, we would need to explain why being able to act rather than merely observing is ever necessary for insight into means-ends relations.
And, third, there is some---admittedly limited---evidence that even momentarily preventing infants from acting can impair their goal-tracking abilities \citep{bruderer:2015_sensorimotor}.
This hints that merely acquiring an ability to act may not be enough for goal tracking: what matters may be your abilities at the moment you are tracking the goal.

There is a further problem for the idea that
knowledge of means-ends relations
might explain why there should be a relation between infants’ abilities to act and to track goals.
Many researchers have contrasted scenarios involving genuine bodily actions such as a hand grasping an object (or films of these) with scenarios involving nonbodily movements such as a mechanical claw seizing an object \citep[for example, ][]{kanakogi:2011_developmental,Woodward:1998dm}.
The means-end relations involved are essentially the same across the two kinds of scenario.
And yet the researchers generally find that behaviours indicative of goal tracking occur only for the bodily actions and not for obviously nonbodily movements.%
\footnote{%
This finding is especially puzzling as we have also seen (in \cref{sec:infants-track-goals}) that infants in the first year of life do appear to provide evidence of goal tracking when confronted with scenarios involving self-propelled balls \citep{Csibra:1998cx} or  cartoon fish \citep{daum:2012_actions}.
}


We are left with a puzzle.
Exactly how is infants’ goal tracking linked to their abilities to act, and why should there be any such link?


\section{Conclusion}
Humans can track the goals of actions from around three months of age or earlier (as we saw in \cref{sec:infants-track-goals}), which is roughly when they first manifest abilities to track the behaviours of physical objects (see \cref{cha:principles-object-perception,cha:simple-view}).
How do infants do this?

In thinking about this question, there are good reasons to focus on \gls{pure goal tracking}, that is, on goal tracking which does not involve ascribing intentions or any other mental states (see \cref{sec:pure-goal-tracking}).
This is not because we know infants cannot track mental states; in fact there may be reason to guess that infants at this age can do so (as we will see in \cref{cha:mind}).
It is rather because pure goal tracking could serve as a foundation, in infants and adults  alike, for mental state tracking and social interaction.

The \gls{Teleological Stance} provides an account of how pure goal tracking is possible in theory.
Just as rough generalisations about objects can in theory enable a thinker to segment them, represent their persistence and track their causal interactions (see \cref{sec:principles-object-perception}), so also rough generalisations relating actions to goals can in theory enable a thinker to track the goals of action (as we saw in \cref{sec:teleological-stance}).
But is the Teleological Stance \gls{descriptively adequate}? Does it accurately describe infants’ goal-tracking abilities?
In attempting to answer this question we run into two puzzles.

The first puzzle arises because infants’ responses are sometimes in line with the Teleological Stance and sometimes appear to be based on statistical regularities. 
By itself this is not puzzling: there are good reasons to make use of statistical regularities in goal tracking (see \cref{sec:statistical-regularlities}).
But in order to have a refutable theory, we need to understand something about when and why infants respond in line with one or another approach.
What factors cause infants in the first nine months of life to prioritise statistical regularities, and when will they respond in line with the Teleological Stance?

The second puzzle arises because infants’ goal-tracking abilities appear to bear a complex and so far unexplained relation to their abilities to act.
Sometimes, but not always, enhancing or impairing infants’ abilities to perform an action correspondingly enhances or impairs their ability to track the goals of those actions (as we saw in \cref{sec:tracking-is-acting-in-infancy}).
Before we can conclude that the \gls{Teleological Stance} is \gls{descriptively adequate} we should understand how abilities to track goals are linked to abilities to act and why there should be any such link.

In short, we seek a theory of goal tracking in the first nine months of life that will enable us to resolve both puzzles.
This will eventually provide a basis for understanding the role of goal tracking in the developmental emergence of knowledge.






%%% Local Variables:
%%% TeX-master: "master"
%%% End:
