Intuitions about joint commitment


Abstract  In what sense is commitment essential to joint action? Attempts to answer this question have so far been hampered by clashes of intuition. Perhaps this is because the intuitions in question have been investigated using informal methods only. To explore this possibility, we adopted a more formal approach to testing intuitions, systematically sampling naive participants’ intuitions about experimentally controlled scenarios.  This approach did reveal significant patterns in participants’ responses, which may hint at potential conceptual links between commitment and joint action. It did not however support the view that commitment is essential to joint action. We conclude that intuitions may be a poor basis for philosophical analysis.


Keywords: joint action, commitment, shared intention, Margaret Gilbert, social obligations


Introduction


In recent decades,  philosophers have devoted considerable effort to investigating the phenomenon (or phenomena) of joint action.[1] Despite this, significant disagreement on the basic features of joint action remains: for instance, on whether joint action essentially involves commitment (for example, \citet{bratman:2014_book} versus \citet{gilbert:2014_book}); on whether joint action essentially involves common knowledge (for example, \citet{blomberg:2015_common} vs \citet{bratman:2014_book}); on whether joint action essentially involves a special kind of reasoning \citep{Gold:2007zd,pacherie:2013_lite}, a special kind of mental state \citep{Searle:1990em,gallotti:2011_naturalistic,gallotti:2013_social}; or a special kind of subject \citep{helm_plural_2008} as opposed to mental states with plural subjects \citep{schmid:2009_plural_bk}.  Here we focus on the issue about commitment. Our aim is to identify obstacles to adjudicating among competing theoretical positions and to explore the prospects for one way of overcoming the obstacles.


Is commitment essential to joint action?  If joint action essentially involves having an ordinary, individual intention, and if having such an intention essentially involves commitment, then there is at least one clear sense in which commitment is essential to joint action. On this there is little disagreement. But whatever kind of commitment is involved here can only be a commitment of the subject of intention to herself; or at least, it cannot be a commitment of anyone but the subject and it cannot be a commitment to anyone other than the subject.  It is therefore neither a contralateral commitment (that is, a commitment of one individual to another) nor a joint commitment (that is, a commitment of two or more individuals to those two or more individuals; \citealp{gilbert:2014_book}).  The controversial question is therefore whether  joint action essentially involves contralateral or joint commitment.


Clashing intuitions appear to inform, in part, disagreement on this question. For instance, Gilbert attempts to support her view that joint action essentially involves joint commitment with an observation:


‘If they are walking together, both Andrea herself and Heinrich will have the understandings [that] by virtue of their walking together Andrea has a right to Heinrich’s continued walking alongside her, together with the standing to issue related rebukes and demands’ (Gilbert 2013, p. 25).


Roth does not share this intuition:


‘Mightn’t one have a noncommittal attitude toward one’s walk with someone if, for example, one suspects that person might turn out to be irritable and unpleasant company?’ (Roth 2004, p. 361)


Gilbert reasons that if joint action essentially involves joint commitment, then when agents performing a joint action are doing so in virtue of one or more intentions, these intentions cannot be unilaterally rescinded. (Several routes to this view are available; see \citet{gilbert:2009shared,gilbert:2014_book}.) To illustrate, Gilbert describes a case:


‘The parties are Ned and Olive, and Olive is speaking: “Our plan was to hike to the top of the hill. We arrived at the hill and started up. As he told me later, Ned realized early on that it would be too much for him to go all the way to the top, and decided that he would only go half way. Though he no longer had any intention of hiking to the top of the hill, he had as yet said nothing about this to me, thinking it best to wait until we were at least half way up before doing so. Before then we encountered Pam, who asked me how far we intended to go. I said that our intention was to hike to the top of the hill, as indeed it was”’ (Gilbert 2013, p. 8).


Gilbert reasons that the intuitive correctness of Pam’s statement supports the view that Ned could not unilaterally rescind an intention, and that this in turn supports the view that joint action essentially involves joint commitment.


How compelling is this line of thought? Bratman’s reply is blunt:


‘As I see it, once Ned has changed his mind they no longer have a shared intention to climb to the top’ (Bratman 2014, p. 117).


How can we account for such stark clashes of intuition? Gilbert is explicit that her views, like those of many philosophers, are based on ‘informal observation including self-observation’ and her ‘own sense of the matter’ (Gilbert 2013, pp. 24, 358). It is no surprise that such methods have not reliably led philosophers with different theoretical positions (and potentially also different experiences and backgrounds) to adopt compatible views. 


In light of this, it appears that the method of relying on one’s intuitions presents two obstacles to adjudicating among competing theoretical positions. First, we as outsiders cannot replicate the procedures by which Gilbert, Bratman and others arrive at certain conclusions. And second, we cannot be confident that there are no extraneous personal factors which are serving as crucial inputs to the procedures. And yet, before writing off attempts to draw conclusions from intuitions altogether, we should give arguments from intuitions the best chance of success. To this end, we can move from informal observation to a more systematic approach, and we can consider the views of a more diverse group of people than the professional philosophers whose theories are at stake. 


In three pre-registered experiments, we set out to discover how naive subjects categorise behaviours as involving shared intentions, commitments and obligations. The pre-registered study information can be found here: https://aspredicted.org/blind.php?x=rb59nw and here: https://aspredicted.org/blind.php?x=dx3ss3










Experiment 1


In experiment 1, we presented three groups of  participants (N = 98) with the following three vignettes:


Baseline Condition: The parties are Ned and Olive, and Olive is speaking. She says that their plan was to hike to the top of the hill. We arrived at the hill and started up.  On the way, we encountered Pam, who asked me how far we intended to go. I said, ‘Our intention is to hike to the top of the hill.’


Test Condition:The parties are Ned and Olive, and Olive is speaking. She says that their plan was to hike to the top of the hill. We arrived at the hill and started up. As he told me later, Ned realized early on that it would be too much for him to go all the way to the top, and decided that he would only go half way. Though he no longer had any intention of hiking to the top of the hill, he had as yet said nothing about this to me, thinking it best to wait until we were at least half way up before doing so. As it happens, I was in the same position as Ned: I’d also decided that I would not go all the way to the top of the hill, though I hadn’t yet broached the subject with him. Before either of us got around to raising this issue, we encountered Pam, who asked me how far we intended to go. I said, ‘Our intention is to hike to the top of the hill.’


Parallel Condition:The parties are Ned and Olive, and Olive is speaking.  She says that she was considering hiking to the top of the hill. I arrived at the hill and started up. As I did so, I saw Ned ahead of me on the path also hiking towards the top of the hill. At some point along the way, I realized that it would be too much for me to go all the way to the top, and decided that I would only go half way. Along the way I ran into Ned sitting down and taking a break. Just then Pam also appeared, who asked how far we intended to go. I said, ‘Our intention is to hike to the top of the hill.’


All participants were then presented with the following three test questions:


Shared Intention Question:To what extent would you agree that Olive’s statement to Pam at the end was accurate (i.e., "Our intention is to hike to the top of the hill")?


Commitment Question: To what extent do you think that Ned and Olive have a commitment to walk to the top of the hill?


Obligation Question:To what extent do you think that Ned and Olive have an obligation to walk to the top of the hill?


Answers were given on a 5-point scale ranging from ‘Strongly disagree’ to ‘Strongly agree’.


We take Gilbert’s position to predict that, on each question, answers in the baseline and test conditions should not differ, but that the test and parallel conditions should differ; in particular, participants’ answers should be closer to ‘Strongly Agree’ in the test than in the parallel condition.[2] 


Results


For the shared intention question, we performed a three-way Anova, which revealed a significant effect of condition, F (94,2) = 19.18, p < .001, ges= .29. We then performed post-hoc pairwise comparisons using a Bonferroni correction (alpha = .017), which revealed that responses in the test condition  (M= 2.46) were significantly lower than in the baseline condition (M= 4.32), t (58) = 6.50, p< .001 and also significantly lower than in the parallel condition (M= 3.43), t(56.68)=3.15 ,  p=.003. This is diametrically opposed to the pattern which Gilbert’s position would predict. 


For the commitment question, we performed a three-way Anova, which revealed a significant effect of condition, F (94,2) = 22.30, p < .001, ges= .32. We then performed post-hoc pairwise comparisons using a Bonferroni correction (alpha = .017), which revealed that responses in the test condition (M= 2.51) were significantly lower than in the baseline condition (M= 4.28), t (57) = 7.98, p< .001, but did not differ significantly from the parallel condition (M= 3.09), t(69)=2.14 ,  p<.036, although the effect was marginal. This pattern is inconsistent with Gilbert’s analysis.


For the obligation question, we performed a three-way Anova, which did not reveal a significant effect of condition, F (94,2) = .02, p = .98. We may speculate that many participants found the term ‘obligation’ to be too strong for such a casual instance of joint action. 


Next, we carried out a battery of regressions.


First, a simple linear regression was performed to predict responses to the shared intention question based on responses to the commitment question. A significant regression equation was found, F (95,1)=47.5, p <.001, with an r-squared of .33. This is just what we should expect on the basis of Gilbert’s analysis.


Next, a simple linear regression was performed to predict responses to the shared intention question based on responses to the obligation question. No significant regression equation was found, F (95,1)=1.07, p=.304, with an r-squared of <.001. This should come as a surprise on the basis of Gilbert’s analysis.


Finally, a simple linear regression was performed to predict responses to the commitment question based on responses to the obligation question. A significant regression equation was found, F (95,1)= 5.13, p= .026, with an r-squared of .04.This is just what we should expect on the basis of Gilbert’s analysis.


Experiment 2


In experiment 2, we changed only the wording of the decisive statement in the vignette ‘Our intention is to hike to the top of the hill’, replacing it with ‘We will walk to the top of the hill.’ The reason for this was that talk about intention is relatively uncommon and may have made the task unnecessarily difficult for our participants. To illustrate this possibility, we consulted a large collection of English language corpora (https://www.english-corpora.org/iweb/).  The phrase ‘our intention’ occurs only 8306 times. In some of these cases it is not used in the way Gilbert requires (for example, ‘our intention was good’, which indicates a divergence between intention and action).  By contrast, the phrase ‘we will’ occurs roughly 220 times more frequently than ‘our intention’. Admittedly many of these uses do not express intention (for example, ‘We will see what happens with Romos real career’), but the phrase ‘we will walk to’ alone occurs 65 times in the corpora.  While not decisive, these considerations led us to suspect that the phrase ‘our intention’ may have reduced participants’ performance.  We reasoned that the more colloquial ‘we will’ phrasing would reduce any variance due to participants’ uncertainty about intention. This change does not alter the predictions which derive from Gilbert’s position.


Results


For the shared intention question, we performed a three-way Anova, which revealed a significant effect of condition, F (89,2) = 15.95, p < .001, ges= .26. We then performed post-hoc pairwise comparisons using a Bonferroni correction (alpha = .017), which revealed that the test condition (M= 2.58) differed significantly from the baseline condition (M= 4.07), t (56) = 5.03, p< .001, but not from the parallel condition (M= 2.30), t(58)=0.96 ,  p=.34. This is opposite to the pattern which Gilbert’s view predicts.


For the commitment question, we performed a three-way Anova, which revealed a significant effect of condition, F (89,2) = 15.95, p < .001, ges= .26. We then performed post-hoc pairwise comparisons using a Bonferroni correction (alpha = .017), which revealed that responses in the test condition (M= 2.32) differed significantly from the baseline condition (M= 3.82), t (56) = 5.88, p< .001, but did not differ significantly from the parallel condition (M= 2.94), t(61)=2.35,  p=.022, although the effect was marginal. Again, this is opposite to the pattern which Gilbert’s view predicts.


For the obligation question, we performed a three-way Anova, which did not reveal a significant effect of condition, F (89,2) = 1.52, p = .22. As noted above, we may speculate that many participants found the term ‘obligation’ to be too strong for such a casual instance of joint action.


Next, we carried out a battery of regressions.


First, a simple linear regression was performed to predict responses to the shared intention question based on responses to the commitment question. A significant regression equation was found, F (90,1)=22.32, p <.001, with an r-squared of .19. This is just what we should expect on the basis of Gilbert’s analysis.


Next, a simple linear regression was performed to predict responses to the shared intention question based on responses to the obligation question. No significant regression equation was found, F (90,1)=9.51, p = .003, with an r-squared of <.09. This should come as a surprise on the basis of Gilbert’s analysis.


Finally, a simple linear regression was performed to predict responses to the commitment question based on responses to the obligation question. A significant regression equation was found, F (90,1)=17.9, p < .001, with an r-squared of .16. This is just what we should expect on the basis of Gilbert’s analysis.


Experiment 3


In order to probe the robustness of our findings, we conducted experiment 3 with new vignettes describing a different scenario (adapted from Gilbert, 2014). The vignettes were as follows: 


Baseline Condition: Roz and Dan have decided to play a 5 set tennis match on Tuesday morning. During the second set, their friend Phil arrives and approaches the court to greet them. Roz tells him, ‘We are playing a 5 set match.’


Test Condition: Roz and Dan have decided to play a 5-set tennis match on Tuesday morning. Midway through the second set, Roz decides that she has had enough tennis and is going to stop after the second set. As it happens, Dan has the very same thought, but neither of them says anything just yet. During the second set, their friend Phil arrives and approaches the court to greet them. Roz tells him, ‘We are playing a 5-set match.’


In view of the higher degree of interdependence in this scenario, we elected not to include a parallel condition.


We take Gilbert’s position to predict that the baseline and test conditions should not differ. Any evidence that they do differ would therefore present a challenge to her theory.


Results


We first performed planned pairwise comparisons for the shared intention question, which revealed that the test condition  (M= 3.47) did not differ significantly from the baseline condition (M= 3.88), t (52) = 1.30, p = .199. This is consistent with Gilbert’s view.


We then performed planned pairwise comparisons for the commitment question, which revealed that responses in the test condition (M= 2.83) were significantly lower than in the baseline condition (M= 3.70), t (60) = 2.99, p = .004. Gilbert’s view provides no reason to expect this.


We then performed planned pairwise comparisons for the obligation question, which revealed that the test condition  (M= 2.81) did not differ significantly from the baseline condition (M= 2.70), t (57) = 0.29, p= .767. As noted above, we may speculate that many participants found the term ‘obligation’ to be too strong for such a casual instance of joint action.


Next, we carried out a battery of regressions.


First, a simple linear regression was performed to predict responses to the shared intention question based on responses to the commitment question. No significant regression equation was found, F (61,1)=2.57, p = .114, with an r-squared of .024. This should come as a surprise on the basis of Gilbert’s analysis.


Next, a simple linear regression was performed to predict responses to the shared intention question based on responses to the obligation question. No significant regression equation was found, F (61,1)=0.39, p = .536, with an r-squared of .001.
This should come as a surprise on the basis of Gilbert’s analysis.


Finally, a simple linear regression was performed to predict responses to the commitment question based on responses to the obligation question. A significant regression equation was found, F (61,1)=10.49, p = .002, with an r-squared of .13
This is just what we should expect on the basis of Gilbert’s analysis.














Conclusions


Attempts to establish which, if any, forms of commitment are essential to, rather than merely commonly associated with, joint action have so far been hampered by clashes of intuition. This may be due in part to the way in which intuitions have been investigated---namely through informal, unrepeatable observations. To improve the chance that progress in adjudicating theories can be made by reflection on intuitions, we adopted the more systematic approach of sampling theoretically neutral, naive participants’ intuitions about experimentally controlled scenarios.  Indeed, this approach did reveal significant patterns in participants’ responses.  To some extent these patterns were consistent with the predictions of Gilbert’s view about the role of joint commitment in joint action. In line with her view, we found (in Experiments 1 and 2, albeit not 3) that participants’ judgements about obligation predicted their judgements about commitment, which in turn predicted their judgements about expressions of intention. 
However, other key predictions we derived from Gilbert’s view were unsupported and, in some cases, even falsified.  In particular, for the three questions in Experiments 1 and 2, Gilbert’s view provides a reason to predict a difference between test and parallel conditions. In the one case in which we found a significant difference, the difference was opposite to the prediction generated by Gilbert’s view.  Further, for the three questions in all experiments, Gilbert’s view provides a reason to predict no difference between baseline and test conditions---and yet we did observe significant differences for the shared intention question in Experiments 1 and 2, and for the commitment question in all three experiments. Overall these results indicate that a systematic approach to sampling naive participants’ intuitions about examples does not provide evidence for Gilbert’s view on commitment and joint action.








Should we therefore reject Gilbert’s view? We believe caution is needed. Although our findings indicate that the method of consulting intuitions does not provide evidence for her position, there may be other ways to support it. Further, consulting intuitions may equally fail to provide evidence for competing views. Indeed, our complex pattern of findings builds upon other recent research (e.g. \citet{starmans:2012_folk,starmans:2013_taking}) suggesting that intuitions may be a surprisingly poor basis for philosophical analysis.





A variety of labels have been used for what we are calling ‘joint action’.  These  include 
‘joint action’ \citep{brooks_joint_1981, Sebanz:2006yq, Knoblich:2010fk,Tollefsen:2005vh,pettit:2006_joint,Carpenter:2009wq,Pacherie:2010fk, brownell:2011_early,sacheli:2018_evidence,meyer:2016_planning},
‘social action’ \citep{tuomela:1985_weintentions},
‘collective action’ \citep{Searle:1990em, Gilbert:2010fk},
‘joint activity’ \citep{baier:1997_joint},
‘acting together’ \citep{tuomela:2000_cooperation}, % ‘Acting together involves sociality in the relatively strong sense that such action must be based on joint intention or shared collective goal.’
‘shared intentional activity’ \citep{Bratman:1999fr},
‘plural action’ \citep{Schmid:2008},
an exercise of ‘joint agency’ \citep{pacherie:2013_lite} or of 
‘small scale shared agency’ \citep{bratman:2014_book}, 
‘intentional joint action’ \citep{blomberg:2015_common},
‘collective intentional behavior’ (which is an exercise of ‘plural agency’) \citep{ludwig:2016_individual}, and 
‘collective activity’ \citep{longworth:2019_sharing}.
We leave open whether these are all labels for a single thing or whether different researchers are targeting different things. 
As we use ‘joint action’, the term applies to everything any of these labels applies to.


We note that the views of opponents of Gilbert, such as Bratman, do not generate relevant predictions which are distinct from these.  This is because they may allow that commitments are commonly associated with joint action even though not essential (see \citet[pp.~110--1]{bratman:2014_book} for discussion).