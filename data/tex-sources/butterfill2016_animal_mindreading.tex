%!TEX root = master.tex
% !TEX encoding = UTF-8 Unicode

\maketitle

\section{Introduction}
Few things matter more than the mental states of those nearby.
Their ignorance defines limits on cooperation and presents opportunities to exploit in competition.
(If she’s seen where you stashed those mealworms she’ll pilfer them when you’re gone, leaving you without breakfast.
And you won’t get that grape if he hears you sneaking past.)
What others feel, see and know can also provide information about events otherwise beyond your ken.
It’s no surprise, then, that abilities to track others’ mental states are widespread.
Many animals including scrub jays \citep{Clayton:2007fh},
ravens \citep{bugnyar:2016_ravens},
goats \citep{kaminski:2006_goats},
dogs \citep{kaminski:2009_domestic},
lemurs \citep{sandel:2011_evidence},
monkeys \citep{hattori:2009_tufted}
% (***Kuroshima et al. 2002, 2003)
 and chimpanzees \citep{karg:2015_chimpanzees} reliably vary their actions in ways that are appropriate given facts about another’s mental states.
What underpins such abilities to track others’ mental states?

There is a quite widely accepted answer.
As in humans, so in other animals: abilities to track others’ mental states are underpinned by representations of those mental states.
Some people seem less confident about lemurs or monkeys than chimpanzees, perhaps in part because these animals’ abilities to track others’ mental states appear less flexible \citep[e.g.][]{burkart:2007_understanding}.
Others caution that there is currently insufficient evidence to accept that any nonhuman animals ever represent others’ mental states \citep[e.g.][]{whiten:2013_humans}.
But overall the view that abilities to track others’ mental states are underpinned by representations of those mental states is endorsed by many of those cited above for at least some nonhuman animals.

The simple answer will appear inescapable if we assume that tracking others’ mental states must, as a matter of logic, involve representing others’ mental states.
But this assumption is incorrect.
Contrast {representing} a mental state with {tracking} one.
For you to \emph{track} someone’s mental state (such as a belief that there is food behind that rock) is for there to be a process in you which nonaccidentally  depends in some way on whether she has that mental state.
Representing mental states is one way, but not the only way, of tracking them.
In principle it is possible to track mental states without representing them.
For example, it is possible, within limits, to track what another visually represents by representing her line of sight only.
More sophisticated illustrations of how you could in principle track mental states without representing them abound \citep[e.g.][pp.~571ff]{buckner:2014_semantic}.
What many
experiments actually measure is whether certain subjects can track mental states:
the question is whether changes in what another sees, believes or desires are reflected in subjects’ choices of route, caching behaviours, or anticipatory looking (say).
It is surely possible to infer what is represented by observing what is tracked.
But such inferences are never merely logical.
% The claim that such inferences are never merely logical depends on an assumption about the nature of representation.
% Since you can track others’ mental states arbitrarily well without actually representing them (as does the behaviourist counterpart of Laplace’s demon introduced below), any adequate theory of representation must allow that tracking does not entail representing.

To learn what underpins abilities to track others’ mental states, we would therefore need to evaluate competing hypotheses.
In recognising this, we immediately face two requirements.
The first requirement is a theoretically coherent, empirically motivated and readily testable hypothesis on which tracking mental states does not involve representing mental states.
This requirement is currently unmet (\citealp[p.~486]{halina:2015_there};
\citealp[p.~322]{heyes:2014_animal})
% \citep[p.~486{halina:2015_there}: ‘Mindreading skeptics have yet to provide a positive account concerning what behavior-reading skills we should expect particular nonhuman animals to possess independently of their performance on mindreading tasks. Instead, the behavior-reading skills proposed are all and only those that can account for the current results of mindreading research.’
%\citep[p.~322]{heyes:2014_animal}: ‘because behavioural strategies are so unconstrained ...it is very difficult indeed, perhaps impossible, to design experiments that could show that animals are mindreading rather than behaviour reading.’
 and, as the next section argues, surprisingly difficult to meet.

\section{Pure Behaviour Reading: Cast the Demon Out}
\label{sec:track}
Pure behaviour reading is the process of tracking others’ behaviours, including their future behaviours, independently of any knowledge, or beliefs about, their mental states.
Can research on pure behaviour reading supply hypotheses on which tracking mental states does not involve representing mental states?

Contrast two approaches to theorising about behaviour reading.
One focusses on the behaviourist counterpart of Laplace’s demon.
The behaviour reading demon has unlimited cognitive capacities, perfect knowledge of history and can conceptualise behaviours in any way imaginable.
Although blind to mental states, it can predict others’ future behaviours at least as well as any mindreader (\citealp[p.~528]{andrews:2005_chimpanzee}; \citealp[p.~483ff]{halina:2015_there}).
Invoking the behaviour reading demon makes vivid the point that the existence of abilities to track others’ mental states does not logically entail representations of mental states.
But the behaviour reading demon is little  use when it comes to generating testable hypotheses.
Not even the most exacting rigour requires excluding the possibility that an animal is a behaviour reading demon before accepting that it can represent mental states.

The other approach to theorising about behaviour reading concerns actual animals rather than imaginary demons.
\citet{Byrne:2003wx} studied a particularly sophisticated case of behaviour reading in Rwandan mountain gorillas.
%
\begin{figure}
\begin{center}
\includegraphics[width=0.6\textwidth]{img/byrne_2003_fig1.png}
\caption{
  \label{fig:byrne_2003_fig1}
\emph{Read this!}  An analysis of the steps performed by the left and right hands in preparing nettles to eat without getting stung.
Source: \citet{Byrne:2003wx}, figure 1.
}
\end{center}
\end{figure}
%
The procedure for preparing a nettle to eat while avoiding contact with its stings is shown in \vref{fig:byrne_2003_fig1}.
It involves multiple steps. Some steps may be repeated varying numbers of times, and not all steps occur in every case.
The fact that gorillas can learn this and other procedures for acquiring and preparing food by observing others’ behaviour suggests that they have sophisticated behaviour reading abilities \citep[p.~513]{Byrne:2003wx}.
If we understood these behaviour reading abilities and their limits, we might be better able to understand their abilities to track mental states too.

We seek an account of pure behaviour reading to generate testable hypotheses about tracking mental states without representing them.
This will involve at least three components: segmentation, categorisation and structure extraction.

First, it is necessary to segment continuous streams of bodily movements into units of interest.
Humans can readily impose boundaries on continuous sequences of behaviour even as infants \citep{Baldwin:2001rs}.
How could such segmentation be achieved?
Commencement and completion of a goal or subgoal typically coincide with dramatic changes in physical features of the movements such as velocity \citep{Zacks:2001vo}.
Baldwin and Baird express this idea graphically with the notion of a ‘ballistic trajectory’ which provides an ‘envelope’ for a unit of action \citep[p.~174]{Baldwin:2001rs}.
Research using schematic animations has shown that adults can use a variety of movement features to group behavioural chunks into units
\citep{Hard:2006gr}.

A second component of behaviour reading is categorisation.
Adult humans spontaneously label units of behaviour as ‘running’, ‘grasping’, or ‘searching’ (say).
This is categorisation: two units which may involve quite different bodily configurations and joint displacements and which may occur in quite different contexts are nevertheless treated as equivalent.
How are categories identified in pure behaviour reading?
One possibility is that some categorisation processes involve mirroring motor cognition.
When a monkey or a human observes another’s action, there are often motor representations in her that would normally occur if it were her, the observer, who was performing the action (see \citealp[][]{rizzolatti_functional_2010} for a review).
Further, in preparing, performing and monitoring actions, units of action are represented motorically in ways that abstract from particular patterns of joint displacements and bodily configurations \citep[e.g.][]{koch:2010_resonance}.
These findings indicate that %there may be a deep connection between processes involved in action performance and one process of categorisation in behaviour reading.
%Perhaps
one process by which units of action are categorised is the process by which, in other contexts, your own actions are prepared.

A third component of behaviour reading is structure extraction.
Many actions can be analysed as a structure of goals hierarchically ordered by the means-ends relation
(see \vref{fig:action} for an illustration.)
%
\begin{figure}
\begin{center}
\includegraphics[width=0.6\textwidth]{img/fig_action.jpg}
\caption{
  \label{fig:action}
A routine action with a complex, hierarchical structure.
}
\end{center}
\end{figure}
%
A behaviour reader should be able to extract some or all of this structure.
But how?
Units of behaviour that are all involved in bringing about a single outcome are more likely to occur in succession than chunks not so related.
This suggests that transitional probabilities in the sequence of units could in principle be used to identify larger structures of units, much as phonemes can be grouped into words by means of tracking transitional probabilities \citep{Gomez:2000jr}.
We know that human adults can learn to group small chunks of behaviour into larger word-like units on the basis of statistical features alone \citep{Baldwin:2008mw}.
A statistical learning mechanism required for discerning such units is automatic \citep{Fiser:2001cx}, domain-general \citep{Kirkham:2002cj} and probably present in human infants \citep{saffran:2007_dog} as well as other species including
songbirds \citep{abe:2011_songbirds} and rats \citep{murphy:2008_rule}.
It is therefore plausible that some animals use statistical learning to extract some of the hierarchical structure of actions.

Our primary concern here with behaviour reading is as a potential basis for abilities to track others’ mental states without representing them.
But behaviour reading matters in other ways too.
In mindreaders, behaviour reading enables mental state ascriptions (\citealp[p.~861]{Newtson:1977dw}; \citealp[p.~708]{Baldwin:2001rn}).
Behaviour reading may also matter for efficiently representing events \citep{Kurby:2008bk}, identifing the likely effects of actions \citep{Byrne:1999jk}, predicting when an event  of interest will occur \citep[p.~121]{Swallow:2008cf},
and learning through observation how to do things \citep{Byrne:2003wx}.
And of course a special case of pure behaviour reading, ‘speech perception’, underpins communication by language in humans.

What are the limits of pure behaviour reading?
It is perhaps reasonable to assume that structure extraction depends on domain-general learning mechanisms.
After all, such mechanisms appear sufficient and there is currently little evidence for domain-specific mechanisms.
%(even in adult humans, the most abundant lab animals)
This assumption allows us to make conjectures about the limits of pure behaviour reading.
One limit concerns non-adjacent dependencies.
There is a non-adjacent dependency in my behaviour when, for example, my now having a line of sight to an object that is currently unobtainable because of a competitor’s presence results in me retrieving the object at some arbitrary later time when the competitor is absent.
In this case, my retrieving the object depends on my having had it in my line of sight, but there is an arbitrary interval between these events.
The hypothesis is that structures involving non-adjacent dependencies are relatively difficult to learn and identify, and that difficulty increases as the number of non-adjacent dependencies increases.%
\footnote{%
Compare \citet[][]{vries:2012_processing}.
Of course, whether non-adjacent dependencies are intrinsically difficult depends on the cognitive architecture \citep{udden:2012_implicit}.
There is evidence that monkeys \citep{ravignani:2013_action} and chimpanzees \citep{sonnweber:2015_non} can learn patterns involving one non-adjacent dependency.
}
More generally, since birdsongs are discriminable and involve diverse behavioural structures \citep{berwick:2011_songs}, we might take the \emph{Birdsong Limit} as a rough working hypothesis: structures not found in birdsong cannot be extracted in pure behaviour reading.
%(See \citet{berwick:2011_songs} on the structures of birdsongs.)

Although not designed to test such limits, some existing experimental designs involve features which plausibly exclude explaining subjects’ performance in terms of pure behaviour reading only.
To illustrate, consider the sequence of events in the ‘misinformed’ condition of \citet[][Experiment 1]{Hare:2001ph}.
A competitor observes food being placed [$A$], the competitor’s access is blocked [$B$], stuff happens [$X^N$], food is moved [$C$], more stuff happens [$Y^N$], and the competitor’s access is restored [$D$].
Finding evidence that chimpanzees can learn to identify patterns of this form [$ABX^NCY^ND$] and use them to predict the conspecifics’ behaviours would represent a major discovery.

% Except possibly in adult humans, pure behaviour reading may involve extracting the sorts of structure characteristic of birdsong but not the more complex sorts of structure characteristic of human language (see \citealp{berwick:2011_songs} for a detailed description of the contrast between these two).

% \begin{quote}
%  ‘Previous comparative animal research has demonstrated awareness of dependencies either occurring at a fixed distance [6,7] or between specific items [5]. Detection of abstract dependencies at arbitrary variable distances (crucially beyond one intervening element, already shown in [4,7]) has never been demonstrated before in a non-human animal (though see [8] for initial hints). The current study tested the hypothesis that a non-human primate species could detect abstract, non-adjacent dependencies in acoustic stimuli, even when dependencies occurred over an arbitrary variable number of intervening sounds.’

% ‘Squirrel monkeys consistently recognized and generalized the pattern ABnA at different levels, showing sensitivity to arbitrary-distance dependencies.’ \citep{ravignani:2013_action} see also \citep{sonnweber:2015_non}
% \end{quote}


While it is probably impossible and certainly unnecessary to exclude the possibility that an animal is a behaviour reading demon, it turns out to be quite straightforward (in theory, at least) to exclude the possibility that its actual behaviour reading abilities are what underpin its abilities to track others’ mental states.
Even in advance of knowing much about the processes and representations involved in pure behaviour reading,
the assumption that structure extraction depends on domain-general learning mechanisms
makes it unlikely that the relatively sophisticated abilities of corvids and great apes (say) to track others’ mental states could be underpinned by pure behaviour reading only.
% But where? No idea.

\section{End False Belief about False Belief}
\label{sec:end-false-belief}

In the absence of an alternative, should we  accept, provisionally, that in at least some nonhumans, tracking mental states does after all involve representing them?
There are at least two obstacles to accepting this.

The first is a false belief about false belief.
The false belief task \citep{Wimmer:1983dz} is sometimes regarded as an acid test of mental state representations (see Bennett's, Dennett's and Harman's influential responses to \citealp{premack_does_1978}).
Awkwardly, chimpanzees and other nonhuman animals have so far mostly thwarted efforts to show that they can track others’ false beliefs \citep[e.g.][]{marticorena:2011_monkeys}.
False belief tasks continue to  yield many important discoveries concerning humans  \citep[e.g.][]{devine:2014_relations,milligan:2007_language}. But there are reasons to doubt that the false belief task, despite its enormous value, is an acid test of mindreading.
First, it is possible to track others’ false beliefs without actually representing them \citep{butterfill_minimal}.
Second, there is evidence that typically developing humans can represent incompatible desires before they can represent false beliefs \citep{rakoczy:2007_desire}.
Having an ability to track false beliefs is therefore not sufficient for being able to represent beliefs
 and probably not necessary for being able to represent mental states.
So whether we accept that any nonhumans can represent others’ mental states should not hinge on whether they can track false beliefs.
 As \citet[p.~622]{premack_does_1978} suggest, a false belief task is ‘another arrow worth having in one's quiver rather than the assured bullseye that the philosophers suggest it is.’

There is a second, more challenging obstacle to accepting that some nonhumans can represent mental states.
After claiming that ‘chimpanzees understand … intentions … perception and knowledge,’ \citet{Call:2008di} qualify their claim by adding that ‘chimpanzees probably do not understand others in terms of a fully human-like belief–desire psychology’  (p.~191).
This is true.
The emergence in human development of the most sophisticated abilities to represent mental states probably depends on rich social interactions involving conversation about the mental \citep[e.g.][]{moeller:2006_relations}, on linguistic abilities \citep{milligan:2007_language},
%\citet[p.~760]{moeller:2006_relations}: ‘Our results provide support for the concept that access to conversations about the mind is important for deaf children’s ToM development, in that there was a significant relationship between maternal talk about mental states and deaf children’s performance on verbal ToM tasks.’
and on capacities to attend to, hold in mind and inhibit things \citep{devine:2014_relations}.
These are all scarce or absent in chimpanzees and other nonhumans.
So it seems unlikely that the ways humans at their most reflective represent mental states will match the ways nonhumans represent mental states.
Reflecting on how adult humans talk about mental states is no way to understand how others represent them.
But then what could enable us to understand how nonhuman animals represent mental states?

%To understand what underpins abilities to track others’ mental states we need to evaluate competing hypotheses.
The view that tracking mental states involves representing them leaves too many options open,
as Call and Tomasello’s nuanced discussion shows.
It is not a hypothesis that generates readily testable predictions.
We need a theoretically coherent, empirically motivated and readily testable hypothesis on which tracking mental states does involve representing mental states \citep[compare][p.~321]{heyes:2014_animal}.
% ‘the core theoretical problem in ... animal mindreading is that ... the conception of mindreading that dominates the field ... is too underspecified to allow effective communication among researchers’
% Heyes (2015, 321)
Identifying such a hypothesis is the second requirement we would have to meet in order to evaluate competing hypotheses about what underpins abilities to track others’ mental states.
And to meet this second requirement we must first reject a dogma.


\section{Reject the Dogma of Mindreading}
\label{sec:dogma}


% ‘the core theoretical problem in ... animal mindreading is that ... the conception of mindreading that dominates the field ... is too underspecified to allow effective communication among researchers’
% Heyes (2015, 321)


% After claiming that ‘chimpanzees understand … intentions … perception and knowledge’ (p.\ 191), \citet{Call:2008di} qualify their claim by adding that ‘chimpanzees probably do not understand others in terms of a fully human-like belief–desire psychology.’
% But they make no attempt to say, positively, what chimpanzees do understand of mental states.

% (*mention Gomez).

% \begin{quote}
% ‘we should be focused not on the yes–no question (do chimpanzees have a theory of mind?), but rather on a whole panoply of more nuanced questions concerning precisely what chimpanzees do and do not know about the psychological functioning of others’
% \citep[p.~149]{Hare:2001ph}

% \end{quote}


% % You can avoid succumbing to the first temptation by identifying abilities to track another’s mental states for there seems to be no alternative underpinning other than representations of mental states.


% % The first temptation is particularly hard to resist in experiments where it is unclear to us what other than representations of mental states might underpin an ability to track others’ mental states.
% % But to argue from the premise that we haven’t yet identified an alternative
% % The antidote may be to reflect on particularly ingenious alternatives \citep[e.g.][pp.~571ff]{buckner:2014_semantic}
% % Mindreading, the process of
% % 	identifying mental states and purposive actions
% % 	as the mental states and purposive actions of a particular subject, isn’t magic.
% % It depends on information about context and events, past and present.
% % It is always possible, in principle, to use such information to track another’s mental states other than by representing them.

Representing physical states, such as the masses or temperatures of things, requires having some model of the physical.
Little follows directly from the fact that an individual can represent weight or other physical properties: everything depends on which model of the physical underlies her capacities.
And if we ask, ‘What model of the physical characterises her thinking?’, we find that there are multiple, experimentally distinguishable candidate answers \citep[e.g.][]{kozhevnikov:2001_impetus}.
Her physical cognition might be characterised by a Newtonian model of the physical, or perhaps on an  Aristotelian model.
% Different models have different implications for limits on her ability to correctly predict events.
And it might involve one or another measurement scheme.
Perhaps, for example, she distinguishes the weights of things relative to her abilities to move them.
Or maybe she relies on a system of comparisons.
Different models of the physical and different systems of measurement generate different predictions about the limits of her abilities to track physical events.


Likewise for mental properties.
The conjecture that someone can represent mental states---that she is a mindreader, or that she has a ‘theory of mind’---does not by itself generate readily testable predictions.
%Here’s where I disagree with \citet{halina:2015_there}.
Everything depends on which model of the mental characterises her mindreading.



In asking which model of the mental—or of the physical—characterises a capacity, we are seeking to understand not how the mental or physical in fact are but how they appear from the point of view of an individual or system.
Specifying a model is a key part of providing what Marr calls a ‘computational description’ of a system \citep{Marr:1982kx}.
The model need not be  something used by the system: it is a tool the theorist uses in describing what the system is for and broadly how it works.

When an animal is suspected of mindreading we must ask, How does she model the mental?
But it will make no sense to ask this question as long as we are in the grip of a dogma.
The dogma is that
all models of the mental comprise a family in which one of the models, the best and most sophisticated model, contains everything contained in any of the models.

This dogma implies that only animals whose capacities approximate those humans exhibit in talking about mental states can be mindreaders.
In rejecting the dogma we also remove any reason to make this assumption.
Different mindreaders may rely on different, incommensurable models of the mental and different schemes for distinguishing mental states.
Mindreading in other animals need not be an approximate version of mindreading in adult humans any more than medieval physical thought approximates contemporary physical theories.

To see how strange endorsing
%these ideas are
the dogma would be,
contrast the mental with the physical.
The briefest encounter with the history of science reveals that there are several models of physical phenomena like movement, mass and temperature.
Some models are more accurate but also relatively costly to apply, while others are easier to apply but less accurate.
And there appear to be different kinds of physical cognition which  involve different—and incommensurable—models of the physical
\citep[e.g.][]{helme:2006_what}.
It would be astonishing to discover that there is one privileged model such that all physical cognition can be understood by reference to that particular model.
The dogma of mindreading tacitly guides discussion only because, by contrast with the rich array of flawed theories of the physical, there is a scarcity of scientific theories of the sort of mental states which animals can track.
But this scarcity can be alleviated.

\section{Construct Models of the Mental}
\label{part:minimal}


What is a model of the mental?
On a widely accepted view, mental states involve  subjects having  attitudes toward contents (see \vref{fig:mental_state}).
Possible attitudes include believing, wanting, intending and knowing.
(Not every model of the mental need include these attitudes.)
The content is what distinguishes one belief from all others, or one desire from all others.
The content is also what determines whether a belief is true or false, and whether a desire is satisfied or unsatisfied.


\begin{figure}
\begin{center}
  \includegraphics[width=0.7\textwidth]{img/fig4n.png}
\caption{
\label{fig:mental_state}
	Mental states involve  subjects having  attitudes toward contents.
}
\end{center}
\end{figure}

There are two main tasks in constructing a model of mental states.
The first task is to characterise some attitudes.
This typically involves specifying their distinctive functional and normative roles by developing a theory of the mental.
The second task is to find a scheme for specifying the contents of mental states.
This typically involves one or another kind of proposition.

One model of the mental is specified by minimal theory of mind \citep{butterfill_minimal,butterfill:2016_goal}, which repurposes a theory offered by \citet{Bennett:1976rg} in building on insights offered by \citet{gomez:2009_embodying}  and \citet{Whiten:1994sw} among others.
This theory—or, rather, series of nested theories—specifies states with stripped down functional roles
%(avoiding normative characteristics altogether)
whose contents are distinguished by tuples of objects and properties rather than by propositions.
These features ensure that, although minimal theory of mind is capable of underpinning abilities to track mental states including false beliefs in a limited but useful range of situations, realising minimal theory of mind need involve little conceptual sophistication or cognitive resources.


The construction of minimal theory of mind enables us to specify some simple models of the mental,
and to generate testable hypotheses about how mindreaders model minds.
One such hypothesis concerns infant humans.
The hypothesis is that a minimal theory of mind describes the model of the mental which underpins mindreading processes in these subjects.
A key prediction of this hypothesis has so far mostly been confirmed (see \citealp{low:2016_cognitive}).
A minimal model of the mental might capture how minds appear from the point of view of some mindreading processes in some humans.
%But even if not, the hypothesis is readily testable and worth testing.

%It may be worth testing
Consider
a related hypothesis about nonhuman animals:
abilities to track mental states in some nonhumans are underpinned by capacities to represent mental states which involve a minimal model of the mental.
(This hypothesis was suggested by \citealp{Apperly:2009ju}.)
This hypothesis avoids objections arising from views on which nonhumans have representational powers whose emergence in human development involves linguistic abilities and communicative exchanges.
It also generates testable predictions about the limits of mindreading in nonhumans, including predictions which distinguish hypotheses about minimal theory of mind from hypotheses about pure behaviour reading.
And there is already a hint that one such prediction is correct (see \citealp[][]{karg:2016_differing}; they don’t mention this, but a signature limit of minimal mindreading is inability  generally to do level-2 perspective taking).


Constructing models of the mental
 enables us to identify
theoretically coherent, empirically motivated and readily testable
hypotheses on which
representations of mental states underpin abilities to track them.
But of course this is just a first step towards understanding varieties  of animal mindreading, one that opens the way for further theorising about the kinds of processes that underpin mindreading.

% This section has focussed on minimal theory of mind only for want of a better alternative.
% The considerations offered do not depend on whether or not minimal theory of mind generates correct hypotheses about nonhumans’ abilities to represent mental states.


\section{Conclusion}
\label{sec:conclusion}
What underpins abilities to track others’ mental states?
To answer this question we would need to evaluate at least two competing hypotheses.
First, we would need a theoretically coherent, empirically motivated and readily testable hypothesis on which tracking mental states does not involve representing mental states.
Although no such hypothesis currently exists (\citealp[p.~486]{halina:2015_there};
\citealp[p.~322]{heyes:2014_animal}),
there is a body of research on behaviour reading from which a theory capable of generating readily testable predictions might be derived (see \cref{sec:track}).
Second, we would need a readily testable hypothesis on which representations of mental states underpin abilities to track them.
The construction of minimal theory of mind enables us to generate one such hypothesis (\cref{part:minimal}).

How plausible are these hypotheses?
Even in advance of having a theory of behaviour reading, we might assume that extracting structure in behaviour reading depends on domain-general learning mechanisms only.
Given this assumption, it seems unlikely that nonhumans' most  flexible mental-state-tracking abilities are underpinned by behaviour reading only (\cref{sec:track}).
This may motivate the search for alternative theories on which tracking others’ mental states does not involve representing them.
It may even justify accepting, provisionally at least, that some animals other than humans represent mental states.

To accept this is not yet to have a theory about mindreading capable of generating readily testable predictions, however (\cref{sec:end-false-belief}).
Understanding abilities to track others’ mental states is not simply a matter of categorising them as involving or not involving representations of mental states.
Instead we need to understand how different mindreaders model the mental.

Because different mindreaders may rely on different, incommensurable models of the mental and different schemes for distinguishing mental states,
we need to identify models of the mental that we can use to generate readily testable hypotheses about different mindreaders’ capacities (\cref{sec:dogma}).
The construction of minimal theory of mind illustrates how to do this.
%It also illustrates how identifying models of the mental enables us to generate readily testable hypotheses.


The hypothesis that
abilities to track mental states in some nonhumans including great apes and corvids are underpinned by capacities to represent mental states which involve a minimal model of the mental
has three things going for it.
It makes precise what researchers should care about in asserting that animals other than humans can represent others’ mental states.
It isn’t already known to be false, and
there is even a hint that its predictions are correct (\cref{part:minimal}).
And it has no theoretically coherent, empirically motivated and readily testable competitors—at least not yet.
So if a minimal model of the mental doesn’t characterise any nonhuman animals’ abilities to track other mental states, what does?


\section{Bibliographical Note}
\label{sec:Bibliographical Note}
Stephen Butterfill researches and teaches on joint action, mindreading and other philosophical issues in cognitive science at the University of Warwick (UK).


% As we saw in \cref{sec:track}, existing research already suggests that, at least for animals with relatively flexible abilities to track others’ mental states, these predictions are probably incorrect.


% In adult humans at least, tracking them sometimes depends on representing others’ mental states.
% But in other animals (lemurs, maybe), it is thought that tracking probably involves representing behaviours only.






% I started by observing that there is a simple answer: in humans and at least some other animals, tracking them depends on representing others’ mental states.
% But because the tracking does not logically entail representing,





%%% Local Variables:
%%% TeX-master: "master"
%%% End:
