% ---
% title: Three Strategies for Shared Intention: Plural, Aggregate and Reductive
% subtitle: 
% ---


% \begin{center}
%   status: submitted
% \end{center}


\begin{abstract}
\noindent
When deciding on a strategy for explicating shared intention, we all face 
two fundamental questions. 
First, can an intention or any other mental state
have more than one subject? 
A positive answer to this allows the plural subject
strategy: shared intention is a matter of there being one mental state with two or more subjects.
Mental states are shared in the same sense that siblings share a parent; no simpler view exists.
A negative answer blocks the plural subject strategy.
This motivates asking the second fundamental question.
Are there aggregate subjects and, if so, can they have intentions?
The aggregate strategy depends on a positive answer to this question: the idea is that 
shared intention is a matter of there being aggregate subjects of mental states, 
that is subjects of mental states with proper parts that include subjects of mental states.
By contrast, a negative answer to this question \addedRone{limits us to} the reductive strategy:
shared intention is a structure of ordinary, individual subjects’ emotions,
intentions and other mental states.
I contribute a limited review of the three strategies.
I also defend a novel thesis.
Whereas these strategies are \addedRone{often} presented as conflicting attempts to characterise a single set of phenomena, my thesis is that 
for each strategy there are phenomena which can be correctly characterised only by following that strategy.
Instead of attempting to find one true strategy,
we may need to seek ways to combine insights from different strategies.


% Previously, the three strategies have mostly been regarded as deployed in competing 
% attempts to characterise a single set of phenomena; indeed, proponents of each strategy
% generally motivate their theories by identifying objections to theories built on other strategies.


% This makes sense on the assumption that we, as researchers, have a shared understanding of the 
% mental states and other phenomena that are the targets.
% But what if this assumption is false?
% What if different researchers are targeting different phenomena?
% It may be that the right answer to the two fundamental questions depends on 
% your background understanding of the mental states and your particular explanatory aim.
% And it may be that neither the understanding nor the aim is something that we, as researchers, all share.
% At least, this is the view we aim to defend here.
\

\

\noindent
abstract: 250 words ;
main text: 8200 words ;
references: 1250 words

% Discoveries about the neuropsychological basis of ethical abilities 
% suggest that intuitions about particular moral scenarios
% involving features not frequently encountered by humans
% are philosophically unreliable.
\end{abstract}

\section{Introduction}
\label{sec:introduction}


Start with the idea that 
shared intention, whatever that is, 
is something which makes things we do together the genuinely joint activities they are.
We manifest shared intention in walking together, playing a piano duet, or painting a house together.
% (to borrow some famous examples).
Philosophers, 
in attempting to elucidate ideas about shared intention, 
have followed three distinct strategies.
One involves plural subjects, one aggregate subjects,%
\footnote{%
\addedRone{Where I use ‘aggregate’, others have used ‘collective’ \citep[p.~274]{bjornsson:2017_corporate} 
and ‘group’ \citet[p.~74]{list_pettit:2011}.
Although more familiar, I have avoided these terms because they seem to me to risk inviting confusing aggregate subjects and plural subjects.}
}
and one a reduction of the apparently plural or aggregate to the merely individual.%
\footnote{%
\label{fn:no-shared-intention}%
\addedRone{This is not an exhaustive division of strategies.} It would also be possible to reject the idea that we need a notion of shared intention at all. 
Although beyond the scope of this essay,
there are interesting discussions which may motivate considering this view in \citet{baier:1997_joint}, \citet{chant_unintentional_2007}, \citet{petersson_collectivity_2007}, and \citet[pp.~13ff]{longworth:2019_sharing}.

\addedRone{It is also not the only way of dividing strategies. \citet[§3]{schweikard:2021_collective} offer a division into ‘content-, mode-, and subject-accounts of collective intentionality.’ These cut across the division into plural, aggregate and reductive. Both the plural subject and the aggregate subject strategies yield ‘subject-accounts’ while the reductive strategy would yield content-accounts.}
}
Whereas these strategies are usually presented as conflicting attempts to characterise a single set of phenomena, my thesis is that 
for each strategy there are phenomena which can be correctly characterised only by following that strategy.
% % CUT BECAUSE DOES NOT MATCH CONCLUSION:
% we can interpret (or usefully misinterpret) proponents of the different strategies as aiming to characterise different things.%
% Perhaps different strategies are needed for different explanatory targets.

\addedRone{This thesis is not entirely novel. 
Both \citet{list_pettit:2011} and \citet{bratman:2022_shared}, for instance,
have developed views on which, roughly speaking, a reductive strategy is applied to small-scale, informal interactions 
whereas the aggregate strategy is applied to corporate or institutional agents.%
\footnote{\addedRone{%
See \citet[p.~33]{list_pettit:2011} on ‘joint intentions’ and 
\citet[pp.~135ff]{bratman:2022_shared} on ‘institutional intentions’.%
}}
My thesis is that, similarly, a combination of strategies may be needed even when restricting 
attention to small-scale interactions involving two or three people.%
}

% This thesis is offered tentatively.
% The considerations I will offer are far from decisive.
% Minimally, my hope is to establish that we do not yet have reason to 
% attempt to pick one true strategy.
% Instead we may need to seek ways to combine insights from different strategies.


What consequences would follow from this thesis if true—if, that is, 
%we can usefully interpret proponents of the different strategies as aiming to characterise different things?
each of the three strategies is needed to characterise some phenomena associated with shared intention?
% Suppose for a moment (looking ahead to the end of this whole discussion)
% that it is true that we can usefully interpret proponents of the different strategies as aiming to characterise different things. 
This would make things easier in one way: it is not necessary for proponents of one strategy to refute the theories invented by those following a different strategy.
This would be a good outcome because, despite much effort, 
few if any existing attempts to refute particular theories of shared intention are widely regarded as successful.
But it would also make things harder in another way.
For if there is not one set of things that the different theories are aiming to fully capture, 
%and if we as researchers do not currently share an understanding of those things, 
then it will not be possible to specify the target of a theory simply by reference to shared intention.
We will need a deeper understanding of what each particular theory is aiming to explain.

Superficially, the need for a deeper understanding of targets of explanation 
might not seem like much of a challenge.
After all, we already know that some theorists primarily target semantics \citep{ludwig_collective_2007},
others ‘the deep structure of our thought about acting together’ \citep[p.~2]{gilbert:2022_simple},
and yet others ‘the explanatory structures that directly underlie [...] cases of acting together’ \citep[p.~7]{bratman:2022_planning}.
% ‘an explanatory sub-structure’ of capacities for acting together \citep[p.~2]{bratman:2022_two}.
There are, then, already things written about what theories of shared intention aim to explain.
Yet little attention has been given to the possibility of co-existence.
The usual assumption is that \addedRone{the various strategies lead to theories which do contradict each other;
or else, more rarely, that multiple strategies can be combined in pursuit of a single explanatory target}.

\addedRtwo{We already know that it is coherent to pursue a combination of strategies because some have already done this.
\citet[pp.~127ff]{bratman:2014_book}, for example, considers the possibility
that his reductive strategy yields a theory on which there are also what he calls ‘group agents’
(for which I shall use the term ‘aggregate agents’; see section \ref{sec:aggregate}).
This is a relatively conservative way of combining strategies.
Not only is there a single explanatory target, but the group agents are merely epiphenomena of the reductive strategy.
The novel feature of what follows is the observation that 
different explanatory targets require different strategies,
which, as we will see, appears to create a challenge to understanding co-existence.
}


% to this assumption makes things worse, not better.
% With the theories contradicting each other, 
% we could at least hope to identify winners and losers eventually.
% If instead we suppose that these theories are compatible attempts to explain different things,
% we are left with 
% %untenable theoretical fragmentation.
% a big jumble of theories and no idea what, if anything, to do with any of them.
% This is why we need a deeper understanding of what each particular theory is aiming to explain:
% it would be a first step towards %sorting out the jumble and 
% understanding how to fit the various theoretical insights into a coherent picture of the domain.

If, as the following argues, multiple different strategies for shared intention really are needed,
then we all face a problem. 
The problem is that we, as researchers, need, but lack, 
a common understanding of what theories of shared intention are theories of.


\section{\addedRone{Living Out a Theory}}
\label{sec:live-it-out}

\addedRtwo{%
My thesis is that,
for each strategy for explicating shared intention,
there are phenomena which can be correctly characterised only by following that strategy.
The argument for this thesis (in section \ref{sec:multiple-strategies}) will hinge on the idea that
it is possible, in some cases, to make a theory true by living it out.
To avoid surprises later, this section introduces that idea.
}

\addedRtwo{%
While laws of mechanics apply no less to us agents than
to anything else, it would usually be futile, perhaps even incoherent, to attempt to move
according to these laws.
By contrast, one of the roles of mental state attribution is to provide norms
which individuals can measure themselves against and aim to live by \citep{mcgeer:2007_regulative,zawidzki:2013_mindshaping}.
}

\addedRtwo{%
To illustrate, consider the norm of agglomeration: 
it is a mistake to knowingly have several intentions if it
would be a mistake to knowingly have one large intention agglomerating the several intentions
(\citealp{Bratman:1987xw}).
Whether this is actually a mistake is controversial—several philosophers have defended views of intention which are incompatible with it
\citep[§4]{setiya:2022_intention}.
But regardless of that, it is possible some people might, however mistakenly, take 
agglomeration as an ideal by which to live.
They check their intentions against the norm and criticise each other for failing to implement it.
It is equally possible that another group of people, having considered the matter deeply, intentionally disregard the norm of agglomeration.
In their view, adhering to this norm would be a mistake.
}

\addedRtwo{%
The possibility that some ordinary agents might adopt or reject the norm of agglomeration in practice
raises a question.  
% complicates the question of whether we should regard the norm as correct.
Are ordinary agents’ views ever relevant to whether the norm is correct?
Imagine we were to say, crudely, that whether the norm holds is just a matter of whether people take it to hold.
Such a view faces myriad challenges.
One is to accommodate the fact that ordinary agents are wrong about norms, at least occasionally (as the present author can attest).
Another challenge is to avoid a regress.
This is not a line I propose to develop.
Alternatively, 
one might take a hard line and insist that ordinary agents’ views are irrelevant to whether the norm of agglomeration holds.
Taking this line is complicated by the fact that ordinary agents’ views shape at least some of their thoughts and actions. 
Their views are not idle speculations about themselves but can form ideals which they attempt to live out.
Someone who takes the hard line cannot therefore claim to be explaining how ordinary agents think or act.
Minimally, then, anyone pursuing this line would have to identify which phenomena their position is supposed to explain. 
A further challenge is that philosophers’ methods involve intuition, imagination and reasoning about consistency. 
These methods are good for identifying possible ways things could be.
But where there are multiple theoretically coherent positions on which fully-informed ordinary agents reasonably differ, these methods are not likely to yield insight into how things actually are.
}

\addedRtwo{%
Resolving the issue of how, if at all, ordinary agents’ views are relevant to the correctness of the norm of agglomeration is beyond anything I can offer here (or anywhere else).
But for our purposes, what matters is a relatively uncontroversial point.
There is a difference between,
on the one hand,
things which are merely described and predicted by a set of attributions
and, on the other hand,
agents who are attempting to live out a set of attributions together with 
some norms governing them.
You might have a view about how combinations of yeast, sugar and heat can be used to influence how dough rises,
but the dough itself has no perspective.
And even if the dough achieved self-awareness, that would matter only insofar as its self-awareness influenced variables you care about. 
When ascribing attitudes and norms to agents, by contrast, philosophers are not required to 
adopt the outsiders’ perspective—they can also take into account the agents’ own perspective.
}

\addedRtwo{%
Characterising intentions and norms from some agents’ own perspective is a familiar and coherent philosophical project.
Where some agents are attempting to live out a theory, it is reasonable to accept,
in the absence of overriding reasons such as incoherence, ignorance or inertness,
that the theory could be true of them.
}

\addedRtwo{%
Perhaps this will seem too hypothetical to be worth taking seriously.
We have no idea which, if any, ordinary agents aim to live by the norm of agglomeration and which, if any, aim not to.
My sense, however, is that philosophical theories are not supposed to depend on any such facts.
They are, after all, usually developed independently of any investigation into what ordinary agents think.}%
\footnote{%
\addedRtwo{%
\citet[p.~175]{gilbert:2009shared} does offer a theory which appears to involve stipulations about ordinary agents’ views.
But this is not supported by investigation, nor is there any explicit suggestion that an investigation would support the theory.
Others have attempted to investigate aspects of how well Gilbert’s theory captures ordinary agents’ views \citep{gomez-lavin:2019_normativity,michael:2022_intuitions}. 
But those researchers are careful to distinguish the aims of their investigations from supporting, or refuting, Gilbert’s philosophical position.
The leading philosophical theories of attitudes and norms are about ways people might reasonably be.
It is all entirely hypothetical.
}
}

\addedRtwo{%
Not that it has to be hypothetical.
Consider the familiar distinction between revealed and stated preferences.
To illustrate, in investigating the value people place on a life,
we could observe how much less people pay to own a house near a known source of carcinogenic pollution.
Or, alternatively, we could give them a questionnaire asking how much they would pay to eliminate the same risk.
A wide range of research has examined how revealed and stated preferences diverge (\citealp{carson:1996_contingent,alberini:2019_revealed}, for example).
There is also research on the factors that ‘often create a wedge between revealed and normative preferences’ \citep[p.~1788]{beshears:2008_how}.
Divergences between revealed and stated preferences matter in practice because they create difficult questions for policy makers on how much to invest in preventing deaths.
For our, more theoretical purposes, the divergence illustrates how understanding agents’ actions requires 
taking into account their own perspectives.
Agents’ stated preferences are views about how a model applies to them.
Because agents sometimes aim to live out these views, they are not inert commentary.
The problem is not fundamentally that this complicates predicting behaviour.
It is that where agents aim to conform to a model,
incorrect predictions do not have the same significance.
% Because the failure can be measured against what the agent was aiming to achive.
}

% ∞idea SP (stated preferences) vs RP (revealed preferences) gets resolved by (i) different methods
% are suitable in practice for different cases; and (ii) there are just multiple kinds of prefenrece
% (link to Dickinson too). While we should not take SP as the last word on what preferences are,
% no general theory of motivation can focus just on RP.


\addedRtwo{%
The distinction between revealed and stated preferences illustrates
the dual role of theories of attitudes and norms, 
in prediction and in offering ideals which people can attempt to live out.
There is a difference between,
on the one hand,
things which are merely described and predicted by a set of attributions
and, on the other hand,
agents who are are attempting to live out a set of attributions together with some laws governing them.
In the latter case, 
the agents’ aiming to live out a theory is a reason not to reject that theory insofar our aims include understanding the agents’ own perspective.
}

\addedRtwo{%
The argument that follows is an attempt to apply this general point 
to philosophical theories of shared intention.
My aim is to 
show that for each of the three
strategies for shared intention—plural, aggregate and reductive—there is at least
one consistent theory following this strategy which it is possible for people to intentionally live out.
This thesis is probably either too odd or too obvious to be interesting in its own right,
but I will suggest that it has consequences which complicate existing attempts to understand shared intention.
As these consequences depend on the aims of a theory of shared intention,
I start with the aims.
}



\section{Background on Shared Intention}
\label{sec:background}

Why do we need a notion of shared intention at all? 
Because it is what distinguishes genuinely joint activities from things people do in parallel but merely individually \citep{gilbert_walking_1990,bratman:2022_planning}.
This is, of course, at most a partial answer. 
The hope is that investigating a notion such as shared intention will enable us, eventually, to ‘discover the nature of social groups in general’ \citep[p.~2]{gilbert_walking_1990} and to understand the conceptual, metaphysical and normative aspects of basic forms of sociality \citep[p.~3]{bratman:2014_book}.
But one route to these lofty goals is to focus on 
distinguishing genuinely joint from merely parallel activities
in mundane cases involving two or three people.

Ayesha and Ahmed have spent the morning in the kitchen washing the dishes together. 
This is a paradigm case of joint activity. 
We can contrast Ayesha and Ahmed’s activities with those of two anti-social people who act in parallel but merely individually. 
The idea is that these other, anti-social people wash the dishes side-by-side,
but their actions are merely performed in parallel and so do not involve any joint activity.
What distinguishes Ayesha and Ahmed’s activities from these other people’s?

A temptingly simple idea is to appeal to coordination.
Could Ayesha and Ahmed’s activities be distinguished by virtue of being coordinated?
The obstacle is that acting in parallel can also involve coordination.
The actions of the other, anti-social people, who are merely acting in parallel, 
may nevertheless need to be tightly coordinated because space in their communal kitchen is limited.
They may also politely anticipate each other’s movements and work around them.
Mere coordination, then, cannot distinguish joint activity.

The failure of this and other simple ideas hints that 
distinguishing joint activities from their parallel but merely individual counterparts
is a deep and difficult problem.
This problem is a variant of one about ordinary, individual action.
The ‘Problem of Action’ is to distinguish a person’s actions from things that merely happen to them \citep{frankfurt1978problem}.
If ordinary, individual intention is key to solving that problem,
perhaps some joint counterpart of intention is the key to solving the problem of joint action.
This motivates using the term \emph{shared intention} to label whatever the normative or psychological
structure is needed to distinguish joint activities from things people do in parallel but merely individually.

Introducing a label for the problem does not take us very far towards a solution.
The problem now becomes to say what shared intention is.
One possibility is the plural subject
strategy: shared intention is a matter of there being one mental state with two or more subjects. 



\section{The Plural Subject Strategy}
\label{sec:plural}
A plural subject is two or more people who are each among the subjects of a single intention or other mental state.

The difficulty of understanding the plural subject strategy is mainly that it is so simple.
Intentions and other mental states involve subjects, attitudes and contents.
The content is what distinguishes two intentions from each other—the intention to cook dinner from the intention to go for a walk, say, differ in content.  
Attitude is what distinguishes intentions from other mental states—the intention to go for a walk differs in attitude from the desire to go for a walk.
And the subject is what distinguishes your mental states from mine.
The idea of the plural subject strategy is just that intentions can have more than one subject.
You and I can share an intention in the same sense that siblings share a parent.
Your intention to walk may also be my intention to walk: you and I are equally subjects of this intention.

It is helpful, 
in thinking about the contrast between individual and plural subjects, 
to draw on a related distinction between distributive and collective interpretations of sentences.
Consider these sentences:
%
\begin{enumerate}
  \item The fans left the stadium.
  \item The fans completely blocked the road. \label{item:fans-blocking}
\end{enumerate}
%
The first sentence is naturally read distributively: it is a matter of each fan individually leaving.
But the second sentence is naturally read collectively. 
As the road is very wide, not even the largest individual fan did much at all to block the road. 
But because so many fans were milling around in the road, it was impossible to traverse it.
So understood, the second sentence’s truth is not, or not only, a matter of each fan individually blocking the road.
This is a collective reading.
%
The distinction seems applicable to sentences about intention:
%
\begin{enumerate}[resume]
\item The twins intended to win the race. \label{item:twins-intention}
\end{enumerate}
%
If we imagine a race that can only have one winner, 
a 100 meter sprint, say,
then it is natural to read this sentence distributively.
Its truth is just a matter of each twin intending to win the race.
But if we imagine the twins running in a three-legged race together, 
it seems possible to read the sentence collectively.
On this reading, there is one intention whose subject is the twins.
They are, to put it colourfully, of one mind.

We can describe the twins as a \emph{plural subject}. 
But note that the plural subject is nothing other than the twins themselves. 
% For comparison, if the twins were to ride a motorbike by each placing one hand on the controls, we might ...
% If you have an intention, you are a subject; and of course the subject is nothing but you.
We must avoid confusion on this point in order to distinguish plural subjects from aggregate subjects, which are fundamentally different (more on this in section \ref{sec:aggregate}).


The plural subject strategy \addedRone{requires} no novel conceptual, metaphysical or normative ingredients over and above those already required in a theory of ordinary, individual action—it \addedRone{can be implemented in ways that} respect Bratman’s continuity thesis.%
\footnote{%
See \citet[p.~8]{bratman:2014_book}: ‘This is the continuity thesis. As we might try saying: once God created individual planning agents and placed those agents in a world in which they have relevant knowledge of each other's minds, nothing fundamentally new-conceptually, metaphysically, or normatively-needs to be added for there to be modest sociality.’
}
Just as the truth of the collective reading of (\ref{item:fans-blocking}) (\vpageref*{item:fans-blocking}) does not require anything other than the fans to mill in the road, 
so the truth of statements about intention collectively read requires nothing other than intentions and their subjects.%
\footnote{%
Here I am assuming Ontological Innocence, which is a controversial claim \citep[§5]{linnebo:2022_plural}.
}


But are there really plural subjects of intention?
One possibility is that there merely seem to be, and are not actually, collective readings of sentences about intentions like (\ref{item:twins-intention}) above.
There are a range of objections along these lines.
Most extreme is the claim that all apparently collective predication is really disguised distributive predication. 
A more limited objection is that statements about actions and intentions merely seem to have collective readings.
\citet[Chapter 9]{ludwig:2016_individual} offers a detailed discussion along these lines.
His conclusion is carefully nuanced:
%
\begin{quote}
  ‘we do not need to accept genuine plural [...]\ agents into our ontology in order to accept what we say about [...]\ collective action, at least insofar as we express this using plural subject terms.’
  \citep[p.~168]{ludwig:2016_individual}
\end{quote}
%
Ludwig might be right that semantic considerations do not force us to accept that plural subjects exist. 
\addedRone{Given his further premise that plural subjects should be avoided if possible,
this would be a compelling argument against the existence of coherent collective readings.
But, importantly, Ludwig finds nothing forcing us to reject their existence either.
% His main line of argument depends on the premise that we can deny their existence unless 
% shown that this is impossible.
So as long as there is either no general presumption against plural subjects or else sufficient reason to 
suppose that they are necessary,
it is not incoherent to imagine statements about intention have true collective readings.
Minimally, collective readings are a helpful tool for clarifying what the plural subject strategy is.}


But is it really coherent to suppose that intentions might have more than one subject?
One might object that to have an intention it is necessary to have a mind;
and that having a mind minimally involves having a range of mental states, and perhaps even being self-aware.%
\footnote{%
\addedRone{Note that this line of objection could be pursued independently of 
whether intentions are mental states.
On some views,
intentions are not mental states
\citep{thompson:2008_life,russell:2018_intention}
but having them might nevertheless require having a mind.}
}
\citet{Schmid:2013_self}, 
who has perhaps the best-developed version of the plural subject strategy,
accepts this constraint but argues that \addedRone{there are no good grounds for supposing that it could not be met}.
Likewise, \citet{helm_plural_2008} argues that there are plural subjects with a range of 
‘emotions and desires in the right sort of rational structure’ \citep[p.~29]{helm_plural_2008}.
Of course this would mean that plural subjects are unlikely to be involved in spontaneous interactions between strangers, as when I am struggling to propel my heavy push chair on to the bus and you helpfully seize the front and we lift together.
Plural subjects on views like Schmid’s or Helm’s would require vastly more intimate, long-term connections between individuals.

\addedRone{If we follow Schmid or Helm, it is possible to wonder how there could be 
plural subjects. And the sense of mystery one might have about this could, perhaps, motivate rejecting
the entire strategy in favour of apparently less mysterious alternatives.
But this would be an error. 
Any of the strategies can be developed in ways that will seem mysterious to at least some philosophers.
But one of my aims is to draw attention to the existence of straightforward, nonmysterious ways of developing each strategy.}

An alternative, potentially less mysterious plural subject view \addedRone{might be based on rejecting the claim that 
having an intention entails having a range of mental states.}
This view could be inspired by reflection that humans are prone to attribute mental states on the slightest of pretexts to things which, as they know, lack not only minds but even physical bodies \citep{Heider:1944ts}.%
\footnote{%
\addedRone{%
Alternative motivation for intentions without much in the way of other mental states
is offered by Bratman’s characterisation of what he calls ‘social-procedural-rule-based institutional Intentions’ \citep[pp.~147–50]{bratman:2022_shared}.
}
}
Whether or not things cannot actually have intentions without having minds, 
no such constraint appears to apply to at least one significant strand of everyday thinking.
Perhaps, then, the plural subject strategy is needed for capturing ways in which some people sometimes think.

\addedRone{We can take this one step further.
Here is a way in which you and I could become the plural subject of an intention.}
We each somehow become convinced, however mistakenly, that you and I are the plural subjects of an intention to cook dinner.
This thought might influence our behaviour: 
thinking, perhaps mistakenly, that having this intention means we are subject to various norms, we might aim to act in ways that conform to them.
We are, by our lights, acting as if we had this intention.
We could also be taking for granted that we were plural subjects of a range of other beliefs, desires and mental states, and perhaps explicitly attributing some as our activity unfolds.
And others, if they became convinced, perhaps mistakenly, that we had this intention, might also think and act accordingly.
In this way, what began as merely a mistake became real enough to shape the social world through being adopted as a normative ideal.%
\footnote{%
\addedRone{I am grateful to an anonymous reviewer for observing that 
\citet[pp.~137ff]{rovane:1998_bounds} offers an extended discussion of the possibility that
very committed people might, over time, achieve such a high degree of rational unity that
‘it would not be possible to engage just one of its human constituents separately’ (p.~141).
(The scenario I am imagining is merely one in which being a plural subject is taken as a normative ideal.)}
}

In this section I have introduced the plural subject strategy and offered a preliminary and superficial case for its theoretical coherence.
Following this strategy can lead to various quite different theories.
On some theories, the existence of plural subjects involves long-term, intimate connections capable of supporting a shared mental life.
On other possible theories, plural subjects can be temporary phenomena arising from the specific needs of a moment.

This falls short of showing that the plural subject strategy is successful.
For it to succeed, 
minimally there must be cases in which the existence of plural subjects is actually what distinguishes joint activities from things people do in parallel but merely individually.
(At least that is what one quite prominent approach requires, as we saw in section \ref{sec:background}.)
I have tried to indicate the difficulties involved in showing that no such cases \addedRone{could} exist.
But of course we have not seen positive grounds to suppose that there are \addedRone{now, or have ever been,} any intentions (or other mental states) which do have plural subjects.
\addedRtwo{Following the approach of section \ref{sec:live-it-out},
the possibility that there is a theoretically coherent plural subject theory 
which some people could aim to live out
is reason to accept that such a theory could capture an aspect of shared intention.}


% TODO same problem as above arises for the reductive strategy (say this in section \ref{sec:reductive})
% UPDATE: decided not to because it’s probably obvious.

For what it is worth, my own sense is that \addedRone{it would be quite hard to establish, in practice, that
a particular situation did involve a plural subject,
and that the difficulty of doing so may have been underestimated}.
To illustrate,
\citet{Schmid:2008} claims that plural agents feature in common sense thinking.
This claim is hard to evaluate because other philosophers would probably reject it (\citealp{ludwig:2016_individual}, perhaps).
Certainly philosophers seem vulnerable to making wrong assumptions about common sense thinking in other cases 
\citep[e.g.][]{starmans:2012_folk,nagel:2013_authentic,starmans:2013_taking}.
One as yet unresolved challenge is to develop an operationalisation which would enable us to distinguish someone thinking \addedRone{and acting} in terms of plural subjects from someone operating with a different conception of shared intention.
% This is harder than it seems ... no superficial linguistic marker (clocks have faces ...)
\addedRtwo{%
Whereas we have multiple methods for identifying stated preferences (including contingent valuation and choice modelling), we currently lack any hint about how we might identify attitudes towards plural subjects.
This challenge is made harder by the need to distinguish plural subjects from aggregate subjects.}

% Helm offers another kind of argument for plural subjects.
% His view is that accounts which do not involve plural subjects ‘fail to make sense of a distinctive and important part of the landscape of social phenomena—important especially for making sense of our most intimate relationships with others’ \citep[p.~21]{helm_plural_2008}.
% Although I admit that I am unsure what the argument is here, this does seem to be 


% *DIFFICULTIES depend on explanatory target:
% \begin{itemize}
%   \item is it really a strand in anyone’s thinking
%   \item would it really be useful to think in this way (what would it add)
%   \item is there a good psychological theory
% \end{itemize}





\section{The Aggregate Subject Strategy}
\label{sec:aggregate}
An aggregate subject is a subject with proper parts which are themselves subjects in their own right.
If you have intentions and some proper parts of you also have intentions of their own, then you are an aggregate subject.


Although our interest is in subjects of intention, and of mental states generally,
a non-mental illustration may be helpful.
The Portuguese man o’war, \emph{Physalia physalis}, %capitalization is correct!
is an animal composed of polyps which are themselves born as animals in their own right.
Why is the man o’war an aggregate subject rather than a plural subject?
Because it is numerically distinct from the animals which compose it.
These may change over its lifetime.
By contrast, in the case of a plural subject, there is nothing that could continue to exist if one of the individuals ceases to exist.
A plural subject is not a thing at all: it is just some individuals.
An aggregate subject, even one which right now consists of nothing but some individuals, is nevertheless a thing that is logically distinct from the individuals which comprise it.

For a non-mental illustration which features both plural and aggregate subjects, consider:
%
\begin{enumerate}[resume]
  \item The protestors formed a barrier which blocked the entrance.
\end{enumerate}
%
Forming the barrier is something the protestors do collectively. 
They (and no other thing) are the plural subject of the forming.
But in talking about the barrier we have introduced an aggregate entity.
Although it is composed of the protestors and nothing else, it is numerically distinct from them.
We know the barrier is distinct from the protestors because one of the protestors might abandon the barrier 
and be replaced by a new protestor.

Not everything true of an aggregate subject is true of a corresponding plural subject,
even when, as in the protestors’ case, the only parts of the aggregate subject are the plural subject.
For example, the barrier may be capable of surviving an assault which would destroy the plural subject.
%the barrier thrives on being assaulted even though the plural subject does not
And, conversely, it is true that the plural subject formed the barrier but false that the aggregate subject did so.
The mental case is similar.
To be an aggregate subject, a thing must have its own intentions (or other mental states) which are at least potentially distinct from those of its parts \citep[p.~274]{bjornsson:2017_corporate}.



How could there be aggregate subjects of intention?
As List and Pettit put it:
%
\begin{quote}
  ‘Let a collection of individuals form and act on a single, robustly rational body of attitudes [...] and it will be an agent.’\footnotemark
\end{quote}
%
\footnotetext{%
\citet[p.~74]{list_pettit:2011}.
While they use the term ‘group’ rather than ‘aggregate’, 
\citet[p.~31]{list_pettit:2011} explicitly stipulate that, as they use the term, being a plural subject is not the same as being a group: ‘Collections of individuals come in many forms. Some change identity with any change of membership. An example is the collection of people in a given room or subway carriage. Other collections have an identity that can survive changes of membership. Examples are the collections of people constituting a nation, a university, or a purposive organization. We call the former ‘mere collections’, the latter ‘groups’. Our focus here is on groups.’ 
}
%



%
Individuals sometimes act in this way only because their interests are so closely aligned, as when a variety of finance professionals all rush to exploit a tax loophole so that state finances appear ravaged by a many-handed beast.%
\footnote{%
See https://correctiv.org/top-stories/2021/10/21/cumex-files-2/
}
Alternatively, individuals may authorise a representative to speak for them as a group.
Such cases are unlikely to be theoretically interesting given our aim of investigating shared intention more broadly \citep[pp.~7ff]{list_pettit:2011}.
% former KGB officers sought control of assets in the chaotic early days of the collapse of the Soviet Union \citet{belton:2020_putin}.
Instead we should focus on cases where an aggregate agent has what Sugden calls \emph{autonomy}:
%
\begin{quote}
  An aggregate subject has autonomy if there is ‘the possibility that 
  every member of the group has an individual preference for y over x
  (say, each prefers wine bars to pubs) while the group acts on an 
  objective that ranks x above y.’
  \citep{Sugden:2000mw}
\end{quote}
%
The challenge, then, is to explain how there can be aggregate subjects which are autonomous from the subjects which compose them.


One approach to meeting this challenge borrows from decision theory. 
It is possible to use decision theory as an ‘elucidation of the notions of 
subjective probability [roughly, belief] and subjective desirability or utility [roughly, desire]’ \citep[xi]{Jeffrey:1983oe}.
Whether or not there are other ways of elucidating attitudes, decision theory provides one coherent way of thinking about them.
But decision theory is also agnostic about what subjects are.
As long as a thing’s behaviour fits a certain pattern, one that is specified by axioms linking attitudes to actions, the thing can coherently be attributed preferences.
This is why decision theory and its derivatives can be applied not only in describing humans but also bacteria, business organisations and countries \citep[chapter 10]{dixit:2014_games}.
Being agnostic about what subjects are makes it a useful tool for constructing a theory of aggregate agents.

In essence, the construction goes like this.%
\footnote{%
The construction is borrowed from \citet{Bacharach:2006fk} and \citet{Sugden:2000mw}.
I am not claiming that their views require aggregate subjects, only that some of their ideas 
can be (mis?)used to develop the aggregate subject strategy.
}
Two or more individuals take themselves, rightly or wrongly, to be components of an aggregate agent.
These individuals each ascribe preferences, and perhaps other attitudes, to the aggregate agent, and they all ascribe the same attitudes.
They then use these preferences to rank combinations of individual actions, and each individual selects an action from a highest-ranking combination.
Given the usual axioms about preferences being transitive and so on \citep{steele:2020_decision}, 
and given some background assumptions about the individuals’ knowledge of their situation, 
it will be possible to use decision theory to model the situation as if there were an aggregate agent.
And if we follow Jeffrey in taking decision theory as elucidating preference and other attitudes,
we can infer that there actually is an aggregate agent.
Further, because the preferences and other attitudes ascribed by the individuals need not be their own,
the aggregate agent has autonomy in the above sense.

How might the aggregate subject strategy provide a notion of shared intention? 
And how might it enable us to distinguish joint actions from things people do in parallel but merely individually (see section \ref{sec:background})?
One possibility is to stipulate that the intentions arrived at by individuals through the process of determining how the aggregate agent will act comprise a shared intention 
(\citealp{Gold:2007zd}; alternative views are offered by \citealp{bardsley:2007_collective} and \citealp{pacherie:2013_lite}).
On this view, one way for an activity to be genuinely joint is for it to issue from reasoning about the preferences of an aggregate subject where each reasoner is a part of that aggregate subject.%
\footnote{%
Whereas \citet{Gold:2007zd} appear to defend their view as the only kind of shared intention,
\citet{pacherie:2013_lite} explicitly offers a view on which the aggregate subject strategy and the reductive strategy each characterise forms of shared intention.
Also, as none of these researchers present their views as involving aggregate subjects,
my suggestion is only that we can use their ideas in pursuing the aggregate subject strategy.
}

None of this shows, of course, that there actually are aggregate subjects of preference or intention.
Even assuming there are, we cannot yet say whether their existence is actually what distinguishes joint activities from things people do in parallel but merely individually.
My aim in this section was merely to defend the theoretical possibility of aggregate subjects.
% And even this modest aim creates a problem.


% \section{Too Many Subjects?}
% \label{sec:too-many-subjects}

% There is a potential difficulty with the notions of plural and aggregate subject as we have been developing them.
% As things stand, it appears that there could not be a plural subject without there also being an aggregate subject.
% More carefully: suppose we follow Schmid or Helm in supposing that a plural subject of intention could exist only when it has a mental life involving a range of attitudes.
% This would appear to guarantee ‘a single, robustly rational body of attitudes’ \citep[p.~74]{list_pettit:2011}.
% And then it seems to follow directly (at least on List and Pettit’s view) that the plural subject is shadowed by an aggregate subject.

% The potential difficulty is that we now have too many subjects.
% Can the aggregate subject differ in its actions or attitudes from the corresponding plural subject?
% If not, their coexistence is worse than redundant: we would need to know what it is that ensures the harmony of their actions and attitudes.
% This is puzzling because the grounds for attributing actions and attitudes to the two kinds of subject are different.
% Alternatively, if the aggregate subject can differ in its actions or attitudes from the corresponding plural subject,
% we would appear to have the potential for disunity in our theory of shared intention. 
% Of course disunity is not necessarily a bad thing—it may be, after all, that the phenomena to be elucidated genuinely lack unity.
% But in this case there is a special problem: it seems difficult to identify any actual phenomena that the theoretical disunity would usefully elucidate.
% After all, no one has previously anticipated that a statement like ‘They intend to cook dinner’ might turn out to be true of a plural subject while false of an aggregate subject whose parts are the plural subject.

% This potential difficulty is pressing because some arguments for postulating plural or aggregate subjects
% work equally well for both. 
% Consider a case for preferring such accounts over reductive accounts:
% %
% \begin{quote}
% On reductive accounts, ‘it makes some sense to say that the result is a kind of shared action: the individual people are, after all, acting intentionally throughout.
% However, in a deeper sense, the activity is not shared: the group itself is not engaged in action whose aim the group finds worthwhile, and so the actions at issue here are merely those of individuals.
% Thus, these accounts ... fail to make sense of a ... part of the landscape of social phenomena.’
% \citep[pp.~20--1]{helm_plural_2008}
% \end{quote}
% %
% If this consideration carries weight, it seems to do so for both plural and aggregate subjects indifferently.%
% \footnote{%
% Strictly speaking if we follow the aggregate subject strategy we have only the actions of individuals.
% This is because aggregate subjects, unlike plural subjects, are individuals \addedRone{in their own right.
% (If you and I form an aggregate subject, the aggregate subject is a third individual.)}
% But, as Helm’s phrase ‘the group itself’ hints, the consideration he offers is really about ordinary, non-aggregate individuals.
% }
% Its proponents therefore face the potential difficulty of having too many subjects.
% They must either explain what ensures plural subjects exist in harmony with the corresponding aggregate subjects
% or else identify phenomena that their potentially divergent activities and attitudes elucidate.

% % Potential resolution: if we follow Schmid or Helm, then aggregate subjects are not theoretically interesting; they are only theoretically interesting where there could not be a plural subject.

% % Potential resolution: explain the possibility of having one kind of subject without the other.

% % Potential resolution: List and Pettit’s criterion is too weak; what is needed for an aggregate subject (but not for a plural subject) is the existence something that will survive the loss of an individual.

% % SHOULD PROBABLY CUT IN RESPONSE TO reviewer.2
% How might we meet this challenge? 
% One thought, inspired by \citet{rovane:1998_bounds}, is that plural subjects might 
% involve individuals committing to achieve long-term rational unity:
% in this case, the basis for the plural subject’s existence would 
% also ensure that the shadow aggregate subject 


% it does not provide a general solution to the potential difficulty of having too many subjects.
% After all, our argument for the theoretical coherence of aggregate subjects involved a variety of ways by which they might come into existence; there seems no reason to suppose that the intentions of plural subjects are always required.
% Further, it seems that plural subjects’ intentions need not invariably involve commitments to emulate a single body, nor invariably be directed to any kind of aggregate subject.
% After all, much of the point of our doing things together is, often enough, to exploit that fact that we have multiple bodies by, for instance, dividing tasks between us in a way that takes advantage of our ability to be in different places at the same time.

% The potential difficulty of having too many subjects may motivate considering a third strategy.




\section{The Reductive Strategy}
\label{sec:reductive}
If you seek to characterise shared intention entirely in terms of ordinary, individual subjects and their ordinary, individual attitudes then you are pursuing the reductive strategy.

Contemporary interest in the reductive strategy starts with 
\citet[p.~203]{sellars:1963_imperatives}’s observation that statements to the effect that we intend that we cook dinner are ‘clearly not the logical sum of’ statements about each of us individually intending that we cook dinner (\citealp{tuomela_we-intentions_1988}).
Apparently, then, our having a shared intention that we cook dinner together cannot consist simply in our each intending this.
% CUT—allusive!
% \footnote{%
% \citet{Bratman:1999fr} offers several considerations relevant to this claim.
% }
A natural question is whether 
there is any combination of ordinary, individual intentions, knowledge states or other mental attitudes 
  our having which could be 
  necessary or sufficient 
  for us to have a shared intention that we cook dinner.

The most extensively developed and widely discussed attempt to provide sufficient conditions for shared intention is \citet{bratman:2014_book}’s. 
The full account is complex but the core idea, put roughly, is this. 
For us to have a shared intention that we cook dinner, 
it suffices that we each intend that we cook dinner,
that we intend to do so by way and because of these intentions,
and that this is all common knowledge among us.

Proponents of the reductive strategy have succeeded in providing 
sets of necessary or sufficient conditions
which have intuitive pull for some and against which none of the published counterexamples have been widely accepted as successful.
% CUT—not relevant
% \footnote{%
% It is perhaps worth noting \citet{kopec:2018_shared}’s as yet unanswered series of counterexamples,
% although these are counterexamples to \citet{gilbert:2014_book}’s theory which is not derived from the reductive strategy.
% }
This is remarkable given that the model for Bratman and several others is Grice’s analysis of meaning \citep[footnote 13 to p.~334]{Bratman:1992mi}, which met a different fate.
\addedRone{A ‘flood’ of counterexamples to Grice’s analysis led to extensive revisions, to which further counterexamples were developed \citep[p.~11]{searle:2007_grice}.
Confidence in the project’s eventual success was shaken when, 
in a dramatic change of direction, Schiffer, who was formerly a leading proponent of the Gricean analysis, 
argued that the whole project was based on a mistake.}%
\footnote{%
\addedRone{See \citet[p.~265]{Schiffer:1987zb}: ‘if one were to make a list of all the things philosophers have 
in mind when they talk of “theories of meaning or intentional content,” then I would claim that 
there are no true theories satisfying the descriptions on that list. The questions being asked [...] that would require positive theories as answers all have false presuppositions.’}}
\addedRone{While Grice’s analysis has continued to inspire various projects (\citealp{moore:2016_gricean}, for example),
we are no closer to a successful reductive analysis.
By contrast, generally accepted counterexamples to reductive sets of necessary or sufficient conditions for shared intention
appear to be rare.
}

% Despite its success in providing theories which have intuitive pull for some and to which there are no published counterexamples, there are some challenges to the reductive strategy.
% For our purposes, the most interesting challenge arises from diversity among its proponents concerning the features of shared intentions.
A diversity of views about the features of shared intentions can be found in the reductive strategy.
The various conditions proposed imply conflicting views about 
whether having a shared intention invariably involves 
% contralateral commitments (for: \citealp{gilbert:2009shared}; against: \citealp[p.~361]{Roth:2004ki}), 
dispositions to help (for: \citealp[56–7]{bratman:2014_book};
  % \citealp[p.~95]{Searle:1990em}; 
  against: \citealp{Bratman:1992mi}, \citealp{ludwig_collective_2007}),
common knowledge (for: \citealp{Bratman:1993je}; against: \citealp{blomberg:2015_common}), 
% awareness of having a shared intention on each subject’s part (for: most; against: \citealp{Gold:2007zd}),
and corresponding individual intentions on each subject’s part (for: \citealp{Bratman:1992mi}; against: \citealp{sellars:1963_imperatives}).
There are also further issues on which theorists could disagree, 
including on whether shared intention invariably involves contralateral commitment (for: \citealp{gilbert:2009shared}; against: \citealp[p.~361]{Roth:2004ki}),
cooperation \citep{salomone-sehr:2022_cooperation} or  nonobservational knowledge \citep{roessler:2020_plural}.
% Did not mention all possible points above because some of the above are proponents of aggregate subjects (or not reductive strategies, e.g. Gilbert, Roth, Searle, Gold)

This gives rise to a challenge to the reductive strategy.
For the diversity makes it unclear when proponents of the reductive strategy can coherently be interpreted as offering competing attempts to characterise a single thing 
and when as offering compatible attempts to characterise different things.%
\footnote{%
Individual theorists have expressed both views. 
For instance, \citet{ludwig_collective_2007} positions his view as characterising something distinct from Bratman’s, while \citet{pacherie:2013_lite} positions her view as a revision of Bratman’s. 
My question is whether interpreting their views contrary to their statements would be theoretically coherent.
}

The reductive strategy allows us to construct many theories, each internally theoretically coherent but inconsistent with other reductive theories that share an explanandum.
For any combination of views about the features of shared intention, it would be possible to construct a coherent reductive theory.
If we simplify and regard the 
% seven
six
features mentioned above as binary, this yields 
% 128 
64
reductive theories.
The scarcity of counterexamples cuts two ways.


%Arguments that the reductive strategy should have priority if it works.  But why? Plural subjects are nothing. And although not nothing, aggregate subjects are no more conceptually or ontologically novel than things like the barrier composed of people (see section XX).

% USE THE TABLE ABOUT CONFLICTING PROPERTIES OF SHARED INTENTION FROM BOCUM?


% The reductive strategy can give rise to aggregate and plural subjects ( see \citet[Chapter 6]{bratman:2014_book} who investigate how a reductive approach may enable the construction
% of plural and aggregate subjects.). 
% FOr the strategy to count as reductive, their existence must not be part of the characterisation of shared intention but a consequence of it. (Or does the strategy still count as reductive in this case? Does it matter? Maybe the point is that the strategies are not exclusive)
% This is not a problem (harmony is ensured). The problem is that aggregate and plural subjects appear to emerge in ways other than through meeting conditions imposed by the reductive strategy.


\section{The Limits of a Metatheoretical Principle}

We have seen that the three strategies each yield theoretically coherent positions,
and, further, that at least one of these strategies alone yields many theoretically coherent positions.
Can we decide between the positions by invoking metatheoretical principles?
This idea has been carefully developed by Bratman:
%
\begin{quote}
  ‘If we can get a plausible model of modest sociality without appealing to a fundamental discontinuity in the step from individual planning agency to such sociality, then there is a presumption against an appeal to such a discontinuity in our theorising.’
  \citep[p.~36]{bratman:2014_book}
\end{quote}
%
If true, this principle provides a good reason to prefer the reductive strategy over theories like that of \citet{Searle:1990em} and perhaps also that of \citet{gilbert:2014_book}.%
\footnote{\addedRone{%
There is room for uncertainty about whether Gilbert’s theory meets the requirement about no fundamental discontinuities.
\citet[p.~55ff]{smith:2015_shared} argues that it does, at least ‘to the extent that Bratman’s’ does.
\citet[p.~75]{bratman:2015_shareda} objects to this claim on the grounds that ‘[t]he capacity to participate in the creation of [...] plural commitments does [...] go beyond capacities that are involved in individual agency.’%
}}
This is because each of those theorists postulates a fundamental discontinuity.
In Searle’s case, this is a novel kind of attitude, the ‘we-intention’, which differs from ordinary intention along the same dimension as desire differs from intention.%
\footnote{%
Searle does not use the term ‘we-intention’, which was notably used by \citet{tuomela_we-intentions_1988}.
(Those authors credit \citet{sellars:1963_imperatives}, although he does not use exactly that term.)
Following \citet[p.~33]{gilbert:2007_searle}, it has become common to use this term in discussing Searle.
}
For her part, Gilbert postulates a novel kind of commitment and associated nonmoral norms.
The novel kinds of intention and commitment are fundamental discontinuities.

Although Bratman’s metatheoretical principle rules against some theories,
it does not exclude many of those derived from the reductive strategy.
Nor does it exclude the plural and aggregate subject strategies outright.
After all, the whole point of plural subjects is that they are nothing but some subjects \citep{boolos:1984_value}.
And the bare idea of an aggregate subject is no more a fundamental discontinuity than is a barrier composed of protestors.
Further, as we have seen (in sections \ref{sec:plural} and \ref{sec:aggregate}),
both plural subject and the aggregate subject strategies can be implemented without appealing to fundamental discontinuities.

Apparently, then, Bratman’s metatheoretical principle is limited. 
There are theories from each of the three strategies—plural, aggregate and reductive—between which it fails to discriminate.

This is why I have presented the strategies in an unusual order.
The usual way is to start with the reductive strategy and then possibly
to justify adopting one of the others by some failure of that strategy
(see, for example, \citealp{helm_plural_2008}% quoted in section \ref{sec:too-many-subjects}
).
In my view that is a mistake.
There is no consensus on attempts to demonstrate failure of the reductive strategy generally.
Quite the opposite: after three decades there is not yet even a successful published counterexample to the most widely discussed reductive theory (\citealp{Bratman:1992mi}; \citealp{bratman:2022_planning}).
But, equally, the mere absence of successful objections to a theory is not enough to establish its truth.
The plural and aggregate subject strategies are significant 
not because the reductive strategy can be shown to fail 
but because they also yield theoretically coherent, as yet unfalsified theories.

These are deep waters. 
Some researchers hold that plural subjects should be avoided if possible, even narrowly logical ones (see \citealp{ludwig:2016_individual} cited in section \ref{sec:plural}).
Some even suggest that ‘all plural locutions should be paraphrased away’.%
\footnote{%
\addedRone{\citet[§5]{linnebo:2022_plural} identifies this as ‘the traditional view in analytic philosophy’ (which Linnebo does not endorse).}
}
Were this true, the plural subject strategy might be ruled out for reasons not specifically psychological.
As this illustrates, the considerations offered here fall far short of demonstrating that it would be impossible to find general principles which do discriminate among the three strategies for shared intention.

But there is also a positive argument for my thesis that all three strategies are needed.




\section{Do We Need Multiple Strategies for Shared Intention?}
\label{sec:multiple-strategies}



It is possible for people to intentionally live out one or another theory of shared intention:
to think and act as if that theory were true of them.
This indicates that no one theory alone could be sufficient to fully characterise shared intention.
Or so I will argue in this section.

For each of the three strategies, there are recipes you and I could \addedRtwo{explicitly aim to follow}.
To illustrate, suppose the time for us to face the growing pile of dirty dishes in our kitchen
has finally come.
Having both been inspired by \citet{Schmid:2008},
we might regard ourselves as the plural subject of an intention to wash the dishes
and act accordingly (see section \ref{sec:plural}).
Or perhaps what comes to mind is instead an idea about 
ascribing preferences to an aggregate subject and doing our parts to fulfill them
(see section \ref{sec:aggregate}).
Or maybe we have both just been reading \citet{bratman:2014_book} and are impressed that
we could benefit by forming and making explicit the intentions he identifies, thereby meeting
his sufficient conditions for shared intention
(see section \ref{sec:reductive}).
This being new to us, we even decide to write everything down so that we can track the attitudes
and actions.
Things go well and we continue to use the recipe for shared intention in our future activities.
Over time our use of the chosen recipe becomes so familiar that we hardly need to think about it at all.

The possibility of our \addedRtwo{aiming to follow} a recipe associated with any one of the three strategies for shared intention,
first explicitly and then with greater skill,
suggests that no one strategy can claim to be uniquely correct.
Instead, capturing the full range of phenomena involving shared intention will 
require theories associated with several different strategies.

\addedRtwo{%
The extent to which we actually succeed in following a recipe may be quite limited,
much as our stated preferences alone may explain only a small part of our behaviour (see section \ref{sec:live-it-out}).
What matters for our purposes, however, is just that the aim of living out the theory is not entirely inert.
It should influence some of our thoughts and actions.
This is what makes it reasonable to accept, in the absence of overriding reasons, that the theory could be true of us.
}
% A final response is that, in setting out to follow a recipe we are already exhibiting shared intention.
% It follows that the 

% Objection: the first time we need some kind of shared intention to get started.
% Concession: yes, recipe following is probably not the most fundamental kind of shared intention, neither ontogenetically nor phylogentically.


It may be helpful to consider possible responses to this position.
One response starts with the observation that following a recipe together may involve us having a shared intention to do so.
Probably, then, not all shared intention is a consequence of our intentionally
living out a theory of shared intention.
We might therefore be motivated to search for phylogenically or ontogenically foundational forms of shared intention
(see, for example, \citealp{Tollefsen:2005vh,Rakoczy:2007ou,pacherie:2013_lite}).
Perhaps it would even be possible, eventually, to relate strategies and theories to different stages and needs.
This is a radical response which breaks from the most extensively developed, best defended theories currently available, which give no such importance to evolutionary or developmental considerations.

\addedRone{%
An alternative response would aim to distinguish recipes that humans actually follow,
noting that these may be fewer than those which could in principle be used.%
\footnote{\addedRone{%
I am grateful to an anonymous referee for suggesting this possibility.%
}}
There are at least two potential sources of inspiration for this response.
One is narrowly philosophical attempts to establish that aggregate or plural subjects of shared intention
are either practically indispensable or required for certain explanatory purposes. 
(\citealp{roth:2014_indispensability} presents both kinds of argument, for example.)
The other source of inspiration for this response could be taken from \citet{gomez-lavin:2019_normativity},
who offer findings which they interpret as showing that everyday thinking
involves distinctive features of \citet{gilbert:2014_book}’s account of shared intention.
Just here we encounter a dilemma.
Narrowly philosophical arguments may establish that humans do follow one recipe but
appear unlikely to show that they do not also follow other recipes.
On the other hand, taking inspiration from experimental research involves a radical departure
from the kinds of consideration usually taken to motivate a theory of shared intention.
This is clear from responses to \citeauthor{gomez-lavin:2019_normativity}’s work,
which include  \citet{lohr:2022_recent} who challenges their interpretation on methodological grounds and \citet{michael:2022_intuitions} who offer apparently contrasting findings.
Those authors’ interest in discovering how people actually think about joint activities
has no counterpart in the work of the leading philosophers.
}

A \addedRone{bolder and} more orthodox response might be to allow that we could coherently follow any of the recipes but deny that all
of them yield shared intention (or, more ambitiously, even that any do).
The challenge for proponents of this response is to identify grounds for rejecting the view that following the recipes yields genuine shared intention.
They would need to enable all of us, as researchers, to know which things a theory 
of shared intention should explain independently of our knowing which theory is true.
As things stand now,
philosophers typically use particular examples of joint activities to introduce the topic
(see section \ref{sec:introduction}, and \citealp[5–6]{bratman:2014_book}, for example).
The usual assumption is that the examples are sufficient to identify ‘shared activity of the sort we are trying to understand’ \citep[p.~6]{bratman:2014_book}.
But intentionally following one of the recipes is, of course, one way of 
walking together, playing a piano duet, or painting a house together.
So if examples of joint activities provide us as researchers with a common understanding of the things to be explained, that common understanding supports the view that more than one strategy’s recipes are needed to capture them.
Opponents of this view need further theory-independent ways of identifying what is to be explained.


% % NOTHING WRONG WITH NEXT PARA BUT:
% %   * unclear if this is a version of 2nd response
% %   * not obviously necessary
% %   * seems a bit weak (why assume Schmid’s theory?)
% A third possible response might be to insist that in following a recipe we are merely acting as if we have a shared intention.
% One way to develop this response would be to invoke \citet{Schmid:2013_self}’s claim that genuine plural subjects are self-aware.
% For all our attempts to live out Schmid’s theory,
% for all our own conviction that we are a plural subject of intentions to wash the dishes and such,
% and for all other people’s agreement with us about this,
% we might yet lack plural self-awareness and therefore, by the theory’s lights, entirely fail to be a plural subject.
% Of course, this way of developing the response is only an option if the theory is correct that plural self-awareness is possible.
% And in that case the recipe is unnecessary; after all, its purpose was to support the possibility of plural subjects.
% Further, comparable responses for the aggregate subject and reductive strategies are unavailable because 
% theories derived from those strategies generally provide a set of sufficient conditions for shared intention
% which ordinary adults could, in principle, intentionally set out to meet.
% For those strategies it is therefore much harder to find considerations internal to the strategy itself which would distinguish genuinely having a shared intention from living out a theory of shared intention.
% The people living out a theory of shared intention, having met the theory’s requirements, reasonably take themselves to have shared intentions.
% The challenge for proponents of this response, then, is to find sufficient grounds to override the agents’ own perspectives and insist that they are merely acting as if they had shared intentions.

Overall, 
it seems plausible that
at least three different strategies for shared intention are needed.
This is 
because each of the plural, aggregate and reductive strategies 
is associated with a recipe people could intentionally follow 
and thereby 
  manifest phenomena 
  for characterising which the corresponding strategy is needed.



\section{Conclusion}
I have explored three strategies for elucidating ideas about shared intention.
The plural, aggregate and reductive strategies are often regarded as competing attempts to characterise a single target.
Proponents of the plural and aggregate subject strategies typically object that the reductive strategies fail,
while proponents of reductive strategies aim to show that the other strategies are not needed
\addedRone{(or, if they are needed, that they can be tacked on to a reductive strategy; see \ref{sec:introduction})}.
Despite much effort, no such arguments currently enable us to determine which strategy is correct.
In contrast, I have appealed to the possibility of intentionally living out different theories to argue that
%for each strategy there are probably phenomena which only that strategy can be used to correctly characterise.
for each strategy there are phenomena which can be correctly characterised only by following that strategy
(section \ref{sec:multiple-strategies}).
There may also be many theories derived from the reductive strategy for which the same is true
(section \ref{sec:reductive}).

If correct, this conclusion marks a collective success.
Whereas it was initially assumed that at most one strategy would work,
the careful development of multiple theories by their various proponents suggests there are multiple theoretically coherent possibilities,
each having intuitive appeal to some.

Despite the success,
this conclusion is not a tenable stopping point.
Whereas progress surely requires that we can discover grounds to reject some theories,
the conclusion that we need multiple strategies 
seems to imply that anything goes in constructing theories of shared intention.
% It also raises the ‘too many subjects’ problem,
% which involves apparently spurious questions about how 
% plural subjects and aggregate subjects co-habit (section \ref{sec:too-many-subjects}).

What to do?
The conclusion that we need multiple strategies rests on the premise that we need at least one theory of shared intention.
One way to avoid it might be to reject this premise 
(see footnote \ref{fn:no-shared-intention} on page \pageref{fn:no-shared-intention} for 
discussions which may motivate considering this option).
But while questioning the premise may lead to fresh insights,
the idea that we might completely do away with theories of shared intention seems unpromising insofar as some existing theories have been fruitfully applied beyond philosophy.%
\footnote{%
See, for example, \citet{Tomasello:2007gl}, \citet{Rakoczy:2007ou}, \citet{Moll:2007gu} and \citet{Grafenhain:2010zl}.
Note that these researchers smoosh together incompatible philosophical theories when introducing notions of shared intention.
The insights being applied are common to many theories and do not depend on the correctness of any one theory. 
}

A more hopeful response 
to the untenable conclusion %that for each strategy there are phenomena which can be correctly characterised only by following that strategy
would be to take inspiration from other domains where multiple apparently incompatible approaches have been discovered.
We might draw a very inexact parallel with the twin possibilities of using sets to
replace plural quantification and of using plural quantification to construct sets \citep[§4.3]{linnebo:2022_plural}.
Perhaps—so the hopeful response—we can show that the strategies for shared intention yield theories which are in some sense equivalent ways of elucidating a single set of ideas about shared intention.%
\footnote{%
For an illustration of how this might begin, see \citet[Chapter 6]{bratman:2014_book} who investigates how a reductive approach may enable the construction
of plural and aggregate subjects.
}
Or perhaps the informal nature of the theories and their diversity means that this response is just too hopeful.

A simpler response is also available.
Several researchers have pointed to things which stand in need of explanation 
and which, they suggest, might be explained using a theory of shared intention.
These include behavioural and neuroscientific findings \citep{gallotti:2013_social},
patterns in cognitive development \citep{Tomasello:2007gl},
and decision making \citep{Sugden:2000mw}.
In some cases this has led to debate on which things theories of shared intention are supposed to explain (see, for example, \citealp{bratman:2014_book} on \citealp{Gold:2007zd}).
One way to make further progress would be, in offering a theory about shared intention,
to specify things which stand in need of explanation 
in a way that can be understood independently of the theory’s truth or falsity;
and to formulate the theory in such a way that makes it possible to determine, eventually, whether it does actually explain those things.

% CUT BECAUSE PREACHY
% In some ways this last response would mark a departure from existing practice.
% It would require more attention 
% not only to potential targets of explanation 
% but also 
% to the limits of theories, how they can be operationalised and how to falsify them.
% The departure is surely worthwhile.
% After all, there is a difference between merely constructing a theoretically coherent model 
% % without regard to what, if anything, it models 
% and enabling us all to know whether the model explains some particular things.

To illustrate how this might go, consider a relatively easy family of questions.
How do 
various groups of individuals represent the activities of some agents acting together in particular situations?
Instead of interpreting existing theories as claims about how shared intention is, 
we can also interpret (or usefully misinterpret) them as theories about how people represent situations involving shared intention.
Generating predictions from existing theories is difficult but there are signs that this might be possible.%
\footnote{%
See \citet{gomez-lavin:2019_normativity}.  Although their study is not directly concerned with shared intention, their approach may illuminate how their participants represent joint activities. 
}
%To the extent that this programme succeeds, having many incompatible theories becomes an advantage.
Relative to this project—that of discovering how individuals represent joint activities—the existence of many theories is not a bad thing.
After all, there may well be differences between species, between infants, children and adults, and between cultures.
Further, a single individual may adopt different models in different situations.
Diversity in the theories of shared intention may enable us to discover
genuine diversity in the things to be explained.

% *The most basic strategy: specify what you are trying to characterise in a way that allows
% us to check whether the theory is true (see mechanistically neutral proceedure)
% *Doesn’t Bacharach do this (here are some games where T1 predicts these decisions, T2 those decisions?
% Question is then what shared intention adds)?
% Bratman responds by distinguishing strategic interactions from cooperation. But what is the positive characteirsation of the cooperation he is attempting to characterise?
% *Also Galotti and Frith do seem concerned with mechanisms (could use them as an example?)


% *None of this undermines Bratman’s programme. 
% The reductive approach does succeed in showing that conceptual novelties are not needed to characterise at least one class of cases (at least assuming that the metatheortic principle holds).  
% So Bratman did what was necessary.

% *Difference between characterising (it could be like this) and explaining (these are the factors which causally explain those events).  Philosophers seem mostly uninterested in explanation given that (i) explanations require empirical testing and (ii) empirical testing requires operationalisation.
% [PROBLEM: Can’t Bratman be seen as offering a conjecture about planning capacities and a model of how they interconnect? So it’s not merely characterising but it’s not explaining either. Bratman recently put it in terms of Cummins functions; but doesn’t that require testing too?]

% MAYBE HAVE TO STOP WITH THE DISATISFYING CONCLUSION? (I’m interested in the mechanistically neutral strategy after all.  How does this even relate to that? Maybe a useful first step would be to recognize that there are different explanatory targets and aim to identify them—then the chaotic pluralism might be more disciplined?)


In conclusion, we researchers need, but lack, a common understanding of what theories of shared intention are theories of.
It has been fruitful to construct different models of how aspects of shared intention might be.
The next step is to find out which models explain which things.



% \section{*Models}

% Can I do anything with the idea that the philosophers are constructing many models?
% Maybe it’s the same problem as one about intention?

% What changes if we buy the idea about models?
% \begin{enumerate}
%   \item more interest in specific applications (e.g. development, understanding role of motor representations)
%   \item more interest in operationalisation and falsification
%   \item more interest in the limits of the models
% \end{enumerate}

% Notable that Gilbert and Bratman clearly aim to offer things useful in social science but
% little concern with issues.
% Need to look outwards more.

% Maybe easiest if we think about which theories correspond to which ways people think about shared intention.



% against plural or aggregate are Convervative Principles

% against reductive is Not Shared Enough / Does Not Capture the Plural Perspective



