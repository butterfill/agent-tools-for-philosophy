%!TEX root = master.tex





% TITLE s
% Motor Representation and Knowledge of Skilled Action

% Instructions
% 5-7000 words. Ideally forward new ideas that advance the field but will remain accessible to a non-expert philosophical audience

% Q1 for chapter: Why are those with skill better able to make judgements about observed actions (as commentators)? (Obstacle: skills are not knowledge [refine].)

% PLAN: Start with basketball prediction study \citep{aglioti_action_2008}.
% Then, explain why the question is challenging (two kinds of outcome representation, motor vs intention) … before offering our proposed answer.

% Q2 why are those with skill able to form different intentions from those without? (Obstacle: skills are not knowledge [refine].)


\section{Introduction}
Most of the chapters in this volume concern the role of skill in doing.  Our focus is on its role in observation.  Why consider observation?  Much evidence suggests that skill is every bit as important for observing as for doing. It is no exaggeration to say that your skills are a foundation of your ability to observe and acquire knowledge about others’ actions. 

% steve todo: watch 3 minutes of NBA semi-final matches
Start with an illustration. A commentator and a former player are at a basketball match observing a player taking a free shot, potentially winning the match. Just as the ball leaves the player’s hands, the lights go out and the stadium is in complete darkness. So information about the early part of the player’s kinematics is available to the two observers whereas they have little or no information about the ball’s trajectory. Each observer, the commentator and former player, is asked to predict whether the ball went into the basket. Who would you bet is most likely to be right? Betting on the commentator might seem safe: she has years of experience watching shots taken from this perspective, while the former player, although skilled, has comparatively rarely observed the action from this perspective. Yet it is the former player, not the commentator, who is more likely to make the correct prediction.%
\footnote{%
 \citet[experiment 1]{aglioti_action_2008} compared three groups: one group of players, one of coaches and sports journalists (‘expert watchers’) and one of novices.
 Each was shown clips of the early stages of a player taking a shot and tasked with predicting whether the shot would land in or out.
 The players made correct predictions significantly more often than the expert watchers did, and the expert watchers were no more likely to be correct than the novices (p. 1111; see figure 1).
 Modelling the impact of shorter and longer clips on correct responses suggests that only the players were able to make full use of bodily information, while expert watchers and novices may have relied more on information about ball trajectory (p. 1111).
 } As this illustrates, being skilled in performing actions of a certain type can enable you to acquire observational knowledge about others’ actions of that type.  But why? Why are those more skilled in performing certain actions (sometimes, at least) better able to acquire knowledge when observing those actions?


% *outline plan for the chapter:
Our chapter aims to answer this question.  Before facing it directly, we shall first review further evidence for the premise that those more skilled in performing certain actions are, at least sometimes, better able to acquire knowledge when observing those actions (in \cref{sec:skills_explain_observation}). We shall then elaborate a conjecture which, if correct, would answer the question. Our conjecture is that performing and observing actions involves a common element, namely motor representations of outcomes to which the actions are directed.  It is this common element which explains why skills matter not only for performing actions but also for gaining knowledge in observing actions.  While a body of evidence supports this conjecture (see \cref{sec:motor_explains_skills}), it faces an objection.  The double life motor representations lead appears to require that they have two, incompatible directions of fit.  After answering this objection (in \cref{sec:objection}), we find ourselves confronted with a further challenge.  If our conjecture is right, whether we know something about the goals of an action sometimes depends on how we represent that action motorically. That is, motor representations can have content-respecting influences on knowledge states.  The challenge, which we call the Interface Problem, is how there could be such influence.  At present many candidate answers are available, as we shall see (in \cref{sec:interface_answers}).  In our view, neither any available evidence nor narrowly theoretical considerations are likely to yield decisive reasons in favour of any one answer.  A deeper understanding of how expertise matters for gaining knowledge of observed actions ultimately requires new discoveries on how motor representations and knowledge states interface.  


% *restriction of our topic to skills like writing and throwing (bodily skills?); skill in being elected to parliament and the like (e.g. programming) are beyond the scope of this chapter :



\section{That Skills Matter For Observational Knowledge}
\label{sec:skills_explain_observation}
The premise of our question is that those more skilled in performing certain actions are, at least sometimes, better able to acquire knowledge when observing those actions.
This is illustrated by \citet{aglioti_action_2008}’s basketball study, as we have seen. But this is much too bold a premise to accept on the basis of a single study. What further evidence supports it?

The method of \citeauthor{aglioti_action_2008}’s study relied on correlation as there was no intervention on the subjects’ level of skill. By contrast, \citet{urgesi:2012_long} included a training component. They studied volleyball fans and players, replicating and extending \citet{aglioti_action_2008}’s findings.  They then studied two groups of nonplayers, giving one observational training and the other training in playing volleyball. They found that those trained in playing volleyball also acquired observational skills not found in those given observational training only. Specifically, only those trained became able to use bodily movement in predicting whether a shot would be in or out.%
\footnote{%
\citet[experiment 2]{urgesi:2012_long} studied three groups who all took part in twelve two-hour sessions over three weeks.  
The three groups’ sessions differed in how subjects were taught about floating serves: one group was shown videos and given verbal instructions (‘observation training’); another group was given training that required them to perform floating serves (‘execution training’); and the control group was given training that did not involve floating serves at all.
Before the training, and then again after the training, each subject was tested on how accurately they were able to predict whether a shot would be in or out.
Subjects made predictions after watching two kinds of clips.
One kind of clip showed bodily movement only (stopping at just the point where the hand contacts the ball); the other kind of clip showed the ball’s trajectory (starting at just the point where the hand contacts the ball).
Only execution training resulted in significantly more accurate predictions on the basis of bodily movements; and only observation training resulted in  significantly more accurate predictions on the basis of ball trajectories (see figure~3 on p.~533). 
}


Acquiring a skill through training usually involves observing actions, even if only one’s own. Could it be observational experiences associated with training, rather than the acquired skills to act, which explain improved observational abilities?  To answer this question, 
\citet{casile:2006_nonvisual} investigated the effects of training blindfolded subjects to acquire a new skill. 
(The blindfold ensure they could not visually observe the actions they were being trained to perform.)
When walking, people typically swing their arms in phase with their legs.  It is surprisingly difficult to adopt different phase relations between arms and legs, although this can be learnt \citep{flying_circus:1970}. Importantly, it can also be difficult to identify walking actions involving unusual phase relations between arms and legs through observation. Exploiting these facts, \citeauthor{casile:2006_nonvisual} trained subjects in performing a silly walk while blindfolded.
Before training, and again after training, each subject was tested on her ability to visually identify walks which differed in the phase relation between arms and legs.
The found not only that training improved accuracy (see figure~2 on p.~70), but also that how well a subject could perform a novel walk after training was correlated with how accurately she could visually identify it  after (but not before) training (pp.~71--2).
This is evidence that the effects of having skills to act on your abilities to observe actions does not, or does not always, depend on gaining sensory experience in observing actions acquired as a side-effect of training.



Further evidence for the effect of skills to act on acquiring knowledge of others’ actions through observation comes from research with human infants. 
When observing a hand that is approaching some objects and about to grasp one of them, infants will, like adults, often look to the target of the action in advance on the hand arriving there \citep{Falck-Ytter:2006dg}.
As in adults, this proactive gaze indicates that the infants are tracking the goals of the observed actions.%
\footnote{%
For an infant to \emph{track the goal of an action} is for there to be a process in the infant such that how this process unfolds nonaccidentally depends, perhaps within limits, on which goal the action is directed to.
We take no view on whether or not the infants have knowledge about the goals of actions. 
Research on their abilities is relevant given that, as we suppose, the goal tracking processes infants manifest somehow matter for knowledge.
}
Critically, though, the occurrence of this proactive gaze in infants is related to their acquisition of the skills needed for performing the actions.
For those infants who are as yet less good at reaching and grasping, their eyes  do not arrive on an object to be grasped in advance of the hand grasping it \citep[compare][]{kanakogi:2011_developmental}.
Further, if we consider proactive gaze for different kinds of observed actions (such as putting objects into containers or various kinds of grasping actions), we find that infants’  gaze to the target of an action becomes more proactive  as they become better able perform the particular kind of action observed \citep{cannon:2012_action}.  
\citet{ambrosini:2013_looking} extended this finding by exploiting the fact that grasping skill can be measured by the minimum number of digits used to grasp from 10 to 2. Remarkably, they found that this fine-grained measure of skill in performing actions correlated with infants’ abilities to track the goals of observed actions.%
\footnote{%
Note that our suggestion is not that infants cannot track the goals of observed actions they cannot perform. 
The key claim for us is that infants are like adults in so far as those more skilled in performing certain actions are also better at extracting information when observing them.
% infants are able to track the goals of actions which they are unable to perform \citep[e.g.][]{skerry:2013_firstperson,bruderer:2015_sensorimotor}.
% This is consistent with the idea that skills in performing action affect abilities to track the goals of observed actions, as having a skill does not imply being able to exercise it.
}

What happens if we intervene on infants’ skills?  
\citet{sommerville:2005_action} put ‘sticky mittens’ on three-month-old infants and allowed them to play with objects.
This allowed them not exactly to grasp objects but to manipulate them with their hands, which likely boosted their skills. 
Following the training (but, in other studies, not mere observation: see \citealp{sommerville:2008_experience,gerson:2014_learning,bakker:2015_enhanced}), infants’ manifested abilities to track the goals of simple object-directed actions which untrained infants appeared to lack.%
\footnote{%
Infants’ goal-tracking abilities were measured using the much-replicated habituation paradigm introduced by \citet{Woodward:1998dm}.
}
This indicates that, as with adults, boosting an infant’s skill in performing an action can have a corresponding effect on the infant’s abilities to track the goals of actions of that type.
The link between skills and action observation is present even as humans are first acquiring skills to manipulating worldly objects.



% conclusion of section:
In this section we seen evidence from a range of studies covering a diversity of skilled actions which supports our premise that those more skilled in performing certain actions are, at least sometimes, better able to acquire knowledge when observing those actions (for a longer review and some bolder claims, see \citealp{shiffrar:2011_athletic}).  

This premise requires qualification.  The cases of skill we have considered all involve very small-scale\footnotemark\ bodily actions: they are skills such as throwing a ball, walking, and grasping an object.  These examples of skill contrast with, say, the skill that a successful politician exercises in winning an election for the fifth time despite having caused several disasters and being transparently corrupt. We should not generalise from one kind of skill to another without evidence.  
\footnotetext{%
In general, a \emph{very small-scale} action is one that is typically distantly related as a descendent by the means-end relation to the actions which are sometimes described as ‘small scale’ actions, such as playing a game of volleyball or rearranging some blocks.
}




\section{The Effects of Skill Depend on Having Capacities to Represent Actions Motorically}
\label{sec:motor_explains_skills}

Why should having a skill required for the performance of an action ever enhance your ability to observe and acquire knowledge about actions of that type? It is, sometimes or always, because there is an element common to having skill in performing an action and being able to acquire knowledge about that action through observation: both involve abilities to represent outcomes motorically. Or so we conjecture.

In this section our aim is to elucidate and support this conjecture.  Let us start by further specifying the conjecture as the conjunction of three subclaims:
%
\begin{enumerate}

\item exercising skills in performing very-small scale bodily actions involves representing certain outcomes motorically;

\item motor representations of these very outcomes sometimes occur when you are merely observing the actions; and 

\item the occurrence of these motor representations can enhance your ability to acquire knowledge about the actions when observing them.

\end{enumerate} 
% First, exercising skills in performing very-small scale bodily actions involves representing certain outcomes motorically. Second, motor representations of these very outcomes sometimes occur when you are merely observing the actions. And, third, the occurrence of these motor representations can enhance your ability to acquire knowledge about the actions when observing them.  
In what follows we review evidence for each of the three subclaims in turn.

% (1) intro to motor representation: [strategy: anticipation-1 (=general postural adjustment to predicted outcomes) then anticipation-2 (=goal-related anticipation) then representation]
Skill in performing an action often depends in many subtle, ordinarily unnoticed ways on anticipating its consequences. 
For instance, to open a sliding draw with any kind of finesse, you need to make anticipatory postural adjustments in order to maintain your balance, and these adjustments need to be tightly coordinated with the timing of your action. 
The difficulty, and the importance, of such anticipatory adjustments can be made vivid by considering developmental changes in infants’ abilities to perform actions such as reaches.  
Infants’ initial anticipatory postural adjustments are not well timed and only gradually develop adult-like temporal precision over a period of around seven months (\citealp{witherington:2002_development}; see also \citealp{hofsten:1991_structuring}).  
These and other minute observations of the development of abilities to perform very small-scale actions reveal that, often enough, the early parts of a skilled action anticipate the future parts in ways that cannot be determined from environmental constraints alone.



Adult-like skill in performing an action also involves anticipatory adjustments related to goals of later parts of the action (rather, than merely to side-effects of action like changes in posture).  To illustrate, consider arranging some new books (the old-fashioned, paper kind) on your shelves.  You take them out of a box on the table and place them on different shelves depending on where they belong in your collection. Some go on to high shelves, some are placed low down. Although mundane, this is a highly skilled activity. One aspect of the skill concerns how you grasp each book. Typically, the higher the shelf for which a book is destined, the lower on its spine you will grasp it \citep{cohen:2004_wherea}. The lower grip is initially more awkward but makes things easier when you come to finally placing the book.%
\footnote{%
This is an instance of what is usually called the ‘end-state comfort effect’ \citep{rosenbaum:1992_time,rosenbaum:1993_plans}. 
While it is important not to conflate end-states with goals, in this instance there is a connection.
Sensitivity to the end-state of the action implies sensitivity to the goal of the last part of the action (the placing of the book) since it is this goal that determines the end-state.
}  
As this illustrates, some skilled actions unfold in ways that anticipate the goals of later parts of the action (see \citet{kawato:1999_internal} for another example).

Given that skilled performance of very small-scale actions requires anticipation, including anticipation reflecting the goals future actions, how is such anticipation achieved?  Many researchers hold that control of very small-scale actions involves motor representations.%
\footnote{%
On what motor representations are and why they are necessary, key sources include \citet{rosenbaum:2010_human}, \citet{prinz:1990_cc}, \citet{wolpert:1995internal}, \citet{jeannerod_motor_2006}  and \citet{rizzolatti_mirrors_2008}.
}
These representations specify outcomes which are quite abstract relative to bodily configurations and joint displacements.  They may represent the movement of a book from one place to another, for example; or they may represent the securing of an object in a way that is neutral as between manual grasping, oral grasping and grasping with a tool \citep[e.g.][]{Rizzolatti:2001ug,umilta:2008pliers,cattaneo:2009_representation}.  
By representing outcomes, motor representations enable some forms of anticipatory control in performing very-small scale bodily actions.
 
A further reason for postulating motor representations of outcomes arises from the need to understand how action control can be computationally tractable given the many degrees of freedom, even after taking bodily synergies into account, afforded by the body’s joints  \citep{tessitore:2013_hierarchical}.  Representing actions in terms of outcomes and the means by which they are achieved contributes to making anticipatory control tractable.

% (2) in observation [‘motor representation of these very outcomes sometimes occur when you are merely observing the actions’]
So much for the first of the three subclaims of our conjecture (item 1 in the list above). What about the second subclaim?
This is the idea that motor representations lead a double life, for they sometimes occur not only when you are performing an action but also when you are merely observing another performing that action. Perhaps the most direct evidence in humans for this subclaim comes from measuring motor evoked potentials on the muscles of an observer.
Several investigations have revealed that when motor activity is amplified in the brain of an observer using transcranial magnetic stimulation, there are minute indicators of muscle activation in the observer in the right muscles at the right time \citep[e.g.][]{fadiga:1995_motor,Fadiga:2002kl, cavallo_grasping_2011,cattaneo:2009_representation}. This is one indication that observing an action can trigger motor representations which would occur in you if you were not merely observing but actually performing the action.  Further evidence for the occurrence of motor representations in action observation has been obtained from many scenarios using a wide variety of neurophysiological and behavioral measures in both human and monkey subjects \citep[][provide  reviews]{rizzolatti_mirrors_2008,rizzolatti_functional_2010,rizzolatti:2016_mirror}.

Importantly for our purposes, what is represented motorically in action observation includes outcomes to which actions are directed.  We know this thanks to experiments which (a) vary features of the action such as kinematics and context while keeping the outcome constant \citep[e.g.][]{rizzolatti:1988_functional,umilta:2008pliers}; and (b) vary the outcome while meticulously exploiting video editing to keep all other features of the action as constant as possible \citep[e.g.][]{Fogassi:2005nf}. The findings from these experiments suggest that there are motor representations whose occurrence in an observer varies specifically with the outcome of the action observed.

Further, the degree to which motor representations are involved in observation depends on the level of your skill in performing the action.  Or at least this is suggested by comparing pianists with nonpianists  \citep{haslinger:2005_transmodal}, and ballet with capoeira dancers \citep{calvo-merino:2005_action}.  In each case, observers skilled in performing the particular kind of action showed stronger brain activation in areas which would be involved in performing that action. This indicates that they represent the observed actions motorically in ways that the unskilled do not. 

But what are those motor representations of outcomes doing in you when you are merely observing another’s actions?
The answer, we suggest, is that they are enhancing your ability to acquire knowledge about the observed actions.
Or, if not the whole answer, this is at least one thing those motor representations are doing.
So the third subclaim of our conjecture (item 3 in the above list).
But why accept this?

% (3) can enhance [‘the occurrence of these motor representations can enhance your ability to acquire knowledge about the actions when observing them’]
Evidence for this subclaim comes from experiments in which subjects’ abilities to represent actions motorically are momentarily impaired. 
This can be done indirectly by constraining body parts, which is thought to impair abilities to represent motorically actions involving those body parts.
Or it can be done more directly using TMS pulses. However it is done, the findings are broadly similar. The proactive eye movements which indicate goal tracking are less likely to occur \citep{ambrosini:2012_tie,costantini:2013_how}, and your judgements about the goals of actions are slower \citep{dausilio:2009_motor} or less accurate \citep{urgesi:2007_representation,vankemenade:2012_effects,michael:2014_continuous} when your ability to represent actions motorically are momentarily impaired.

% Perceptual after-effects provide further support for this subclaim.  \citet{cattaneo:2010_ones} had subjects perform either pushing or pulling actions. This induced an after-effect (or something analogous to one) on their perceptual judgements. When judging whether another’s hand was pushing or pulling an object, those who has been pushing were biased to judge that the hand was pushing (and conversely). But when TMS was applied to relevant motor areas, the after-effect vanished. We assume that these after-effects make it more likely that the judgements are true, at least in more typical situations when no curious scientist is inducing them. Given this assumption, this influence of motor representations on perceptual judgements is evidence that the occurrence of motor representations can enhance your ability to acquire knowledge about actions you are observing.
%
% paragraph cut in response to reviweer’s comment:
% One might worry here that this assumption is not well-founded, since quite often the action that one is observing is unrelated to the action that one has just performed. In addition, it’s not quite clear what it means for an “after-effect” to be induced on one’s perceptual judgments. Is it the same as claiming that one’s responses were primed in a certain direction?
%


If the effect of performance skills on action observation is sometimes or always mediated by motor representations as we are suggesting, then we might guess that how accurate your observations are would depend on how closely you and the agent observed are matched in skill. You would be best at using observations to predict outcomes of actions performed by someone very like you in skill. Accordingly, \citet{knoblich:2001_predicting} and \citet{knoblich:2002_authorship} tested how matches in skill between observer and agent influence predictions. They had subjects watch videos of actions and predict their outcomes, where the agent in the video could  be either a stranger or the subject herself (thus ensuring an ideal match in skill between observer and agent). They found that, for throwing darts and also for handwriting, subjects could best predict the outcomes of their own actions, even when not informed that the observed actions were their own.\footnote{%
In \citet{knoblich:2001_predicting}, the subjects were informed about the identity of the agent observed; in \citet{knoblich:2002_authorship} they were not.
}


The subclaim that motor representations can enhance your ability to acquire observational knowledge of actions needs further specification.  For it should not be assumed that the influence of motor representations on knowledge states is a matter of one-off triggering: it may be more plausible to suppose that motor representations dynamically influence processes by which knowledge states are arrived at and maintained (as \citealp{fridland:2016_skill} suggests in a parallel case).  And, further, the influence of motor representations on these epistemic processes should probably not be understood as a matter of determining their outcomes in the sense that you could predict exactly which knowledge states will be arrived at from knowledge of what is represented motorically (compare \citealp{burnston:2017_interface}).
Finally, note that we are here offering no suggestions about how motor representations may enhance abilities to acquire knowledge: the view defended so far is silent on processes.

% conclusion of section:
So why are those more skilled in performing certain actions (sometimes, at least) better able to acquire knowledge when observing those actions?
In this section we have elaborated and defended our conjecture that it is  sometimes or always because both involve abilities to represent outcomes motorically (see \citet{shiffrar:2010_people} for a broader review). Your skills matter for action observation because you exercise these skills in observing much as you do in performing an action.


While a body of evidence appears to support this conjecture (as we have just seen), there is an objection to its coherence.
The objection concerns the idea that motor representations both control actions (so must have a world-to-mind direction of fit) and are involved in tracking the goals of others’ actions (which requires a mind-to-world direction of fit).


\section{An Objection: Motor Representation and Direction of Fit}
\label{sec:objection}
Our conjecture entails that there is a single kind of representation, motor representation, which features both in performing actions and in gaining  observational knowledge of them. 
Let us take a further step and suppose that this is no accident but rather reflects the functions motor representations serve.
It appears to follow that representations of this kind have different directions of fit.  Apparently, some are world-to-mind insofar as they are supposed to lead to performing actions, while others are mind-to-world insofar as they are supposed to enable predictions of others' actions. 

This objection arises from the idea that motor representations lead a double life.  On the view introduced earlier, what is represented motorically in action observation includes outcomes to which actions are directed. Further, these representations have the function of enhancing your ability to acquire knowledge about the observed actions. It seems, therefore, that these representations must be mind-to-world. That is, their function depends on the motor representation representing only outcomes which are actually goals of the observed action.  Because the idea that motor representations lead a double life involves saying that there is just one kind of representation here and in action control, it appears to entail that different instances of a single attitude can have different directions of fit: some are world-to-mind, others are mind-to-world.  But this is arguably incoherent.  So the objection.

One response to the objection would be to embrace the idea that a single representation can have multiple directions of fit, as others have done (see, for example, \citet{millikan:1995_pushmi-pullyu} and \citet[Section~7.2]{shea:2018_representation}).
As not everyone accepts this,%
\footnote{%
For example, \citet[][p.~546]{artiga:2014_teleosemantics} argues that ‘the main motivation for embracing the Pushmi-Pullyu account is flawed’.
}
we propose a reply to the objection which does not depend on contradicting it.%
\footnote{%
Our reply to the objection also indicates a way of defanging some of \citet{millikan:1995_pushmi-pullyu}’s original arguments for the existence of representations with both directions of fit.
}

As a preliminary to replying to this objection, consider an analogy.  There is a rotary dial on your oven which enables you to initiate and control the oven's activity. We might think of the dial as having an oven-to-instrument direction of fit: the oven temperature is supposed to adjust to the setting on the dial.  But now suppose, further, that there is an indicator light on your oven which is illuminated unless the oven has reached the temperature specified by the dial. This enables you to use the dial to discover the temperature of the oven: if the light is on, you turn the dial down until just the point where the light goes off.  Now the setting on the dial tells you the temperature of the oven.  So we might think of the dial as having an instrument-to-oven direction of fit.

This analogy will guide our response to the objection.  The key point can be put like this. There is a core system featuring the dial, thermostat, heating element and oven.  Relative to this core system, the dial always has an oven-to-instrument direction of fit.  However, there is a larger system which embeds the core system and exploits it for novel ends.  This larger system includes you and your capacity to temporarily prevent significant changes in the temperature of the oven (perhaps by moving the dial between settings too quickly for the heating element to respond).  Relative to this larger system the dial has an instrument-to-oven direction of fit.  So to  understand the dial's functions, we do need two directions of fit, oven-to-instrument and instrument-to-oven.  But this is not quite to say that the dial has both directions of fit.  For something has a direction of fit only relative to a particular system.  Which direction of fit we see depends on which system we are considering.  Understanding the dial does not require supposing that anything has two directions of fit relative to a single system.

Our response to the objection is similar.  If we consider planning-like motor processes only (the core system), then each motor representation’s function is linked to initiating and controlling action.  From this perspective, only a world-to-mind direction of fit is in view.
But these planning-like motor processes can occur in the context of a larger system, one which involves something that somehow prevents performance of action.  The functions of this larger system concern predicting which outcomes actions will be directed to.  If we consider this larger system, it is natural to describe the motor representations as having a mind-to-world direction of fit.  So, as in our analogy, which direction of fit we see depends on which system we are considering.  We need never have two directions of fit in view simultaneously. Our reply to the objection, then, is that our conjecture involves no incoherence when properly understood.

So much for the objection.
From here on we will assume that it is at least coherent to conjecture,
as we have, 
that having skills enhances your observational abilities wholly or in part because it enables you to better represent the goals of others’ actions motorically.
But the truth of this conjecture would raise a  theoretical challenge.  
The challenge concerns how motor representations could have content-respecting influences%
\footnote{%
For one state’s influence on another to be \emph{content-respecting} is for whether or how the first influences the second to nonaccidentally reflect some relation between the two state’s contents.
} 
on knowledge states. 
We call meeting this challenge the Interface Problem.


\section{How Do Motor Representations Influence Knowledge States?}
\label{sec:interface_answers}
Suppose, as we have been considering, that having skills enhances your observational abilities because it enables you to better represent the goals of others’ actions motorically. 
Then we are sometimes in this situation: in observing an action, we represent a certain outcome motorically; and partly in virtue of this motor representation, we come to know that this  outcome or a matching\footnote{%
Two outcomes \emph{match} in a particular context just if, in that context, either the occurrence of the first outcome would normally constitute or cause, at least partially, the occurrence of the second outcome or vice versa.
} 
outcome is a goal of the observed action.
It is not just that motor representations influence knowledge states: whether we know something can depend, in some way, on what we represent motorically. 
That is, motor representations can have content-respecting influences on knowledge states.
How is this possible?

There have been millennia of discussion about knowledge and, more recently,  quite a bit of research on motor representation. By contrast, there is comparatively little written on how the two might be connected.  For this reason it seems to us worthwhile to consider a range of candidate answers at this stage.  As the question is ultimately a scientific one, our hope is that some or all of the rough candidate answers reviewed in this section can be turned into hypotheses generating readily testable predictions.

% [Answer 1: identity]
One candidate answer might involve identifying motor representations with knowledge states.  Perhaps, for example, motor representations which occur in action observation are one kind of knowledge state.  This candidate answer implies that there is no issue concerning how motor representations can have content-respecting influences on knowledge states; or, if there is, it is just a special case of the more general issue of how knowledge states can have content-respecting influences on each other. One attraction of this candidate answer is that it would eliminate the Interface Problem, which might otherwise be difficult to solve. But there is also a cost. 

The first candidate answer (‘Identity’, as we might call it) involves an unorthodox view of on the nature of knowledge. After all, motor representations have a double life, as discussed earlier. In control of action, they specify an outcome to which your actions are directed; in observation, they specify an outcome to which someone else’s actions are directed. So there is a sense, apparently (but not actually---see \cref{sec:objection}) paradoxical, in which motor representations have two directions of fit. By contrast, knowledge states are usually thought to lead more boring lives and to have just one direction of fit.

To be clear, we are not suggesting that its unorthodox implication comprises an argument against Identity.  Philosophers have put forward a range of incompatible claims about knowledge and its varieties. As we lack grounds for deciding among these claims and are unsure what considerations might constrain philosophical claims about the nature of knowledge states, we can say only that Identity’s unorthodox implication counts as a reason for considering alternative candidate answers to our question.
Determining the correctness of Identity may depend on deriving and testing its predictions. (To this end, we derive a prediction from Identity below.)

% [Answer 2: inference]
An alternative candidate answer is that motor representations are connected to knowledge states via inferential processes.
There is no mystery about how one knowledge state can have content-respecting influences on another knowledge state: inference is paradigmatically a process by which mental states have content-respecting influences.
It might in principle be that motor representations can similarly lead to knowledge states via a process of inference combining the two kinds of state.
One attraction of this candidate answer (call it ‘Inference’) is that, if correct, it would appear to solve the Interface Problem without conceptual novelty or reliance on untested conjectures.

An obstacle to accepting this answer arises from one respect in which motor representations and knowledge states differ.
It is a familiar idea that representations have format as well as content.
For example, something about the arrangement of streets in a region can be represented both by a map and by a text.
The different kinds of representation are subject to different constraints on what can be represented.
For example, the map cannot be used to represent that there are at least two streets leading from the Central Cafe without thereby representing something more specific.
Such constraints are a consequence of representational format.

Whereas the formats of artifactual representations are open to view, the formats of mental representations are harder to identify.
One way to distinguish between representational formats is to use the method of constraints.
If something can be represented by a state of one kind but not the other, this may be due to a difference in representational format. 
And there are limits on what can be represented motorically which are not limits on what you can know. To illustrate, consider interacting with a genuine and a fake Armani jacket.  Assuming the fake is a good one, interacting with it will be no different motorically from interacting with the genuine one; even though you may well know which jacket you are interacting with. Likewise, imagine boarding a plane from New York to Caracas or to Reykjavík. These are importantly different actions and you may be fortunate enough to know which you are performing; but such differences cannot be represented motorically.
This indicates (nondemonstratively) that knowledge states and motor representations differ in representational format.\footnote{%
There are complementary considerations based on performance characteristics which support the same conclusion \citep{butterfill:2012_intention}.
}

How does this bear on the second candidate answer, Inference? Since they differ in format, motor representations and knowledge states could not feature in a single inferential process unless there were also a process of translation from one representational format to the other.%
\footnote{%
Why accept this claim?
Sometimes people can successfully make use of representations in either of two different formats while being unable to determine directly (that is, independently of making further use of the representations) whether two representations in different formats specify the same or different things.
So being able to use representations in either format is not sufficient for combining representations of different formats in a single inference.
Something more is needed: the ability to translate between them.
} 
It follows that  accepting Inference would require postulating some process of translation between motor representations and knowledge states.
But nothing is known about such a translation process, and such a process is not known to be needed other than for solving Interface Problems.
So the second candidate answer, Inference, relies on an untested, novel postulation.

This consideration is no argument against Inference.  After all, the postulation may turn out to be correct.  Indeed, \citet{shepherd:2018_skilled} proposes that something like translation between representations in different formats is necessary, among other things, for generating ‘sophisticated … action concepts’ (p.~14).  On Shepherd’s account, this is either enabled by motor representations having a kind of conceptual structure, or by there being a capacity to conceptualise information represented motorically.%
\footnote{%
Note that we deliberately avoid relying on any counterpart of \citet[p.~10]{shepherd:2018_skilled}’s claim that ‘an intention can take both propositional and motoric contents’ here.
This claim is not necessary for his view that  
‘a translation process may be just what we need’ (p.~3).
} 
Although details remain to be discovered, Shepherd suggests that researchers already know that practical reasoning about how to act involves a combination of motor representations and intentions (p.~16). If his argument succeeds, similar considerations may well justify a related claim about motor representations and knowledge states in reasoning about what has been, or is being, done.

Whether Inference seems initially attractive probably depends on your general sense of how the mind works.  If monistic theories of mental processes and representations appeal, Inference will likely appeal too.  If, on the other hand, you are both inclined towards pluralism about mental processes and representations and also tempted to explain resilience and adaptation by appeal to encapsulation or modularity, then Inference and Identity alike may seem unattractive.
Does this render it infeasible to determine the correctness of these views?
We think not.
A striking prediction can be generated from either Identity or Inference. Suppose you represent the goal of some observed action motorically. Then you should be in a position to know what the goal of the action is.
How could you not?
Given Identity, motor representations which occur in action observation are one kind of knowledge state, so failure to know would presumably involve having inconsistent beliefs or some other knowledge-preventing failure of rationality.
And given Inference, only this or failure to make a simple inference could preclude you from knowing. 
So if it turned out that humans can represent the goals of observed actions motorically while lacking corresponding knowledge despite there being no inconsistency nor failure to make a simple inference, then we would have evidence against both Identity and Inference.


Is this prediction testable? %
% \footnote{%
% We are grateful to a reviewer, Myrto Mylopolous, for objecting that it is difficult to see how the prediction can be tested.
%}
Consider that 
there are parallels to Identity and Inference for a question about how motor representations relate to intentions rather than knowledge states.  One consideration against these parallel views arises from ways motor representations and intentions can fail to match, as illustrated by Anarchic Hand Syndrome \citep[compare][p.~7]{mylopoulos:2016_intentions}.  
Subjects with Anarchic Hand Syndrome, which usually follows lesions of the anterior part of the corpus callosum and of the supplementary motor area, perform actions incompatible with their avowed intentions. These patients may also refer the their anarchic hand as having a mind of its own \citep{della_sala:1994fk}.
Consider a subject with Anarchic Hand Syndrome who intends not to drink some hot tea until it cools.
As one hand attempts to pick up the cup and bring it to her mouth, she needs to intervene with her other hand to put the cup back onto the table \citep[p.\ 1114]{della_sala:1991_anarchic}.
There is clearly some mismatch between her intentions and her motor representation.
But there is no reason independent of accepting Identity to suppose that the mismatch involves anything like the kind of irrationality that occurs when one knowingly has incompatible intentions.
And since Anarchic Hand Syndrome is not specifically linked to failures in reasoning, nor is there  reason independently of accepting Inference to conjecture that the mismatch is due to failure to make a simple inference.
Our suggestion is that parallel considerations could be used to show that any discoverable mismatches between knowledge (rather than intention) and motor representation provide evidence against Identity and Inference.
If so the striking prediction is readily testable.

% Answer 3: executive concepts (Pacherie \& Mylopoulos)
A third candidate answer, due to \citet{mylopoulos:2016_intentions},
involves the notion of an executable action concept.%
\footnote{%
\citet{mylopoulos:2016_intentions} focus on how motor representations relate to intentions rather than to knowledge states; however, their view can be generalised in a natural way. 

The notion of an executable action concept may be related to \citet[p.~19]{pavese:2015_practical}’s suggestion that ‘operational semantic values are kinds of practical senses’ (Pavese, personal communication): in both cases, a key idea is that entertaining propositions or possessing concepts concerning ways of acting enable agents to act in those ways.
}   
This is a concept which ‘could guide the formation of a volition, itself the proximal cause of a corresponding movement’ \citep[p.~8]{mylopoulos:2016_intentions}. To illustrate, in humans MANUAL REACH would typically be an executable action concept whereas WAG TAIL would not typically be. Further, they propose that concepts are executable action concepts in virtue of an association between the concept and a motor schema%
\footnote{
Motor schema ‘are internal models or stored representations that represent generic knowledge about a certain pattern of action … that is the organization and structure common to a set of motor acts’ \citep[p.~14]{mylopoulos:2016_intentions}.
}
such that when a thought involving the concept occurs, the associated motor representation is activated. We might suppose that the converse also occurs. That is, when a motor representation occurs in action observation, any associated executable action concepts are somehow activated, biasing the observer to think about the corresponding action. 

Could this idea explain, in principle, how motor representations have content-respecting influences on knowledge states without there being any inference or translation process? Suppose that the concept MANUAL REACH is associated with a motor schema for manual reaching. Then when observing someone reach for a cup, this outcome is represented motorically. Such a motor representation involves both the specific outcome involving this particular cup and a motor schema. Because the motor schema is associated with the concept, activation of the motor schema increases the probability that the concept will be activated too. 


One limit of this answer, call it Association, is that, as it stands, it leaves many issues open \citep[as][note]{mylopoulos:2019_intentions}. One dilemma concerns whether there can in principle be a mismatch between the concept and the motor schema. That is, could the concept MANUAL REACH (say) in principle be associated with a motor schema for grasping with the mouth (say)? If it could, there would be a sense in which executive action concepts are blind. That is, the concept-user is not necessarily in a position to know both what the concepts she uses are concepts of and which action they enable her to perform.  If, on the other hand, there could not be a mismatch, then Association requires an account of how such rogue associations are impossible in principle.

Like Inference, Association indicates a way of developing a candidate answer to our question about how motor representations could have content-respecting influences on knowledge states. And the two directions are not obviously incompatible: in attempting to construct a candidate answer, both ideas might in principle be combined. Despite the promise of these ideas, we believe little enough is known that it remains worth considering further alternatives.

Our fourth (and last) candidate answer invokes experience. When observing an action you may have an experience which provides you with reasons for a judgement about the goals of that action. Call this an \emph{experience revelatory of action}.  Now we know that motor representations can influence perceptual processes \citep{bortoletto:2011_action}, and there is even some evidence that motor expertise may influence whether an experience reveals an action \citep{funk:2005_hand}.  To make a leap, we might guess that which outcomes are represented motorically can influence which goals are revealed in experiences revelatory of action: that is to say, motor representations can \emph{shape} experiences revelatory of action. Perhaps this is how motor representations can have content-respecting influences on knowledge states. 

This candidate answer, call it Experience, suggests a rough parallel between two interface problems. One concerns representations which feature in perceptual processes, the other motor representations. In both cases, the question is how these representations can have content-respecting influences on knowledge states. And in both cases the answer is that the representations shape experiences, which in turn provide reasons for judgements.  There is a reason why some animals have experiences: it provides a link between representations with different kinds of formats in cases where binding things together with inferential processes would be suboptimal (perhaps because it would break otherwise useful encapsulation, for example).

Of course the parallel needs careful development. Whereas there are perceptual modalities, it is perhaps unlikely that there is also a motor modality.  So the closest parallel for motor representations’ influence on knowledge states may be with amodal perceptual representations such as object indexes (as \citealp[p.~12]{sinigaglia:2015_puzzle} suggest).
It would also be possible to develop Experience while rejecting the parallel with a perceptual interface problem entirely (see \citealp[pp.~156--158]{sinigaglia:2015_goal_ascription}).

However exactly it is developed, Experience faces a methodological objection.  As a candidate answer to a question about one problem (How can motor representations have content-respecting influences on knowledge states?) it raises two questions which appear no less puzzling.  After all, it seems no easier to understand how motor representations can have content-respecting influences on experiences, nor how experiences can have content-respecting influences on knowledge.  Our view is that such objections carry little weight when we are so far from understanding how to solve the Interface Problem.  Any candidate answer that can be turned into a hypothesis capable of generating readily testable predictions is worth considering. 

But could Experience really be turned into a testable hypothesis?  We know that motor representations can influence judgements about the trajectories of bodies in motion \citep{shiffrar:1990_apparent,blake:2007_perceptiona}. But how? The Visual Hypothesis says these judgements are based are visual experiences of movements only. By contrast, the Action Hypothesis says that motor representations can influence experiences associated with bodily trajectories in ways which are not exhaustively visual. Note that the Action Hypothesis does not flow directly from Experience (which is, strictly speaking, consistent with the Visual Hypothesis).  However the Action Hypothesis could be regarded as one way of developing Experience.  If the Action Hypothesis is right, there should be situations in which subjects can rationally distinguish bodily trajectories in ways not fully explained by their visual experiences of movements.  To illustrate, consider placing a solid barrier somewhere along a possible hand trajectory.  Suppose (as might in principle happen) that subjects were to judge, on the basis of observation, that the hand follows this trajectory and that they also report not seeing the hand pass through the solid barrier.  If the same combination were not obtained concerning the movements of mere shapes (rather than hands), we might conclude that the judgement about the hand trajectory is not, or is not entirely, a consequence of visual experiences of movement.  This would be evidence for the Action Hypothesis.  Of course, it is unlikely that things would turn out so neatly. We mention the possibility merely to illustrate one virtue of the fourth candidate answer, Experience: although perhaps implausible and complicated, it is a source of hypotheses which generate readily testable predictions.  


\section{Conclusion}
We started with the discovery that those more skilled in performing certain very-small scale bodily actions are sometimes better able to acquire knowledge when observing those actions (see \cref{sec:skills_explain_observation}).  But why?  We conjecture that it is because performing and observing actions involves a common element, namely motor representations of outcomes to which the actions are directed. This conjecture is supported by a range of evidence (\cref{sec:motor_explains_skills}). It is also theoretically coherent (\cref{sec:objection}).  However, its correctness would leave us with a deeper and more puzzling question than the one it aims to answer.

If our conjecture is right, whether we know something about the goals of an action sometimes depends on how we represent that action motorically. That is, motor representations can have content-respecting influences on knowledge states. How is this possible? As we have seen (in \cref{sec:interface_answers}), there are at least four distinct candidate answers to this question.  Further, at least three of these are consistent with each other in the sense that, in principle at least, any combination of them could be correct.  There is a gap in our understanding of how expertise matters for gaining knowledge of observed actions.  

In our view, progress could be made in two ways.  The first is to explore links between different interface problems.  An interface problem arises wherever it is challenging to explain how representations of one kind can have content-respecting influences on representations of another kind. This is challenging not only for how motor representations influence knowledge states but also, moving in the opposite direction, for how intentions influence motor representations \citep{butterfill:2012_intention}. And, more broadly, there appear to be related challenges for understanding how perceptual representations can influence knowledge states \citep[e.g.][]{jackendoff:1996_architecture}.  Of course, the solutions to different interface problems may turn out to have little in common. But our guess is that good solutions will be reused.  Linking different interface problems may therefore constrain the range of candidate solutions that need be considered.  This is likely to leave several candidates in play, of course. After all, the interface problem is a question about how minds work and so not one that could be answered on the basis of narrowly theoretical considerations.  We therefore need a second way of making progress: we need to turn rough ideas into hypotheses and to test their predictions. 


%%% Local Variables:
%%% TeX-master: "master"
%%% End:
