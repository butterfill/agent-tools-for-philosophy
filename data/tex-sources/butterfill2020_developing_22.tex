%!TEX root = master.tex

\part*{Conclusion}
\addcontentsline{toc}{part}{Conclusion}

\chapter{Conclusion}
\label{cha:conclusion}

% still TODO : *** (FB not age is important; restore preface with Hannah dialogue?.)


% \section{Conclusion: TODO}

% \begin{enumerate}
% \item inferential and intentional isolation
% % \item replace 'joint action' with 'social interaction'? Need to be clear that I'm not talking about primary intersubjectivity by more complex kinds of social interaction.
% % \item whether to talk about social interaction at all?
% \item Make logic of social interaction clearer. My view: why is social interaction important at all? Not: social interaction drives all cognition. Nor: social interaction is completely irrelevant. Rather: social interaction is important for the step from core knowledge to knowledge proper. I.e. it is important for the rediscovery step. It can facilitate rediscovery --- and is maybe even essential for it. So distinguishing between core knowledge and knowledge proper creates a potential role for social interaction. I.e. the second breakthrough provides an opportunity for understanding why the first breakthrough is so significant.
% \end{enumerate}



I started with a simple question.
How do humans first come to know about 
objects,
% causes,
% colours,
actions,
and minds?
On the view elaborated in this book,
the answer, whatever it ultimately turns out to be, will involve two primary ingredients, core knowledge and joint action, where the role of core knowledge is to make joint action possible.

This view falls out of an exploration of what has been discovered about the developmental emergence of knowledge  in two domains, physical objects (in \cref{part:physical-objects}) and minds and actions (in \cref{part:minds-actions}).
% We seem to be far from having settled answers to even basic questions about infants cognition of objects, minds and actions. 
My conclusion is that infants’ surprisingly sophisticated abilities concerning objects, minds and actions do not involve knowledge at all.
Instead they involve a combination of broadly perceptual, motoric and metacognitive states and processes, perhaps among other things.
These states and processes do not appear to change over development: they are also found in adults, and they underpin much the same abilities in adults as in infants.
The fact that these states are distinct from knowledge enables us to invoke them without circularity in explaining the developmental emergence of knowledge.

The early-developing abilities must somehow provide a basis for the later acquisition of knowledge.
But how?

Any attempt to answer this question faces a problem.
The states which underpin early-developing abilities are cut off from knowledge: in both adults and infants, they exhibit \gls{intentional isolation} from, and lack of \gls{inferential integration} with, knowledge.
These states influence the rest of the mind only indirectly, by their influences on metacognitive feelings, behaviours and other \glspl{intentional isolator}.
There is no prospect, therefore, of postulating operations on the contents of these states which transform them into the contents of knowledge states.
There can be no direct representational connections between the states underpinning early-developing abilities and knowledge states.
The emergence of knowledge has to be a process of rediscovery, of discovering as if for the first time things about objects, minds and actions which in some sense infants have long relied on in their interactions with the world.
While we know almost nothing about how rediscovery occurs, it is probably  not something infants achieve alone.
Rediscovery must be a joint action.
Just here we face an objection: on the leading, best developed views, joint action presupposes knowledge (and much further sophistication besides) and so could not explain its emergence in development.
One candidate solution is to reject the leading views in favour of an account of joint action in terms of \glspl{collective goal}.
Thinking in terms of collective goals allows us to see how infants’ early-developing abilities to track aspects of minds and actions provide a basis for the social skills that make rediscovery possible.

The developmental emergence of knowledge depends on nonsocial, early-developing abilities concerning objects, minds and actions; so cannot be understood in entirely social terms.
But nor can it be understood by ignoring infants’ interactions with others entirely, for the states underpinning those early-developing abilities are intentionally and inferentially isolated from knowledge.
This creates a special, as yet barely explored role for joint action in development.
Coming to know your first facts about particular objects, minds and actions is a process of rediscovery, which is a joint action.

These conclusions are controversial, and rest on some quite wild conjectures whose predictions are yet to be tested.
But even were all of them accepted, we would still be some way from answering the question about the developmental emergence of knowledge.
It is therefore worth highlighting some themes running through the discoveries and puzzles we have considered, with a view to how these might eventually inform constructing an answer to the question about how humans first come to know simple facts about particular objects, actions and minds.



\section{Infants Rely on Minimal Models \ldots}
A model is a way some part or aspect of the world could be.
Models can be used not only to characterise the world, but also to characterise how the world appears from the point of view of a particular individual (or processes within her).
A significant part of understanding the developing mind involves constructing models.
Spelke’s \gls{Principles of Object Perception} characterise a model of the physical (see \cref{cha:principles-object-perception}), and Gergeley and Csibra’s \gls{Teleological Stance} characterises a model of action (see \cref{cha:action}); similarly, minimal theory of mind identifies a model of minds and actions (see \cref{cha:mind-solution}).
Insofar as these models are \gls{descriptively adequate}, they enable us to see objects, minds and actions from an infant’s point of view.

One striking feature of all the models we have considered is that they are unsophisticated, being characterised by a few principles which are at best rough approximations to the truth.
These are minimal in the sense that no less sophisticated models have been described.

Unsophisticated and inaccurate models are often be useful in a limited range of circumstances. 
In the case of the physical, an impetus model of the physical works as well as any other as long as you remain on the surface of this planet and avoid launching things vertically; and a Newtonian model is fine if you are not travelling too fast.
Less sophisticated models often have an advantage for finite minds under time constraints: their use demands less cognitive power.
A minimal model can make possible more extreme speed--accuracy trade-offs that would otherwise be available.

Compared to adults, infants have limited memory, attention and inhibitory control. 
They also lack some of the external supports to thinking that adults benefit from: they cannot talk to themselves, write notes or make diagrams, for instance.
Further, their experience of the world is more circumscribed and they have had fewer opportunities to automatize patterns of acting and thinking.
These are all reasons why an infant might ideally be able to make a different speed--accuracy trade-off than would an reflective adult who is not under any time pressure.
This explains why we find that models involved in infants’ early-developing abilities concerning objects, minds and actions are minimal.

\section{\ldots\ As Do Adults, Sometimes}

Development does not seem to be a process of elaborating these models, turning the minimal into the more sophisticated.
In considering objects (\cref{sec:object-indexes-adults}), actions (\cref{sec:motor-theory-goal-tracking}) and minds (\cref{sec:signature-limits-mindreading}), we found evidence that some processes in adults rely on the very minimal models that underpin infant’s abilities.

These findings can also be explained by appeal to the role of models in enabling speed--accuracy trade-offs.
If minimal models merely became more elaborate throughout development, this would restrict the range of speed--accuracy trade-offs available to older children and adults.
Retaining the minimal models alongside more sophisticated models enables greater cognitive flexibility.

Just identifying a model counts as progress in understanding developing minds because it enables us to see things from an infant’s (and a cognitively-constrained adult’s) point of view.
But any time we identify a model, we are confronted with a challenge.
What links the model (or the principles we use to characterise the model) to the mind of an infant?
Answering this question is a challenge because of the puzzles developmental discoveries consistently throw up.



\section{Puzzles Matter}
\label{sec:puzzles-matter}
In tracing many discoveries about development, we have consistently been confronted by puzzles.
Where a scientist might view puzzles as a sign of failure, a philosopher is likely to see identifying and elucidating puzzles as an end in itself.

Any attempt to answer questions about when humans first know simple facts about particular objects, actions or minds by considering the existing discoveries is quickly confronted by a puzzle of this form:
some observations indicate that knowledge is in place in the first months of life and that there is no developmental transition between infancy and later years; whereas other observations indicate the contrary, and provide evidence for a developmental transition; further, the difference between the observations arises from apparently irrelevant factors such as a difference in response type measured (from pupil dilation to anticipatory looking, for example), timing constraints imposed,  or scenario used (occlusion to endarkening, or location to numerical identity, for example).

In each of the domains we considered---objects, minds and actions---there is a clear temptation to dismiss the puzzle by insisting that one or the other side involves extraneous methodological defects.
But attempts to identify these extraneous methodological defects have not been successful, and each side can point to a variety of impressive findings.

We should resist this temptation.
The findings appear puzzling only as long as we  assume, perhaps tacitly, that whatever states underpin the abilities being measured are like knowledge in that they must, in the absence of an impediment, be capable of guiding any response.
But this assumption is neither theoretically mandatory nor supported by much (if any) evidence.
The alternative, as we have seen, is to treat the puzzles as informative  both about the nature of the states and processes involved and also about the models that underpin these abilities.
The apparent puzzles, properly understood, may point to a signature limit of motor representation (which is unperturbed by brief endarkening but strongly affected when occlusion events involve impenetrable barriers; cref{sec:motor-representation-objects}), or of minimal models of mind (\cref{sec:signature-limits-mindreading}), or of perceptual animacy (which may enable pupil dilation but not anticipatory looking; \cref{sec:perceptual-animacy}), among other things.

The puzzles are so valuable because mere successes taken in isolation could be explained in indefinitely many ways, while an isolated observation of mere failure can be hard to distinguish from a null result.
The puzzles provide insights about limits on success. 
These limits are what allow us to falsify hypotheses about which models, states or processes underpin the successes.
This is the key to solving linking problems.


\section{Linking Problems Abound}

One theme running across all domains we have considered is that infants demonstrate surprisingly sophisticated abilities surprisingly early in life.
These enable them to manifest symptoms associated with knowledge,
which makes it tempting to conjecture (as many have) that infants’ abilities are based on knowledge or belief.
Such conjectures are versions of what I called the \gls{Simple View}.

One attraction of a Simple View is that it provides a straightforward answer to the problem about what links a model to the mind of an infant:
the principles which characterise the model are things the infant knows.

No Simple View is to be discarded lightly because there is invariably much going for it. 
In each domain we considered, the Simple View required no theoretical novelty, generated readily testable predictions and was supported by some evidence.
But versions of the Simple View do quite consistently generate incorrect predictions (as we saw in \cref{cha:linking-problem,cha:action,cha:mind}).
Infants manifest symptoms associated with knowledge. 
But whatever underpins these manifestations cannot be knowledge because there are limits on which responses it can guide or because it lacks the \gls{inferential integration} characteristic of knowledge (see \cref{sec:knowledge}).

At this point we are confronted with a linking problem (see \cref{cha:linking-problem}).
We have identified some principles that are \gls{descriptively adequate} to infants’ early-developing abilities in a domain.
If not knowledge or belief, what does link these principles to their minds?

As background to this question, we have been working with a crude picture of the mind (introduced in \cref{sec:crude-picture}).
The mind comprises at least three kinds of states and processes:
\begin{enumerate}
    \item epistemic (that is, knowledge-related), 
    \item motoric, and 
    \item perceptual.
\end{enumerate}
Although crude, this picture of the mind has the virtue that all three kinds of state postulated are needed by large bodies of theoretical work. 
If we seek to go beyond it by adding further kinds of state, we need to be sure that these novel kinds of state are characterised in ways that are both theoretically coherent and empirically motivated.

The failure of a Simple View invites us to go beyond the crude picture of the mind.
After all, infants’ abilities did not initially seem to be entirely a consequence of perceptual or motor representations.
So we can easily get the impression that the crude picture is insufficient for us to solve linking problems.
Yet existing attempts to go beyond the crude picture have been challenged on theoretical grounds, and they do not seem to generate novel predictions (as we saw in \cref{cha:linking-problem}).
That a large part of why linking problems are so difficult: they seem to require a way of going beyond the crude picture.


% Recognising that infants’ abilities concerning physical objects, minds and actions do not involve knowledge of particular facts concerning these this pushes us to go beyond the crude picture because 
% But does understanding the link between principles and infants’ minds  really require postulating a fourth kind of mental state?


\section{Core Knowledge Isn’t What You Think It Is}
% ∞TODO : replace with point that core knowledge is not uniform nor unified.
One attempt to solve linking problems is associated with the notion of \gls{core knowledge}.
This is often understood as a fourth kind of mental state, something distinct from knowledge proper as well as from perceptual and motor states.
In this case, postulating core knowledge means adding a novel kind of mental state, a ‘third [fourth] type of conceptual structure’ \citep[p.~10]{carey:2009_origin}.

I have argued against such a construal on the grounds that existing theoretical accounts of core knowledge do not pull their weight by generating novel predictions (\cref{sec:against-core-systems}); nor are things improved by appealing to standard theories of modularity.
None of these arguments are conclusive, of course.
But they do motivate seeking alternatives.
And, as we saw in \cref{sec:core-knowledge-minimal-view}, there is at least one alternative way of characterising core knowledge. 
This is a lighter (theoretically modest) approach which appears to require fewer bold commitments while being no less useful for tackling the puzzles.

On the lighter approach, ‘core knowledge’ is initially just a label for whatever it is that underpins early-developing abilities in a domain.
Issues like whether core knowledge is innate, encapsulated or unchanging across development are not treated as a matter of stipulation but left over for discovery.
We make progress by identifying core knowledge with states whose existence could be independently established.

On the lighter approach, core knowledge is not a solution to any linking problem.
It is a label for a solution yet to be identified (and perhaps a guess at the rough shape the solution will take).

None of this means that theories of core knowledge have nothing to offer.
Innateness aside (for reasons explored in \cref{cha:innateness}), many of the insights which prompted introducing core knowledge appear to be robust.
One key insight is that what underpins infants’ abilities also plays a role in cognition throughout life, and so can be found in adults \citep{Spelke:1994oz,Carey:1996hl}.
Another key insight is that core knowledge is not inferentially integrated with knowledge proper.
Perhaps these insights do not hold good across every domain in which infants manifest abilities, but they do seem to characterise those domains we have considered.
Accepting the lighter approach to core knowledge does not entail rejecting these insights: the point is just that the insights do not by themselves require or enable us to go beyond the crude picture of the mind by postulating novel kinds of mental states or processes.
The puzzles and the linking problems remain unresolved.


\section{How to Solve Linking Problems}
The inspiration for the approach to solving linking problems pursued here is the \gls{CLSTX Conjecture} (from \cref{sec:CLSTX-conjecture}). 
According to this conjecture, what links the \gls{Principles of Object Perception} to infants’ (and adults’) minds is a system of object indexes.
This conjecture does not require postulating novel kinds of mental states or processes.
The postulation of object indexes is independently motivated by discoveries and theories in nondevelopmental areas of cognitive science, and they are relatively well understood.

The success of the CLSTX conjecture suggests  an approach to solving linking problems: go beyond the crude picture only when there is independent reason to postulate a novel kind of state or process.
In line with this approach, we have considered conjectures invoking object indexes, motor representations of objects’ affordances, metacognitive feelings, motor representations of outcomes in action observation, and perceptual animacy.
Although none of these is strictly part of the crude picture of the mind, their postulation is independently motivated, they are relatively well understood, and recognising their existence  involves no radical departure from the crude picture.

The CLSTX Conjecture turned out to generate incorrect predictions, of course (\cref{sec:clstx-conjecture-incorrect}).
But the approach it inspired continued to work: successive revisions introduced \glspl{motor representation} concerning affordances (\gls{Conjecture O} from \cref{sec:revised-CLSTX-conjecture}) and \glspl{metacognitive feeling} (\gls{Conjecture Om} from \cref{sec:metacognitive-feelings-object-indexes}).
Similarly, it was possible to explain how principles about actions are linked to infant’s minds by appeal to a combination of motor representations and perceptual animacy.

We do not (yet) need to postulate novel kinds of representation or cognitive process in order to solve linking problems.


Whereas proponents of core knowledge standardly construe it as a novel ‘type of conceptual structure’ \citep[p.~10]{carey:2009_origin},
the evidence so far suggests that we can decompose core knowledge into perceptual and motoric constituents plus \glspl{metacognitive feeling} that are already familiar from research in nondevelopmental cognitive science (see \cref{sec:crude-picture}).%
\footnote{%
The exception is mindreading: we did not yet see evidence linking mindreading to perceptual or motoric processes.
My guess is that this is because the research is yet to be done.
We may eventually discover that the earliest developing abilities to track mental states are in fact a consequence of a combination of broadly perceptual and broadly motoric processes.
}
This implies that core knowledge lacks unity.
Rather than labelling one thing, core knowledge of objects turns out to comprise a variety of things including, among others, object indexes and motor representations.
Core knowledge also turns out to lack uniformity: the states and processes it labels in one domain are not the states and processes it labels in every other domain.
This lack of unity and uniformity is part of what makes theorising about development so hard.
What seems to be a single ability rarely involves just one kind of state or process (there is a lack of unity), and what goes for one domain rarely goes for another (there is lack of uniformity).


As I mentioned back in \cref{cha:intro,cha:linking-problem}, Davidson expresses scepticism about the possibility of solving linking problems:
\begin{quote}
    ‘if you want to describe what is going
    % //p.~128//
    on in the head of the child when it has a few words which it utters in appropriate situations, \emph{you will fail}
    for lack of the right sort of words of your own. We have many vocabularies for describing nature when we regard it as mindless, and we have a mentalistic vocabulary for describing thought and intentional action; what we lack is a way of describing what is in between’ 
    \citep[pp.~127--8; my emphasis]{Davidson:2001sm}.
\end{quote}
%
I think Davidson is wrong, although I appreciate the encouraging prognosis.%
\footnote{%
You might suspect Davidson’s target is not scientists at all, but the context indicates otherwise. Just after this passage (p.~128) he reports, ‘I am thankful that I am not in the field of developmental psychology.’
}
Combining minimal models with insights into perceptual, motor and metacognitive processes  enables us to describe even what is going on in the heads of wordless infants.


\section{Representation: Handle with Care}

The conjectures we have considered all involve representation in one way or another. 
% Or (in case object indexes and metacognitive feelings are not representational) things associated with representation, at least.
As we saw in \cref{sec:representation-not-knowledge},
\citet{Haith:1998aq} claims that ‘no concept causes more problems in discussions of infant cognition than that of representation.’
I disagree; or at least I don’t think it is at all necessary for the concept of representation to cause problems.
We can avoid problems by holding in mind three uncomplicated theoretical points.

First, representing is distinct from tracking. Not all tracking involves representing (\cref{sec:can-we-reject}). 
What is more directly observed in an experiment is usually tracking rather than representing. 
Supporting conclusions about representation using  observations concerning tracking is difficult but not impossible.
To make such an inference, your hypothesis about representation has to generate novel predictions. That is, it has to add to the theory rather than merely providing a way of restating a claim about tracking.
The method of signature limits (from \cref{sec:signature-limits}) provides one way to get from tracking to representing.

Second, hypotheses about representation should avoid, whenever possible, postulating representation without postulating a particular kind of representation (see \cref{sec:representation-not-knowledge}). 
Representations are always perceptual, motoric, epistemic, photographic, cartographic, or whatever.
The mind can no more contain bare representations than you can hang a shape on your wall without hanging a rectangle, circle or some other particular shape.
It is sometimes but only rarely necessary to introduce an unspecified kind of representation.
Doing so is always a mark of ignorance.
We can see this by contrasting the comparatively rich predictions that can be made about infants’ abilities concerning actions, where conjectures about particular kinds of representation are possible (recall \gls{Conjecture MP} from \cref{sec:theory-of-goal-tracking}), with the sparser predictions that can currently be made about mindreading because we do not yet know which kinds of representations underpin mindreading abilities.

Third, there is a three-fold distinction between formally, descriptively and \gls{explanatorily adequate} (\cref{sec:for-simple-view}).
When characterising some infants’ abilities it is often useful to have a model of the domain.
The model is often identified by a set of principles or a theory.
When thinking about the model in relation to the infant’s abilities, we can regard it as merely \gls{formally adequate}; that is, someone who used the model as a guide to reality, was otherwise omniscient and had unlimited cognitive resources could use it to manifest the ability.
Or we may go further and state that the model is \gls{descriptively adequate}; this is, it it enables us to predict the extent and limits of the infants’ abilities.
Nothing yet follows concerning representation.
It is only in considering the {explanatory adequacy} of the model that we may need to invoke representation.
Of course things will quickly get messy if we try to make claims about representation on the basis of findings which speak only to formal or descriptive adequacy. 
But that is just a consequence of introducing theoretical claims which are not empirically motivated.
None of this suggests there is anything especially problematic about the concept of representation.






\section{Inferential and Intentional Isolation}
Solving puzzles about the patterns of infants’ and adults’ successes and failures, and the associated linking problems, might enable us to describe what is going on in infants’ minds; but it leaves our overall question unanswered. 
The question was how humans first come to know simple facts about particular objects, actions and minds.

In attempting to answer this question, we are confronted by a dilemma concerning \gls{intentional isolation} and lack of \gls{inferential integration}.

On one horn, core knowledge (whatever exactly that turns out to be) must somehow contribute to the developmental acquisition of knowledge.
It is tempting to think that this might involve some process of inference from core knowledge to knowledge, or at least an operation that transforms contents of core knowledge states into contents of knowledge states.
Existing theories proceed along just these lines (\cref{sec:objects-rediscovery}).
How else a transition from core knowledge to knowledge proper might occur remains a mystery.

On the other horn, core knowledge is invoked in part to solve puzzles about the patterns of infants’ and adults’ successes and failures (\cref{sec:puzzles-matter}).
This indicates (but does not demonstratively require, of course) that core knowledge is cut-off from knowledge and other states: its influence on them must be limited in ways that preclude \gls{inferential integration}.
Otherwise the apparently puzzling patterns of performance could hardly persist into adulthood; and theories invoking core knowledge would generate the same incorrect predictions that versions of the Simple View do.
Further, the effects of core knowledge may involve metacognitive feelings, behaviours and other intentional isolators.
If, as I suggest, only intentional isolators can connect core knowledge to knowledge proper, then core knowledge is both inferentially and intentionally isolated from knowledge.
This is inconsistent with the idea that a transition from core knowledge to knowledge proper could involve an operation transforming the contents of one kind of state into the (parts of) the contents of another.

My guess is that inferential and intentional isolation is real.
Further, it is an architectural virtue and not a deficit.
Any complex system with human-like capacities for error faces a trade--off between robustness and flexibility.
Robustness requires a conservative attitude towards learning and change, whereas flexibility requires not being constrained by heuristics and assumptions in responding to new information even at the risk of introducing large and potentially fatal errors.
By having \glspl{core system} isolated from knowledge systems, it is possible to make two different trade-offs simultaneously.
If a philosopher can convince herself that tables, chairs and people are merely artefacts of her own mind, or even that the universe consists of herself alone, this will not directly impair her thanks to inferential and intentional isolation.%
\footnote{%
But if you invite her to dinner, be careful who you put next to her.
}
The isolation allows core knowledge to be ultra conservative, ensuring a basic level of functioning in the world, while knowledge proper makes the complementary trade-off.

Understanding the developmental emergence of knowledge requires understanding how core knowledge contributes despite its isolation.
Achieving this takes us outside the infant’s head and into its social world.



% \section{Rediscovery and Joint Action}

% Here we face an obstacle.
% There are some sophisticated and well-supported theories of infants’ abilities to engage in joint actions, communicate referentially and use words.
% But on almost any of these theories, infants’ abilities presuppose much of the knowledge whose emergence in development might otherwise be explained by these abilities.
% On these theories, infants’ heads must be filled with knowledge or beliefs before they can perform joint actions and communicate referentially.

% This obstacle can be overcome.
% For it is possible to construct
% minimal models
% of joint action
% and referential communication
% which do not presuppose that infants already have the relevant knowledge,
% and therefore allow us to conjecture that these abilities may play a role in the developmental emergence of knowledge.

\section{Rediscovery Is Joint Action}
Dramatic discoveries about core knowledge are rarely thought about in connection with  infants’ social skills and the role these play a role in cognitive development. 
This is understandable.
After all, whatever the truth about innateness turns out to be, there is no indication in any of the discoveries we have considered that core knowledge is in any interesting sense a social phenomenon.
This is a major challenge to any suggestion that all human cognitive development depends on social interaction in ways that development in other eusocial animals does not.

Reflection on inferential and intentional isolation motivates
reconsidering.
If the role of core knowledge in the emergence of knowledge proper can only be indirect, what could core knowledge do?
Perhaps core knowledge of objects, minds and actions enables infants not only to interact with physical objects but also to perform joint actions involving \glspl{collective goal} with the people around them.
And perhaps these joint actions are somehow implicated in explaining how humans first acquire knowledge of simple facts about particular things.

This idea is so underspecified it is only just worth mentioning.
To work it out in any detail we would need to consider many further domains in which infants manifest abilities, including the moral, chromatic, numerical, spatial, communicative, and linguistic.
We would also to know much more about joint action in development.
But we have made a step forwards in seeing the shape a story about the developmental emergence of knowledge might take.
Core knowledge and joint action both need to feature in this story, and  one role for core knowledge is to enable joint actions, which in turn allow humans to rediscover what is in some sense already encoded in their core knowledge.

% It is conceivable, just, that the processes of rediscovery which explain the  emergence of knowledge in development are joint actions.




% There is just one small obstacle to attempt to tell a story along these lines.
% On the best-developed, leading theories, infants’ social abilities presuppose quite sophisticated knowledge and intentions.
% On these theories, there is no hope of appealing to social interaction or communication to explain much about the developmental emergence of knowledge because they already presuppose so much.

% This obstacle can be overcome by invoking some of the research associated with core knowledge.
% The combination of perceptual, motor and other structures make possible a limited but useful range of joint actions activities.
% And these structures do not involve knowledge and are not inferentially integrated with knowledge.
% So it is a mistake to assume that developmentally basic social skills depend on knowledge, belief or intention.
% Recognising this allows us to invoke one-year-olds’ abilities to do things together without presupposing that they already have sophisticated bodies of knowledge of physical and social aspects of the world.



% \section{The Approach: Modest and Conservative}
% \label{sec:modest-conservative-minimal}

% In attempting to solve developmental puzzles, I have been proposing an approach that is theoretically modest.
% Bold claims about modularity, core knowledge and innateness are yet to be supported by much developmental evidence, as we saw in \cref{part:physical-objects} and the Interlude.
% Nor do they seem necessary for making theoretical progress, as they generate few distinctive predictions (\cref{sec:against-core-systems}). 
% Most of the benefits can be had at lower cost by adopting a \gls{dual process theory}.
% Such a theory aims implies that two or more processes exist but is not committed in advance to any particular view about how the processes differ: this is left as a matter for discovery rather than stipulation.

% I propose that the theoretically modest dual process theories of object cognition, goal-tracking and mindreading capture what is useful in invoking core knowledge or modularity while excluding things that are not empirically motivated by any of the considerations we have in view.
% Of course we will eventually need less theoretically modest theories.
% But the route to establishing these will involve discoveries about the processes, like the discovery that some of infants’ earliest abilities concerning physical objects involve a system of object indexes (see \cref{sec:object-indexs-princ}).
% And whereas theories of core knowledge or modularity involve claims about the mind in general, we may find that there is a lack of uniformity across domains.

% The approach I have proposing is not only modest: it is also theoretically conservative.
% We do not (yet) need to postulate novel kinds of representation or cognitive process. 
% Whereas proponents of core knowledge construe it as a novel ‘type of conceptual structure’ \citep[p.~10]{carey:2009_origin},
% the evidence suggests that we can decompose core knowledge into perceptual and motoric constituents that are already familiar from the crude picture of the mind (see \cref{sec:crude-picture}) plus \glspl{metacognitive feeling}.%
% \footnote{%
% The exception is mindreading: we did not yet see evidence linking mindreading to perceptual or motoric processes.
% My guess is that this is because the research is yet to be done.
% We may eventually discover that the earliest developing abilities to track mental states are in fact a consequence of a combination of broadly perceptual and broadly motoric processes.
% }

% The conservative approach may eventually reach its limit, forcing us to postulate novel kinds of representation.
% (I hope this happens sooner rather than later: it will be fascinating.)
% But it is likely to leave one legacy:
% our initial, naive division of abilities into domains such as representing objects as persisting or tracking actions may turn out to bear little relation to the underlying cognitive systems.
% What appears to be one thing may turn out to be two (as I argued in the case of mere target tracking contrasted with proper goal tracking; see \cref{sec:perceptual-animacy});
% and what appear to be distinct things may turn out to be one (if, for example, abilities to track targets of action and to represent briefly unperceived objects are both consequences of a system of object indexes).

% There is no reason to think that the best developmental theories will eventually turn out to be modest or conservative.
% The only reason for favouring theories with these characteristics now is methodological.
% Too little is known to justify bold claims about innateness, core knowledge or modularity; nor are we in a position to identify novel kinds of representation in ways that generate readily testable predictions about development.
% For now we can learn most by pushing theoretically modest and conservative approaches to their limits.
 













%%% Local Variables:
%%% TeX-master: "master"
%%% End:
