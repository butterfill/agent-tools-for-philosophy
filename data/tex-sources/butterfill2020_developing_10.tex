%!TEX root = master.tex

\chapter{Conclusion to Part I}
\label{cha:conclusion-part-objects}


There are two main theories about the nature of infants’ earliest capacities concerning physical objects.
According to one, these capacities involve beliefs about, or knowledge of, general principles governing objects’ behaviours.
According to the other, understanding these capacities requires postulating a novel kind of representation (or ‘conceptual structure’), something knowledge-like which is not actually knowledge (\citealp[][p.~10]{carey:2009_origin}; see \cref{sec:core-knowledge}).
Both theories are wrong, or so the arguments of \cref{part:physical-objects} attempted to show.

Instead, four-month-olds’ abilities are based on a combination of object indexes, motor representations of objects and metacognitive feelings of surprise.
None of these three things necessarily involves having any beliefs about, or knowledge of, physical objects.
But nor are they novel kinds of representation---each of the three is familiar from comparatively established theories about adult cognition.
Infants’ sophistication is real but does not involve knowledge: there is still so much to be explained about the developmental emergence of knowledge of physical objects.
On the other hand, the puzzling patterns of performance infants exhibit do not justify postulating novel kinds of representation.
The novel feature of infants’ abilities concerning objects is the way object indexes, motor representations and metacognitive feelings work together.

None of this quite answers the question we started with: how do humans make the amazing transition from not knowing any facts about particular physical objects to knowing some facts?
Before evaluating our progress with respect to this question, I first want to return to a question that has been outstanding since \cref{sec:segmentation}.


\section{What Is an Expectation?}
\label{sec:expectation-surprise}

Much developmental research hinges on assumptions about infants’ expectations and things that surprise them, as we saw throughout \cref{cha:principles-object-perception,cha:simple-view}.
But what are these things?
What is an expectation, and what is surprise?

Many philosophers take surprise to involve awareness of your own beliefs.
For example, Davidson stipulates that to be surprised that there is no coin in my pocket,
\begin{quote}
  ‘[i]t is not enough that I first believe there is a coin in my pocket, and after emptying my pocket I no longer have this belief. Surprise requires that I be aware of a contrast between what I did believe and what I come to believe. Such awareness \ldots\ is a belief about a belief’
  \citep[p.~104]{Davidson:2001sm}.
\end{quote}
%
By contrast, developmental scientists use the term ‘surprise’ with no commitment to any such view.
For example, \citeauthor{wang:2004_young} stipulate that
\begin{quote}
  ‘[t]he term  surprise is used here simply \ldots\ to denote a state of heightened
    attention or interest caused by an expectation violation’ \citep[p.~168]{wang:2004_young}.
\end{quote}
%
On this view, surprise is a subjective consequence of an expectation being violated and need not involve any awareness that the expectation has been violated.
The scientists’ characterisation of surprise raises more questions than it answers.
What is this state caused by an expectation violation?
And what is an expectation, in this context?

Given the argument for \gls{Conjecture Om} (see \cref{sec:metacognitive-feelings-object-indexes}),
we can answer these questions by appeal to object indexes, motor representations of objects and metacognitive feelings.
Operations involving object indexes or motor representations depend on physical objects appearing to behave roughly in accordance with \glslink{Principles of Object Perception}{certain principles}.
For infants to have an \emph{expectation} about the ways physical objects behave is just for the thing expected to be required if physical objects are to behave in accordance with these principles.
And \emph{surprise} is a metacognitive feeling which occurs when a violation of this expectation causes an error in the system of object indexes.

This way of characterising expectations and surprise is at odds with ordinary ways of using the words ‘expectation’ and ‘surprise’.
In everyday life, an expectation might be something like a belief: to expect rain is to believe it will rain, or will probably rain, for example.
By contrast, the technical notion of expectation used in developmental science is best understood as nothing like a belief.
In that technical sense, for an event to be expected by you is just for its occurrence to be accordance with principles characterising perceptual or motor processes in you.
Similarly, surprise in everyday life involves there being something you are surprised that.
You are surprised that it did not rain, for example.
By contrast, the metacognitive feeling of surprise has no comparable intentional object: it is not surprise about anything (see \cref{sec:intentional-isolation}).
Just as you can experience feelings of anger without there being anything you are angry about, so having a metacognitive feeling of surprise need not involve being surprised that anything is the case.

This (admittedly speculative and unorthodox) view about expectations and surprise has a radical consequence.
Suppose \gls{habituation} and \gls{violation-of-expectation} experiments really do measure metacognitive feelings of surprise, as just suggested.
Then these methods should not be regarded as revelatory about what subjects believe or know.
Why not?
Because all the available evidence about metacognitive feelings indicates that they are triggered by monitoring the fluency of processes like facial recognition, action selection, and the like; none of it points to reflective inferential processes operating on beliefs or knowledge states as potential causes of metacognitive feelings.
This suggests that habituation and violation-of-expectation results are likely to reflect what is going on at a deeper level than that at which humans are making inferences and acquiring beliefs or knowledge.




\section{Core Knowledge: A Lighter Account}
\label{sec:core-knowledge-minimal-view}

One consequence of \gls{Conjecture Om} concerns \gls{core knowledge}.
If the conjecture is correct, we must either abandon the claim that infants have \gls{core knowledge} of objects or else recognise that core knowledge of objects lacks unity, being a composite of at least three things that are, to an interesting degree, independent of each other: object indexes, motor representations of objects, and metacognitive feelings of surprise.
Which path should we take?
This may be a largely terminological matter.
Carey, one of those responsible for coining the term ‘core knowledge’, takes the \gls{CLSTX Conjecture} about \glspl{object index} to be consistent with her views on core knowledge.
A proponent of core knowledge could take the same attitude towards \gls{Conjecture Om}.

Recall that \gls{core knowledge} is characterised by invoking a list of properties including innateness, encapsulation, and the rest (see \cref{sec:core-knowledge}).
The view that core knowledge actually has these properties faces three lines of objection.
First, there is no compelling theoretical reason to assume that these properties should hang together (as we saw in \cref{sec:core-knowledge-is-badly-defined}).
Second, there is scarce evidence that the things comprising core knowledge actually have these properties (for consideration of innateness, see \cref{cha:innateness}).
Third, and most pressing, not all of the properties appear to be relevant to explaining infants’ abilities and the developmental emergence of knowledge (see \cref{sec:against-core-systems}).

We can avoid all three lines of objection by taking a theoretically lighter approach to characterising \gls{core knowledge}.
Start with a minimally informative definition of core knowledge.
In the domain of objects, let core knowledge of physical objects be whatever it is that underpins infants’ abilities to segment objects, represent them as persisting and track their causal actions.
And similarly for other domains.
Discover the things that actually underpin the abilities---which, in the case of physical objects, are object indexes, motor representations and metacognitive feelings, at least according to \gls{Conjecture Om} (see \cref{cha:causation}).
Call these \emph{constituents} of core knowledge.
% dual process
Next, check we are justified in distinguishing core knowledge from knowledge proper.
To this end, establish whether processes operating on the constituents are distinct from the inferential processes in which knowledge states feature.
They should be distinct in this sense: the conditions that influence whether these processes occur or what outputs they generate, and the conditions under which these process result in certain kinds of behaviours, do not completely overlap.
In the case of physical objects, there is abundant evidence for distinctness (see \vref{table:occlusion-vs-endarkening} for a partial summary).
Finally, investigate the constituents of core knowledge to discover whether core knowledge exhibits characteristics like encapsulation and innateness.

On the theoretically lighter approach, whether \gls{core knowledge} is innate, exhibits informational encapsulation or arises from phylogenetically old systems is a matter for discovery rather than something assumed in advance.

In the case of physical objects,
we do already have evidence that core knowledge exhibits an interesting degree of informational encapsulation.
This is particularly clear for one of its constituents, object indexes, assignments of which can conflict with a subject’s beliefs (see \cref{sec:is-CLSTX-core-knowledge}).
The conjectured role of \glspl{metacognitive feeling} in connecting operations on object indexes (and motor representations) to knowledge states also indicates that \gls{inferential integration} will be limited or absent. 
As \glspl{intentional isolator},    \glspl{metacognitive feeling} provide a noninferential link between core knowledge of physical objects and knowledge proper.

Taking the lighter approach means we should be cautious in drawing conclusions about core knowledge generally. 
There is no guarantee that, assuming it exists at all, core knowledge in other domains will exhibit the same properties as core knowledge of physical objects.

%By contrast, there is scarce evidence concerning the innateness (or not) of these things.

% The truth of \gls{Conjecture Om} reveals what this core knowledge is.
% To fully make sense of patterns of success and failure in tracking briefly unperceived objects, it seems that we need to recognize that core knowledge of objects is a hybrid phenomenon.
% It involves at least three things that are, to an interesting degree, independent of each other: motor representations, object indexes and metacognitive feelings.

% As mentioned earlier (in \cref{sec:core-knowledge}), Carey suggests that core knowledge is a ‘third type of conceptual structure’ \citep[p.~10]{carey:2009_origin}.
% But if \gls{Conjecture O} is right, then
% either there is no such thing as core knowledge of physical objects, or else it is a hybrid of object indexes and motor representations which influence thought and action indirectly, for instance by way of metacognitive feelings.
% Either way, no ‘third type of conceptual structure’ is needed to explain development.



\section{Development Is Rediscovery}
\label{sec:objects-rediscovery}
How do humans get from the abilities concerning physical objects manifested at four- and five months of age to being fully fledged knowers?
How do they make the transition to knowing simple facts about physical objects, such as the fact that there are two mice behind that screen?

Leading contemporary answers to this question rely on
what I will call \emph{\gls{Assumption of Representational Connections}}:
%
\begin{quote}
The transition from not knowing any simple facts about physical objects at all to knowing some such facts
involves operations on the contents of core knowledge states, which transform them into (components of) the contents of knowledge states.
\end{quote}
%
This Assumption is required for
Spelke’s suggestion that mature understanding of objects, number, and mind derives from core knowledge by virtue of core knowledge representations being assembled \citep{Spelke:2000nf}.
It is also required for claims by Leslie and others
that modules provide conceptual identifications of their inputs \citep{Leslie:1988ct};
and for Karmiloff-Smith’s representational re-description
\citep{Karmiloff-Smith:1992lv}.
It is even required for Mandler’s claim
that ‘the earliest conceptual functioning consists of a redescription of perceptual structure’ \citep{Mandler:1992vn}.

% Most proposals rely on this assumption, including:
% (i) Spelke’s suggestion that mature understanding of objects derives from core knowledge by virtue of core knowledge representations being assembled (\citeyear{Spelke:2000nf}); (ii) claims by Leslie and others that modules provide conceptual identifications of their inputs \citep{Leslie:1988ct}; (iii) Karmiloff-Smith’s representational re-description (\citeyear{Karmiloff-Smith:1992lv}); and (iv) Mandler’s claim that ‘the earliest conceptual functioning consists of a redescription of perceptual structure’ (\citeyear{Mandler:1992vn}).

The view currently under consideration, \gls{Conjecture Om},  requires us to reject the \gls{Assumption of Representational Connections}.
On  the present view,
four- and five-month-old infants’ abilities to track briefly unperceived depend on three things:
%object indexes, motor representations and metacognitive feelings.
a system of object indexes, a capacity to represent objects motorically, and metacognitive feelings.
So if there is core knowledge of physical objects at all, it is a hybrid of these three (see \cref{sec:core-knowledge-minimal-view}).
Further, having metacognitive feelings is not a matter of being intentionally related to physical objects (other than perhaps oneself, of course).
And only metacognitive feelings and other \glspl{intentional isolator} link operations on object indexes to beliefs and knowledge states.
If this is right, it is not true that operations on the contents of core knowledge states somehow transform them into components of knowledge states.
Metacognitive feelings insulate any representational features of core knowledge from intentional features of beliefs and knowledge states.
The Assumption of Representational Connections must therefore be rejected.

This makes the question about development particularly
difficult to answer.
It means that rather than being a matter of assembling or redescribing representations, development must be a process of rediscovery.
Coming to know simple facts about physical objects is a matter of rediscovering things that are already core knowledge, or at least already implicit in the operations of a system of object indexes.



Some might object that development could not require such rediscovery because it would be hopelessly inefficient to require things already encoded to be learnt anew.
But rediscovery is an elegant solution to a practical problem.
If you are building a survival system you want quick and dirty
heuristics that are good enough to keep it alive: you don’t
necessarily care about the truth.
If, by contrast, you are building a thinker, you want her to be able to think things that are true irrespective of their survival value.
This cuts two ways.
On the one hand, you want the thinker’s thoughts not to be
constrained by heuristics that ensure her survival.
On the other hand, in allowing the thinker freedom to pursue the
truth there is an excellent chance she will end up profoundly
mistaken %(Malebranche?)
or deeply confused %(Hegel?)
about the nature of physical objects.
So you don’t want thought contaminated by survival heuristics and you don’t want survival heuristics contaminated by thought.
Or  if some contamination is inevitable, you at least want to limit it.
You want beliefs and knowledge states to be inferentially isolated from survival heuristics including any core knowledge.
This is beautifully achieved by giving your thinker systems for tracking objects and their interactions which appear early in development, and also a mind which allows her
to acquire knowledge of physical objects gradually over months or years, taking advantage of interactions with objects as well as social interactions about objects—providing, of course, that the
two are not directly connected but rather linked only very loosely, via \glspl{intentional isolator} like metacognitive feelings.


\section{How Does Rediscovery Occur?}

Discoveries that information about particular physical objects is already somehow carried in four-month-olds’ core knowledge make it initially tempting to guess that it will be easy to get from core knowledge to knowledge proper.
But matters are complicated by the conjecture that only \glspl{intentional isolator} such as \glspl{metacognitive feeling} link core knowledge to beliefs (at least in the case of physical objects).

The step from core knowledge to knowledge proper is like the step from feeling electric shocks to understanding electricity.
Receiving electric shocks might alert you that there is something to be discovered but does not reveal much about the nature of electricity.
Like electric shocks, metacognitive feelings of surprise may function to trigger ‘stop-and-think’ responses to events---except that, unlike electric shocks, metacognitive feelings occur where there is an unexpected lack of fluency in mental processing and so probably potential for learning.
But in isolation, a metacognitive feeling does not tell you anything much about what could be learnt.

 
I mentioned earlier (in \cref{sec:metacognitive-feelings-ex1}) Koriat’s proposal that
 ‘metacognitive feelings \ldots\ allow a transition from the implicit-automatic mode to the explicit-controlled mode of operation’
\citep[p.~150]{koriat:2000_feeling}.
Although he wasn’t talking about development, I think we can productively misinterpret what he says as about development.
Metacognitive feelings have a dual role. By triggering ‘stop-and-think’ responses to
events which interfere with automatic processing, they may create opportunities for learning.
But because they are intentional isolators, they also serve to keep 
the later-developing, less automatic processes separate from the more automatic, 
early-developing processes.
So they both ‘allow a transition’ of one kind and prevent a transition of another kind.

The challenge we face, then, is to explain how rediscovery might occur.
Coming to know simple facts about particular physical
objects may begin with object indexes and the
metacognitive feelings these give rise to, but it does not end there.
Coming to know simple facts about physical objects may involve interacting with others around you who already have knowledge.
This is one reason for investigating infants’ abilities to track others’ minds and actions, and to act together with others.

% objects, learning to use tools, and perhaps engaging with others and objects simultaneously in joint actions.




% \section{*TODO: consider integrating}
% % ∞TODO

% In particular, it predicts that \glspl{signature limit} of a system of object indexes are also \glspl{signature limit} of four-month-olds’ abilities to segment objects, represent them as persisting and track their causal interactions (see \cref{sec:signature-limits}).
% In short: no object indexes, no physical objects.
% It also predicts that where object indexes do not give rise to metacognitive feelings, infants should be unable to manifest their abilities concerning physical objects in \gls{violation-of-expectation} experiments.
% In short: no metacognitive feelings, no physical objects.
% And it predicts that patterns of success and failure in representing briefly unperceived objects will be influenced in specific ways by differences in mode of disappearance as well as whether an occluder or a barrier to action is used (see \cref{sec:revised-CLSTX-conjecture})


% ALSO some general lessons:

% 1. not only abilities but also limits matter (you can’t solve the Linking Problem or distinguish systems without limits)

% 2. identify infant capacities with known adult mechanisms to make progress

% 3. the Linking Problem is not  entirely a conceptual problem; contra Davidson we do not lack the concepts to describe the mind of an infant [already in conclusion]



%%% Local Variables:
%%% TeX-master: "master"
%%% End:
