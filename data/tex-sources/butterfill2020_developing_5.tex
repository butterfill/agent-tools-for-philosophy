
%!TEX root = master.tex

% DONE use : it’s not a conceptual problem, contra Davidson; or if it is, it’s one that has already been solved.
% \citep[pp.~127--8]{Davidson:2001sm}:
% The difficulty in describing the emergence of mental phenomena is a conceptual problem: it is the difficulty of describing the early stages in the maturing of reason, the stages that precede the situation in which concepts like intention, belief, and desire have clear application.
%In both the evolution of thought in the history of mankind, and the evolution of thought in an individual, there is a stage at which there is no thought followed by a subsequent stage at which there is thought. To describe the emergence of thought would be to describe the process which leads from the first to the second of these stages. What we lack is a satisfactory vocabulary for describing the intermediate steps. We are able to describe what a preverbal child does by employing the language of neurology, or in crude behavioristic terms we can describe movements and the sounds emitted.
%You can deceive yourself into thinking that the child is talking if it makes sounds which, if made by a genuine language user, would have a definite meaning. (It is even possible to do this with chimpanzees.) But words, like thoughts, have a familiar meaning, a propositional content, only if they occur in a rich context, for such a context is required to give the words or thoughts a location and a meaningful function. If a mouse had vocal cords of the right sort, you could train it to say 'Cheese'. But that word would not have a meaning when uttered by the mouse, nor would the mouse understand what it 'said'. Infants utter words in this way; if they did not, they would never come to have a language. But if you want to describe what is going
% //p.~128//
% on in the head of the child when it has a few words which it utters in appropriate situations, you will fail for lack of the right sort of words of your own. We have many vocabularies for describing nature when we regard it as mindless, and we have a mentalistic vocabulary for describing thought and intentional action; what we lack is a way of describing what is in between. This is particularly evident when we speak of the 'intentions' and 'desires' of simple animals. We have no better way to explain what they do.

\chapter{The Linking Problem}
\label{cha:linking-problem}

\noindent
How do humans first come to know simple facts about particular physical objects?
Knowing any such facts probably involves being able to segment physical objects, to represent them as persisting, and to track their causal interactions.
In \cref{cha:principles-object-perception} we saw that infants can do all of these things from around four months of age or earlier.
We also saw that there is a single set of principles, the \gls{Principles of Object Perception}, which describe how infants do these things.
This led us to ask, What links these principles to infants’ minds?
Our current answer is the \gls{Simple View}.
According to the Simple View, infants know or believe the Principles of Object Perception.
And they use these principles in inferring facts about physical objects, their locations, movements and interactions, from sensory information about features.
The Simple View promises an attractively straightforward answer to the question about how humans first come to know facts about particular physical objects: they do so in a way that they might later come to know facts about economic values or legal obligations, namely by inference.

This chapter is about why the Simple View is wrong, and about the problem we face when we reject the Simple View, which I’ll call the
\gls{Linking Problem}.
The Simple View is wrong because it makes incorrect predictions (\crefrange{sec:against-simple-view}{sec:even-worse-for-the-simple-view}).
The problem arising from its failure, the Linking Problem, is to provide an alternative account of what links the \gls{Principles of Object Perception} to infants’ and others’ minds and so explains their abilities to segment objects, represent them as persisting and track their casual interactions (\cref{sec:the-challenge}).
If not belief or knowledge states, what does link the principles to particular minds?

Devoting a whole chapter to explaining a problem might seem excessive, even given that some theoretical foundations will be laid along the way.
But versions of the Linking Problem arise for all domains of knowledge.
It is a pervasive but often overlooked obstacle to understanding how knowledge emerges in humans.




\section{Against the Simple View}
\label{sec:against-simple-view}

Some philosophers have used intuitions to develop sophisticated arguments against the Simple View.
\citet{Bermudez:2003dj}, for instance, argues that those without the ability to use a language cannot make inferences;
and Davidson argues that those without language cannot think at all.%
\footnote{%
According to Davidson, knowledge, intention, desire and all forms of thought depend on belief \citep[pp.~210--1]{Davidson:1995lk}; ‘[h]aving a belief demands \ldots\ appreciating the contrast between true belief and false ’
\citep[p.~209]{Davidson:2001sm}; and ‘we grasp the concept of truth only when we can communicate the contents---the propositional contents---of the shared experience, and this requires language’ 
\citep[p.~27]{Davidson:1997wj}.
}
It may be hard to accept that four-month-old infants are in the business of inferring truths about particular objects’ locations from abstract principles.
(And perhaps it is no less hard to accept that adults typically do this in segmenting objects.)
%It is not obvious that the same process is occurring in infants.
But scientific and mathematical discoveries sometimes require us to reject intuitions, perhaps even deeply held intuitions about very fundamental things like space and time.
For this reason there seems to be slim prospect of effectively challenging the Simple View on the basis, ultimately, of intuitions about the nature of knowledge, belief and inference.
In any case, doing so is also unnecessary.
For there are excellent scientific reasons for rejecting the Simple View.

% *(The basic idea is to say there's a discrepancy regarding BOTH (a) permanence and (b)
% causal interactions)

Recall \citeauthor{baillargeon:1987_object}’s experiment which used a rotating drawbridge to show that four month olds can represent objects as persisting even while briefly hidden from their view.
(The scenarios used in this experiment are depicted in \vref{fig:baillargeon_1987_fig1}).
As we saw, converging evidence comes from other experiments using different scenarios and different means of detecting infants’ responses to them.
The means include \glslink{habituation}{dishabituation}, looking times as a marker of \gls{violation-of-expectation}, and anticipatory looking; there is also a study involving neural measures (\citealp{kaufman:2005_oscillatory}; this study will concern us in \cref{cha:causation}).
But what happens if instead of measuring infants’ looking or neural responses, we instead measure how they search for objects?

\citet{Shinskey:2001fk} did just this.
They created apparatus much like that used by \citet{baillargeon:1987_object} in her drawbridge studies.
There was an opaque screen that could rotate between lying flat on the ground and being raised to conceal a toy behind it (as depicted in the bottom of \vref{fig:shinskey_2001_fig1}).
The key difference between this study and Baillargeon’s was that the infants could choose to pull the screen forwards, towards themselves and thereby reveal whatever is behind it.
\citeauthor{Shinskey:2001fk} also used a second piece of apparatus just like the first except that the screen was transparent rather than opaque.
They reasoned that infants would quite often pull the screen forwards just for fun, regardless of what is behind it.
However, they also guessed that when infants know there is an interesting toy behind the screen, then they will pull it forwards more often than when they know that there is nothing behind the screen.
This is just what happened when infants were presented with the apparatus involving a transparent screen:
they sometimes pulled the screen forwards when there was no toy behind it, but they pulled it forwards significantly more often when the toy was behind it (see the left bars, labelled ‘transparent’, in \vref{fig:shinskey_2001_fig2a}).
What happened  when infants were presented with the opaque screen?
Here infants pulled the screen forwards no more often when they had observed a toy being placed behind it then when they had observed that there was nothing behind it  (see the right bars, labelled ‘opaque’, in \vref{fig:shinskey_2001_fig2a}).
This is evidence that  seven-month-old infants do not know that a toy they have very recently seen hidden behind a screen is behind the screen.
After all, since knowledge guides action we would expect infants who know that a toy is behind an opaque screen to pull the screen forward more often than infants who know there is nothing behind the screen, just as they do when the screen is transparent.

%
\begin{figure}
\begin{center}
\includegraphics[width=0.7\textwidth]{fig/shinskey_2001_fig1.png}
\caption{
  \label{fig:shinskey_2001_fig1}
  The transparent and opaque barrier apparatus used in an experiment showing that seven-month-olds will not search for an object after it is hidden from their view.
  Source: \citet[][figure 1]{Shinskey:2001fk}.
}
\end{center}
\end{figure}
%

%
\begin{figure}
\begin{center}
\includegraphics[width=0.5\textwidth]{fig/shinskey_2001_fig2a.png}
\caption{
  \label{fig:shinskey_2001_fig2a}
  Seven month old infants’ actions indicate that they are unaware of a toy after it is hidden behind an opaque screen.
Note that what matters is not the absolute percentage of screen pulls but whether there is significant increase in screen pulls when the toy is present behind the screen.
When the screen is opaque, there is no such increase.
  Source: \citet[][figure 2, part]{Shinskey:2001fk}.
}
\end{center}
\end{figure}
%

This is a problem for the Simple View.  The Simple View predicts, correctly, that infants’ looking behaviours will reveal sensitivity to the locations of physical objects that are momentarily out of view.
But the Simple View also predicts, incorrectly, that the same infants’ searching behaviours will likewise reveal sensitivity to the locations of physical objects.
In fact, even much older infants, seven month olds, do not search for hidden objects.
The Simple View generates an incorrect prediction.

Our knowledge that infants fail to act on objects hidden behind a barrier or a screen does not rest on just one study.
I described \citeauthor{Shinskey:2001fk}’s experiment because it is so elegant,
but more than two decades of research strongly supports the view that infants fail to search for objects hidden behind impenetrable barriers or screens until around eight months of age \citep[p.~202]{Meltzoff:1998wp} or maybe even later \citep{moore:2008_factors}.
% (We will encounter some apparently conflicting evidence in \cref{cha:causation}.)

Someone might attempt to rescue the Simple View by postulating factors which would prevent infants’ knowledge from manifesting itself in their searching for the hidden object.
The idea is that infants do know that the toy is behind the opaque screen but they are prevented from searching for it by some extraneous factor.
It is difficult to identify any such extraneous factor, however.
Could it be that infants are unable to prepare or perform the searching action?
\citeauthor{Shinskey:2001fk} rule out this possibility by including the transparent screen apparatus.
Since both opaque and apparent apparatus demand the same actions, the difference in infants’ behaviours cannot be due to difficulties in acting.
As Shinskey argues on the basis of some further experiments, ‘action demands are not the only cause of failures on occlusion tasks’ \citep[p.~291]{shinskey:2012_disappearing}.
Further evidence that infants’ failure to search for a hidden object are not due to difficulties preparing or performing a search action is provided by \citet{moore:2008_factors} who compare infants’ abilities to search for partially hidden and fully hidden objects.

Could it be that infants do know the toy is behind the opaque screen but are prevented from reaching for it by the twin of demands of having to hold their knowledge in mind and to prepare an action?
For comparison, suppose you are asked to hold an eight digit number in mind just a moment after your phone has rung.
Even if you know where the phone is, you might be briefly prevented from reaching for it by the sheer difficulty of holding the digits in mind.
Could something similar be true of infants?
Could the nonvisibility of the object place demands on memory that consume cognitive resources also required for preparing action?
\citet{moore:2008_factors} reasoned that if this were right, having the toy make a noise continuously should help infants to remember it and so boost their performance.
But they found that eight-month-olds failed to search for a toy irrespective of whether it made a noise \citep[Experiment 2]{moore:2008_factors}.
This suggests that memory is unlikely to be the only limiting factor.

There is also evidence that, when an object is not hidden by an impenetrable screen, infants even younger than the seven-month-olds \citeauthor{Shinskey:2001fk} tested can cope with the twin of demands of having to hold their knowledge in mind and to prepare an action.
%We know this can’t be right because infants even younger than the seven-month-olds \citeauthor{Shinskey:2001fk} tested can reach for objects they can’t currently perceive.
To demonstrate this, \citet{shinskey:2012_disappearing} hid objects in a tray of milk.
She found that six-month-olds reach into the milk significantly longer after observing as a toy is hidden in the milk than after not observing this.
Relatedly,
\citet{jonsson:2003_infants} compared infants’ reaching to an object in darkness with their reaching to an object momentarily hidden by an impenetrable barrier.
They found that six-month-olds will not reach for an object hidden by a barrier but will reach for one hidden by darkness (see also \citealp{hespos:2009_occlusion,babinsky:2011_infants}).
%(As you might guess, infants’ reaches to currently unperceived objects are not as well controlled as adults’; see \citealp{babinsky:2011_infants}.)
The fact that seven-month-olds can, in some conditions, reach for an object they are not currently perceiving indicates that demands on holding knowledge of an unperceived object’s location in mind do not explain why they fail to retrieve objects from behind a barrier.
It seems, then, that infants’ failures to act really do  indicate that they do not know where an object hidden behind a barrier is.
And this means we must reject the Simple View.


To recap, according to the Simple View four-month-olds know where an object recently hidden behind an opaque screen or barrier is by four months of age at the latest.
This view correctly predicts infants’ responses to the presence of an object behind an opaque screen in studies involving \gls{habituation}, \gls{violation-of-expectation} and anticipatory looking methods.
But it also predicts that the same infants will search for an object recently hidden behind an opaque barrier.
This prediction is false.
Infants will not do this until around eight months of age at the earliest.
We could rescue the Simple View if we could identify an extraneous factor which prevented infants from manifesting their knowledge in searching for objects hidden behind an opaque barrier.
But identifying any such extraneous factor is difficult because infants have no difficulty in searching for an object behind a barrier when the barrier is transparent, and they have no difficulty searching for an object when it is hidden by milk or darkness rather than by an impenetrable barrier.
Despite much effort, no one has published a convincing explanation of infants’ failures to search that is consistent with the claim that they know where an object recently hidden behind an opaque screen is.
For this reason we should reject the Simple View.

Or should we?
Even without being able to explain away the Simple View’s failure, you might refuse to abandon it on the basis of just one incorrect prediction.
%But there is more.

% *** TODO : check this!  Haven’t even read in pdf! ***

\section{Further Evidence Against the Simple View}
\label{sec:furth-evid-against}

We have just seen that the Simple View makes an incorrect prediction about infants’ abilities to represent objects as persisting.
The Simple View also makes an incorrect prediction about infants’ abilities to track causal interactions, or so this section will argue.
Taken together, the two incorrect predictions provide strong grounds for rejecting the Simple View.

Consider a scenario in which a ball rolls down a ramp.
The ball is stopped by a barrier, which can be placed in various positions along the ramp.
In front of most of the ramp and barrier there is a screen with four doors in it.
The screen is tall enough to hide the last part of the ball’s journey, but the barrier sticks up over the screen making it easy to predict where the ball will stop---at least, this is easy for adults.
The barrier is always placed after one of the four doors in the screen, as depicted in \vref{fig:hood_2003_fig1}.
Where will infants expect the ball to stop?

%
\begin{figure}
\begin{center}
\includegraphics[width=0.9\textwidth]{fig/hood_2003_fig1.jpg}
\caption{
  \label{fig:hood_2003_fig1}
  A ball rolls down a ramp and hits a barrier.  Which door will it end up behind?
  Source: \citet[][figure 1]{Hood:2003yg}.
}
\end{center}
\end{figure}
%

If you recall \citeauthor{spelke:1992_origins}’s experiment in which a ball drops vertically behind a screen with a solid bench in its path (as depicted in \vref{fig:spelke_1992_fig2a}),
you might guess that even four-month-old infants will correctly expect the ball to stop in front of the barrier.
This is probably half right.
In one experiment \citet{Hood:2003yg} showed two- and three-year-olds the ball rolling down the ramp.
They then opened a door to reveal the ball.
Some of the time they contrived to retrieve the ball from the wrong door, as if by magic.
In these cases the children looked significantly longer than when the ball was retrieved from the correct door.
This indicates that these children expected the barrier to stop the ball.
\citet[Experiment 1]{mash:2006_what} refined \citeauthor{Hood:2003yg}’s procedure.
Rather than sometimes retrieving the ball from an incorrect location, they simply opened doors.
But they contrived things in such a way that sometimes the ball was not behind the correct door.
They found that when the correct door was opened, infants looked longer when no ball was behind it than when the ball was behind it.
This is evidence that infants anticipate the ball’s location.
It is not just that seeing the ball in the wrong place enables them to realise something is wrong: merely seeing the ball’s absence is enough to confound their expectations.

So far these findings accord with those discussed earlier (even if the subjects are  two or three years old rather than four months), and with the predictions of the Simple View.
But the useful  feature of the rolling ball scenario is that it allows us to investigate infants’ expectations using two different measures.
What happens if we test where two-year-olds expect the ball to stop not by measuring looking times but instead by asking them to open a door to retrieve the ball?

When \citet{Berthier:2000eu} did this, they found that two-year-olds did not open the correct door more often than someone opening doors at random would.
Some children selected a favourite door which they always opened; others always opened a door adjacent to the barrier but had no preference for the door on the correct side of the barrier.
This failure to search for the ball in the correct location is evidence that two-year-olds do not expect the solid barrier to stop the moving ball and do not know where the ball is at the end of its journey.

Because this is such an unexpected finding, \citeauthor{Berthier:2000eu}’s experiments have been replicated and extended by several labs.
\citet{Hood:2003yg} gave the same individuals both \gls{violation-of-expectation} and search tasks using the same apparatus.
All children looked longer at apparently impossible events in which a ball was retrieved from a wrong location,
but few of their two-and-a-half-year-olds reliably searched for the ball in the correct location.
\citet{Butler:2002bv} modified the screen in front of the ramp so that it was entirely transparent except for the doors.
This meant that children could observe the ball rolling between the doors.
Amazingly, this had only a modest effect on two-year-olds’ performance: most (70\%) still did not reliably search for the ball in the correct location.
\citet{keen:2008_toddlers} explored three different ways of getting children to focus on the barrier, reasoning that younger children might fail the search task simply because they were not paying attention to the barrier.
It turned out that none of these much improved the children’s performance in searching for the ball.
The truth is that while four-month-olds’ looking behaviours indicate sensitivity to the fact that solid barriers stop moving objects, two-year-olds’ searching actions indicate ignorance of this fact.

The same discrepancy between looking and searching as evidence for abilities to track causal interactions has been found in adult nonhuman primates, specifically cotton-top tamarins \citep{santos:2006_cotton-top}.
Related discrepancies have also been found in other adult nonhuman primates \citep{gomez:2005_species, santos:2009_object}.%
\footnote{%
Not all nonhumans have difficulties in searching for unperceived objects.
Dogs have no difficulty using solidity when searching for an object  \citep{kundey:2010_domesticated};
and young chicks, unlike human infants \citep{Shinskey:2001fk}, will search for an object hidden behind a barrier \citep{chiandetti:2011_chicks_op}.
Primates may be special in finding it difficult to search for currently unperceived objects.
}
This suggests that discrepancies between looking and search behaviours are not a consequence of extraneous factors or capacity limits.
After all, such discrepancies should never occur in fully developed adults of any species if they were.
Discrepancies between looking and search behaviours in some adult primates therefore strengthen the case against the Simple View.

% ‘A similar permanent dissociation in understanding object support relations
% might exist in chimpanzees. They identify impossible support relations in looking tasks,
% but fail to do so in active problem solving.’
% \citep{gomez:2005_species}

% ‘to date, adult primates’ failures on search tasks appear to
% exactly mirror the cases in which human toddlers perform poorly.’
% \citep[p.~17]{santos:2009_object}



\section{Things Get Even Worse for the Simple View}
\label{sec:even-worse-for-the-simple-view}
All the evidence against the Simple View considered so far arises from patterns in infants’ abilities to perform purposive actions.
By contrast, much of the evidence in favour of the Simple View comes from paradigms which involve eye movements that are either very simple purposive actions or not purposive actions at all.
This invites a possible defence of the Simple View.
Could tasks requiring nonocular purposive actions be less sensitive measures of infants’ capacities?
If so, the Simple View might still be broadly supported by the available evidence.

The problem with this line of defence is that infants’ abilities are sometimes manifest in tasks requiring reaching but not in looking time measures.
% Things are not quite so straightforward.
% A further challenge for the Simple View comes from evidence that five-month-olds not only sometimes fail to search for hidden objects but also sometimes fail to look longer when a momentarily hidden object fails to reappear as if by magic.
As mentioned earlier (in \cref{sec:against-simple-view}), infants will reach for an object hidden in darkness; given the Simple View, this indicates that they are able to know the object’s location even while it is hidden.
But what happens if instead of measuring reaching we measure looking times?
\citet{charles:2009_object} compared what happens when an object is momentarily hidden behind a screen with what happens when an object is momentarily  hidden by darkness.
They used a trick with light and mirrors so that for some of the infants, the object did not reappear when the screen came up or the light returned.
Surprisingly, five-month-old infants’ looking times indicated that an expectation had been violated only when the object was hidden behind a screen and not when it was hidden by darkness.
% This finding is a further difficulty for the Simple View:

Apparently, then, tasks requiring eye movements are not more sensitive than tasks which require manual actions.
Depending on the scenario used, infants in their first five or so months of life will fail to reveal capacities to track physical objects in either kind of task.

The Simple View therefore generates incorrect predictions not only about manual search behaviours but also about performance on violation\-/of\-/expectation tasks, as \vref{table:occlusion-vs-endarkening} summarises.
If, as infants’ searching behaviours suggest, they know roughly where the object hidden by darkness is, why is this knowledge not manifest in their looking times?
And if, as infants’ looking behaviours indicate, they know roughly where an object occluded by an impenetrable barrier is, why is this not manifest in their manual actions?


\begin{table}

\begin{center}
\footnotesize	%shrink for better spacing

% add space between rows
\extrarowsep=7pt

\begin{tabu} to 0.8\linewidth {X[3,l] X[1,c] X[1,c]}

\toprule

& occlusion & endarkening
\\
\cmidrule(r){2-3}
violation-of-expectation & \checkmark & $\times$
\\
manual search & $\times$ & \checkmark
\\
%
\bottomrule
%
\end{tabu}
\caption{Limits on five-month-olds’ abilities to track briefly unperceived objects.}
\label{table:occlusion-vs-endarkening}
\end{center}	%careful -- position of this affects distance between table and caption(!)

\end{table}
\normalsize


The case against the \gls{Simple View}, although not decisive, is compelling.
According to the Simple View,
humans know or believe principles which partially characterise the boundaries, movements and interactions of physical objects, and they are able to use these principles to infer truths about the locations of objects.
This view does make correct predictions in some cases,
but it also makes systematically incorrect predictions about infants’ and children’s purposive actions.
If the Simple View were correct,
seven-month-olds should be able to retrieve objects from behind an opaque barrier,
and they should be able to correctly locate a ball whose path is manifestly blocked by a solid barrier.
They should also look longer when a briefly endarkened object fails to reappear.
That they cannot do any of these things is a compelling reason to reject the Simple View.

Infants in the first months of life do manifest symptoms associated with knowledge about particular physical objects. 
But this is not the same thing as their actually knowing things (\cref{sec:knowledge}).

% If this is right, how far does the evidence take us? 
% Should we not only reject the Simple View but also doubt even the claims that infants can segment objects, represent them as persisting and track their interactions?
% I think this would be a mistake.
% The evidence against the Simple View comes from different methods and paradigms than those which provided positive evidence for infants’ abilities.
% Hypothesising that five-month-olds can represent objects as persisting (say) does not mean predicting that this ability will be manifest in everything those infants do.
% It is only the Simple View that generates this prediction and causes all the trouble.


% ∞todo : come back to these questions
Rejecting the Simple View leaves us with a cluster of questions.
How else can we explain four-month-olds abilities to segment objects, represent them as persisting and track their causal interactions?
Why do infants manifest these abilities when they are measured in some ways but not others?
How else are the \gls{Principles of Object Perception} linked to infants’ minds?
And since it is not by inference from principles known or believed, how else is it that humans first come to know simple facts about particular physical objects?
These are all aspects of the \gls{Linking Problem}.



\section{The Linking Problem}
\label{sec:the-challenge}
Infants’ abilities to segment physical objects, to represent them as persisting even while  briefly unperceived, and to track their causal interactions can be \glslink{descriptively adequate}{described} by the \gls{Principles of Object Perception} (see \cref{sec:principles-object-perception}).
But what links these principles to infants’ minds in such a way as to explain these abilities?

Consider the three most familiar kinds of mental representation, which are perceptual, motoric and epistemic. (These were introduced in \cref{sec:crude-picture}).
Perceptual representations are those involved in perceiving.
It may seem obvious, and it is often more or less taken for granted, that infants’ abilities to represent physical objects as persisting even while  briefly unperceived cannot be explained by appeal to perceptual states.
After all, much of the point of considering  briefly unperceived objects is to rule out the possibility that infants’ abilities are merely perceptual.

What about \glspl{motor representation}? Could these explain infants’ abilities concerning physical objects?
Motor representations are the states involved in preparing and performing very small-scale actions like reaching for a cup, grasping it, transporting it to your mouth and drinking from it (see \cref{sec:crude-picture,sec:motor-theory-goal-tracking} for more on motor representations).
Objects are indeed sometimes represented motorically (see \cref{sec:motor-representation-objects}),
but usually not when they are unavailable to act on, as, for instance, when they are behind an impenetrable barrier.
This makes it unlikely that four-month-olds’ abilities concerning physical objects could be entirely a consequence of motor representations.


Eliminating perceptual and motor representations leaves us with the third familiar kind of mental representation, epistemic.
This includes things like knowledge, belief, desire and intention.
The \gls{Simple View} is the view that four-month-old infants’ abilities are underpinned by epistemic states.
We know this is probably false because, as we have just seen, the \gls{Simple View} generates multiple incorrect predictions.

Our question is, What links the \gls{Principles of Object Perception} to infants’ (and perhaps others’) minds?
Identifying the link is essential for explaining how they are able to segment physical objects, represent them as persisting and track their causal interactions.
And it is a problem---call it the \emph{\gls{Linking Problem}}---because it seems we cannot identify the link by invoking any of the three familiar kinds of mental representations.

Davidson might be interpreted (or constructively misinterpreted) as describing the Linking Problem in a passage reflecting on philosophical challenges posed by development:
%
\begin{quote}
  ‘The difficulty in describing the emergence of mental phenomena is a conceptual problem: it is the difficulty of describing the early stages \ldots\ that precede the situation in which concepts like intention, belief, and desire have clear application. \ldots\

   We have many vocabularies for describing nature when we regard it as mindless, and we have a mentalistic vocabulary for describing thought and intentional action; what we lack is a way of describing what is in between’
    (\citealp[pp.~127--8]{Davidson:2001sm}; compare \citealp[p.~11]{Davidson:1999ju}).
  \end{quote}
%
The Linking Problem arises in attempting to explain the emergence of knowledge, which is a mental phenomenon.
The failure of the \gls{Simple View} indicates that Davidson is right insofar as solving the \gls{Linking Problem} requires a way of identifying something that is ‘in between’ mindless behaviour and epistemic states such as knowledge or belief.
But Davidson takes it to be a ‘conceptual problem’.
And he takes the problem to be unsolved, perhaps even unsolvable.
In what follows we will see grounds for thinking that the Linking Problem is not a narrowly conceptual problem as it can be solved, in at least one domain, by careful attention to scientific discoveries (see \cref{cha:causation}).

% ∞todo move this?
% The problems we face in thinking about knowledge of physical objects are ones that will recur in thinking about other domains.
% In many domains, a variety of evidence is most straightforwardly interpreted by ascribing knowledge to infants of a certain age---and yet further evidence shows that infants of this age cannot have such knowledge,
% and nor, apparently, can their abilities be explained by appeal to perceptual or \glspl{motor representation}.
% This is true not just for knowledge of physical objects, which is our current concern, but also for knowledge of minds, actions and other domains besides.
% Perhaps correctly interpreting the evidence requires us to postulate states in addition to the most familiar kinds, namely perceptual, motoric and epistemic. 




\section{Representation Not Knowledge}
\label{sec:representation-not-knowledge}
It may be tempting to think that the \gls{Linking Problem} is trivial or already solved.
Don’t scientists already have a perfectly good vocabulary for describing what is in between mindless behaviour and knowledge or belief?  Isn’t talk about representation supposed to serve just this purpose?

Thinking along these lines, we might aim to retain many of the virtues of the \gls{Simple View} by switching from talking about knowledge and belief to talking about representation.
Instead of saying, as the Simple View does, that the \gls{Principles of Object Perception} are things four-month-olds know or believe, we can say instead that they represent these principles.
And, relatedly, we can say that these principles feature in inference-like cognitive processes that lead to representations of facts about particular physical objects, such as facts about their boundaries, locations and interactions.
Since not all kinds of representation are linked to purposive action in the way that knowledge and belief are, this more cautious view does not generate the false predictions that the Simple View generates.

% Note that the issue of representation comes up twice for us.
% There is a question about whether the principles of object perception are represented.
% And there is a question about whether objects, their locations, properties, and interactions are represented.
% The problem raised by the discrepancy between looking and acting is a problem for two claims: (i) the simple view (the principles of object perception are knowledge \&c); and also (ii) the claim that the representations of objects which derive from the principles of object perception are knowledge states.

Invoking representation instead of knowledge or belief is a good move insofar as it enables us to avoid a view that clearly generates incorrect predictions.
But it is not a move that, all by itself, will solve the problems arising from the failure of the Simple View.
This is because invoking representation instead of knowledge or belief merely amounts to switching from a more concrete view that provides candidate explanations and generates testable predictions to a less concrete view that explains little and predicts less.

Representation is a generic notion.
Photographs, maps, blueprints and sentences are all representations, and all have mental counterparts.
If you wish, want, hope, intend, imagine, guess, fear, believe or know that Sam will have an easy birth, then you represent this.
Little can be predicted about your behaviour from the bare fact of your representing that Sam will have an easy birth.
\citet{Haith:1998aq} claims that ‘no concept causes more problems in discussions of infant cognition than that of representation.’
The first step towards avoiding these problems is to recognise how little can be predicted or explained with the bare fact that someone represents something.

Imagine observing as Ayesha knocks her would-be mugger out cold with a single, well-aimed swing of her handbag.
Guessing that Ayesha has loaded her handbag with rocks, you pick it up yourself but discover that it feels much too light to contain rocks.
You must change your guess about what is in her handbag (unless, of course, you are willing to postulate some extraneous factor interfering with the normal effects of gravity on rock).
Changing your guess from rocks to physical matter is a step forward insofar as you are no longer obviously wrong.
But it is also a step backward insofar as your new guess does not provide an interesting candidate explanation for the knock-out blow.
It is barely controversial that Ayesha’s handbag contains something physical.
Of interest is which kind of physical thing has this combination of  low density and clout.
Representation is a generic notion in the way that physical matter is.

Where psychologists tend to use ‘representation’, philosophers are perhaps more likely to talk about ‘tacit’ or  ‘implicit’ knowledge.
Despite much research on this topic stretching across decades (\citealp{Stich:1978wb} is a particularly brilliant, pioneering example)
and some sustained attempts taking a variety of different approaches (including \citealp{Dienes:1999lh}, \citealp{Davies:1989gg}, \citealp{Sperber:1997io} among many others), it appears that we have all made quite limited progress in characterising such notions.
My own sense is that the most we can safely say is that tacit or implicit knowledge is like knowledge but lacks some of the features associated with it.
If this is our position, switching from a claim about knowledge to one about tacit or implicit knowledge is like switching from knowledge to representation.
These switches are ways of marking our ignorance about what links the Principles of Object Perception to the minds of individuals.
They do not solve the Linking Problem.

What kind of representations (if any) do four-month-old infants have of physical objects?
What is it about this kind of representation that enables them to give the appearance of knowing things about physical objects on some tasks even while failing other tasks which, apparently, test for the same knowledge (see \cref{table:occlusion-vs-endarkening} on page \pageref{table:occlusion-vs-endarkening})?
What is the relation between the representations of this kind, which infants have by at least four months of age, and the knowledge of physical objects which they lack until months or years later?
Until we can answer these questions, we have not solved the \gls{Linking Problem}.




\section{Graded Representations?}
\label{sec:grad-repr}
Faced with the difficulty of solving the \gls{Linking Problem}, we might reasonably attempt to duck it.
Does explaining four-month-olds’ abilities to segment objects, represent them as persisting and track their causal interactions really require any kind of mental representation that is not knowledge?

A radical way to duck \gls{Linking Problem} would be to attempt to explain four-month-old infants’ abilities without relying at all on  conjectures  about knowledge or mental representations.
This idea is outlined by \citet{smith:2005_cognition} and \citet[][]{schoner:2007_dynamic}.
As these authors admit, the approach they are aiming to elaborate is not currently well understood and their theories are at a comparatively early stage of development.
Appeal to mental representation may eventually turn out to be as mistaken as invoking vital forces or aether.
But while there is much intriguing theoretical and experimental research covering isolated phenomena, at present there is probably no good alternative to starting with mental representations if our aim is to understand how knowledge emerges in development.

% (By the way, \citet{Munakata:2001ch} is a nice review of dissociations, not only developmental dissociations.)

A different way to duck the \gls{Linking Problem}
has been proposed by \citeauthor{Munakata:2001ch} (\citeyear{Munakata:2001ch}; see also \citealp{munakata:1997_rethinking}).
She suggests that knowledge can be ‘graded’: some knowledge states are ‘stronger’ while others are ‘weaker’.
She also holds that weaker knowledge states can drive looking time behaviours but not control purposive action.
This allows her to hold that four-month-olds know principles about objects generally and facts about particular objects,
just as adults do.
On this view, infants’ representations of objects are not different in kind from adults’ knowledge of facts about particular physical objects.
% \citep{shinskey:2010_something} specify that the graded representations idea is supposed to be an alternative to postulating core knowledge.

The idea that knowledge can be graded is initially attractive.
It appears to avoid the incorrect predictions of the \gls{Simple View}, yet provides a simple solution to the \gls{Linking Problem}.
But this idea also faces a challenge.
Talk about ‘strength’ in this context needs to be anchored in a theory of representation.
When talking about a radio signal (say), it is possible to specify what signal strength amounts to, and it is clear that signals of varying strength can all carry the same message.
But what does strength amount to in the case of a mental representation or knowledge state?

Without an answer to this question, invoking graded representations will not explain anything.
It amounts merely to retrospectively postulating a novel aspect of representation to characterise findings about what four-month-olds can and cannot do.

To see the force of this challenge, consider that proponents of graded knowledge hold that weaker representations can guide many looking behaviours, whereas manual search behaviours generally require stronger representations.
Why is it this way around?
Why is a stronger representation needed for manually searching than for looking?
And why is a stronger representation needed when objects are occluded rather than merely endarkened?
Until we can answer such questions, postulating graded knowledge will not explain the developmental puzzles about knowledge of objects.

Another problem is that infants at around four or five months of age do not always fail to manually search for objects, nor do they always succeed in manifesting an appearance of knowledge in their looking behaviours (see \cref{table:occlusion-vs-endarkening} on page \pageref{table:occlusion-vs-endarkening}).
In solving the Linking Problem is it not enough to explain why infants sometimes fail to search.
We must also explain why they sometimes succeed in searching yet fail in looking.

One approach to explicating what strength is might be to equate it with confidence.
What is confidence?
To illustrate, consider Ayesha who knows she has never visited Milan and also that she has never been to the moon.
While she knows both things, she is much more confident about the second.
Her confidence is reflected in the fact that she would risk more if offered an opportunity to bet on whether she has ever been to the moon.
Can we equate strength with confidence?
This is not the view of any proponent of graded representations, but since we already know that it is necessary to postulate degrees of confidence, equating strength with confidence would avoid introducing a new aspect of representation.

Unfortunately equating strength with confidence is unlikely to work.
This is because differences in confidence are unlikely to explain why infants’ representations of momentarily hidden objects influence many of their looking behaviours but have no effect on many of their manual searching behaviours.
Consider, for instance, a two-year-old infant who can open one of several doors and will get a reward if she opens the door with an object behind it (as in the paradigm of \citealp{Berthier:2000eu} discussed in \cref{sec:furth-evid-against}).
If she knows which door the ball is behind, then she should not be systematically opening the wrong door however low her confidence might be.
So equating strength with confidence would mean that invoking graded representations generated incorrect predictions.

How else might the notion that representations vary in strength be explicated?
An ambitious move would be to anchor the notion that representations can be stronger or weaker in a connectionist model of infants’ performance on some tasks involving briefly hidden objects \citep[see][p.~698]{munakata:1997_rethinking},
or in facts about the neural basis of representations \citep[see][p.~309]{Munakata:2001ch}.
Success in doing this has the potential to provide a robust theory of graded representations, one likely to generate many distinctive predictions.
Unfortunately there are several challenges facing this approach.
The first is to explain which patterns in neurons’ firing determine strength.
The second challenge is to explain why, when strength is understood in terms of patterns in neurons’ firing, more strength should generally be necessary for manual search behaviours than anticipatory looking.
And the third challenge is to explain which feature of representations corresponds to the patterns in neurons’ firing.
Currently none of these challenges have been addressed in detail.
% Recognising this, \citet[p.~309]{Munakata:2001ch} suggests that ‘[a]t a conceptual level, representations can be graded in terms of how ‘clean’ they are for signaling the appropriate information, as opposed to being corrupted by noise or damage.’

% * Is \citet{shinskey:2010_something} offering a prediction generated by the view?  It is really hard to work out why they think changes in preferences concerning novelty are predicted by the graded representations view and a problem for the core knowledge alternative!

As things stand, appeal to the notion that knowledge states can vary in strength does not enable us to duck the \gls{Linking Problem}.
Nor is there yet any substantial prospect of explaining infants’ abilities without appeal to mental representations of some kind.
Of course the challenges facing approaches to ducking the \gls{Linking Problem} might eventually be overcome.
But for now our best hope of understanding how knowledge of simple facts about physical objects emerges in development is probably to face the \gls{Linking Problem} head on.

% talk about graded representations may point in the direction of what will eventually be a better alternative to postulating core knowledge.

% * Despite the difficulties, no better alternative to postulating core knowledge is currently available.
% (*but aren’t I about to suggest an alternative?  Just say that graded representations aren’t a better alternative.)
% This is not good given the challenges theories of core knowledge face.

\section{Conclusion}

In order to understand how humans first come to know simple facts about particular physical objects we need to understand infants’ abilities to segment objects, represent them as persisting and track their causal interactions.
An initial attempt, the Simple View, generates incorrect predictions.
The problem with the Simple View is that it entails that infants who can segment objects, represent them as persisting and track their causal interactions thereby also know facts about the locations, movements and interactions of particular physical objects.
But there is considerable evidence that they do not.
The former abilities can exist in the absence of  any corresponding knowledge about physical objects, as we saw in sections
\ref{sec:against-simple-view} and \ref{sec:furth-evid-against}.

The failure of the Simple View means we are confronted with the \gls{Linking Problem}.
We have to identify a link between principles that describe infants’ abilities concerning physical objects and their minds.
How can we do this?

A first idea was to switch from knowledge to representation: instead of saying that infants know or believe things about particular physical objects, we might say merely that they represent them.
This probably enables us to avoid falsehood, but it won’t enable us to explain the origins of knowledge.
We need more (see \cref{sec:representation-not-knowledge}).
What is the nature of infants’ earliest representations of physical objects?
In particular, what is it about their representations in virtue of which they appear to manifest knowledge when tested in some ways while appearing to lack knowledge when tested in other ways?
And what is the relation between these early representations and the knowledge of facts about particular physical objects which appears later in development?

It is just conceivable that we can duck the \gls{Linking Problem} by invoking the idea that knowledge can have different grades of strength and weakness.
But, as we saw (in \cref{sec:grad-repr}), the challenges that would have to be overcome in providing an explanatory account of graded representations capable of generating useful predictions are probably at least as daunting as the challenges involved in facing the \gls{Linking Problem} head on.

What next? The most influential, best developed attempt to solve the Linking Problem involves the notion of core knowledge. 

%%% Local Variables:
%%% TeX-master: "master"
%%% End:
