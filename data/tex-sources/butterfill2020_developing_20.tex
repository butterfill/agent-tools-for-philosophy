%!TEX root = master.tex


\chapter{Communication with Words}
\label{cha:words}


How do humans first come to communicate with words?
Very young infants do not use words to communicate at all.
But from around their first birthday you can often observe infants using single words like ‘mama’, ‘nana’, ‘throw’, and ‘cold’.
% \citet[p.~45]{Tomasello:2003fk}: ‘The first words that children learn and use include exemplars from almost all of the major parts of speech from adult language: proper nouns, common nouns, pronouns, verbs, adjectives, adverbs, prepositions, and so forth.’
Infants typically use even their first words not only to make demands but also to provide spontaneous commentary, which is sometimes accompanied by pointing and other nonverbal forms of communication.
We want to understand how humans make the developmental transition  from communicating entirely without words to communicating with words.


In this chapter we will consider two influential proposals.
By the end you should have a sense of why they are both wrong.

\section{Training Theories}
\label{sec:davidson-thought-language}
According to Davidson, if someone can think then she can communicate with words.
Davidson never quite offered an argument for this position but it follows from two ideas that are central to his thinking.
The first idea is that thinking requires understanding the possibility of being wrong and, in particular, understanding the possibility of having false beliefs.
As Davidson puts it,
%
\begin{quote}
‘Having a belief demands \ldots\ appreciating the contrast between true belief and false \ldots\ Someone who has a belief about the world—or anything else—must grasp the concept of objective truth, of what is the case independent of what he or she thinks’
\citep[p.~209]{Davidson:2001sm}.
\end{quote}
%
Here Davidson is writing about belief, but all forms of thought depend on belief.
You cannot know that something is the case, desire or intend it without having a belief \citep[pp.~210--1]{Davidson:1995lk}.
% \footnote{%
% Compare \citet[pp.~210--1]{Davidson:1995lk}:
% ‘These mental attributes are \ldots\ equivalent: to have a concept, to entertain propositions, to be able to form judgements, to have command of the concept of truth. \ldots\  In order to be right or wrong, one must know that it is possible to be right or wrong.’
% }
Davidson’s second idea is that understanding the possibility of having false beliefs requires being able to communicate by language:
%
\begin{quote}
‘we grasp the concept of truth only when we can communicate the contents---the propositional contents---of the shared experience, and this requires language’ \citep[p.~27]{Davidson:1997wj}.
\end{quote}
%
Putting Davidson’s two ideas together tells us that you cannot think unless you can communicate with words.%
\footnote{%
Incidentally, Davidson also holds that you cannot act unless you can communicate with words.
On his view, all actions are intentional \citep{Davidson:1971fz},
and
‘[i]ntentional action cannot emerge before belief and desire, for an intentional action is one explained by beliefs and desires that caused it’
\citep[p.~127]{Davidson:2001sm}.
}

But how could someone who cannot think acquire her first words?
% Many philosophers suggests that acquiring your first words is a matter of forming habits an associations.
% According to Russell, ‘A child learning to speak is learning habits and associations which are just as much determined by the environment as the habit of expecting dogs to bark and cocks to crow’ (\citeyear[p,~71]{Russell:1921ww}).
%
Perhaps through being trained.  Wittgenstein seems to have thought this was at least possible:
\begin{quote}
  ‘The child learns this language from the grown-ups by being trained to its use. I am using the word ‘trained’ in a way strictly analogous to that in which we talk of an animal being trained to do certain things. It is done by means of example, reward, punishment, and suchlike’
  \citep[p.~77]{Wittgenstein:1972lj}.
\end{quote}
%
More boldly, Quine held that children actually acquire words through training.
He asserted that
\begin{quote}
‘the child’s early learning of a verbal response depends on society's reinforcement of the response in association with the stimulations that merit the response’
\citep[p.~82]{Quine:1960fe}.
% (Quine 1960, p. 82)
\end{quote}
%
So for each word there is a set of ‘stimulations’ in response to which an utterance of that word would be appropriate.
For instance, we might suppose there is a set of banana stimulations in response to which an utterance of the word ‘banana’ would be appropriate.
The child then comes to use the word ‘banana’ in response to the banana-stimulations by means of being trained.
She is rewarded for using ‘banana’ correctly or punished for using it incorrectly (or both).
In this way she gradually conforms to what her trainers think of as the correct pattern of use.


Davidson also assumes this view of how children acquire their first words:
\begin{quote}
‘nothing more is necessarily involved than verbal responses increasingly conditioned to what the teacher thinks of as appropriate circumstances, and the child finds satisfying, often enough’
 (\citealp[pp.~70--1]{Davidson:2000mt}; \citealp[pp.~111-2]{Davidson:1999wg}).
\end{quote}
%
On this view, a child acquiring her first words is like a pigeon learning to peck in response to a target appearing.
When the pigeon happens to peck the target, it gets a little food pellet as a reward.
This reinforces the pigeon’s tendency  to peck next time the target appears.
On a view like those mentioned by Quine, Wittgenstein and Davidson, this is also how children acquire their first words.
They happen to make a certain sound in response to the appearance of some ‘target’ and are rewarded for doing so.
This reinforces their tendency to make the sound next time the target appears.
Call any such view a \emph{\gls{training theory}} of \gls{lexical acquisition}.

Why are training theories of language acquisition at all attractive?
As we saw (at the start of this section), Davidson holds that anyone who can think can communicate with words.
This implies that the process of acquiring your first words cannot require thinking.
A proponent of Davidson’s position therefore faces a challenge.
How could children acquire their first words without already being able to think?
The \glspl{training theory} associated with Davidson, Quine and (maybe) Wittgenstein provide a neat answer:
acquiring your first words is a matter of forming habits to use words in response to stimuli as a consequence of reinforcement.
On such a theory, no thinking at all is required.
% Further, there does not appear to be any alternative.
% If you hold that thinking requires communication with words, it appears that you also have to hold that acquiring your first words is a matter of being trained to respond to situations unthinkingly.


It is striking that philosophers appear willing to assume the correctness of a training theory of \gls{lexical acquisition} without considering evidence for it.
How do children actually acquire their first words?


\section{Lexical Innovation}
\label{sec:lexical-innovation}
Children acquiring language often create their own words before they learn to use those of the adults
around them.
For some, their very first words are novel creations
(\citealp[p.~122]{Bloom:2000qz}; \citealp{Clark:1981bi,Clark:1982hj}).
% ‘Some children are so impatient that they coin their own demonstrative pronoun. For instance, at the age of about 12 months, Max would point to different objects and say “doh?,” sometimes with the intent that we do something with the objects, such as bring them to him, and sometimes just wanting us to appreciate their existence’ (\citealp[p.~122]{Bloom:2000qz}; see further \citealp{Clark:1981bi,Clark:1982hj}).
For example, one child’s first words included ‘bimbam’ for oranges and ‘mipmap’ for a type of colourful large plastic clip.  %Hannah
Adults and older children sometimes respond not by attempting to train the child to use words as they do, but instead by borrowing the child’s word.
If there is training in acquiring your first words, it is at least as much a matter of you training those around you as it is them training you.

These points are beautifully illustrated by a conversation with June, a 15-month-old infant, recorded by \citet{Higginson:1985wy}:%
\footnote{%
\citeauthor{Higginson:1985wy}, like many others, has made these recorded conversations publicly available in the CHILDES database \citep{MacWhinney:2000lu}.
You can find the data online at \url{http://childes.talkbank.org}.
}
% see ‘childes.doc.pdf’ in notes/
%
\begin{quote}
\begin{dialogue}
  \speak{June}	puttaputta.
\speak{Mother} puttaputta \lips ok.
\speak{Mother} 	this puttaputta?
\speak{Mother} 	Peter Piper picked a peck of pickle peppers
\par\lips\par
\speak{June}		puttaputta.
\speak{Mother} 	puttaputta?
\speak{Mother} 	Where's puttaputta?
\speak{Mother} 	Can you show me puttaputta?
\par\direct{June turns the page.}
% \medskip
\speak{June}		puttaputta.
\speak{Mother} 	That's not puttaputta.
\end{dialogue}
\end{quote}
%
At this point, June’s mother clearly thinks that ‘puttaputta’ refers to Peter Piper.
She attempts to correct June.
But June resists correction.
The conversation continues:
\begin{quote}
  \begin{dialogue}
  \speak{June}		puttaputta.
\speak{Mother} 	puttaputta?
\speak{June}		puttaputta.
\speak{Mother} 	ok \lips Doctor .
\par\direct{June takes the book, looks at it and then hands it back to her mother.}
\speak{Mother} 	Foster went to Gloucester in a shower of rain … he stepped into a puddle right up to his middle and never went there again.
\speak{June}	puttaputta.
\speak{Mother} 	ok \lips the late Madame Fry wore shoes a mile high and when she walked by me I thought I should die.
\end{dialogue}
\end{quote}
%
June is attempting to use ‘puttaputta’ to get her mother to read to her.
She quickly clears up her mother’s misunderstanding and gets what she wants.
June continues to use ‘puttaputta’ in recorded conversations for roughly the next six months when she wants something read to her or to be told what something is.
During this time her mother also uses  ‘puttaputta’
in the way June initiated:
\begin{quote}
  \begin{dialogue}
\speak{June}		putta.
\speak{June}		puttaputta.
\speak{Mother} 	Am I supposed to read that?
\speak{Mother} 	You have to come over here then.
\speak{June}		puttaputta.
\speak{Mother} 	What do you want me to puttaputta?
\end{dialogue}
\end{quote}
%
As these conversations illustrate, acquiring your first words it at least sometimes not a matter of being trained: it is a matter of creating words that others then have to learn.



Even where children are using words acquired from those around them, they will often creatively misuse (from the adults’ point of view) these words in order to get a point across.
This is perhaps clearest in the ways children intelligently ‘overextend’  even their very first words.
That is, they will use words for things that no adult would apply them to, but in ways that somehow fit and can often be readily understood.
For example, \citet[p.~35]{Clark:1993bv} mentions a child who used ‘apple’ for  ‘balls of soap, a rubber-ball, a ball-lamp, a tomato, cherries, peaches, strawberries, an orange, a pear, an onion, and round biscuits.’
On the view Davidson, Wittgenstein and Quine propose (see \cref{sec:davidson-thought-language}),
adults are not supposed to reinforce such uses of ‘apple’ by, for example, responding as if they understood children.
If anything, they are supposed to respond by punishing children who use words in this way.%
\footnote{%
Do not try this at home.
}
But an adult who wants to communicate with a child needs to do just the opposite of what Davidson’s, Wittgenstein’s and Quine’s views imply is required: she needs to adapt to the way the child uses words.

Adults readily overextend and misuse words even when no children are around, often without realising that they are doing so.
On hearing that an expert team is negotiating with the hostages for their captors’ release,%
\footnote{BBC Radio 4, News at 5pm, 12/11/03}
people will often ask for more information without spotting the ‘mistake’ (compare \citealp{Sanford:2002tc}).
It may be that adults are no more flexible towards children learning their first words than towards anyone else: lexical innovation may turn out to be a fundamental characteristic of all communication by language.

Davidson suggests that interpreting what seem to be children’s early attempts to communicate with words as genuinely communicative actions  involves self-deception:
%
\begin{quote}
‘You can deceive yourself into thinking that the child is talking if it makes sounds which, if made by a genuine language user, would have a definite meaning. \ldots\ If a mouse had vocal cords of the right sort, you could train it to say ‘Cheese’.
But that word would not have a meaning when uttered by the mouse \ldots\
Infants utter words in this way; if they did not, they would never come to have a language’
\citep[pp.~127]{Davidson:2001sm}.
\end{quote}
%
Is it self-deception?

On the \glspl{training theory} of \gls{lexical acquisition} associated with Davidson, Quine and Wittgenstein, early word use is a consequence of training.
This implies that the sole source of systematicity in children’s early uses of words is training.
What predictions does this generate?
One prediction is that children will not systematically use words in ways they have not been trained to use them in.
But children create new words of their own, and they creatively misuse others’ words.
In both cases, they use words systematically.
But they do not initially use them in accordance with any kind of training they have received from others.
Contrary to what \glspl{training theory} predict.

You might be tempted to object that reinforcement obviously matters.
After all, if using a particular word never got a child what she wanted, she would surely stop using it and try something else.
So which words a child ends up using must to some extent depend on how adults respond to her uses of them.
This is true, but it is not an objection.
On the view associated with Davidson, Quine and Wittgenstein, the systematic use of a word is supposed to be a consequence of reinforcement.
What the research on \gls{lexical innovation}---the creation and creative misuse of words---shows is that some children sometimes use words systematically in ways that do not accord with any training they have yet received.
So even for children acquiring their very first words, not all systematic use of a word is a consequence of training.
This is consistent, of course, with allowing that how frequently a child uses a word in future is likely to be influenced by others’ responses.

A further prediction of \glspl{training theory} of \gls{lexical acquisition} is that children should not be able to acquire their first words without being trained in their use.
As \citet[p.~8]{Bloom:2000qz} notes, there are children who cannot speak and ‘produce only a few sounds’, so could receive little if any training.
Some of these children nevertheless acquire typical vocabularies and can understand complex sentences, contrary to what \glspl{training theory} would predict.

Key predictions of Davidson’s, Quine’s and Wittgenstein’s views about language acquisition are readily falsified.
\Glspl{training theory} of \gls{lexical acquisition}  should therefore be rejected in the absence of strong positive support for them.
What is the alternative?


\section{The Shipwreck Survivor}
\label{sec:shipwreck-survivor}
Imagine you are a shipwreck survivor washed up on a beach.
Strangers surround you.
They are talking among themselves but you cannot understand their words, and no one seems interested in talking to you.
As days turn into weeks, you live your life alongside these strangers and gradually figure out what some of their words mean.
You do this by forming hypotheses and testing their predictions.
At first you thought \emph{nasi} meant food.
But that hypothesis led you to predict that the people around you would use it as a superordinate in all food contexts, which they do not.
In fact, \emph{nasi} appears to be on a level with other words like \emph{bakmi} in food contexts.
Eventually, locking onto the correlation between rice and \emph{nasi}  leads you to the view that this is a word for rice specifically rather than for food more generally.
As you become relatively confident about a few tens of words, you start trying to use them.
The strangers around you gradually begin to take notice and occasionally respond.
They sometimes respond in ways that seem consistent with your hypotheses about meaning, bringing rice in response to your request for \emph{nasi}, for example.
These responses strengthen your confidence in some of the hypotheses, enabling you to learn their words a bit faster.

Is your experience as a shipwreck survivor like that of a child acquiring her first words?

Some theorists appear to imply that is is.
According to Higginbotham, acquiring your first words is a matter of  ‘coming to know the meanings of words, where at a given stage the learner’s conception is an hypothesis about the meaning’ \citep[p.~153]{Higginbotham:1998rm}.

This view is elaborated and defended by appeal to a rich variety of observational evidence by Clark.
Her position involves two key claims.
The first concerns a description of the problem to be solved in acquiring your first words:
\begin{quote}
  ‘One of the first problems children take on is the MAPPING of meanings onto forms … They must identify possible meanings, isolate possible forms, and then map the meanings onto the relevant forms’ (\citealp[p.~14]{Clark:1993bv}; compare \citealp[p.~242]{Bloom:2000qz}).
\end{quote}
For example, on this view children need to map the concept RICE (assuming meanings are concepts) onto the word ‘rice’ (or whatever word is used locally).

Clark’s second key claim concerns how children solve the problem.
It is a matter of formulating and testing hypotheses about what particular words mean:
\begin{quote}
  ‘Adults offer both words and information about word meanings, and children try out their hypotheses about word meanings in their own uses of the words, with adults offering corrections when needed’
\citep[p.~122]{clark:2009_first}.
\end{quote}
%
 A variant of this view is also developed by Bloom.
 %who, like Higginbotham, holds that ‘children learn words through the exercise of reason’ \citep[p.~1103]{Bloom:2001ni}.
% He
As a key additional element, he proposes that children’s abilities to track others’ \glspl{communicative intention} enable them to work out which words refer to which things \citep[p.~60]{Bloom:2000qz}.

These theories---call them \emph{\glspl{shipwreck survivor theory}}---of how children acquire their first words have a long history.
Wittgenstein, who opposes shipwreck survivor theories,
 famously attributes such a theory to Augustine \citep[§1]{Wittgenstein:1953mm}, rightly or wrongly.
\citet[p.~19]{Clark:1993bv} identifies Brown as a source for contemporary interest in \glspl{shipwreck survivor theory}:
\begin{quote}
‘The tutor names things in accordance with the semantic customs of the community. The player forms hypotheses about the categorical nature of the things named. He tests his hypotheses by trying to name new things correctly. The tutor compares the player's utterances with his own anticipations of such utterances and, in this way, checks the accuracy of fit between his own categories and those of the player. He improves the fit by correction’
(\citealp[p.~194]{Brown:1958tg} as quoted by \citealp[p.~19]{Clark:1993bv}).
% Brown (1958, p. 194) as quoted by Clark (1993, p. 19)
\end{quote}
%
Clark and others propose that this is how children acquire their first words.
The child is like you as the shipwreck survivor.
%(Bloom holds that this is inessential.)

The case for shipwreck survivor theories of how children acquire their first words is impressively detailed \citep{Clark:1993bv,clark:2009_first,Bloom:2000qz}.
But, as you can probably guess from how I have labelled them, it seems to me that they face substantial objections.

One objection is simple, although not decisive.
Typically developing children with experiences of others’ languages learn to associate words with things many months before they first produce any words, at around six month of age or earlier \citep{tincoff:1999_beginnings_,tincoff:2011}.
Presumably they have a rich network of associations involving words by the time they are just starting to produce them themselves.
So the key problem that children face according to \glspl{shipwreck survivor theory} seems to be one that they have largely solved months before they produce any words at all.
This indicates that these theories may be missing something important about the nature of lexical acquisition.

%One consequence of this view is that learning words cannot ever be a route to acquiring concepts \citep[p.~258]{Bloom:2000qz}.
% \citep[p.~258]{Bloom:2000qz}: ‘Language may be useful in the same sense that vision is useful. It is a tool for the communication of ideas. It is not a mechanism that gives rise to the capacity to generate and appreciate these ideas in the first place.’


\section{Language Creation}
\label{sec:language-creation}
Try to imagine you had never communicated linguistically with anyone and had no experience of others’ languages at all.
From the time of your birth, everyone around you has voices that work on frequencies you are unable to detect.
You realise that other people interact much more easily than you can.
They are obviously getting something out of interactions with each other that you are not.
But what?
What is it that they are doing and how are they doing it?

Some profoundly congenitally deaf children in North America are brought up in purely oral environments without any sign language and therefore do not experience language at all.
These children invent their own sign languages, which are called \glspl{homesign}.
Although their invented languages are not as rich in every respect as those of children who experience other people's languages, they have many of the same features (Goldin-Meadow 2002, 2003).  These deaf children have somehow worked out for themselves what linguistic communication is, and they have found a way of doing it.  They have invented languages with no prior experience of language, and they have invented languages in a modality that people around them barely use in linguistic communication.
They have even trained others around them to understand them, even though the others cannot use sign languages as well as they do.

What are these homesign languages like?
Consider some examples from Goldin-Meadow’s groundbreaking research.  % \citet{Goldin-Meadow:2003pj}.
In \vref{fig:goldin-meadow_2003_fig1}, David explains that this father is asleep by pointing to the chair where he usually sits and then gesturing sleep.
In the next figure (\vref{fig:goldin-meadow_2003_fig11}) you can see David producing ‘Snack eat Susan.’
He is communicating that Susan should eat by pointing at some snacks, gesturing eating and pointing at Susan.
Finally, in \vref{fig:goldin-meadow_2003_fig22},
Qing, who is using a book, explains that a ‘swordfish can poke a person so that the person becomes dead, that they have long, straight noses, and that they swim’ \citep[p.~170]{Goldin-Meadow:2003pj}.
As these examples illustrate, homesigners use gestures for a variety of ends including making comments, requesting things and storytelling; they also ask questions, of course.

\addFigure{goldin-meadow_2003_fig1}{%
Source: \citet[figure 1, p.~74]{Goldin-Meadow:2003pj}}

\addFigure{goldin-meadow_2003_fig11}{%
Source: \citet[figure 11, p.~110]{Goldin-Meadow:2003pj}}

\addFigure{goldin-meadow_2003_fig22}{%
Source: \citet[figure 22, p.~171]{Goldin-Meadow:2003pj}}



Homesign gestures are unlike the gestures which many speakers produce while talking.
Those gestures vary across time and from one context to another.
By contrast, homesigns involve gestures that are stable across time and context, much as words are \citep[p.~1389]{Goldin-Meadow:2002dq}.
Homesign gestures are also unlike those involved in miming actions.
Miming characteristically involves producing what looks like an approximation of the movements involved in actually performing an action.
By contrast, homesigns include some gestures which are not related to their meanings in this way.
This includes gestures for breaking and giving, for instance \citep[p.~76]{Goldin-Meadow:2003pj}.
Finally, homesign gestures are unlike those which hearing adults or children might invent spontaneously, when they unexpectedly need to communicate with someone and do not share a language.
Such gestures are tied to the needs of the moment and usually quickly forgotten.
Homesign gestures are quite different.
Homesigners’ repertoires, like those of children in more typical environments \citep{clark:1988_logic}, tend to comprise sets of gestures which contrast with each other systematically, so that the creation of a new gesture takes into account which gestures are already part of the emerging homesign \citep[p.~1389]{Goldin-Meadow:2002dq}.
In short, homesigns have key features of language.

%Homesigners can use their gestures to communicate about past, future and hypothetical events.

The existence of homesigns shows that children can create their own first languages without experience of others’ languages.
This has many interesting consequences.
%  should this be part of the section on lexical innovation instead?
One concerns a common and probably quite tempting assumption about language acquisition.
According to Clark,
%
\begin{quote}
‘Logically, comprehension must precede production.
How else can speakers know which words to produce to convey a particular meaning?’
\citep[p.~246]{Clark:1993bv}.
\end{quote}
%
The existence of homesigns shows that this is a mistake (as does the development of Idioma de Señas Nicaragüense, which I shall mention in a moment).
Sometimes the task in acquiring your first words is not one of learning about others’ words but of creating your own.
Creating words is one way of coming to know which words to produce to achieve a certain end.

Another consequence of the fact that children can acquire their first words by creating them concerns the claim that anyone who can think can communicate with language
(see \cref{sec:davidson-thought-language}).
This claim implies, of course, that thinking---and knowledge, belief, desire and intention---cannot play any role in acquiring your first words.
Proponents of this claim therefore face a challenge.
The challenge is to explain how humans acquire their first words without thinking (see \cref{sec:davidson-thought-language}).
The response Davidson, Quine and Wittgenstein might suggest to this challenge is simple.
Children are first trained to use words unthinkingly, in just the way that a pigeon might be trained to peck at a certain kind of target.
We already saw that this response to the challenge does not appear adequate because it generates an incorrect prediction about acquiring your first words in very ordinary circumstances (see \cref{sec:lexical-innovation}).
But the existence of homesigns poses a further, independent challenge.
What needs to be explained is not merely how humans acquire their first words without thinking.
We need an explanation of how, without thinking, they could create words.

Perhaps the most important consequence of the fact that children can create their own first languages without experience of others’ languages concern the  \glspl{shipwreck survivor theory}.


\section{Can the Shipwreck Survivor Be Rescued? No.}
According to \glspl{shipwreck survivor theory}  (see \cref{sec:shipwreck-survivor}), a key problem facing children is to map the words those around them use onto the meanings they take them to have.
Children solve this problem by formulating hypotheses and testing them against patterns in how others use words (or intend to use them).
Additional evidence may come from others’ reactions when the child uses a word in accordance with a hypothesis.
How should findings concerning lexical innovation and homesigns shape our evaluation of these theories?

This shipwreck survivor model manifestly fails to apply in case of homesigns.
Carers and other adults are in no position to correct their homesigning children’s use of words since the child is more expert in the language than the adults.
And as the homesigner is completely deaf to others’ words, no one provides her with information about the use of words or their intentions regarding them.
Unlike a (hearing) shipwreck survivor, there is no prospect of a homesigner testing a hypothesis about someone else’s words.
The very premise of \glspl{shipwreck survivor theory} is wrong: homesigners acquiring their first words are simply not in the mapping-meanings-onto-words business.

The case against \glspl{shipwreck survivor theory} is strengthened by research on the development of sign languages in Nicaragua.
In 1977 a group of \gls{homesign}ers were brought together in the first Nicaraguan elementary school open to them.
The school did not employ teachers with knowledge of sign languages and ‘[l]ipreading was not systematically taught and is not commonly used by deaf Nicaraguans’ \citep[p.~182]{Kegl:1999es}.
Instead the teachers attempted to teach in spoken Spanish.
We can therefore be confident that these children lacked experience of established languages.
Instead their linguistic experiences arose from attempts to communicate with each other.
With time, a mix of entirely independent homesigns coalesced into a shared language with vocabulary and structures.
As new cohorts of homesigners entered the school, they did not simply learn the existing language.
Instead they refined and extended it, giving rise to a more sophisticated, distinct language called \emph{Idioma de Señas Nicaragüense} \citep{Senghas:2001zm,Kegl:1999es}.
The upshot of this is that, in this group, there was an extended period in which ‘the most fluent signers are the youngest, most recent’ members \citep[p.~1780]{senghas:2004_children}.
No \gls{shipwreck survivor theory} of lexical acquisition can explain this phenomenon.
Any such theory depends on the premise that older, less recent group members (for example, parents) provide the raw materials which younger, more recent members (for example, children) learn from.
Homesigns and the development of Idioma de Señas Nicaragüense shows that this premise is at least not universally true.
Sometimes lexical acquisition is not a matter of testing hypotheses and mapping words to meaning: it is a matter of creating words.

Does this mean we must abandon \glspl{shipwreck survivor theory}?
Recall Bloom’s argument against \glspl{training theory} (see \cref{sec:lexical-innovation}).
Bloom argued that since some children’s acquisition of words is not a consequence of training, training cannot be necessary.
Similarly, evidence from homesigns and the development of  Idioma de Señas Nicaragüense shows that neither hypothesis testing nor gaining information about communicative intentions is necessary for lexical acquisition.
One response to both arguments is that theories of lexical acquisition do not need to identify necessary conditions in order to succeed.
This is true, at least as long as success does not involve defending the claim that anyone who can think can use language (see \cref{sec:davidson-thought-language}).
But there is a deeper challenge to \glspl{shipwreck survivor theory}.

We saw earlier that even hearing children surrounded by spoken language create words, and do so even as they are acquiring their first words (see \cref{sec:lexical-innovation}).
They also misuse the words of those around them by, for example, overextending ‘apple’ to talk about cherries and rubber balls.
Could the ordinary way of coming to acquire your first words have more in common with creating a homesign than doing the detective work required by the shipwreck survivor?

This question raises a deep and difficult issue about the nature of communication with words.
I have just described what children acquiring their first words often do as ‘misusing’ others’ words.
But in what sense is using an old word for something new misusing it?
Adults, regardless of linguistic expertise, routinely do much the same, and their doing so may be essential for successful communication (compare, for example, \citealp{Carston:200ip,Recanati:2003ar}).
Of course, you and I are surrounded by social \glspl{convention} and norms governing how words are used, some of which are quite fiercely enforced.
And using the word ‘apple’ for cherries is a clear violation of these.
But what do these conventions and norms have to do with communication with words?
To many the answer seems all too obvious:
%
\begin{quote}
  ‘[W]hat makes terms conventional is that everyone in a speech community agrees with everyone else on which words \ldots\ or expressions denote which kinds. \ldots\
  In learning a new language, then, speakers learn a new set of conventions’
  (\citealp[pp.~67--8]{Clark:1993bv}; compare \citealp[pp.~17--8]{Bloom:2000qz} and \citealp[p.~1120]{Tomasello:2001ic}).
  % Tomasello (2001: 1120): “A language is a set of historically evolved social conventions by means of which intentional agents attempt to manipulate one another’s attention.” As the context and (Tomasello 2000: 358) makes clear, Tomasello’s conventions include word–object mappings.
  % Bloom (2000: 17–8): “two things are involved in knowing the meaning of a word—having the concept and mapping the concept onto the right form. \ldots\ Saying, for instance, that a two-year-old has mixed up the meanings of cat and dog implies that the child has the right concepts but has mapped them onto the wrong forms.” Although Bloom does not explicitly mention convention here, it’s clear from the context that the ‘right form’ involves a convention.
\end{quote}
%
There is also fairly widespread consensus among philosophers that social conventions governing how words are used are somehow essential to communication by language (see, for example, \citealp[p.~220]{Wright:1986wi}; \citealp[p.~106]{Fodor:1975pb}; \citealp[pp.~100--1]{Dummett:1978cf}).
% “It is a convention of English that ‘red’ in its most basic, literal sense, is correctly predicated only of things which are red. Speakers of English who are credited with an understanding of ‘red’ in its most basic and literal sense are thereby credited, inter alia, with the intention to uphold this pattern of predication as a matter of convention” (Wright 1986: 220).
% Fodor (1975: 106): “a natural language is properly viewed in the good old way: viz., as a system of conventions for the expression of communicative intentions. One might think of the conventions of the language as a sort of cookbook which tells us, for any C that can be communicated by an expression of the language, ‘if you want to communicate C, produce an utterance (or inscription) which satisfies the descriptions D1, D2 \ldots\ Dn’ where specimens Ds might be syntactic, morphological, and phonological representations of the utterance. The converse remarks hold for the hearer: To know the conventions of a language is at least to know that an utterance which satisfies D1, D2 \ldots\ Dn also standardly satisfies the description ‘produced with the intention to communicate C’.”
% “it is because two speakers take the language as governed by the same \ldots\ theory of meaning that they can communicate with one another by means of that language” (Dummett 1978: 100-1)
But not everyone accepts this view.
In a widely ignored series of papers,%
\footnote{%
\citet{Davidson:1994ol,Davidson:1982uu,Davidson:1992pl,Davidson:1991ic,Davidson:1989if,Davidson:1999il}.
}
Davidson argues that conventions
\begin{quote}
   ‘do not \ldots\  have anything to do with meaning or communication.
  Using a word in a non-standard way out of ignorance may be a faux pas in the same way that using the wrong fork at a dinner party is, and it has as little to do with communication as using the wrong fork has to do with nourishing oneself, given that the word is understood and the fork works’ \citep[p.~9]{Davidson:1994ol}.
\end{quote}
%
If Davidson is right, the possibility of acquiring your first words by creating a language of your own together with the ubiquity of lexical innovation would imply that \glspl{shipwreck survivor theory} are inadequate.
These theories focus on something that merely speeds up and may simplify the process of language acquisition,
namely social conventions regulating word use,
rather than on its central feature,
namely starting to use, misuse or invent words.

One apparent reason for rejecting this claim and maintaining that convention is essential for communication by language arises when we return to the issue of how a \gls{homesign} differs from the manual gestures produced by speakers of oral languages (see \cref{sec:language-creation}).
The difference, you might think, is this.
Only in the case of a homesign does a set of conventions about how manual gestures are to be used become established.
A related idea is that conventions about words are essential for being understood:
%
\begin{quote}
‘speakers must be consistent in the conventional meanings they assign from one occasion to the next, and they must maintain the same contrasts in meaning from one occasion to the next’
\citep[p.~83]{Clark:1993bv}.
 \end{quote}
%
But must speakers do this?
And is convention the ingredient that separates the homesign from gestures which merely accompany linguistic communication?

These claims only seem obvious as long as we assume that violating convention leads to anarchy.
Of course you cannot assign different meanings to words on different occasions at random.
How words have been used before must constrain in some way how they are used in the future.
Such constraints are what separates producing words from performing any other kind of gesture (whether oral or manual).
But why assume the constraints in question are conventions?
An alternative requirement is that future uses must be intelligible to one’s audience in the light of past uses and the current context.
When meeting this requirement requires lexical innovation and therefore violating a convention, expert speakers weigh the demands of etiquette against the demands of fluent communication.
What separates homesigns from manual gestures which merely accompany oral communication is the simply  fact that, in homesigns, the ways a gesture was used in the past is typically part of what makes a future use of that gesture intelligible.
Adherence to \glspl{convention} is not the essence of communication with words; innovation is.

If this is right, we should reject both \glspl{training theory} and \glspl{shipwreck survivor theory}.
Neither approach captures the social aspect of \gls{lexical acquisition}.
On a training theory, lexical acquisition is all about an expert training an unthinking animal.
On  \glspl{shipwreck survivor theory}, it is all about the novice figuring out what experts are up to.
In neither case is there any form of joint action.
But if lexical innovation is an essential part of the process of acquiring your first words,
that process can only be a genuinely joint action.


\section{Conclusion}
So how do humans first come to communicate with words?
In this chapter we have explored two prominent approaches, \glspl{training theory} and \glspl{shipwreck survivor theory}.
I offered an argument, not decisive but at least worth refuting, that neither is right on the grounds that lexical innovation is a characteristic part of acquiring your first words.
Starting to communicate with words is a more creative process than either of the main approaches allows, and it is also more of a joint activity too.

In \cref{cha:communication}, I tentatively offered a minimal model of the production of communicative actions including pointing and one-word utterances.
The model is an attempt to outline how possessing abilities to initiate {joint action} could enable an individual to perform \glslink{referential communication}{referential}, \gls{purposively communicative} actions.
While that model is not supposed to characterise the ways more expert speakers use words,
it does suggest a view about how children come to acquire their first words.
Doing so is a matter of discovering that the action of producing this word can contribute to achieving an \gls{extra-communicative purpose} by initiating or contributing to a joint action.

% Typically developing children with experiences of others’ languages associate words with things many months before they first produce any words, at around six month of age or earlier \citep{tincoff:1999_beginnings_,tincoff:2011}.
% Presumably they have a rich network of associations involving words by the time they are just starting to produce them themselves.
% This indicates that the challenge they overcome in acquiring their first words is not figuring out which words others use to talk about shoes or the daddy.


Lexical acquisition is neither merely a matter of training, nor merely a matter of reasoning about the meanings of words.
Rather, it involves lexical innovation in the context of joint action from the first utterances of words.

In focussing on acquiring words we have been ignoring what many take to be the central feature of language, namely syntax or patterns in the ways words are combined to form phrases and sentences.
Research on this topic turns out to require fundamental revisions to the models of communication considered so far.  %I.e. core knowledge of syntax


%%% Local Variables:
%%% TeX-master: "master"
%%% End:
