%!TEX root = master.tex

\chapter{The Simple View}
\label{cha:simple-view}

We have just seen (in \cref{cha:principles-object-perception}) evidence for Spelke’s bold conjecture that humans, from the first months of life, segment objects in accordance with the \gls{Principles of Object Perception}. %---cohesion, boundedness, rigidity, and no action at a distance.
% We need not focus on the details of the principles, which probably need refining; we will also add principles later.
% Even allowing this, 
This is a bold conjecture: it cites a small number of principles which generate many testable predictions about how infants segment objects in different situations.
The success of these predictions (so far, at least) motivates us to ask two questions.
First, What is the link between these principles and the mind of an individual infant?
Second, Are whatever abilities four-month-olds have to represent objects as persisting or to track their causal interactions also described by the same principles?
A positive answer to this second question will hint that infants’ abilities concerning physical objects may all be consequences of a single system. 
Answering the first question will be the key to characterising this system.
In answering these questions we are aiming to identify the basis for infants’ earliest knowledge of particular physical objects, so approaching our ultimate goal of understanding the transition from knowing nothing about particular physical objects to knowing something.


\section{The Simple View}
\label{sec:simple-view}



% In thinking about this question it is useful to distinguish descriptive from explanatory claims.
% \glsadd{descriptively adequate}
% \glsadd{explanatorily adequate}
% Consider the claim that humans do in fact segment objects in accordance with the Principles of Object Perception.
% This is a merely descriptive claim.
% It says nothing at all about how humans do this.
% A further claim---the explanatory claim---is that the Principles of Object Perception explain how humans segment objects.
% This explanatory claim is both indispensable and a major source of problems.
% It is indispensable because the Principles are not supposed to be merely heuristics for describing and predicting infants’ performance on preferential looking tasks.
% Rather, these Principles are supposed to explain why infants look longer at some things than at
% others.
% But the explanatory claim is also a major source of problems because it is not easy to understand how it could be true.
% In what way could the Principles of Object Perception explain how humans segment objects?

What is the link between the \gls{Principles of Object Perception} which describe four-month-olds abilities to segment physical objects and the mind of an individual infant?
Start with an answer I shall call the \emph{\gls{Simple View}}:
The Principles of Object Perception are things that we believe or know, and we acquire beliefs or knowledge about particular physical objects from these principles by a process of inference.
% (We can't say the principles are known because strictly speaking they are not truths but only heuristics.)
How does this process of inference work?
Perceptual processes provide information about the locations of surfaces in space.
This information together with the \gls{Principles of Object Perception} provides constraints on what physical objects there are.
We form beliefs about particular objects, their locations, movements and interactions in such a way that if our beliefs were true, all of the constraints would be met.

On the Simple View,  coming to know facts about objects is a bit like solving a sudoku or crossword puzzle. 
You start out with some general constraints and a little information.
You then try to fill in the gaps in a way that satisfies the constraints.

In what senses is the Simple View simple?
It invokes only states and processes that are already familiar in characterising adults’ abilities, namely knowledge, belief and inference.
So it involves no conceptual novelty.
%It provides an explicitly epistemological characterisation of infants’ abilities.
And it specifies a simple relation between infants’ and adults’ abilities, namely identity.

The Simple View is worth considering because it is so, well, simple.
And although Spelke herself would probably no longer endorse it, she does appear to have accepted the Simple View at one point in her thinking:
%
\begin{quote}
‘objects are conceived: Humans come to know about an object’s \ldots\ boundaries \ldots\ in ways like those by which we come to know about its material composition or its market value’
\citep[p.~198]{Spelke:1988xc}.
\end{quote}
%
As we will see (in \cref{sec:extending-the-simple-view-to-persistence}), several other researchers also appear to endorse the Simple View.
\footnote{%
It is hard to be sure whether Spelke or others endorse the Simple View because there is always room for uncertainty about what they mean by 
terms like ‘knowledge’.
As I use the term, if it is locked to an arbitrarily limited range of actions, or if it used in response to an arbitrarily limited range of events, then it is not knowledge (see \cref{sec:knowledge}).
}

The Simple View is inspired by two famous cognitive scientists, Marr and Chomsky.
Marr, building on Helmholtz’ and others’ ideas, showed that many visual processes can be described as inferences (see \citealp{Marr:1982kx}).
And Chomsky pioneered the idea that humans’ knowledge of language depends on their knowing of a small number of
principles (see \citealp{chomsky:1965_aspects}). %\cref{cha:syntax}).
Similarly, the Simple View implies that human infants (and adults) come to know facts about particular physical objects by virtue of making inferences from a small number of principles which they know or believe.
% What unites these three cases, Spelke on object segmentation, Marr on vision and Chomsky on
% syntax?
% It’s that they are straightforwardly cognitivist in appeal to knowledge and inference.
% Principles are known, and they are used via a process of inference.
% (There’s a nice quote from Fodor underlining this point.)
% ‘Chomsky’s nativism is primarily a thesis about knowledge and belief; it aligns problems
% in the theory of language with those in the theory of knowledge.  Indeed, as often as not,
% the vocabulary in which Chomsky frames linguistic issues is explicitly epistemological.
% Thus, the grammar of a language specifies what its speaker/hearers have to know qua speakers
% and hearers; and the goal of the child’s language acquisition process is to construct a
% theory of the language that correctly expresses this grammatical knowledge.’
% \citep[p.~11]{Fodor:2000cj}

% Marr’s and Chomsky’s views have been opposed on the grounds that a visual processes cannot literally involve inference but must be something merely inference-like, or that what underpins syntax cannot literally be knowledge but must be something tacit or implicit rather than knowledge proper.
% The Simple View arouses similar opposition, especially from philosophers (see, for example, \citealp{Bermudez:2003dj}).
% But it is nevertheless a good starting point.
% None of the arguments against Marr’s or Chomsky’s views seem to succeed in showing that there is anything theoretically incoherent in their positions.
% Nor does the Simple View obviously suffer from theoretical incoherence.
% (And simplicity is often a defence against nonobvious theoretical incoherence.)
% This together with the endorsement of developmental rockstars including Spelke (at least in one phase of her thinking) and Baillargeon strongly recommend the Simple View as a starting point.


The Simple View matters for understanding how humans first come to know facts about particular physical objects.
If the Simple View is right, segmenting objects involves coming to know facts about them.
So, three- and four-month-olds already know some facts about physical objects (specifically, about where one ends and another begins).
And humans do not have an ability to segment objects that is developmentally and conceptually more primitive than their ability to know facts about physical objects.
Rather, to segment objects is to know some facts about them.
Or so the Simple View implies.



So far the case for the Simple View is based entirely on findings about how humans first segment objects.
Having an ability to segment objects (to determine where one ends and another begins) was only one of three abilities associated with knowledge of objects that we identified.
It is now time to turn to the second of these, the ability to represent objects as persisting.
Thinking about persistence will strengthen the case for the Simple View.
It will also bring us one step closer to understanding how humans first come to know facts about objects.



\section{Persistence}
\label{sec:persistence}

Our overall aim is to discover how humans first come to know simple facts about particular physical objects.
Because this is so difficult, we are approaching it by considering three abilities associated with knowledge of particular physical objects (as explained in \cref{sec:three-abilities}).
We covered the first, the ability to segment objects, in \cref{sec:segmentation,sec:principles-object-perception}.
The next  is the ability to represent things as persisting.

When you are looking after a young child, your representations of the child’s location, size and other simple properties do not depend on your continually perceiving the child.
Even if the child momentarily disappears from view, you may well continue to represent her approximate location so that, for instance, you would be surprised to find her sneaking up behind you.
Without any ability to represent things as persisting, your world would be limited to the objects you are currently perceiving.
Let the child hide behind the logs and, from your point of view, she would cease to exist.
Turn around and the logs would cease to exist too.
%The ability to represent things as persisting is essential for your world to be one where things don't go out of existence when unperceived.
When and how are humans first able to represent objects as persisting?

To answer this question, \citet{spelke:1995_spatiotemporal} took two groups of four-month-old infants.
The infants in one group were habituated to a display involving continuous movement which adults would naturally describe as a single object moving behind two barriers
(see the top left panel of \cref{fig:spelke_1995_fig2} on page \pageref{fig:spelke_1995_fig2}).
Infants in the other group were habituated to a display involving discontinuous movement which adults would typically describe as one object disappearing behind a barrier followed by another object appearing from behind a second barrier (as depicted in  the top right panel of \vref{fig:spelke_1995_fig2}).

Why is this significant? Imagine for a moment that you lacked any ability to represent objects as persisting.
Then the two displays would differ only in that there is a short interval without anything moving in the second display.
But since you are not representing anything as persisting, from your point of view, the two displays would not differ with respect to the number of objects involved.
By contrast, if we think in terms of persisting objects, then the two displays differ in a more substantial way: whereas the first involves a single moving object, the second must involve two moving objects.
The hypothesis that four-month-olds can represent objects as persisting (even while unperceived) therefore implies that infants should represent the continuous movement as involving one object and the discontinuous movement as involving two.

How can we tell whether this is so?
Take those infants who were habituated to the continuous movement.
After habituation, each was shown either a single moving object (as depicted bottom left in \vref{fig:spelke_1995_fig2}) or two moving objects (as depicted bottom right in the same figure).
Because \gls{dishabituation} is stronger for more novel displays, the hypothesis that the infants can represent objects as persisting predicts that these infants should show greater dishabituation to the two moving objects.
After all, they were habituated to one object moving and are now seeing two.
And, by similar reasoning, the hypothesis also predicts that those infants habituated to the discontinuous movements should show the opposite pattern of dishabituation:
that is, after their habituation they should look longer if shown one object moving than if shown two objects moving.
And this is just what \citet{spelke:1995_spatiotemporal} found.
(\citet{Aguiar:2002ob} extended this study and obtained convergent findings.)
Since the hypothesis that infants cannot represent objects as persisting and live in a world of mere features does not make these predictions,
this is evidence that infants, from four months of age or earlier, can represent objects as persisting even while they are momentarily out of view behind a barrier.

This conclusion is also supported by experiments involving anticipatory looking rather than habituation.
\citet{rosander:2004_infants} showed infants an object moving along a path.
At some point it disappeared completely behind a barrier and, predictably, reappeared again as it reached the far side of the barrier.
In this situation adults will show anticipatory looking.
That is, they will not usually be tracking the object with their eyes while it is fully hidden, but they will position their eyes to just where the object will be at almost exactly the time it should emerge from the far side of the barrier.
\citet{rosander:2004_infants} found that, from around four months of age, infants show similar anticipatory looking  (see also \citealp{bertenthal:2013_differential}).
Their timing is not quite as good as adults’ (it improves with age), but the pattern of anticipatory looking is compelling evidence that even four-week-olds are in some sense representing objects that are fully hidden from their  view as persisting.
% \citep{rosander:2004_infants}: `The obtained results are in general agreement with the numerous
% habituation studies that have investigated infants' emerging ability to represent temporarily
% occluded moving objects. The individual data show that 9–12-week-old infants begin to predict
% the reappearance of the object towards the end of the centrally occluded trials.'



%
\begin{figure}
\begin{center}
\includegraphics[width=0.9\textwidth]{fig/spelke_1995_fig2.neg.png}
\caption{
  \label{fig:spelke_1995_fig2}
  Can four-month-olds represent objects as persisting?  \citet{spelke:1995_spatiotemporal} habituated one group to the continuous event (depicted top left), and another to the discontinuous event (depicted top right).  Those habituated to the continuous event showed greater dishabituation to the two-object display (depicted bottom right) than to the one-object display (depicted bottom left); those habituated to the discontinuous event showed the converse pattern of dishabituation.
  Source: \citet[][figure 2, part]{spelke:1995_spatiotemporal}.
}
\end{center}
\end{figure}
%

How could infants be doing this?
\citeauthor{spelke:1995_spatiotemporal} invoke the \gls{Principles of Object Perception}, adding a further principle, \emph{continuity}.
According to the principle of continuity, ‘an object traces exactly one connected path over space and time’ and objects’ paths cannot cross \citep[p.~113]{spelke:1995_spatiotemporal}.
Why is this principle relevant to describing how infants represent unperceived objects in the experiments just discussed?
Consider again the discontinuous motion depicted top right in \vref{fig:spelke_1995_fig2}.
To interpret this scenario as involving one object would require interpreting that object’s movement as not tracing a connected path (after disappearing behind one barrier the object would be nowhere until it reappears from behind the other barrier), and therefore as violating the principle of continuity.
So supposing that humans, including infants, represent persisting objects in accordance with this principle implies, correctly that they will interpret the discontinuous motion scenario as involving more than one object.

There is a second way the principle of continuity could in principle be violated.
Rather than having one object trace a discontinuous path through space and time, we could have two objects whose paths through space and time cross.
This would result in two objects briefly occupying the same space at the same time.
Call this a ‘solidity violation’ of the principle of continuity.
If the principle of continuity describes how four-month-olds represent objects as persisting, then they should also react to solidity violations of the principle of continuity.
But do they?

Imagine that you are about to walk into a castle by crossing its moat.
Just before you get to the drawbridge over the moat, it is drawn up through 90 degrees so that the formerly horizontal bridge is now vertical; it then continues rotating in this direction so that the bridge ends up horizontal again but no longer spans the moat.
What you have just observed is a large-scale version of the scenario to which \citet{baillargeon:1987_object} habituated four-month-old infants.
This is depicted in the top part of \vref{fig:baillargeon_1987_fig1}.
After infants have been habituated to this scenario, a tiny change is made.
Everything is just as before except that before the drawbridge moves, an object is placed
behind it.
Infants can see the object being placed, but once it is in position it is hidden from them by the drawbridge.
One group of infants then sees the drawbridge rotate
through 180 degrees exactly as before, as depicted in the middle of \vref{fig:baillargeon_1987_fig1}.
This is impossible given that there is a solid object behind it.
But if infants do not represent objects as persisting, they should be entirely unaware of this impossibility.
The other group of infants were shown the scenario depicted bottom in \vref{fig:baillargeon_1987_fig1}.
In this scenario the drawbridge stops earlier, at 112 degrees, as if it were impeded by the box; call this the ‘possible scenario’.
\citeauthor{baillargeon:1987_object} asked, Which scenario will cause greater \gls{dishabituation}?
If infants do not represent objects as persisting then they should be unaware of the impossibility of the drawbridge moving through an object.
In that case, the possible scenario should be more novel to them than the impossible scenario because, from their point of view, it differs more from the habituation scenario---something that formerly rotated through 180 degrees is now stopping earlier.
By contrast, if infants can represent objects as persisting then the converse is true:
they should find the impossible scenario more novel because this involves something like a magic trick whereas the in possible scenario the movements of the drawbridge are just as expected.
Across three experiments, \citet{baillargeon:1987_object} found that four-month-old infants showed greater dishabituation to the impossible scenario.
This is further evidence that they can represent objects as persisting even while hidden from view.


%
\begin{figure}
\begin{center}
\includegraphics[width=0.6\textwidth]{fig/baillargeon_1987_fig1.part.png}
\caption{
  \label{fig:baillargeon_1987_fig1}
  Side-on views of the drawbridge scenarios in an experiment showing that 4-month-olds represent objects as persisting.
  Source: \citet[][figure 1, part]{baillargeon:1987_object}.
}
\end{center}
\end{figure}
%


Baillargeon’s experiments have been the topic of much discussion.
Some have replicated her findings \citep{durand:2002_object}, and even found related effects with dogs rather than infants \citep{pattison:2010_case}.
But others have attacked Baillargeon’s drawbridge study on methodological grounds.
% I’m skipping the 2000 infancy papers --- too boring, narrow.  Only interest is that there is an opposition between two extremes, (a) the Simple View and (b) perceptual effects that are irrelevant, or only very distantly relevant, to knowledge of objects.
\citet{sirois:2012_pupil} ran a version of her experiments without finding evidence that infants represent objects as persisting even at 10 months of age.
They argue that the apparent novelty of the impossible scenario to infants may be an artefact of the statistics Baillargeon used.
As controversies like this abound, it is worth taking a moment to ask how
we should respond to it.
Can we hold on to Baillargeon’s conclusions or should they be rejected?

\citet{sirois:2012_pupil}’s methods are rigorous and their criticisms are  initially persuasive, so it may be tempting to think that the experiments they target must be dismissed.
But things are rarely so straightforward.
\citeauthor{sirois:2012_pupil} used computer generated stimuli whereas Baillargeon had a physical set-up,  they studied 10-month-olds rather than four-month-olds, and they used a different method (‘children were \ldots\ not habituated by the time testing began’).
So what can we conclude from the fact that \citeauthor{sirois:2012_pupil} did not find evidence for an ability to represent objects as persisting?
This certainly justifies caution in relying on any single experiment.
Taken alone, \citet{baillargeon:1987_object}’s studies are inspiring but not fully convincing.
However, many further experiments involving different groups of researchers, different scenarios and different methods provide converging evidence for the same conclusion: even four-month-olds can represent objects as persisting (for reviews see \citealp{Spelke:2001pg} or \citealp{Baillargeon:2002hb}).
The initial, groundbreaking studies are probably methodologically imperfect, but the balance of evidence from subsequent experiments suggests that the discovery they illuminate is probably real.%
\footnote{%
For an opposing view see \citet{schoner:2006_using}; for critical discussion of measures involving looking times generally, see \citet{aslin:2007_whats}.
}



\citeauthor{baillargeon:1987_object}’s drawbridge experiments strengthen the case for saying that infants represent objects as persisting in accordance with the principle of continuity.
They suggest that four-month-olds are sensitive to solidity violations of this principle; that is, to violations which involve two objects briefly occupying the same space at the same time.

The question for this section was, When and how are humans first able to represent objects as persisting?
The ability to represent objects as persisting is strikingly similar to the ability to segment objects.
Both abilities appear early in development, being clearly present by four months of age or earlier.
And both abilities can be characterised, at least partially,%
\footnote{%
\citet{wang:2004_young} provide evidence that, in addition to recognising boundaries and patterns of movement characteristic of objects, four-month-olds also appear sensitive to conditions under which one object should hide another from their view.
This indicates that their abilities to represent objects cannot be fully characterised by the principles we have considered.
} 
by a small number of principles, namely the \gls{Principles of Object Perception}.
So although abilities to segment objects and to represent them as persisting are conceptually
distinct, it may turn out that, in humans at least, there is really just one ability.



\section{Extending the Simple View to Persistence}
\label{sec:extending-the-simple-view-to-persistence}
We are far from fully understanding how humans are first able to represent objects as persisting, of course.
But the fact that the ability appears so early in development entails that it does not demand language, nor much conceptual sophistication.
This view is supported by the fact that the ability to represent objects as persisting is found in a wide variety of nonhuman animal including
monkeys \citep{santos:2006_cotton-top},
lemurs \citep{deppe:2009_object},
dogs \citep{kundey:2010_domesticated}, % replication of Baillargeon’s drawbridge with dogs
wolves \citep{fiset:2013_object}, %this is actually dogs and wolves
cats \citep{triana:1981_object},
crows \citep{hoffmann:2011_ontogeny},
chicks \citep{chiandetti:2011_chicks_op},
and
dolphins \citep{jaakkola:2010_what}.%
\footnote{%
If you read these studies you will find that some of the authors talk about Piaget's stages of object permanence, and about visible and invisible displacements.
For our purposes few of these details matter;
% $glossary:object permanence NO, not in glossary because it’s not a term I actually use.
the main thing you need to know is just that having \emph{\index{$object permanence$}object permanence} is being able to represent objects as persisting even when they are briefly hidden from your view.
}
It is possible that humans’ abilities to represent objects as persisting are unrelated to some or all of these other animals’ abilities, of course.
Nevertheless, the fact that chicks can represent objects as persisting does show that doing this is not necessary something that requires much cognitive effort or conceptual sophistication.

Can we say more about how humans first represent objects as persisting?
The position we are currently considering is the Simple View (which was introduced in \cref{sec:simple-view}).
According to the Simple View,
there is a set of principles about physical objects and their movements, the \gls{Principles of Object Perception}; these principles are things that we know or believe; and we generate expectations from these principles by a process of inference.
Earlier (in \cref{sec:simple-view}) we saw that the Simple View provides a candidate explanation of humans’, including infants’, abilities to segment objects.
But it now seems plausible that a single set of principles will characterise both how humans segment objects and how they represent them as persisting (as we saw in \cref{sec:persistence}).
If so, the Simple View also provides a candidate explanation of how humans represent objects as persisting.
They believe or know certain principles and they form expectations about the number and location of objects by making inferences from these principles.


The Simple View is quite widely endorsed.
Commenting on findings about infants’ abilities to represent objects as persisting,
\citeauthor{Aguiar:2002ob} write:
%
\begin{quote}
‘To make sense of such results, we … must assume that infants, like older learners, formulate … hypotheses about physical events and revise and elaborate these hypotheses in light of additional input’ \citep[p.~329]{Aguiar:2002ob}.
\end{quote}
%
If you are inclined to doubt the Simple View---perhaps you doubt that four-month-old infants can really have beliefs or make inferences---you might suppose that this talk about formulating and revising hypotheses should not be taken literally. %, as implying that infants are doing this in the sense that you or I might.
You might look for an interpretation  on which  \citeauthor{Aguiar:2002ob} do not really mean that infants formulate, revise and elaborate hypotheses.
Could they be invoking some kind of tacit or implicit state, and some kind of inference-like process which falls short of actually being inference?
We will indeed eventually consider views along these lines (in \cref{cha:core-knowledge}).
But for their part, \citeauthor{Aguiar:2002ob} explicitly specify that infants formulate and revise hypotheses ‘like older learners’.
This suggests that their position is probably better captured by the full-fat \gls{Simple View} than by invoking some semi-skimmed, not-knowledge-but-a-bit-like-it state.

It is also worth noting that merely stepping back from knowledge by invoking an unspecified notion of tacit or implicit knowledge would not amount to providing an alternative to the \gls{Simple View}.
A genuine alternative to the Simple View needs to identify which states and processes link an individual’s mind to the Principles of Object Perception.
As even articulating (let alone defending) a genuine alternative to the Simple View turns out to be surprisingly difficult, we should hold on to it as our working hypothesis until we encounter evidence  against it.


% \section{Beyond the Principles?}
% \label{sec:beyond-the-principles}

% Can the Principles of Object Perception fully explain four-month-olds’ abilities to represent objects as persisting? 

% However the principle of continuity, even together with the principles of rigidity, no action at a distance,  boundedness and cohesion that were introduced in \cref{sec:principles-object-perception}, is not sufficient to fully characterise infants’ abilities to represent objects as persisting.
% To see why not, consider a further experiment.


% % I'm presenting this experiment as showing that infants represent objects as persisting, and do so
% % in accordance with the Principle of Continuity.  However, the experiment is also about
% % causal interactions between objects.  After all, infants are demonstrating sensitivity to
% % the fact that a solid object must stop the drawbridge from rotating all the way back.


% \citet{wang:2004_young} showed four-month-olds a scenario involving objects being placed behind barriers.
% To make this a test of the ability to represent objects as persisting, they had a screen come up and hide the key phase of hiding event.
% Their experiment was extremely simple.
% They studied infants’ reactions to a wide object which seemed, impossibly, to disappear behind a narrow barrier while the screen was raised.
% This scenario is depicted in the bottom half of \vref{fig:wang_2004_fig1a} and labelled the ‘wide-occluded event’.
% Since the wide object is briefly hidden by the screen, if infants cannot represent objects as persisting then they should be unsurprised when the screen comes down and the wide object is not visible.
% So if infants seem especially interested in this scenario, that would be evidence that they can represent objects as persisting.
% But what would count as being especially interested?
% One good candidate is looking significantly longer at this scenario than at one involving a barrier wide enough to conceal the object (see the top half of \vref{fig:wang_2004_fig1a}).
% And in fact \citet{wang:2004_young} did find that four-month-olds look significantly longer at the apparently impossible scenario.
% Just to be sure, \citet{wang:2004_young} also used a second pair of scenarios which were exactly like the first pair of scenarios except that the wide object was replaced with a narrow object that could fit behind either the narrow or the wide barrier; in this case infants did not look longer at either scenario.
% This is good evidence both that infants can represent objects as persisting and that their abilities to do so cannot be fully characterised by the principles we have considered.
% In addition to recognising boundaries and patterns of movement characteristic of objects, infants also appear sensitive to conditions under which one object should hide another from their view.


% %
% \begin{figure}
% \begin{center}
% \includegraphics[width=0.9\textwidth]{fig/wang_2004_fig1a.png}
% \caption{
%   \label{fig:wang_2004_fig1a}
%   Two scenarios used in an experiment on four-month-olds’ abilities to represent objects as persisting even while not in view.
%   Source: \citet[][figure 1, part]{wang:2004_young}.
% }
% \end{center}
% \end{figure}
% %


% % [*TODO*] integrate this on reaching: \citep{vanwermeskerken:2011_anticipatory}
% % (Interpretation is a bit out there, but it nicely illustrates how occlusion duration
% % can affect reaching at around 7 months of age.)

% % [*TODO*] integrate this ERP measure of permanence: \citep{kaufman:2005_oscillatory}



\section{Causal Interactions}
\label{sec:causal-interactions}

Anyone who knows even the simplest fact about a particular physical object---knows its location, say---can probably segment objects (see \cref{sec:segmentation}),  represent them as persisting (see \cref{sec:persistence}) and track their causal interactions.
When and how do humans first track causal interactions?

In this section we will see that humans can do this from four months of age or earlier, and that the Simple View once again provides a candidate explanation of how they do it.

% part of the answer is that human’s abilities to track causal interactions can be characterised by the principles that enable them to segment objects and represent them as persisting.
% Consequently the Simple View provides a candidate explanation of all three abilities associated with knowledge of objects.

% In this section we will examine the evidence for the Simple View as an account of how humans first track causal interactions; afterwards we will have to face some objections to the Simple View.

What sort of causal interactions might you need to track in order to know simple facts about particular physical objects?
You probably do not need to be able to track particularly complex causal interactions.
But it is plausible that you might need able to track very simple causal interactions, such as the collision of two balls or the interaction of a ball with a barrier.


Consider observing the scenario depicted in the middle of \vref{fig:spelke_1992_fig2a}.
There is a bench above the ground.
A screen appears, hiding most of the bench from view and then a ball drops down from above, moving behind the screen.
Finally the screen is lowered.
Where do you expect to see the ball?
Adults mostly expect to see it on the bench.
After all, the ball cannot pass through the bench as both it and the bench are solid.
Expecting the ball to be on the bench manifests an ability to track causal interactions: the bench’s stopping the ball is a causal interaction, so if you are unable to track causal interactions there is no reason why you should expect the ball to be on the bench.
Do infants also expect the bench to stop the ball?
To find out, \citet[][]{spelke:1992_origins} first habituated four-month-olds to a scenario with no bench.
In this scenario, infants watch as a screen goes up, a ball falls behind it and then the screen comes down to reveal the ball on the ground (as depicted in the left part of \vref{fig:spelke_1992_fig2a}).
Having been habituated to this scenario,
infants were then shown a new scenario involving a bench.
There were two versions of the new scenario.
Some infants were shown a version of the new scenario in which the ball appeared on the bench, as adults would expect it to.
This is depicted in the middle of \vref{fig:spelke_1992_fig2a}; call it the ‘consistent scenario’.
Other infants were shown a version of the new scenario exactly like the other one except that when the screen came down, the ball appeared under the bench (as depicted in the right part of \cref{fig:spelke_1992_fig2a}); call this the ‘inconsistent’ scenario.
To see whether four-month-olds, like adults, expect the ball to be on the bench, \citeauthor{spelke:1992_origins} measured how much \gls{dishabituation} each scenario produced.
They reasoned that if infants were unable to track the ball’s causal interaction with the bench, then the consistent scenario should produce greater dishabituation because when considered purely in terms of features, the consistent scenario is more different from the scenario to which infants were habituated than the inconsistent scenario is.
Accordingly, if the inconsistent scenario produces greater dishabituation, that would be evidence that infants are not only representing causally inert features but can track the ball’s causal interaction with the bench.
And this is just what they found:
the inconsistent scenario was significantly more effective in arousing infants’ interest than the consistent scenario.

%
\begin{figure}
\begin{center}
\includegraphics[width=0.9\textwidth]{fig/spelke_1992_fig2a.neg.png}
\caption{
  \label{fig:spelke_1992_fig2a}
  Scenarios used in an experiment on four-month-olds’ abilities to track causal interactions.  A ball falls behind a screen and then the screen is removed to reveal the ball in various positions.
  Source: \citet[][figure 2]{spelke:1992_origins}.
}
\end{center}
\end{figure}
%



This is evidence that, by four months of age at the latest, infants can track simple causal interactions among objects, even when those causal interactions are occluded.
Further evidence is provided by Baillargeon’s drawbridge study \citep{baillargeon:1987_object}.
I introduced this study in \cref{sec:persistence} as showing that infants can represent objects as persisting (see page \pageref{fig:baillargeon_1987_fig1}).
But the study simultaneously supports the view that infants can track causal interactions: after all, it is not the mere presence of the object behind the drawbridge that matters but also its ability to constrain the drawbridge’s movement.
(For further evidence, see \citealp{saxe:2006_five}.)
% Dogs can do this too.
% This experiment used a search measure rather than a looking time measure.
% 'Dogs correctly searched the near location when the barrier was present and the far location when the barrier was absent. They displayed this behavior from the first trial' \citep{kundey:2010_domesticated} (from the abstract).
%
% Chimpanzees also understand something of physical interactions insofar as their looking times show sensitivity to support relations \citep{cacchione:2004_recognizing}.

How do four-month-old infants track causal interactions among objects?
The \gls{Principles of Object Perception} may be adequate to describing which causal interactions they detect.
To illustrate, recall the principle of continuity.
According to this principle, each object traces a connected path over space and time without crossing any other objects’ paths \citep[][p.~113]{spelke:1995_spatiotemporal}.
The possible positions of a ball falling downwards when there is a bench in its path are limited by this principle.
For the ball to move through the bench would involve a solidity violation of the principle of continuity, so the principle tells us that the ball cannot end up under the bench.

But to say that the principle of continuity and other principles describe how infants track  causal interactions is not yet to explain how infants do this.
The fact that certain principles can be used to describe infants’ behaviours does not force us to accept that those principles are in any sense guiding the behaviours.
However our current working hypothesis, the Simple View, commits us to taking the leap from description to explanation.
According to the Simple View,
the Principles of Object Perception (which include the principle of continuity) are things that infants know or believe, and infants can form beliefs about particular physical objects and their movements and interactions by using the principles in inferences.
The Simple View therefore offers us a candidate explanation of how it is that infants (and adults) are able to track some causal interactions.
But is the explanation correct?


\section{The Case for the Simple View}
\label{sec:for-simple-view}

The main question for this part (\cref{part:physical-objects}) is, How do humans first come to know simple facts about particular physical objects, facts such as that this mug is over there?
All the evidence we have considered so far points to an attractively uncomplicated answer in the form of the \gls{Simple View}.
According to this View, from four months of age or earlier, infants know a small number of principles about physical objects, their movements and Interactions.
And they are able to combine these principles with sensory information inferentially, thereby acquiring beliefs or knowledge about particular physical objects’ boundaries, locations and causal interactions.

Before considering a challenge to the Simple View (in \cref{cha:linking-problem}), let us pause to analyse the leaps of reasoning that take us to it.
This is helpful both for evaluating the strength of the case for the Simple View and for working out what rejecting the Simple View would entail.

We started by asking a relatively simple question (in \cref{sec:three-abilities}).
When and how do humans first manifest the three abilities associated with knowledge of objects?
As we have seen, a variety of evidence indicates that humans manifest all three abilities by four months of age.
Four-month-olds can segment objects, represent them as persisting and track some of their causal interactions.
This is an extraordinary discovery.
It used to be quite widely held that the infants lived in a world of mere features until much later in their development, and that these abilities were a consequence of learning through purposive interactions with physical objects \citep{flavell:1963_piaget}.
The grounds for that view have been almost entirely swept away by the newer research we have been considering.
But the new discoveries leaves us with a question.
Since it is not plausibly a consequence of learning through purposive interactions with physical objects, how do four month old infants segment objects, represent them as persisting and track some of their causal interactions?

An important step towards answering this question is Spelke’s discovery that all three abilities in infants---to segment objects, represent them as persisting and track some of their causal interactions---can be described by invoking to a single set of principles, the \gls{Principles of Object Perception}.
(The principles discussed in this chapter are listed in \cref{table:principles-of-object-perception}.)
This is a revolutionary discovery insofar as infants’ abilities to segment objects were previously thought to depend on information about shape and texture rather than being characterised by facts about physical objects’ nature, movements and causal interactions.
This discovery also hints that infants’ abilities to segment objects, to represent them as persisting and to track their causal interactions may all be consequences of a single mechanism.
As Carey and Spelke put it, ‘[a] single system \ldots\ appears to underlie object perception and physical reasoning’ \citep[][p.~175]{Carey:1994bh}.



\begin{table}

\begin{center}
\footnotesize	%shrink for better spacing

% add space between rows
\extrarowsep=7pt

\begin{tabu} to 0.8\linewidth {X[1,l] X[3,l]}

\toprule

no action at a distance & ‘objects are interpreted as moving independently of one another’
\\
rigidity & ‘objects are interpreted as moving rigidly’
\\
cohesion & ‘two surface points lie on the same object only if the points are linked by a path of connected surface points’
\\
boundedness & ‘two surface points lie on distinct objects only if no path of connected surface points links them’
\\
continuity & ‘an object traces exactly one connected path over space and time’ and objects’ paths cannot cross
\\
\bottomrule
%
\end{tabu}
\caption{Some principles which partially characterise how infants segment physical objects, represent them as persisting, and track their causal interactions.
Source: \citet{Spelke:1990jn} and  \citet{spelke:1995_spatiotemporal}}
\label{table:principles-of-object-perception}
\end{center}	%careful -- position of this affects distance between table and caption(!)


\end{table}

\normalsize



But of course we want more.
We want to understand what kind of ‘system’ this is.
To this end, it is useful to distinguish three claims about the \gls{Principles of Object Perception}.
(These are summarised in \vref{table:levels-claims-about-the-principles}.)
The first is that they are \emph{\gls{formally adequate}}.
That is, someone who took the principles to be true, was omniscient about the arrangement of surfaces in space and had unlimited cognitive resources could, within limits at least, use the principles to segment objects, represent them as persisting even while briefly hidden from view and track their causal interactions.
While it is unlikely that anyone has yet formulated principles that are formally adequate in this sense, the principles we do have suffice to give us a sense of what a formally adequate set of principles would look like.
To say that the Principles of Object Perception are formally adequate is not yet to say anything at all about how these principles relate to infants’ abilities.

A further claim is that the principles are \emph{\gls{descriptively adequate}} for capturing infants’ and others’ abilities.
That is, these principles allow us to generate correct predictions about how infants, adults and many nonhumans will interpret particular scenarios.
They tell us, for instance, how many moving objects someone will interpret a scenario as containing, and where she will locate those objects at different times.
If, as I assume, the principles are not too far from being formally and descriptively adequate, then we could in principle use them to build a machine whose abilities regarding physical objects matched those of infants.
But in building such a machine we would not necessarily be replicating the inner workings of an infant.
The claims about formal and descriptive adequacy are not candidate answers to questions about what underlies infants’ abilities.
That certain principles are descriptively adequate for capturing infants’ abilities to segment objects, represent them as persisting and track some of their causal interactions does not yet tell us how it is that infants are able to do these things.

This is where the third claim comes in.
To say that the principles are \emph{\gls{explanatorily adequate}} for capturing infants’ or others’ abilities is to say that there is a link between the principles and their minds, and that it is partly in virtue of this link that they have these abilities.
Where such a link exists, the principles do not merely describe the infants’ or others’ abilities: they explain them by virtue of their role in characterising processes, representations or systems underlying the abilities.
% For instance, if the principles characterise the contents of knowledge states (as the \gls{Simple View} implies), then they are explanatorily adequate.
While researchers disagree on how the principles are linked to infants’ minds (as we will see in \cref{cha:linking-problem}), many would probably accept the broad claim that the Principles of Object Perception are {explanatorily adequate}.
We too should accept this claim, at least provisionally, not because the available evidence  overwhelmingly supports it but rather because there is currently no better supported alternative.


\begin{table}

\begin{center}
\footnotesize	%shrink for better spacing

% add space between rows
\extrarowsep=7pt

\begin{tabu} to 0.8\linewidth {X[1,l] X[3,l]}

\toprule

Formal Adequacy & If someone took the principles to be true, was omniscient about the arrangement of surfaces in space and had unlimited cognitive resources, to what extent would she be able to segment physical objects, represent them as persisting and track their causal interactions?
\\
Descriptive Adequacy & Do the principles enable us to generate correct predictions about infants’  and others’ abilities to segment physical objects, represent them as persisting and track their causal interactions?
\\
Explanatory Adequacy &  Is there a link between the principles and infants’ or others’ minds, and does this link partly explain how it is they are able to segment physical objects, represent them as persisting and track their causal interactions?

%Do the principles play a role in characterising processes, representations or systems underlying infants’ and others’ abilities to segment physical objects, represent them as persisting and track their causal interactions?
\\
%
\bottomrule
%
\end{tabu}
\caption{Three questions about the Principles of Object Perception.}
\label{table:levels-claims-about-the-principles}
\end{center}	%careful -- position of this affects distance between table and caption(!)


\end{table}

\normalsize

The claim that the Principles of Object Perception are explanatorily adequate is too schematic to be satisfying.
Accepting this claim requires us to ask,
What links the principles and infants’ minds in such a way that infants are able to segment objects, to represent them as persisting and to track their causal interactions?
\glsadd{Linking Problem}

The \gls{Simple View} is one answer to this question.
It is not the only possible answer:
accepting that the Principles of Object Perception are  explanatorily adequate does not compel us to accept the Simple View.
But the Simple View is the only way of answering the question we have yet considered.
On the Simple View, the relation between the principles and infants’ minds is one of knowing or believing.
It is in virtue of infants’ knowledge of, or belief in, the Principles of Object Perception that they are able to segment objects, represent them as persisting and track their causal interactions.

As I keep saying, the Simple View provides an answer to our main question,
How do humans first come to know facts about particular physical objects, such as their locations?
According to the Simple View, they do this by a process of inference which combines principles they know or believe with sensory information about arrangements of shapes in space.
This is an attractive answer insofar as it postulates only states and processes that are familiar and that are needed to explain many other things.
It also has the advantage of entailing an uncomplicated thesis about how infants’ earliest abilities relate to adults’ abilities: they are identical.
Unfortunately, as we are about to see, the Simple View is unlikely to be true.


% These, then, are Spelke’s brilliant insights.
% First, the three requirements on knowledge of objects---having abilities to segment objects, to represent them as persisting and to track their causal interactions---are met by infants in a way that reflects ‘basic constraints on the motions of physical bodies’ \citep[p.~51]{Spelke:1990jn}; these constrains are what the Principles of Object Perception codify.
% Second, attributing knowledge of, or belief in, these principles together with a capacity to draw inferences from them would explain both how infants (and adults) perceive objects and how they reason about them.










% \subsection{slide-200}
% So far we can draw two conclusions about infants' and adults abilities to track
% interactions.  My \textbf{first conclusion} from this section is that infants from around
% 4 months of
% age or younger and nonhuman animals are able to track simple causal interactions.

% \subsection{slide-201}
% I started by identifying three abilities associated with knowledge of physical objects:
% abilities to segment objects, to represent them as persisting, and to track their causal
% interactions.
% My \textbf{second conclusion} is that a single set of principles likely underlies these
% abilities.  The ability to segment objects is bound up with the ability
% to represent them as persisting and with the ability to track their interactions.

% \subsection{slide-202}
% My \textbf{third conclusion} is that we have a problem.
% The problem is that we have to reject the simple view.
% Recall that the simple says that the principles of object perception are things that we know
% or believe.
% We must reject this view because it makes systematically incorrect predictions about actions
% like searching for objects.

% \subsection{slide-204}
% But why is this a problem? Because, as we'll see, it is hard to identify an alternative.




%%% Local Variables:
%%% TeX-master: "master"
%%% End:
