%!TEX root = master.tex

% ∞todo : chapter summary.  What are the assumptions about development that this chapter challenges?  (1) Infants’ capacities are underpinned by knowledge (but chapter 2 already suggested this); (2) concepts precede words (maybe they do in other cases, but they do not in the case of colour); (3) ??? (Maybe drop claim that it does challenge assumptions about development from the introduction?)

\chapter{Colour}
\label{cha:colour}

\noindent
Why colour?
Knowledge of colour isn't foundational in the way that knowledge of objects or of mental states is.
Dress sense aside,
missing out on such knowledge would not dramatically impair a human.
But there is some amazing research on  colour cognition that has been widely ignored by both scientists interested in how minds develop and philosophers of perception.
Reflection on this research is probably indispensable for understanding the developmental precursors of knowledge.
And thinking about how knowledge of colour emerges will challenge almost any prior assumption you might make about the emergence of knowledge.
This is why the present chapter is about how humans first come to know simple facts about colour.
How does a human develop the ability to know that, say, that this tomato is red or that that berry is blue?


\section{Categorical Colour Properties}

Three patches of colour are shown in \vref{fig:three_colours}.
%
\begin{figure}
\begin{center}
\includegraphics[scale=0.3]{fig/three_colours.png}
\caption{
	\label{fig:three_colours}
	Two blues and a green.  Relative to one measure of perceptual similarity, the centre blue is as similar to the left colour as to the right colour.  Yet they are categorised differently.
}
\end{center}
\end{figure}
%
The patches are all different colours, but the two leftmost are both the same colour---they are both blue.
This sounds contradictory but isn't.
In saying that the patches are all different colours, we're talking about the particular shades of colour;
in saying that two of the patches are the same colour, we're talking about \glspl{categorical colour property}.
This chapter is about  categorical colour properties like blue and red, not about particular shades of colour.
It is these categorical colour properties that matter for distinguishing in conversation the edible from the poisonous plants, and the harmless from the dangerous animals.

Our question, then, is not about all kinds of knowledge of colour.
Instead it is specifically about how knowledge of simple facts about the categorical colour properties of things emerges in development.
Depending on your colour vision and linguistic background, you may have come to know on the basis of visual input that the left two patches in \cref{fig:three_colours} (\vpageref*{fig:three_colours}) are both blue whereas the other is green.
Clearly you were not born knowing this or any other facts about the categorical colour properties of particular objects.
So we can ask how it is that you and other humans first come to know about the categorical colour properties of things.
How does the capacity to acquire such knowledge emerge in development?



\section{The Simplest Possible Story: Visual Appearances}
\label{cp:sec:no_appearance}

Start with the simplest possible story.
Red things visually appear one way to you, namely as red, and blue things visually appear another way, namely as blue.
As Byrne and Hilbert put it:
%
\begin{quote}
‘If someone with normal color vision looks at a tomato in good light, the tomato will appear to have a distinctive property—a property that strawberries and cherries also appear to have, and which we call ‘red’ in English’ \citep[p.~4]{byrne:2003_color}.
\end{quote}
%
These facts about visual appearance enable you to lock onto colour properties like \emph{red} in thought.
This in turn enables you to come to know that certain objects are red.
Fodor relies on a story along these lines.
He writes:
%
\begin{quote}
‘all that’s required for us to get locked to redness is that red things should reliably seem to us as they do, in fact, reliably seem to the visually unimpaired’
%  Correspondingly, all that needs to be innate for RED to be acquired is whatever the mechanisms are that determine that red things strike us as they do’
\citep[p.~142]{Fodor:1998ap}.
\end{quote}
%
This is perhaps the simplest story it is possible to tell about the developmental emergence of knowledge of categorical colour properties.
But is it a true story?


\section{Red Things Do Not Visually Appear Red}
\label{sec:red-not-visually-appear}
%Byrne and Hilbert and Fodor are only guessing.
%They offer no evidence or argument.
It turns out that the simplest story, for all its intuitive appeal, is spectacularly wrong.
Red things do not visually appear red;
nor do any categorical colour properties have characteristic visual appearances.

In case this claim seems obviously wrong, please note that it is about categorical colour properties.
These are the kind of colour properties which things of noticeably different shades, like cherries and tomatoes, can share.
We are not debating whether a particular tomato can visually appear to you to have a particular shade of colour.
The issue is rather whether it can visually appear red.

How do we know?
Visual appearances have various characteristic effects.
Where things share a visual appearance, people are likely to judge that they are more similar than when they do not.
They are also more likely to visually group things with a common visual appearance than things without one, all else being equal.
The claim that all red things share a distinctive visual appearance in virtue of being red therefore generates testable predictions about judgements and visual grouping.
In studies with human adults, these predictions have been carefully tested and convincingly falsified.

To test these predictions we would ideally use pairs of things that differ only in that one is red whereas the other is not.
Unfortunately no such pairs exist: where two things differ in whether they are red, they will also differ in which particular shade of colour each has, of course.
We therefore need a way of separating effects due to differences in particular shades of colour from effects due to differences in categorical colour properties.
To this end we need to find sequences of three objects like those in \vref{fig:three_colours}:
first, with respect to particular shades of colour, each object in the sequence should be as different from its neighbours as any other;
and, second, with respect to categorical colour properties, the first two things should share a categorical colour property which the third thing lacks.
(That such sequences exist is well established; see, for instance, \citealp{witzel:2013_categorical}.)
If the claim that categorical colour properties like red have characteristic visual appearances were right,
then the first two things in any such sequence should differ from each other less in visual appearance than the last two things differ from each other.

Witzel and Gegenfurtner set out to test this.%
\footnote{%
See \citealp{witzel2014category} for a brief report; at present the full report of these experiments exists only in Christoph Witzel’s PhD thesis.
\citet{kay_what_1984} made an earlier attempt to test the same prediction, and reached the same conclusion.
However their critical Experiment 2 faces an objection which Witzel and Gegenfurtner’s does not.
\citeauthor{kay_what_1984}  instructed their subjects, ‘I’d like you to tell me which is bigger: the difference in greenness between the two chips on the left or the difference in blueness between the two chips on the right.’
This instruction might in principle have caused subjects to focus on hue and ignore other features.
}
In one of their experiments, they showed adults sequences of three colours as just described.
These adults were then asked to say which of the two outer things was more like the middle thing.
If things which are both red (or both have the same categorical colour property) thereby share a visual appearance,
then adults should judge that the two red things are more similar.
But they do not.%
\footnote{%
‘[T]here were literally no effects at boundaries’ (Witzel, personal communication).
}

Visual appearances, if they exist, should affect not only how people make perceptual judgements of similarity but also how they perceptually group objects.
Consider, for example, the left panel of \vref{fig:perceptual_grouping}.
Do you see a line of shapes running diagonally?
Which way is the diagonal line running?
People shown this image tend to see a diagonal running from the top left corner.
By contrast,  the right panel of this figure is usually seen in the opposite way; and the middle panel would probably be seen in neither way.
This is an illustration of perceptual grouping.
The similarity in visual appearance due to shape influences how objects are grouped together.


%
\begin{figure}
\begin{center}
\includegraphics[scale=0.4]{fig/perceptual_grouping.png}
\caption{
	\label{fig:perceptual_grouping}
	An illustration of perceptual grouping.
}
\end{center}
\end{figure}
%


If categorical colour properties like red and blue have characteristic visual appearances,  these visual appearances should likewise affect perceptual grouping.
\citet{webster:2012_color} tested whether categorical colour affects perceptual grouping by asking adults to make judgements about orientation.
Their stimuli are depicted in \vref{fig:perceptual_grouping_colour}.
Overall, their results indicate that differences in categorical colour property have no  measurable effect on perceptual grouping.
% They found that almost no indicators of an effect of colour category on perceptual grouping (there are a variety of measures and one was signifiant:  p. 381: ‘the participants’ settings thus trended toward a (very weak) CP effect, with an average bias of 0.10 (which was nevertheless significantly different from zero; t (7) = 3.73, p < .01).’). They also suggested that these effects may be due to the use of  CIELAB as a colour space (p. 382: ‘ the small biases we found in the observer’s settings may in part include an artifact of the stimulus space, weakening further the evidence for a clear CP effect in the grouping task.’)

\begin{figure}
\begin{center}
\includegraphics[scale=0.5]{fig/webster_2012_fig1.png}
\caption{
	\label{fig:perceptual_grouping_colour}
	From \citet[figure 1]{webster:2012_color}.
}
\end{center}
\end{figure}

As we have just seen, the view that categorical colour properties have characteristic visual appearances generates multiple incorrect predictions.
Indeed, a further prediction of this view is falsified by \citet{davidoff:2012_perceptual}.
And, as far as I know, no experiment convincingly supports a prediction of this view.
A further, indirect consideration against the view comes from neurophysiological evidence that detection of differences in categorical colour properties does not involve visual processes (\citealp{he:2014_color}; more on this later).
% NB: this claim is not secure; see \citep{zhong:2015_shortterm}.
Taken together, the available evidence suggests that categorical colour properties do not have characteristic visual appearances.

This is a striking conclusion to reach.
That all red things share a characteristic visual appearance is one of the few claims about colour that philosophers agree on; it is taken for granted in at least one encyclopaedia \citep{maund:2012_color} and entailed by what Jackson calls a ‘subject-determining platitude’ in a much-cited discussion \citep[pp.~199--200]{Jackson:1996zz}.
But the claim is probably false.
Being red probably does not entail having a visual appearance characteristic of red things.

This is not to deny that a red ball differs in appearance from a blue ball, of course.
Being red entails being a different shade of colour from anything that is blue.
The view we are considering is consistent with particular shades of colour having characteristic visual appearances.
What it denies is only that categorical colour properties have characteristic visual appearances.

The evidence we have been discussing all comes from research on adults.
Can we extend the conclusion that categorical colour properties lack characteristic visual appearances to infants?
This is not entirely straightforward given that infants’ categorical perception of colour may differ from adults’ (see \cref{cp:sec:development}).
It would be good, although fiendishly complicated,  %complicated  because their categories are not the same
to know whether infants’ behaviours also falsify predictions of the view that categorical colour properties have characteristic visual appearances.
But in the absence of such evidence, it seems reasonable to guess that for infants, as for adults, categorical colour properties do not have characteristic visual appearances.

We must therefore reject the simplest possible story.
We cannot explain how humans first come to know things about the categorical properties of particular objects by appeal to visual appearances because, despite the claims of many philosophers, red things do not visually appear red.

Given how hard finding a replacement for the simplest possible story will turn out to be, you might at some point be tempted to insist that, whatever the best available research indicates, categorical colour properties do have characteristic visual appearances.
But note that even if this were true, the problem would still arise.
This is because the simplest possible story rests on a further false assumption.
Let me explain.

Suppose for a moment, incorrectly, that blue things (say) have a characteristic a visual appearance.
Clearly this visual appearance is only  relevant for explaining the emergence of knowledge about the blueness of particular objects where someone has corresponding visual experiences---where, that is, she can visually experience the blueness of blue things.
Equally clearly, anyone entirely incapable of visually discriminating blue things from non-blue things probably cannot visually experience blueness either.
But there is compelling evidence that the ability to visually discriminate blue things is a consequence of learning to comprehend and correctly apply the word ‘blue’ or some other verbal label for blue.  (This evidence includes \citet{Wiggett:2008xt}, \citet{Winawer:2007im}, \citet{clifford:2012_neural} and \citet{Franklin:2005hp}; see \cref{sec:cp_redescription,cp:sec:development}.)
Further, someone who can correctly apply a colour word like ‘blue’ is likely already to have colour concepts and so  be in a position to know some things about the colours of particular objects \citep{Kowalski:2006hk}.
Putting all this together indicates that the simplest possible story has things the wrong way around.
It is not that visual appearances enable us to get locked onto categorical colour properties and thereby first come to know things about the colours of particular objects.
Rather, visual appearances are---or would be, if they existed---a \emph{consequence} or component of getting locked onto categorical colour properties.
This is why even someone who held that categorical colour properties have characteristic visual appearances---that things can visually appear to be blue or red, say---must reject the simplest possible story.

Rejecting the simplest possible story leaves us with a problem.
How else can we  explain the  emergence in development of knowledge of facts about the categorical colour properties of things?


\section{The Mystery of Categorical Perception of Colour}
\label{cp:sec:the_mystery}

To investigate this question it is essential that we first understand something about \emph{\gls{categorical perception of colour}}, the process or processes by which humans (and perhaps other animals) detect categorical colour properties like yellow, orange and red.
We will start  by thinking about human adults before eventually (in  \cref{cp:sec:development}) working back  to infants.

There are various systems for arranging particular shades of colour in space.
For instance, they can be arranged by wavelength, as in a rainbow.
Alternatively, shades of colour can be arranged in accordance with a model of how the several types of cone in the retina respond to different stimuli.
A system for arranging shades results in a colour space of three or more dimensions which can be used to predict which particular shade of colour a light source or surface will appear to have.
Importantly, in none of these systems for arranging shades of colour is there anything to mark all the boundaries between categorical colour properties: there is no line separating blue from green, say.
%Suppose your knowledge were restricted to the physical properties of light and the operations of the retina.
%Then you would be entirely unable to understand what is special about a particular set of categorical colour properties; this would be just one among indefinitely many equally good ways of dividing the a colour space into categories.
%note
%(This is not to say that all ways of dividing it up are equally good; some argue that the structure of the retina places constants on the variety of ways in which humans could divide a colour space into categories.)
%The fact that any humans know about the blueness of the sky or the redness of blood would be entirely mysterious to you.
Categorical colour properties like blue and green are not generally marked out physically, nor in the way the retina operates.

Yet when monolingual speakers with normal vision are tasked with discriminating particular shades of colour, they are faster and more accurate in discriminating between shades that they would label differently, such as a blue and a purple for English speakers, than shades they would give the same label to, such as two blues for English speakers (faster: \citealp{Bornstein:1984cb}; more accurate: \citealp[p.~22--7]{Roberson:1999rk}; both: \citealp{witzel:2014_categorical}).



%\Vref{fig:two_colour_sequences} represents two sequences of three colours.

%%
%The vertical sequence shows three greens and the uppermost horizontal sequence shows a blue, a purple and a pink.
%Each colour differs from its neighbours by the same amount according to a standard measure of perceptual similarity.
%%*the topic of the footnote might come up later
%\footnote{%
%Ideally the distances between the colours would be equal in a cone-opponent colour space, that is, one in which distances correspond to retinal responses.
%In fact most experiments use colour spaces intended to be uniform relative to measures of perceptual distance, like the one depicted in \cref{fig:two_colour_sequences}.
%This complicates things theoretically, but does it matter in practice?
%\citet{webster:2012_color} and \citet{witzel:2013_categorical} both analysed their experiments using first perceptually uniform and then cone-opponent colour spaces and got similar results in both cases.
%(This is not to say that a poorly chosen colour space never affects experimental results; compare \citealp{brown:2011_color}.)
%}
%Technically the space is intended to be \emph{perceptually uniform}: that is, there is a constant $r$ such that, for any colour in the space, the colours which the retina cannot distinguish from that colour are those within a circle of radius $r$.)

These differences in speed and accuracy cannot all be explained by  physical properties of the stimuli, nor by the way they are processed by the retina.
Nor are differences reflected in humans' abilities to make extremely fine-grained distinctions between shades of colour \citep{witzel:2013_categorical}.
%\citet{witzel:2013_categorical} have shown this by establishing that if you measure just noticeable differences in colour for an observer, you will not typically find that just noticeable differences are larger at the boundaries her colour words mark than elsewhere.
The fact that the colour categories marked out by someone's colour words predict differences in the speed and accuracy of her visual discrimination is therefore something of a mystery.
What explains these differences?

Could it be memory?
For all that we've seen so far, this appears plausible.
Speed and accuracy of discrimination is usually tested using something called a two alternative forced-choice task. % (2AFC).
In such a task, you are first presented with a uniformly coloured circle (say).
Then the circle is removed and you are now shown two uniformly coloured circles.
One circle, the \emph{target}, has the same colour as the first; the other circle, the \emph{distractor}, has a different colour.
Your task is to say which of these two circles is has same colour as the first.
The results just mentioned imply that you will be faster and more reliable on this task when the target and distractor would be given different colour names in your language (perhaps one is blue and the other purple, say) than when target and distractor would be given the same colour name (they are both blue, say).
This task may demand memory because there is always a delay, however short, between your seeing the first circle and your having to make the choice.
This is why it is tempting, at least initially, to think that the  mysterious effects of categorical colour properties on speed and accuracy of discrimination can be explained away by invoking memory.
But can they?

They cannot.
Direct attempts to measure the influence of categorical colour properties on how colours are remembered suggests that there is no such influence \citep{wright:2014_whorfian}.
Further, performance on a task which minimises demands on memory by presenting all the stimuli simultaneously nevertheless reveals effects of categorical colour properties.
This task hinges on \gls{pop out} effects.
Informally, pop out occurs when something jumps out at you.
If you are a monolingual English speaker looking at a sea of blue and pink dots in which there is just one purple dot, the purple dot will pop out at you.
How can we characterise pop out if we want to measure it?
Operationally, pop out is defined in terms of visual search tasks \citep[p.~117]{Treisman:1986pm}.
In a visual search task, you are shown an array of objects and asked to identify the one with a certain property; call the  one you are to identify \emph{target} and call the others \emph{distractors}.
For example, you might be shown an array of sixteen animal pictures and asked to identify the cat.
%comment: linear separability results in easy searches, this is why we need distractors to have two colours.
Or you might be shown sixteen coloured dots, where  the distractors are the colour of the top or bottom green patches from \vref{fig:two_colour_sequences} and the target is the colour of the middle green.
Normally the time it would take you to find the target increases linearly as the number of objects in the array increases.
This is just what you would expect if searching involves processing each object in turn.
But what if the target and distractors differ in their categorical colour properties: say, the target is the colour of the purple patch from \cref{fig:two_colour_sequences} and the distractors are the colour of the blue or pink patches from that figure?
Then, as illustrated in \vref{fig:pop_out}, you would be able to find the target just as quickly when it is one of 36 dots as when it is one of only four dots \citep[Experiment 1]{Daoutis:2006qk}.%
\footnote{%
See also \citealp{wright:2012_categoricala}.
The use of three (rather than two) colours in this experiment is essential to avoid the possibility that the results could be explained as an effect of linear separability \citep{bauer:1996_visual}.
Also note that, as \citet{lindsey:2010_color} show, pop out probably does not occur for all differences in colour that colour names mark.
}
%\citep[p.~1213]{lindsey:2010_color}: ‘we found no evidence that variations in visual search RT are related to the lexical color categories to which the stimulus colors belong.’
%Clearly, in this case visual search cannot be a matter of processing the objects sequentially.
Categorical colour properties \gls{pop out}.

%
\begin{figure}
\begin{center}
\includegraphics[scale=0.3]{fig/two_colour_sequences.png}
\caption{
	\label{fig:two_colour_sequences}
	Two sequences of three colours in a perceptually uniform colour space (CIELUV).  The vertical sequence represents three greens, the horizontal a blue, purple and pink.  This figure is modified from \citet[][p.~239 figure A1]{Daoutis:2006ij}.
}
\end{center}
\end{figure}

\begin{figure}
\begin{center}
\includegraphics[scale=0.3]{fig/daoutis_2006_fig2mod.jpg}
\caption{
	\label{fig:pop_out}
	Categorical colour properties \gls{pop out}.  This figure is modified from \citet[][figure 2]{Daoutis:2006ij}.
}
\end{center}
\end{figure}


Let us take stock.
We have seen that which colours are named in a language predicts not only differences in how rapidly and accurately monolingual speakers of that language can discriminate various shades of colour
but also when colours will pop out at them.
Since pop out is not  plausibly a consequence of memory effects,
clearly we cannot explain away all of these facts by invoking memory.
Worse still, there is plausibly a uniform explanation for the facts about discrimination and pop out, which are presumably consequences of a single process.
If so, we cannot explain away any of these facts as mere memory effects.
(See \citealp[][]{Franklin:2005hp} for further arguments that do not depend on pop out.)
But then what does explain these facts?
Why do differences in which categorical colour properties people have names for predict not only differences in the speed and accuracy with which they can discriminate particular shades of colour but also when things will \gls{pop out} at them?

Could it be the way things visually appear?
Some psychologists and philosophers clearly think so, for they have defined categorical perception in terms of differences in visual appearances (\citealp[p.~288--9]{Bornstein:1987vv}; \citealp[p.~190]{Matthen:2005sc}).
Certainly it is natural to suppose that pop out is a consequence of the fact that  blue and purple (say) are associated with different characteristic visual appearances.
In principle, such visual appearances could also explain effects on speed and accuracy.
There is just one problem.
As we saw in \cref{sec:red-not-visually-appear}, all the available evidence uniformly indicates  that categorical colour properties like blue and purple do not have characteristic visual appearances.
We cannot appeal to the way things visually appear to explain the effects of categorical colour properties if there are no such visual appearances.


If neither memory nor visual appearance explains why categorical colour properties can predict  differences in speed and accuracy of  visual discrimination as well as when colours will \gls{pop out}, what does explain all this?
Some have sought explanations by appeal to structural features of colour.
For example, Matthen claims that the pattern of heightened discrimination is due to the saliency of unique hues \citep{Matthen:2005oq}.
Others have considered the possibility that some or all of the effects on discrimination and pop out are really experimental artefacts arising from the difficulty of controlling for the perceptual similarity of particular shades of colour (\citealp[for example,][]{brown:2011_color}; for discussion see \citealp{clifford:2012_neural} and \citealp{wright:2012_categoricala}).
This is initially plausible because controlling for perceptual similarity is challenging, and differences in perceptual similarity are exactly the sort of thing that could explain things like speed and accuracy of discrimination.
It should also be noted that in earlier research on how categorical  colour properties are detected, researchers were much less careful than they have now become in considering explanations that appeal to inequalities in the perceptual similarity of particular shades of colour.
Nevertheless, both attempted explanations---the appeal to structural features of colour and the challenge concerning perceptual similarity---are clearly incorrect.

To see why, suppose we invent a word for an arbitrary new categorical colour property, one that is not named in any language, and train some people to use this novel colour word.
Suppose the effects of categorical colour properties on discrimination and \gls{pop out} could be explained away as an artefact of structural features of colour or of researchers' failure to equate perceptual similarity.
In that case, being trained to use the new colour name should make no difference.
The people we have trained should no be faster or more accurate in detecting things when there are differences with respect to this newly named categorical colour property.
Nor should training affect whether colours pop out and any of the other responses associated with these.
After all, the training does not alter structural facts like which hues are unique, nor does it alter how perceptually similar particular shades of colour appear.
Nevertheless, the effect of such training is indeed to alter people's responses so that they behave with respect to the newly named categorical colour property as they behave with respect to other categorical colour properties they have names for \citep{Ozgen:2002yk,zhou:2010_newly,clifford:2012_neural,zhong:2015_shortterm}.
%Four: train new categories; compare speakers of different languages; compare left and right visual fields; consider the effects of verbal interference.

This, then, is the mystery of categorical perception.
We know that human adults can detect some categorical colour properties because differences in such properties predict various effects including speed and accuracy of discrimination as well as \gls{pop out} (and we will encounter  more effects later in this chapter).
The mystery is how they do this.
As we have seen, it is neither a matter of mind-independent physical properties nor of retinal processes.
It is not a consequence of memory, nor of how categorical colour properties visually appear.
And it is neither a matter of structural features of colour, nor of certain shades of colour being more perceptually similar than others.
Categorical perception, the process or processes by which humans and perhaps others detect categorical colour properties, is unlike any of the mental phenomena philosophers standardly consider in thinking about the mind.
This matters: neglect of categorical perception and other similarly anomalous processes has been a major obstacle to understanding the origins of mind.





\section{Categorical Perception: A Fresh Start}
\label{cp:sec:fresh_start}

The question for this chapter is, How do humans first come to know facts about the categorical colour properties of particular things?
How, that is, do they first come to know that this thing is red and that one is green (say)?
A first attempt to answer this question was the simplest possible story, which appeals to visual appearances (see \cref{cp:sec:no_appearance}).
The catastrophic failure of this attempt means that
making progress with the question demands understanding \gls{categorical perception of colour}, the process or processes by which humans and perhaps other animals detect categorical colour properties.
But we have just seen (in \cref{cp:sec:the_mystery}) that there is an obstacle to understanding what sort of process categorical perception is.
It doesn't fit neatly into philosophical frameworks standardly used to characterise minds, and there is currently no good account of what it is.

% You might be thinking that this is all very negative.
% But not understanding things is what philosophers do best.
% If you read scientific articles about categorical perception,
% you may notice that the definitions typically given lack depth in that they merely refer to patterns of discrimination rather than the processes which explain them.
Understanding what categorical perception is poses the same kind of problem that understanding what \gls{core knowledge} is  poses (see \cref{sec:core-knowledge,sec:against-core-systems}).
For both categorical perception and core knowledge,  it is easier to be convinced that it must exist than to understand what it is.
This is an obstacle to our project of understanding how knowledge of colour emerges in development.
How can we characterise categorical perception?

We need an approach to understanding what categorical perception of colour is that starts with evidence rather than assumptions.
Rather than assuming we know what categorical perception is---which, basically, no one yet does---we should take nothing for granted and make no assumptions about what sense, if any, categorical perception involves either categories or perception.
Instead we should start by asking, What is categorical perception commonly taken to explain?

We have already seen part of the answer to this question.
For an individual, there is a way of dividing a perceptually uniform colour space into  a small finite number of equivalence classes---corresponding to categorical colour properties---that enables us to predict:
%
%\begin{itemize}
%\item
which colour words she will use to describe things;
%\item
otherwise unexpected patterns in the speed and accuracy  with which she can discriminate shades of colour; and
%\item
when colours will \gls{pop out} at her.
%\end{itemize}
%
The fact that a single division into categories predicts all of these things suggests that they are explained by a single process.
Categorical perception is that process, whatever it is.

This way of identifying categorical perception is not very informative.
%(It is a little bit informative because it might in principle turn out to be merely a coincidence that the same way of dividing colours into categories predicts all four effects; the various effects might in principle be explained by four separate processes.)
That is a virtue---categorical perception is something to be discovered, not defined.
And we can infer things about categorical perception from the things that it explains.
For instance, colours popping out at you will typically alter the overall phenomenal character of your experience in some way.
So from the fact that it explains pop out, we can infer that categorical perception can affect phenomenology.
More carefully, we can infer that, even though there is no visual experience of categorical colour properties (see \vref{cp:sec:no_appearance}), differences in categorical colour properties can affect the overall phenomenal characters of perceptual experiences, at least indirectly.
This fact about categorical perception will later turn out to be essential to understanding its role in the developmental emergence of knowledge.

To discover more about categorical perception we need to identify further  things to be explained.
Consider one last such thing, one which will show that categorical perception, although it probably does not involve early visual processes, is automatic.
A process is \emph{\gls{automatic}} to the degree that whether it occurs or not is independent of the subject's motives and tasks.
To illustrate, take the processes that enable humans  to balance on two legs, or to recognise whether a string of words is a sentence.
These are (mostly) automatic: whether they occur is to a large degree independent of your current motives and tasks.
How can we tell whether categorical perception is an automatic process?

One approach involves something called an oddball effect.
Suppose you see a series of things that are all the same; and then, unpredictably, you see something different.
An \emph{\gls{oddball effect}} occurs when your brain responds differently to the different thing.
This response indicates, of course, that you have detected the difference, or that some part of you has detected it.
(How are brain responses measured? \Cref{sec:brain-methods} at the end of this chapter is an appendix this.)
Where an oddball effect occurs independently of your task and motivations, we have a powerful indicator that the processes responsible for the oddball effect are automatic.
 We can therefore exploit oddball effects to test whether categorical perception of colour is automatic.
 But how?

Suppose you are asked to perform the following simple task repeatedly:
fixate on a cross and press a button when you see a rectangle.
Rectangles come and go, and you press the space bar when one appears: that's all there is to your task.
But it happens that the rectangles are accompanied by coloured blocks.
These coloured blocks have no relevance to your task and you are instructed to ignore them.
The irrelevant coloured blocks are nearly always the same colour, but just occasionally one is a different colour---call this the \emph{oddball colour}.
We are interested in what happens when the oddball colour differs in category from the usual colour (one is blue and the other green, say).
More carefully, we are interested in colour differences that cross the boundaries of the particular categories for which you use colour words.
Do changes in colour give rise to an \gls{oddball effect}?
It turns out that they do.
Can we conclude that your brain detects the change in categorical colour property?
Not yet.
After all, the effect might a consequence just of the difference in particular shades of colours.
To investigate this possibility, we need a new type of oddball, one which involves a difference in shade but no difference in categorical colour property.
If your brain’s responses to the first type of oddballs were just a consequence of the change in shade, then it should respond in the same way to this new type of oddball.
But, as it turns out, it does not.
This indicates that the oddball effects really are a consequence of changes in categorical colour properties.

Oddball effects provide a window on how categorical colour properties are processed in the brain.
The main findings are that categorical colour properties are generally detected around 200 milliseconds after they appear,
and that their detection is automatic \citep{fonteneau:2007_neural,thierry:2009_unconscious, athanasopoulos:2010_perceptual, clifford_color_2010, mo:2011_electrophysiologicala}.
Further, although it is currently difficult to be sure about this,
it is possible that the detection of categorical colour properties  involves processes that occur after early visual processes.%
\footnote{%
\label{fn:is_cp_early_vision}%
While there is evidence that some categorical colour properties are detected by early visual processes (reviewed by \citealp{czigler:2013_visual}; see also \citealp{zhong:2015_shortterm}),
there is also evidence that these do not include the categorical colour properties that subjects have basic colour words for  \citep{clifford:2012_neural,he:2014_color}.
This combination of findings may be due to methodological difficulties, or it may be that there are multiple processes which detect different sets of categorical colour properties (see further \cref{cp:sec:development}).
A reasonable guess is that the process which explains patterns in speed and accuracy of discrimination and \gls{pop out} in human adults involves things that happen after early visual processing.
}


%The detection occurs so quickly that it could not involve thought.
%And since it  occurs even though you were told to ignore the colour and even though the colour shouldn't matter to you, it is plausibly automatic.
%
%NB: also occurs when testing different languages and colour boundaries

We now have a further item, \glspl{oddball effect}, to add to our list of things categorical perception of colour is commonly taken to explain.
It is taken to explain how people will use colour words,
otherwise unexpected patterns in the speed and accuracy with which they can discriminate shades of colour, when colours will pop out, and when oddball effects will occur.
If, as few would seriously question, a single process explains all of these things,
then categorical perception exists and each thing on this list reveals something about what it is.
Categorical perception may not be perception in either of the usual senses: it probably does not involve visual experience of categorical colour properties (see \cref{cp:sec:no_appearance}), and it may not involve early visual processes either (see \vref{fn:is_cp_early_vision}).
But categorical perception does have some features in common with perceptual processes broadly understood.
As we have seen,
oddball effects are evidence that it is automatic,
and \gls{pop out} is evidence that categorical perception sometimes affects, at least indirectly, the overall phenomenal character of your experience.

Categorical perception is something unlike the mental phenomena philosophers and developmental psychologists usually consider.
It is not perception in any ordinary sense, but nor can it be memory or thought (as we saw in  \cref{cp:sec:the_mystery}).
It lacks essential features of visual experience, yet it resembles  perception in being automatic, and perhaps even in having consequences for the phenomenal character of experience.
Recognising the existence of processes with this combination of properties is the key to understanding how knowledge of the colours of things emerges in development.
And, as we shall see (in \cref{cp:sec:cp_is_core_knowledge}), these anomalous processes  probably play also a fundamental role in explaining the developmental emergence of knowledge not only of colour but of other things too.

Having got a fix on what categorical perception is,
%---an automatic process which, although it does not involve visually experience of categorical colour properties, does ensure that differences in categorical colour properties can modify the overall phenomenal character of an experience and  which underpins effects including speed and accuracy in discriminating by categorical colour properties as well as verbal labelling---
it is now time to turn our attention to its role in development.


%
%*This implies that adults have categorical perception of colour, at least according to one definition of categorical perception \citep{Repp:1984ko}.
%Take an adult and find out which categorical colour properties she has words for.
%Then ask her to discriminate repeatedly among sample colours.
%To say that she has categorical perception of colour is just to say that she is faster and more accurate in distinguishing stimuli that she gives different colour names to than those which she gives the same name to,
%
%




%*NB: \citep[p.~27]{witzel:2013_categorical}: ‘the pattern of JNDs was not specific to the categories that correspond to the basic color terms’  I don't understand this experiment --- uses DKL space, which is modelling the colour space associated with second-stage mechanisms to test whether JNDs are smaller between than within categories.  But how is the DKL colour space constructed?  Actually I do get it.  DKL is a model of the outputs of the second stage colour discrimination mechanism on the retina (at which ‘the activations of the cones are combined in the retinal ganglion cells to produce a three-dimensional (3-D) color-opponent space’ \citep[p.~1]{witzel:2013_categorical}), whereas the experiment concerns perceivers judgements in a 4AFC.  The procedure makes sense because ‘cortical color vision is governed by higher order mechanisms’ \citep{hansen:2013_higher}; the question is, in effect, whether these mechanisms are involved in the 4AFC discrimination task and whether they exhibit categorical effects.





%
%What then does determine where the boundary between blue and green falls for you or anyone else?
%What is the nature of your abilities to detect categorical colour properties?
%They are the consequence of  a phenomenon called categorical perception, which imposes seemingly arbitrary boundaries onto a perceptually uniform colour space.
%But what is categorical perception?
%
%Humans enjoy categorical perception of a wide range of phenomena including orientation,  speech and facial expressions of emotion---and, crucially for us, colour.
%To claim that there is categorical perception of colour is, very roughly and to a first approximation, to claim that perceptual processes distinguish not the particular shades of things but also  categorical colour properties like blue and red.
%To understand how knowledge of colour emerges in development we need to think more carefully about categorical perception.
%This will not only help us with colour; it will also guide us in wrestling with core knowledge.
%%In fact, understanding what categorical perception is and how it relates to knowledge and its emergence may even provide  the basis for a theory of core knowledge.
%
%
%
%There are two questions we can ask about such knowledge of facts about the categorical colour properties of things.
%First, how does it emerge in development?
%Second, how do you come to acquire such knowledge on particular occasions?
%Answering the second question will help us to answer the first, and it will also help us in thinking about the notion of core knowledge and development generally.

%
%\section{Categorical Perception of Colour}
%It is difficult to understand what categorical perception of colour is.
%When introducing categorical perception, philosophers tend to start by making doubtful assumptions.
%For instance, Mohan Matthen introduces categorical perception of colour by saying:
%%
%\begin{quote}
%`The colours to which human languages give names are experienced […] as sharply different from one another'
%\citep[p.~190]{Matthen:2005sc}.
%\end{quote}
%%
%This assertion may initially seem to offer a simple approach to understanding categorical perception, but it involves making two assumptions that we should not be taking for granted at the outset.
%Matthen assumes that categorical perception always involves experiencing differences in categorical properties,
%and also that the categorical colour properties humans perceive are the categorical colour properties they give names to.
%Unfortunately things are not so simple.
%Much of the evidence scientists routinely offer for categorical perception does not concern how colours are experienced at all,
%none of the evidence clearly supports the view that humans experience colours they name differently as sharply different,
%and some of the evidence  suggests the contrary \citep{witzel2014category,webster:2012_color}.
%On the link between colour names and categorical perception, what Matthen says is at most a half truth. %\citep{Franklin:2005hp}; \citep[p.~25]{witzel:2013_categorical} ‘sensitivity to color differences may be considered as categorical in the broad sense. However, the global JND pattern is not specific to the categories that correspond to the basic color terms.’
%
%We need a better approach to understanding what categorical perception of colour is, one that starts with evidence rather than assumptions.
%Rather than assuming we know what categorical perception is---which, basically, we don't---we should take nothing for granted and make no assumptions about what sense, if any, categorical perception involves either categories or perception.
%Instead we should start by asking, What is categorical perception commonly taken to explain?
%
% **
%
%There is a way of dividing a colour space into  a small finite number of equivalence classes---the categories---that enables us to make predictions about all three of these effects and more.
%When the members of a sequence of colours fall into different categories, the colours are given different verbal labels, and monolingual English speakers can discriminate them faster and more accurately than when they don’t fall into different categories (all other things being equal).
%The existence of each effect is a fact that stands in need of explanation.
%And these are the facts categorical perception of colour is supposed to explain.
%We have just seen, for instance, that pairs of colours from different categories can be more accurately discriminated than pairs from within the same category.
%Why is this?
%Categorical perception, whatever it is, is supposed to be the answer.
%
%So what is categorical perception?
%The fact that a single division into categories predicts all four effects---naming, accuracy, speed and verbal reports of similarity---indicates that there is a single explanation for all of them.
%Categorical perception is the process, whatever it is, that explains these facts.
%Now this way of defining categorical perception is not very informative.
%%(It is a little bit informative because it might in principle turn out to be merely a coincidence that the same way of dividing colours into categories predicts all four effects; the various effects might in principle be explained by four separate processes.)
%This is a virtue---categorical perception is something to be discovered, not defined.
%But we should still try to find out more.
%What sort of process is categorical perception?
%Can we even be confident that it is a perceptual process?
%
%To discover more about categorical perception we need to look for further  patterns to be explained.
%One of these involves \gls{pop out}.
%Informally, pop out occurs when something jumps out at you.
%If you are looking at a sea of pink dots in which there is just one blue dot, the blue dot will pop out at you.
%How can we characterise pop out if we want to measure it?
%Operationally, pop out is defined in terms of visual search tasks \citep[p.~117]{Treisman:1986pm}.
%In a visual search task, you are shown an array of objects and asked to identify the one with a certain property, the target object.
%For example, you might be shown an array of sixteen animal pictures and asked to identify the cat.
%%comment: linear separability results in easy searches, this is why we need distractors to have two colours.
%Or you might be shown sixteen coloured dots, where all the dots are the colour of the top or bottom green patches from \vref{fig:two_colour_sequences} apart from one which is the colour of the middle green.
%Normally the time it would take you to find the target increases linearly as the number of objects in the array increases.
%This is just what you would expect if searching involves processing each object in turn.
%But what if the target and the other objects belong to different colour categories: say, the target is the colour of the purple patch from \cref{fig:two_colour_sequences} and the others are the colour of the blue or pink patches from that figure?
%Then, as illustrated in \vref{fig:pop_out}, you would be able to find the target just as quickly when it is one of 36 dots as when it is one of only four dots (\citealp[Experiment 1]{Daoutis:2006qk}; see also \citealp{wright:2012_categoricala}).
%Clearly, in this case visual search cannot be a matter of processing the objects sequentially.
%Colour pops out.
%
%Why do we care about pop out?
%Until now we had no compelling reason to suppose that categorical perception is a perceptual process.
%Recall that we identified categorical perception as whatever process explains four types of  effect: speed, accuracy, labelling, and verbal reports.
%Those types of effect are not obviously consequences of perceptual processes.
%But now we can add pop out to the list of effects to be explained.
%The same way of dividing colours into categories that predicts verbal labelling, accuracy and the rest also  predicts when pop out will occur.
%This gives us more insight into what categorical perception is.
%Since pop out is plausibly a consequence of perceptual processes,
%this suggests that categorical perception of colour is indeed a perceptual process.
%
%*NB: \citep[p.~979]{wright:2012_categoricala} discusses an alternative, non-perceptual explanation of pop out, the ‘code interference theory’; according to Wright, pop out is consistent with both code interference and guided search.
%
%*NB: \citep[p.~141]{clifford:2012_neural}: ‘the current study found no category effects in early ERP components, category effects were present during a post-perceptual ERP component (P3). This suggests that acquired category effects can exist without the involvement of early perceptual processing and can be solely governed by cognitive mechanisms such as memory and language (for example, McCarthy \& Donchin, 1981). This study therefore provides further confirmation that category effects are not necessarily perceptual, and that the term ‘categorical perception’ which is traditionally used to describe such effects is misleading and should only be used where there is clear evidence for the involvement of perceptual processes.’
%
%NB: Separate perception and appearance. Consider Experiment 1 of \citep{webster:2012_color} which is about perceptual grouping (they only consider the blue--green boundary) and so illustrates appearance.
%Their results are unclear (they talk about a ‘weak’ effect of categories).
%Compare Witzel \& Gegenfurtner: in neither judgements of perceptual similarity nor hue matching (adjusting the hue of a colour so that it appears mid-way between two samples) are there categorical effects!
%
%cf.  \citep{witzel:2013_categorical} where there is categorical effects for discrimination only at some boundaries.  Both findings question the evidence that there is a single set of equivalence classes
%
%Our topic is still categorical perception, but I want to pause here to reflect on our approach to categorical perception.
%We started by assuming as little as possible and asking what categorical perception is supposed to explain.
%Identifying the facts that categorical perception  is  supposed to explain gives us a handle on the phenomenon.
%We should think of core knowledge using the same method.
%At the outset we don't know what core knowledge is.
%I just said it's like knowledge in some ways but not knowledge.
%We'll get a handle on it by looking at what it is commonly supposed to explain.
%More on this later; now back to categorical perception.
%
%
%
%
%\section{cut}
%Let me first say something about what categorical perception isn't.
%Bornstein, whose groundbreaking research we'll rely on later, introduces categorical perception by incorrectly asserting that:
%%
%\begin{quote}
%‘The category question concerns whether observers perceive qualitative similarities \ldots \ Discriminable wavelengths seem to be categorized together because they appear perceptually similar’ \citep[p.~288--9]{Bornstein:1987vv}.
%\end{quote}
%%
%And Matthen, one of the few philosophers to have considered categorical perception at all, introduces categorical perception of colour by claiming, also incorrectly, that:
%%
%\begin{quote}
%`The colours to which human languages give names are experienced […] as sharply different from one another'
%\citep[p.~190]{Matthen:2005sc}.
%\end{quote}
%%
%This is all wrong.
%As we have just seen (in the previous section),
%observers do not perpetually experience two blue things as perceptually similar in virtue of their being blue, nor do differences in categorical colour properties cause people to experience things as sharply different.
%Further, not all categorical perception is bound up with languages (see sections *development and *two forms).
%
%If, as I have been suggesting, it is a mistake to characterise categorical perception in terms of visual appearances and experiences of similarity, how should we characterise it?
%What is the nature of the process by which humans detect categorical colour properties like yellow, orange and red?
%
%
%
%
%\section{Categorical Perception: Going Deeper}
%Use the Wright challenge: ‘differences in performance of different trial trials types may be attributed to unequal perceptual separations, rather than to a category effect.’
%
%Also challenge in \citep[p.~142]{clifford:2012_neural}: ‘some have argued that supposed color category effects on tasks such as visual search, are better explained by cone-excitation or inequalities in the color metric used to equate same- and different-category color pairs (Brown et al., 2011; Witzel \& Gegenfurtner, 2011).’
%
%Strategies to avoid this: train new categories; compare speakers of different languages; compare left and right visual fields; consider the effects of verbal interference.
%
%
%\citep{bird:2014_categorical} : different neural encoding of category and hue in a target detection task (no judgements about colour were made)


\section{Infants’ Categorical Perception}
\label{cp:sec:development}
Categorical perception of colour emerges early in infancy, from around four months or earlier.
How do we know?

Early studies used \gls{habituation}.
Infants were habituated to the visual presentation of one colour, green say.
The colour was then changed, either to a colour within the same category (for example, a different green), or else to a colour in a different category (yellow, for example).
The two changes in colour were as similar as possible other than that one involved a change in category.
The researchers found that infants dishabituated more when the change in colour involved a change in category than when it did not  (\citealp{Bornstein:1976of}; this finding has since been confirmed by \citealp{franklin:2004_new} with improved methods).
Apparently humans can discriminate categorical colour properties from four months of age or earlier.

The view that infants have categorical perception of colour is indirectly supported by the fact that infants, like adults, have categorical perception of many things other than colour.
These include orientation \citep{franklin:2010_hemispheric}, speech \citep{Kuhl:1987la,Jusczyk:1995it} and facial expressions of emotion  \citep{Etcoff:1992zd,Kotsoni:2001ph,Campanella:2002aa}.
Taken together this is compelling evidence that many forms of categorical perception, including categorical perception of colour, are present by four moths of age or earlier.


But does categorical perception of colour depend on the same mechanism in infants and adults?
It may seem tempting to suppose that it does.
After all, we identified categorical perception as whatever explains otherwise unexpected patterns of discrimination, \gls{pop out}, \glspl{oddball effect} and the rest.
Partial overlap in the effects to be explained in infants and adults might make it tempting to guess  that the mechanisms underlying them are identical.
% And some philosophers have committed themselves to this guess.
% Fodor, for example, is committed to the view that ‘the mechanisms \ldots\ that determine that red things strike us as they do’ are innate
% \citep[p.~142]{Fodor:1998ap}.
% Since the colours infants perceive categorically are not generally the colours named by those around them \citep{Franklin:2005hp,franklin:2014_case},
% the mechanisms underpinning infant categorical perception do not determine that red things strike us as they do.
% Only adult categorical perception determines this.
% Fodor's view therefore requires that the mechanism underpinning adult categorical perception be innate.
% And since the mechanism underpinning adult categorical perception of colour is not plausibly innate unless it also underpins infant categorical perception (note, by the way, that there is no evidence on whether of infant categorical perception of colour is innate),
% Fodor's view requires that infant and adult categorical perception depend on a single mechanism.
But this guess turns out to be risky.
%
%Why?
%In many experimental paradigms, the colours for which an adult manifests categorical perception are only those for which she has colour words.
%Further, training her to use new colour words can affect which colours she categorically perceives (as we saw in \vref{cp:sec:the_mystery}).
%But infants lack any colour words (of course),
%and the colours they perceive categorically are not generally the colours named by those around them \citep{Franklin:2005hp,franklin:2014_case}.
%So what happens in development?
%Is categorical perception in infants and adults is underpinned by a single mechanism whose operation is altered by the learning to use colour terms?
%Or does adult categorical perception of colour depend on a mechanism distinct from that involved in infant categorical perception, one which depends in some way on capacities involved in using language?
%
%In adults, categorical perception does not occur when there is predictable verbal interference.
%When adults are induced to read words (but not when they are asked to do similarly demanding nonverbal tasks),
%differences in categorical colour properties do not influence discrimination \citep{Roberson:2000ge}.
%Careful attention to the effects of verbal interference on categorical perception in adults suggests that a common categorising process underpins both categorical perception of colour and the use of words  \citep{Pilling:2003bi,Wiggett:2008xt}.
%This is a hint that categorical perception may involve different processes in preverbal infants and adults.
%%The process by which human infants detect categorical colour properties is not the same as that by which adults detect them.  %no because the infant process may occur in adults too.
%
%% don't need footnote: about does not say that CP depends on covert use of language  but that there is a common dependence which is consistent with holmes:2012_does
%%\footnote{%
%%This interpretation of the findings is challenged by \citet[p.~442]{holmes:2012_does}.
%%However, their challenge relies on results which would provide no less compelling support for the conclusion that categorical perception involves different processes in preverbal infants and adults.
%%}
%

Why?
It is possible that infant and adult categorical perception  involves different sides of the brain.
This is almost certainly true for categorical perception of orientation
%but the hemispheres are the other way around.
 (\citealp{franklin:2010_hemispheric}; see also \citealp{Gilbert:2008rs,holmes:2012_does,vanderham:2014_lateralized}).
And there is some evidence  that it is also true for categorical perception of colour  (\citealp{franklin:2008_lateralization}; see also \citealp{Franklin:2008jo,zhou:2010_newly,mo:2011_electrophysiologicala,zhong:2015_shortterm}),
although this is currently controversial.%
\footnote{%
Attempts to replicate two of the earliest investigations into which parts of the brain are involved in categorical perception of colour, those of
\citet{Gilbert:2006yb} and \citet{Drivonikou:2007fu}, have failed \citep{witzel:2011_there,brown:2011_color}.
Interpretation of these two replication studies is complicated by the fact that they appear to support incompatible conclusions.
Whereas \citet{brown:2011_color} report finding no categorical perception at all using green and blue, \citet[p.~21]{witzel:2011_there} find that categorical perception did occur (they merely found no sign that it depends on mechanisms located in the left half of the brain).
% \citep{witzel:2011_there} p.~21: ‘For all our sets of stimulus colors, our results exhibited the classical pattern of reaction times considered to be a category effect in the original studies.’
The failures to replicate, which involve extremely careful and extensive testing, warrant caution in interpreting what appears to be evidence that categorical perception depends on different mechanisms in infants and adults.
However, given that  others have found evidence for this view in different experiments (as cited above), the failures to replicate are not yet sufficient grounds to rule that view out.
}
%
That is, the categorical perception of colour manifested by adults  in many experimental paradigms may depend on mechanisms linked to one %the left
hemisphere of the brain, whereas categorical perception in infants appears to involve mechanisms linked to the
other %right
hemisphere.
We also know that, in adults, learning colour words alters categorical perception of colour (see \cref{cp:sec:the_mystery}).
Further, categorical perception in adults appears to rely on processes that are also involved in using colour words \citep[see][]{Pilling:2003bi,Wiggett:2008xt},
Putting all this together suggests that, for colour at least, categorical perception is not constant throughout development.
Instead it may involve two distinct mechanisms, one present from the early months of life and another which appears in development only sometime after colour words have been acquired.

As we cannot rely on the assumption that infants’ categorical perception of colour depends on the same mechanism that adults’ does, we need to consider what is known about the effects of categorical perception in infants.
%
% An ability to discriminate categorical colour properties might in principle be underpinned by any of several different kinds of cognitive process (see \vref{cp:sec:the_mystery}).
% As in the case of adults,
% characterising the process by which infants distinguish the categorical colour properties of things requires investigating different effects of this process.
%
Two particularly revealing effects of categorical perception in adults are \glspl{oddball effect} and \gls{pop out} (see \vref{cp:sec:fresh_start}).
Are these effects also found in infants?
\citet{Clifford:2009fo} showed that changing categorical colour properties gives rise to an oddball effect in seven-month-old infants.
This suggests that, as in adults, categorical perception in infants is rapid (resulting in detection in half a second or perhaps less) and automatic.

What about pop out?
This is less straightforward because pop out involves a \gls{visual search task}, which would normally require verbal instructions.
But \citet{Franklin:2005xk} discovered an infant analogue of pop out.
They tested how quickly four-month-olds could visually detect a target
by inducing them  to gaze at the centre of a display before showing them a coloured target on a coloured background.
The question was how long it would take infants to detect this target,
as measured by the time it took them to move their eyes to it.
The key manipulation was whether the background and target differed in their categorical colour properties---whether, say, both were shades of blue, or whether one was blue and the other green.
Of course, things were set up so that the difference between the two blues was as similar as possible to the difference between the blue and green pair.
Like adults, infants were faster to move their eyes to the target when it differed in categorical colour property from the colour of the background  \citep{Franklin:2005xk,ozturk:2013_language}. %\citet{ozturk:2013_language} is a more more rigorous follow up with more contrasts
This pop-out-like effect suggests that, as in adults, categorical perception in infants may influence the overall phenomenal character of experiences, at least indirectly.

% don't say this because it could just be memory for colour: franklin 2014 review chapter says infants like adults have good memory for exact colours.
%Slightly older infants can also make use of colour properties such as red and green to recognise objects. For instance, nine-months-olds can determine whether an object they saw earlier is the same as a subsequently presented object on the basis of its colour \citep{Wilcox:2008jk}.
%And by the time they are two years old, toddlers who do not comprehend any colour terms can use colour categories implicitly in learning and using proper names; for instance, they are able to learn and use proper names for toy dinosaurs that differ only in colour \citep[][Experiment 3]{Soja:1994np}.
%The fact that infants and toddlers can use categorical colour properties in this way is further evidence that they really do enjoy categorical perception just as adults do.



What can we conclude about categorical perception of colour in infancy?
As things stand we do not know whether or not categorical perception of colour in infants involves the very mechanism that underpins categorical perception in adults.
But we do know that categorical perception of colour (and of other things) is present from four months of age or earlier.
We also know that it has two features which are going to be critical for understanding how knowledge of colour emerges in development.
Categorical perception of colour in infants is an \gls{automatic} process (that is, whether it occurs is to a significant degree independent of the infant's current project and motivations),
and, directly or indirectly, it alters infants' experiences in such a way that differences in categorical colour properties can cause difference in the overall phenomenal characters of their experiences.



%This raises the possibility that in adults two independent processes of categorical perception, detecting different categorical colour properties, can occur.





%Now for the main question.
%[removed because CP=CK interjected]



\section{Is Categorical Perception a Form of Core Knowledge?}
\label{cp:sec:cp_is_core_knowledge}

\Gls{categorical perception of colour} has much in common with \gls{core knowledge}.
It is not just that both are initially mysterious because so different from the mental phenomena that philosophers normally focus on.
Both categorical perception and core knowledge appear early in development, well before referential communication or social interaction involving objects.
Neither categorical perception nor core knowledge are straightforwardly a matter of vision or of perceptual experience---as \citet[p.~141]{clifford:2012_neural} write, ‘categorical perception’ is a misleading label.
And yet both categorical perception and core knowledge can affect, at least indirectly, the overall phenomenal character of perceptual experiences.

Categorical perception further resembles core knowledge in that it exists for a variety of domains.
In addition to colour, humans also enjoy categorical perception of low-level features such as orientation \citep{Wolfe:1992yl}, and of some higher-level features including speech \citep{Eimas:1971cp,Dehaene-Lambertz:1997oi,Liberman:1957cm,Repp:1987xo},  human faces, and both vocal and facial expressions of emotions such as happiness and fear \citep{Kotsoni:2001ph,laukka:2005_categorical,hoonhorst:2011_categoricala}.
Further, categorical perception, like core knowledge, has counterparts in animals other than humans.
To take the best studied case, categorical perception of mating calls and perhaps other acoustic signals is widespread in non-human animals including monkeys, mice and chinchillas \citep{Ehret:1987xg,Kuhl:1987la}, and is even found in cognitively unsophisticated animals such as frogs \citep{Baugh:2008nc} and crickets \citep{Wyttenbach:1996le}.


Categorical perception and core knowledge also share  some key features which distinguish them from knowledge.
Core knowledge is supposed to be
 judgement-independent \citep[p.~1235]{Spelke:2000nf}.
 Unlike knowledge, it is specific to quite narrow categories of event and does not grow by means of generalization (\citealp[p.~57]{Baillargeon:2002hb}; \citealp[p.~344]{Baillargeon:2001ll}).
It is also best understood as a collection of rules rather than a coherent theory \citep[p.~82]{Baillargeon:2002hb}.
And it has limited effects on behaviour, being usually manifest in the control of attention (as measured by \gls{dishabituation}, gaze, and looking times) and rarely or never manifest in purposive actions \citep[pp.~441--2]{Spelke:1994oz}.
% Conversely, some forms of categorical perception require significant cognitive flexibility.
% In categorical perception of speech, for instance, how gestures are categorised depends on the speaker’s dialect \citep{Repp:1987xo}, her rate of speaking \citep[pp.~72--5]{Nygaard:1995po} and the context \citep[p.~44]{Jusczyk:1997lz}; categorisation also depends on integrating visual with auditory information \citep{McGurk:1976ta}.
% In fact, categorical perception of speech and core knowledge of objects both involve representations which are highly abstract related to the surfaces and sounds presented visually and acoustically.
Categorical perception plausibly shares all these features.

These considerations justify us in regarding forms of categorical perception as instances of \gls{core knowledge}, at least provisionally.
This is not a deep discovery.
It is the recognition that theories of core knowledge (and of associated phenomena, like modularity) are simply too crude at present to justify fine distinctions.

Since the existence of several forms of categorical perception in infancy is well established, identifying core knowledge and categorical perception strengthens the case for the existence of core knowledge.
Further, since categorical perception can affect the overall phenomenal character of experiences, at least indirectly,
and can do so in ways that do not straightforwardly reflect how things perceptually appear (see \cref{cp:sec:no_appearance}),
reflection on categorical perception may even be broaden our understanding of \glspl{metacognitive feeling} (see \cref{sec:metacognitive-feelings-ex1}).



Now for the main question.


\section{Categorical Perception and Building Blocks}
\label{sec:cp-building-blocks}
Our main question is about how humans first come to know things about colours.
How is categorical perception of colour relevant to this question?
One influential idea is that categorical perception provides ‘building blocks’ for knowledge \citep[p.~3]{Harnad:1987ej}.

Our first problem is to understand the ‘building blocks’ metaphor.
Start with the simplest possible way of construing building blocks: they are concepts.
On this view, having categorical perception of colour amounts to having colour  concepts.

Some researchers would endorse this view.
Take Leslie, for instance:
%
\begin{quote}
‘The module … automatically provides a conceptual identification of its input for central thought … in exactly the right format for inferential processes’ \citep[pp.~193--4]{Leslie:1988ct}.
\end{quote}
%
Since categorical perception is a modular process if anything is (on modularity, see section \cref{sec:modularity}), this implies that categorical perception of red or blue results in conceptual identification of these properties.
But does it?

Consider what is involved in having a colour concept.
It is not just a matter of being able to respond differently to red things than to non-red things.
It is also a matter of being able to think about those properties as such.
Are children who have categorical perception of certain colours thereby able to think about those colour properties as such?

We can approach this question indirectly by first considering words rather than concepts.
The ability to use colour words accurately has a protracted development \citep{roberson:2004_development}.
A common first stage is realising that questions about the colours of things should be answered by using a colour word while completely failing to use colour words to distinguish colours.
If asked about the colour of something, children at this first stage might say ‘blue’, but they will not use this term consistently. %source: \citep{pitchford:2001_conceptualization}
The second stage is being able to use one or more colour words accurately.
There is much individual variation; some children achieve this sometime around their second birthday, others between the four and seven years of age.
This second stage seems to occur  after \citep{Soja:1994np},
or just around  \citep{pitchford:2001_conceptualization},
the time that a child  can accurately distinguish other properties like shape and size.
In, or shortly after, learning to use one colour word accurately, children become able to use several accurately.
However, some colour words, like ‘brown’, are much harder to learn and are not typically used accurately until long after a child can first use colour words accurately  \citep{Pitchford:2005hm}.

Our question is whether children who have categorical perception of certain colours necessarily possess concepts of those colours, as would have to be the case if categorical perception were conceptual identification.
How do the facts about colour words bear on this?
Those who endorse a particularly crude view linking concepts to words might insist that since children lack any colour words, they could not possess any colour concepts either.
But it would be a mistake to suppose that issues about relations between concepts and words can all be decided without experiment.
Instead we can look at the findings of those who have devised ways of measuring the emergence of colour concepts in development.

To test for possession of colour concepts, \citet[Experiment 2]{Kowalski:2006hk} devised a simple paradigm.
They showed children a uniformly coloured toy.
Then they put the toy away, out of sight.
Finally, they showed children two uniformly coloured discs and asked, ‘Which one of these is like the toy I just put away?’
By putting the toy away before asking the question, they ensured that children could not answer correctly just on the basis of perceptual similarity: passing the task required the ability to think about a colour property as such.
%So Kowalski and Zimiles' paradigm provides an excellent operationalisation of concept possession.
And their results show that children do not possess colour concepts until they are able to use at least some colour words accurately.

Before turning to the significance of this result, consider an objection.
You might suspect that Kowalski and Zimiles' paradigm underestimates children's conceptual abilities because it imposes extraneous demands.
Note, for instance, that putting the toy away before showing the colour discs means that children have to hold the colour of the toy in memory for a short time.
Could it be that children fail to succeed in the paradigm not because they lack concepts but because of memory difficulties?
To rule out this sort of objection,  Kowalski and Zimiles added a further task.
This was as similar as possible to the colour task except that it concerned shape rather than colour.
If memory or some other extraneous demand affected performance in the colour task, it should also affect performance in the shape task.
Crucially performance on the shape task was not related to the ability to use colour words correctly.  %YES: colour words!
This strengthens the case for concluding that possessing colour concepts depends on being able to use some colour words correctly.

A further potential objection is more subtle.
Little is known about which colours human infants and young children perceive categorically---as we will see (in \cref{sec:cp_redescription}), the colours they perceive categorically are not the same as those which most adults perceive categorically.
Is it possible that Kowalski and Zimiles' paradigm underestimates children's conceptual abilities because it requires them to choose between things that, according to their perceptual categories, have the same colour?
This is unlikely for two reasons.
First, stimuli were constructed so that there were relatively large perceptual differences between the colours used.
Second, if children's performance was affected by the colours they have categorical perception of, we would expect their errors to show a certain pattern: the more perceptually similar colours are, the more likely they should be to make mistakes.
For instance, mistaking green for blue or yellow should be more common than mistaking green for red.
In fact no such pattern was observed. %this refers to experiment 1 which asked children to choose from six, not two, discs.
So even without knowing which colours each child has categorical perception of, we can be confident of the result.
Those who were not yet able to use any colour words accurately did not possess any colour concepts.

We now have a strong argument against the view that categorical perception of colour involves conceptually identifying colours.
Children manifest categorical perception of colour early in life, from seven months of age or earlier.
But they do not acquire colour words until around 18 months later at the earliest; and \citet{Kowalski:2006hk} provide convincing evidence that they no not possess any colour concepts until they can use some colour words correctly.
Therefore categorical perception of colour in infants does not involve conceptually identifying colours.

This tells us three things.
First, having categorical perception of colour is insufficient for knowing facts about the colours of things.
Second, if we want to hold on to the idea  that categorical perception provides ‘building blocks’ for knowledge, we must reject the idea that building blocks are concepts.
Third, since categorical perception of colour is a modular process (if anything is), it is not true in general that a module provides ‘a conceptual identification of its input for central thought’ \citep[p.~193--4]{Leslie:1988ct}.
The relation between categorical perception and knowledge of colour is not so straightforward.




\section{Building Blocks and Redescription}
\label{sec:cp_redescription}
Our overall aim is to understand how humans first come to know things about colours.
The suggestion introduced in the previous section is that categorical perception of colours provides ‘building blocks’ for knowledge.
The first problem we face in evaluating this suggestion is that of understanding the metaphor of building blocks.
We have just seen that this can't be done by identifying building blocks with concepts.
What else might the ‘building blocks’ metaphor mean?

Mandler is famous for a theory of development according to which
%
\begin{quote}
‘the earliest conceptual functioning consists of a redescription of perceptual structure’ \citep{Mandler:1992vn}.
\end{quote}
%
In mentioning ‘perceptual structure’, Mandler has in mind not  only things like categorical perception but also perceptual prototypes; and she even applies her view to perception of simple causal interactions (\citealp[p.~69]{Mandler:2000uu}; see \cref{sec:causal-interactions}).
But here our concern is only whether redescription will help us to understand how categorical perception might facilitate the development of knowledge about the colours of things.

Mandler's view is an advance on our first attempt to understand the building blocks metaphor.
Instead of conjecturing, incorrectly, that categorical perception involves concepts, we can conjecture that categorical perception provides categories of some kind, and that the process of acquiring knowledge involves transforming these categories into colour concepts by ‘redescribing’  them.

Even without figuring out what Mandler's notion of redescription might involve, we can see that two predictions follow from this conjecture.
First, since concepts are supposed to come about through a redescription of categories, these two should be systematically related.
In particular, there should be a systematic way of mapping the colours infants categorically perceive onto the extensions of the colour concepts they subsequently acquire.
Second, humans should typically have categorical perception of a colour prior to having a concept of that colour. %Franklin's toddlers
Are these predictions correct?

Both appear to be incorrect.
To understand why, we need to step back and think about cultural variation in colour words for a moment.
Different languages categorise colours differently, and how a language categorises colours can vary over time.
For example, Turkish and Russian have two basic colour terms which mark a boundary somewhere in the category associated with monolingual English speakers' word ‘blue’.
Meanwhile, some Celtic languages have, or used to have, a single basic term covering both green and blue \citep{lazar-meyn:2004_color}.
These differences may seem minor, amounting only to dividing or merging categories.
However, where communities have had little interaction, cultural variation in colour terms can be more radical.
\Vref{fig:berinmo_english} shows the extensions of colour words in English and Berinmo.
The mapping of Berinmo colour words comes from a study of villages in Papua New Guinea by \citet{Roberson:2000cc}.
While some have argued that there are universal constraints on how languages name colours \citep{kay:2003_resolving,Regier:2009ve}, there can be no doubt that different languages divide colours into strikingly different categories.

\begin{figure}
\begin{center}
\includegraphics[scale=0.3]{fig/colour_terms_berinmo.neg.png}
\includegraphics[scale=0.3]{fig/colour_terms_english.neg.png}
\caption{
	\label{fig:berinmo_english}
	The extensions of colour words in two languages, Berinmo (above) and English (below), mapped in a cross-section of Munsell colour space.
	Source: \citet[figure 1]{Roberson:2000cc} %\citet[figure 1]{roberson:2010}.
}
\end{center}
\end{figure}

Why is this variation in colour words across languages relevant?
The colours that a monolingual adult categorically perceives are the colours that her language gives names to \citep{Kay:2006ly,Roberson:2007wg,Winawer:2007im,thierry:2009_unconscious}.
This means that, for example, the colours that Berinmo speakers categorically perceive are not the colours that English speakers categorically perceive.

You might be tempted---as some researchers clearly are---to try to use this striking fact to support relativistic ideas: to infer, for instance, that differences in language create differences in experience.
But recall that categorical perception almost certainly does not involve  perceptually experiencing categorical colour properties (see \cref{cp:sec:no_appearance}) and probably does not involve early visual processes either (see \cref{cp:sec:fresh_start}).
This makes the connection between linguistic relativism and categorical perception more tenuous than is usually acknowledged (as \citealp{kay_what_1984} argue).

The striking fact that which colours are categorically perceived varies with colour words has a different, no less interesting implication.
Since infants in their first months of life do not comprehend or produce colour words (or indeed any words), which colours they categorically perceive cannot depend on which colour words those around them happen to be using.
Indeed, children typically start using colour words accurately some time before their categorical perception becomes influenced by their language \citep{Franklin:2005hp,franklin:2014_case}.
It follows that, in general, the colours infants categorically perceive are not the colours that they name or acquire concepts of.


This is a problem for the conjecture that redescription explains the transition from categorical perception to concept possession and knowledge of colour.
As we saw, that conjecture generates two predictions.
The first was that there should be a systematic way of mapping the colours infants categorically perceive onto the extensions of the colour concepts they subsequently acquire.
Because colour concepts can vary between adults as radically as colour words between languages vary, it appears unlikely---though not impossible, of course---that any such mapping exists.
This is one consideration against the conjecture about redescription.

A further consideration against the conjecture is linked to its second prediction, which is that humans should typically have categorical perception of a colour prior to having a concept of that colour.
While this prediction has not been directly tested,
there is some indirect evidence which speaks against it.
As just mentioned, toddlers come to use some colour words accurately prior to having categorical perception of the colours these words name \citep{Franklin:2005hp}.
So  words precede categorical perception in this sense: for each colour you have a word for, you were probably able to use the word accurately before you could categorically perceive the colour.
This suggests that redescription is unlikely to play a major role in explaining how colour concepts are acquired,
and it makes it at least conceivable that humans sometimes acquire a concept for a particular colour before being able to categorically perceive it.

Given these considerations against the conjecture about redescription together with the lack of direct evidence for it, I shall not consider redescription further.

Disregarding redescription confronts us with a problem.
What are we to make of the ‘building blocks’ metaphor?
We have examined two ways of making it concrete. %too bad pun?
The first was the conjecture that building blocks are concepts; the second was the conjecture that building blocks are categories, which are transformed into concepts by redescription.
Neither looks promising.
There is direct evidence against the first; and the second generates two predictions, both of which are probably incorrect.
It may be time to reject the very idea that categorical perception provides building blocks for knowledge.
This leaves us with a different problem.
If we reject the building blocks metaphor, what alternative account can be given of how humans first come to know things about the categorical colour properties of particular objects?


\section{Development Is Rediscovery}
The different conjectures linked to the idea that categorical perception provide ‘building blocks’ share an assumption with many contemporary theories about how development works.
They all assume that the transition from earlier- to later-developing cognitive abilities involves direct representational connections between the two abilities.
That is, the contents of representations underpinning earlier-developing abilities are  assumed to be transformed into (components  of) the contents of representations underpinning later-developing abilities.
This is the {\gls{Assumption of Representational Connections}} (see \cref{sec:objects-rediscovery}).
What happens if we drop this assumption?
Can we develop a plausible conjecture about the relation between categorical perception and knowledge of colour without it?

%crude illustration

%Suppose there are no direct representational connections between the early-developing cognitive ability to categorically perceive colours and knowledge about the particular colours of things, which emerges later in development.
%How could someone get from the early-developing ability  to the later-developing knowledge?

Here is one story about how categorical perception might facilitate the development of knowledge about colour even without direct representational links between categorical perception and knowledge.
Imagine yourself in the position of  a child who is familiar with the forms (sound or gestural shape) of some colour words but cannot yet use words to distinguish among colours.
You know that these words, ‘red’ and ‘pink’ say, can be used to talk about some property of things but you don't yet know which property.
You face two problems.
The first is to work out which dimension the properties vary along: you need to work out that they are words for colours rather than, say, for shapes or functions.
The second problem is to work out the extensions of the properties associated with these words so that, for instance, you know roughly where red ends and pink begins.
Early-developing categorical perception cannot help with this second problem because the colours categorically perceived by infants are not generally the colours adults name (as we saw in \cref{sec:cp_redescription}).
But can categorical perception help with the first problem?

As we saw, categorical perception in infants enables them to respond to differences in categorical colour properties and causes colour to \gls{pop out} at them, and it does this months or even years before they have any colour words or concepts.
%So categorical perception allows them to distinguish things by properties on the colour dimension well before they can think about colours.
This suggests that
early-developing categorical perception provides a phenomenal marker of categorical colour properties.
It alters the overall phenomenal character of experience in such a way that, often enough, changes in categorical colour property bring about changes in your experience.
(Whether these changes in overall phenomenal character are a direct or indirect consequence of categorical perception remains a topic for investigation.
We do know that they are not a matter of visual appearances characteristic of categorical colour properties---see \cref{sec:red-not-visually-appear}.)
This phenomenal marker could help you, eventually, to lock onto the relevant dimension and so to solve one of the problems you face when acquiring colour concepts and words.

Once you have locked onto the relevant dimension, colour, the next problem is to work out the extensions of particular colour words.
This is a relatively simple problem to solve once you can identify the forms (sounds or shapes) of some colour words and providing that people around you use these colour words systematically.
When these prerequisites are met, your experiences of others' uses of colour words enable you to re-identify a colour property in advance of knowing which colour it is.
To illustrate, consider the simplest sort of case.
You observe that people around you call certain things ‘red’ and others ‘pink’.
You assume that one colour property is common to all the things called ‘red’ and another colour property is common to all those called  ‘pink’.
Given enough observations of this sort you can take a  shot at working out roughly where red ends and pink begins (see \cref{fig:categorical_perception_to_concepts}).

\addFigureWidth{1}{categorical_perception_to_concepts}{A conjecture about the developmental emergence of knowledge of colour. (* --- by the lights of adults around you)}



On this view, early-developing categorical perception and social interaction combine to explain the emergence in development of knowledge of the colours of things.
Social interaction in the form of conversation involving talk about the colours of things is essential because early-developing categorical perception does not enable you to identify the extensions of colour words or properties.
But categorical perception is also essential because merely talking about colours would not enable you to lock onto the relevant dimension, colour rather than (say) shape or function.
% *Is the last point right?  Couldn't we say that differences in colour correlate with differences in colour words; and so talking about colour could perform the same function as CP?

Should we generalise the details of this conjecture from colour to other domains of knowledge.
As we will see, reflection on what we know about how children acquire their first words (see \cref{cha:words}) suggests we should not.
Colour words are probably a special case.

Development is \gls{rediscovery} on this conjecture about how knowledge of colour emerges.
For according to the hypothesis, there is no direct representational relation between the early-developing ability to categorically perceive colours and the later-developing knowledge of colours.
The early-developing ability matters because it influences orienting, enables sorting and can affect the overall phenomenal character of experiences, thereby enabling individuals to solve one of the problems they face in first acquiring knowledge of the colours of things.
But the early-developing ability does not in any sense provide building blocks for concepts or knowledge.
It is true that humans lock on to categorical colour properties long before they have words or concepts for them;
but coming to know even the simplest facts about the colours of things involves rediscovering colour.




\section{Appendix: Methods for Measuring What Brains Detect When}
\label{sec:brain-methods}
In this chapter we saw that differences in categorical colour properties---differences like that between, say, red and blue---are detected by the brain in 100 to 200 milliseconds (see \cref{cp:sec:fresh_start}).
How do we know?
Short of asking her, or looking for a sign in her behaviour, how can we know what someone's brain is doing?

The answer is functional neuroimaging.
Two main techniques are currently used by scientists.
One is fMRI, which stands for functional magnetic resonance imaging and involves being stuck in an expensive scanner, usually in a hospital; the other is EEG, which stands for electroencephalography and involves wearing a cap studded with electrodes. %do I need a figure? (see Figure *).
In fMRI  powerful magnets are used to detect minute changes in magnetisation; from these changes you can infer how much blood is flowing to different parts of the brain, and so how active the neurons in different parts of the brain are.
In EEG you measure electrical currents on the scalp to make inferences about neuronal activity.

The two techniques  involve different trade-offs.
Using fMRI you can get relatively precise information about where things are happening in the brain at the expense of knowing when they are happening.
You also need to stay still in an fMRI scanner, which is one reason why this technique is not widely used with babies and children.
By contrast, EEG enables you to know when things happen to within fractions of a second but provides only rough clues as to where in the brain they occur.
Our focus is EEG, which is starting to be used quite widely in developmental science.

Using EEG to measure what someone's brain detects is often but not always too difficult because the upshot of EEG measurements is a map of  a large part of the brain's changing activity.
Fortunately there is a way of abstracting from the map to single out particular processes such as those involved in the visual detection of something unexpected.
Suppose that we have a sequence of images and know that people who see them will visually detect a change at a particular point.
Analysing the EEG measurements from multiple runs through the sequence and multiple people may allow us to identify some pattern in the measurements which  characteristically occurs when people visually detect something unexpected.
This pattern is an ERP component (ERP stands for event-related potential).
Discovering an ERP component from EEG measurements is a tiny bit like discovering how to identify the shadow of a moving tiger by measuring changes in how much light is reflected at different points of the ground beneath and around it.
%In order to discover a particular event-related potential and to identify its significance, you need to know, of course, what the brains you are measuring are doing.
%After discovery, the presence of an event-related potential can be used to gain knowledge of what brains do in particular situations.

So how can anyone that your brain can detect differences in categorical colour properties early in  processing visual stimuli?
There are ERP components called the N2 and P3.
These are associated with processes that occur after early vision and may reflect evaluation of what is seen and its context.
Comparing changes in colour that do cross a category boundary (blue to purple, say) with otherwise similar changes in colour that do not cross a category boundary (one green to another, say) reveals differences in the N2 and P3 only when the change in colour involves a change in categorical colour property.
This is the key to discovering that at least some of the processes involved in categorical perception are automatic (see \vref{cp:sec:fresh_start}).






%%% Local Variables:
%%% TeX-master: "master"
%%% End:
