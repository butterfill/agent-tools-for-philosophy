%!TEX root = master.tex

\chapter{Principles of Object Perception}
\label{cha:principles-object-perception}

\noindent
How do humans first come to know simple facts about particular physical objects?
To illustrate, you probably know some facts about the approximate location and shape of the book you are reading or the device you are using to read it.
But no one is born knowing any facts about particular physical objects.
So there was a time when you knew no such facts about any particular physical objects at all,
and then, sometime later, you came to know some such facts.
What makes this amazing transition possible?

% *todo In this chapter we will *.


% (For the rest of this chapter
% I shall drop the qualifier ‘physical’ since this is all about physical objects as opposed to,
% say, abstract objects like numbers or forms.)

This chapter introduces the experiments and discoveries that will guide our thinking in working out how humans first come to know facts about particular physical objects.
Another aim of this chapter and the next is to introduce some methods most commonly used in research on human infants.
By the end of these chapters you should be familiar with experiments involving \gls{habituation}, \gls{violation-of-expectation}, anticipatory looking and search behaviours.
(Methods involving neuroscience will not concern us until later chapters.)


\section{Knowledge of Objects Involves Three Abilities}
\label{sec:three-abilities}
What does knowledge of physical objects involve?
To answer this question it is helpful to think about features common to  all physical objects.

One  feature  is boundedness: there is a fact of the matter about where one physical object ends and another begins.
Since physical objects have boundaries, knowledge of physical objects plausibly involves an ability to identify where one object ends and another begins---that is, to \emph{segment} them.
As \vref{fig:segment_geese} illustrates,
the ways objects are ordinarily arranged in space prevents us from doing this in any very simple way.
Sometimes one object occludes parts of another from your point of view, and sometimes one object contains another.
So distinct objects do not always occupy clearly separate  regions of space.
To count the number of geese in a scene, or even just to distinguish which of several geese-parts belong to the same goose, involves being able to segment objects.

%
\begin{figure}
\begin{center}
\includegraphics[width=0.6\textwidth]{fig/DSC_1028cropbw.jpg}
\caption{
	\label{fig:segment_geese}
	The ways objects are ordinarily arranged in space means that there is no simple way to segment them.}
\end{center}
\end{figure}
%

Another feature of physical objects is persistence through time.
% To illustrate, you can sensibly ask whether the leg kicking you under the table is the same leg you kicked a moment ago.
Having any knowledge of physical objects probably involves an ability to represent things as persisting.
But what does this ability involve?
Suppose you are temporarily looking after a child, Hannah, who hides behind a pile of logs where you cannot see her.
When you need to find her, you know to go to the pile of logs even though you cannot see or hear her.
And if a girl pops up from behind the pile of logs, you might wonder momentarily whether it is Hannah again or another girl.
Knowing where to find her and wondering whether this girl is her both involve being able to represent objects as persisting.


Physical objects can causally interact with each other.
% Being kicked alters you, leaving your leg marked by a bruise long after the dinner is over and your host has left.
Knowing any facts about physical objects is likely to involve an ability to track causal interactions.
Minimally, this involves being able to distinguish interactions from mere contingencies.
Seeing how well the deformed shape of the barrier matches the dent in a nearby car, you might wonder whether this is mere coincidence or a consequence of the car hitting the barrier.
Seeing how one moving ball stops just as another begins to move, you might guess that the second ball’s movement is caused by that of the first.
In doing these things you are relying on an ability to track causal interactions among objects.

So far we have seen that reflection on features common to all physical objects
suggests that knowing even simple facts about physical objects
plausibly involves three abilities: to segment objects, to represent them as persisting, and to track some of their causal interactions.
% These three abilities are more primitive than knowledge in the sense that you could have all three abilities without yet knowing any facts about particular physical objects, at least in principle.
Our main question is, How do humans first come to know any facts about particular physical objects?
We can approach this question by asking a simpler one.
When and how do humans first manifest the three abilities associated with knowledge of objects? %, and what is the nature of their earliest abilities to segment physical objects, to represent them as persisting and to track their causal interactions?




\section{Segmentation}
\label{sec:segmentation} 

How do humans first segment physical objects?
% As we will see, there is evidence that even three- or four-month-olds can use perceptible clues like texture and shape to segment objects.
% We will also see that such tiny humans---who are typically months away from being able to move around by themselves---can, in addition to relying on shape and texture, use surprisingly sophisticated strategies to segment objects.
% In fact  this discovery will turn out to be the key to understanding how humans come to have all three abilities  on knowledge of objects mentioned in the previous section.
% But for now our concern is just the first ability, segmentation.
%
Textures and patterns provide one possible clue to where boundaries between objects lie.
As mentioned earlier, because one object often touches, occludes or contains parts of another object, objects do not reliably occupy clearly separate regions of space.
But a sharp change in texture is an indicator---not decisive but perhaps reliable enough---of a boundary between two objects.
Consider the top left panel in \vref{fig:needham_1998_fig3}.
The abrupt change in texture indicates that there are two similarly sized objects, one on its side and the other standing up.
That adult humans can use texture to segment objects is hardly controversial, of course.
But when in their development can humans first do this?

%
\begin{figure}
\begin{center}
\includegraphics[width=0.8\textwidth]{fig/needham_1998_fig3.neg.png}
\caption{
  \label{fig:needham_1998_fig3}
  Stimuli from an experiment showing that 4.5 month olds can use information about patterns or textures to segment objects.
  Source: \citet[][figure 3]{needham:1998_infants}.
}
\end{center}
\end{figure}
%

To answer this question, Needham devised a simple test \citep[experiment 2]{needham:1998_infants}.
She first showed her subjects, who were 4.5 month old infants, a display like that depicted in the top left panel of \vref{fig:needham_1998_fig3}.
A hand then appeared, grasped part of a visible object and pulled it.
For some of the infants, the hand’s pulling resulted in the moving apart of two objects, as depicted in the top row of pictures in \cref{fig:needham_1998_fig3}.
For the other infants, the hand’s pulling resulted in everything moving together, as depicted in the bottom row of pictures.
Would one group of  infants look at the display for significantly longer than the other group?
Needham reasoned that different hypotheses about segmentation predict different answers to this question.
Consider the hypothesis that infants do use textural information to segment the objects.
If this is true, infants would represent the scene as containing two objects and so would expect these objects to move independently when a hand pulls one of them.
Given that infants generally look longer at things which violate their expectations, those infants who see the two objects moving together should look at the display for longer than those who see the objects moving apart.
By contrast, the hypothesis that infants do not use textural information to segment objects does not generate any such prediction about relative looking times.
Accordingly, finding that those infants who see everything moving together look longer than those who see objects moving apart would be evidence that infants can use textural clues to segment objects.
And this is just what Needham found, as you can see in \vref{fig:needham_1998_fig4}.

%
\begin{figure}
\begin{center}
\includegraphics[width=0.6\textwidth]{fig/needham_1998_fig4.png}
\caption{
  \label{fig:needham_1998_fig4}
  Infants can use texture to segment objects from around 4.5 months of age.
  Source: \citet[][figure 4]{needham:1998_infants}.
}
\end{center}
\end{figure}
%

Needham’s experiment is relevant not only as evidence that infants can use texture to segment objects from their first months of life.
It also illustrates a widely used method that will come up frequently.
The method is called
\emph{\gls{violation-of-expectation}}.
It relies on the premise that when something violates an infant’s expectation, she will typically look at it for longer than she otherwise would have.
In designing a violation-of-expectation experiment, you first identify a hypothesis which generates predictions that certain individuals have particular expectations.
(In Needham’s experiment, the hypothesis was that 4.5-month-olds use texture to segment objects; this generated predictions about when they would expect things to move apart.)
% Needham: hypothesis = segment using texture; prediction: expect things to move apart
Importantly, no other relevant hypothesis should generate these predictions.
You then contrive two scenarios which are a similar as possible except that the putative expectation is violated in one but not the other scenario.
If you find that subjects look significantly longer at the scenario in which the putative expectation is violated, you have discovered evidence for the hypothesis.
With careful use of the violation-of-expectation method,
infants’ looking times enable us to discover many surprising things about what they expect.

Talk of expectations being violated may leave you wondering, What is
an expectation? This is an important question, and a surprisingly deep one.
We will return to it in \vref{sec:expectation-surprise}.

The main question for this section is how humans first segment physical objects.
We have seen that infants in their first months of life can use texture to segment objects;
further research also shows that they can also use shape as a clue \citep{needham:1999_role}.
But do humans’ abilities to segment objects initially rely entirely on superficial aspects of objects like their textures and shapes?

To show that they do not,
\citeauthor{kellman:1983_perception} asked what happens when infants are shown a stick partially occluded by a box, as illustrated in \vref{fig:kellman_1983_fig3part1}.
The stick is moving while the box remains still.
How do infants represent this scenario?
Does it look to them as if there are three objects, two short sticks either side of a box?
Or do they represent the scenario as most adults would---that is, as involving one long object partially occluded by the box?
To answer this question, \citet{kellman:1983_perception} showed a group of 4-month-old infants the stick moving behind the box repeatedly, until it no longer held their interest.
% $glossary: habituation
% $glossary: dishabituation
(That is, the infants were \emph{\glslink{habituation}{habituated}} to the display.)
After this, some of the infants were shown a new scenario involving a
single stick moving; and the other infants were shown a different new
scenario involving two unconnected sticks moving simultaneously (see \vref{fig:kellman_1983_fig3part2}).
In both new scenarios, the visible parts of stick from the first scenario were unchanged.
How different are each of the new scenarios from the original scenario?
That depends on how you represent the original scenario.
If you represent the original scenario as containing one stick partially occluded by a box, then the new scenario with two sticks is more novel than the new scenario with just one stick---not only has the box gone, but now there are two sticks whereas before there was just one.
(By saying that one of the new scenarios is ‘more novel’, I mean that it differs more from the original scenario than the other new scenario differs from the original scenario.)
But if you represent the original scenario as containing two sticks either side of a box, then the converse is true: the new scenario containing just one stick is more novel.
So if we had some way of finding out which of the two new scenarios infants find more novel, then we could work out whether infants represent the original scenario as involving one stick or two.
But finding out which of the new scenarios infants find more novel is surprisingly simple (in theory at least---little involving infants is simple in practice).
In general, after infants have been habituated to one scenario and shown new scenarios,  the more novel new scenario will produce greater \emph{\gls{dishabituation}}.
That is, the more novel scenario will be more effective in regaining infants’ interest.% (‘produce greater dishabituation’).
\footnote{%
This is a simplification; see \citet{sirois:2002_models} for a more detailed discussion of habituation.
}
Accordingly, \citeauthor{kellman:1983_perception} measured which of the two stick scenarios depicted in \cref{fig:kellman_1983_fig3part2}  produced
greater dishabituation.
You can see their results in \vref{fig:kellman_1983_fig4a}.
The left part, labelled ‘habituation’, describes how infants spent less time looking at the original scenario after seeing it repeatedly.
The first data points on the right part, labelled ‘test’, reveal a sharp increase in looking time for those infants shown the two-stick scenario (labelled ‘broken’) contrasting with no discernible increase in looking time among those infants who saw the one-stick scenario (labelled ‘complete’).
That is, the two-stick scenario produced greater dishabituation.
This is evidence that this scenario is more novel to the infants, which indicates that infants represent the original scenario as containing a single stick partially occluded by a box.


%
\begin{figure}
\begin{center}
\includegraphics[width=0.3\textwidth]{fig/kellman_1983_fig3part1.png}
\caption{
  \label{fig:kellman_1983_fig3part1}
  A stick moving behind a static box.
  Source: \citet[][figure 3, part]{kellman:1983_perception}.
}
\end{center}
\end{figure}
%


%
\begin{figure}
\begin{center}
\includegraphics[width=0.4\textwidth]{fig/kellman_1983_fig3part2.png}
\caption{
  \label{fig:kellman_1983_fig3part2}
  Two scenarios, both possible consequences of removing the box from \cref{fig:kellman_1983_fig3part1}.
  Source: \citet[][figure 3, part]{kellman:1983_perception}.
}
\end{center}
\end{figure}
%

%
\begin{figure}
\begin{center}
\includegraphics[width=0.8\textwidth]{fig/kellman_1983_fig4a.png}
\caption{
  \label{fig:kellman_1983_fig4a}
  Evidence that 4-month-olds segment objects using cues other than feature and shape.
  Source: \citet[][figure 4, part]{kellman:1983_perception}.
}
\end{center}
\end{figure}
%

What is exciting about a stick moving behind a box?
The fact that infants represent this scenario as involving one stick rather than two is a hint that they are not segmenting objects just on the basis of superficial properties like shape or texture.
%Instead they appear to use movement as a clue to where one object begins and another ends.
This idea is supported by evidence from a further experiment.
Kellman and Spelke modified their stick-moving-behind-box apparatus so that the visible portions of the object moving behind the box were markedly different in shape and texture, as depicted in \vref{fig:kellman_1983_fig13a}; and they made a corresponding change to what infants saw in the test phase of the experiment.
Although the two parts of the moving object are so different, 4-month-olds’ patterns of dishabituation again showed that they represent the scene as involving a single, connected object behind a box \citep[Experiment 6]{kellman:1983_perception}.
Taken together, these experiments provide good evidence that infants’ abilities to segment objects are not based entirely on recognising shape and texture (see also \citealp{Spelke:1990jn}).
%Instead infants can also use objects’ motions, and motion cues can override cues provided by shape and texture.



%
\begin{figure}
\begin{center}
\includegraphics[width=0.4\textwidth]{fig/kellman_1983_fig13a.neg.png}
\caption{
  \label{fig:kellman_1983_fig13a}
  An object with parts that differ in shape and texture moving behind a box.
  Source: \citet[][figure 13, part]{kellman:1983_perception}.
}
\end{center}
\end{figure}
%

\section{Principles of Object Perception}
\label{sec:principles-object-perception}
If infants do not rely only on shape and texture, how do they segment the objects in the displays we have just been considering?
\citet{Spelke:1990jn} suggests that infants rely on a set of principles to segment objects.

One of her principles is \emph{no action at a distance}.
This principle says that distinct ‘objects are interpreted as moving independently of one another.’%
\footnote{%
\citet[p.~50]{Spelke:1990jn}. This principle might need to be refined to accommodate cases where objects are stuck together---indeed, we can regard all of Spelke’s principles as initial guesses subject to revision as more is discovered about cognitive processes concerning physical objects.
}
Because the two moving parts of the apparatus represented in \vref{fig:kellman_1983_fig13a} do not move independently, the principle indicates that they are not distinct objects.
So the hypothesis that this principle describes in part how infants segment objects correctly predicts that they will treat the moving occluded stick as a single object.

The fact that the principle of no action at a distance describes how infants segment objects in one case (the stick-moving-behind-a-box case) is interesting but not by itself terribly informative.
What is more interesting is that this principle describes how infants segment objects in a wide variety of cases (see \citealp{Spelke:1990jn} for a more comprehensive review).
To get a feel for this variety, consider another study involving different stimuli and a different measure---instead of looking times, this one uses infants’ reaching actions.

Let me explain the stimuli first.
The study involves the four scenarios depicted in \vref{fig:spelke_1989_fig3}.
In the scenario depicted in the bottom right of this figure, two spatially separated regions move rigidly together.
The no-action-at-a-distance principle implies that there is just one object in this scenario.
Now consider the scenario depicted in the top left of \cref{fig:spelke_1989_fig3}.
In this scenario, two spatially contiguous things are moving in opposite directions.
Intuitively, these movements imply that there are two objects in this scenario.
This intuition is codified in a further principle Spelke calls \emph{rigidity}.
This principle is the converse of the no-action-at-a-distance principle; it states that ‘objects are interpreted as moving rigidly’ \citep[p.~50]{Spelke:1990jn}.
As there is no way of interpreting the scenario in such a way that one object moves rigidly, the principle of rigidity implies there are two objects.
\citet{spelke:1989_reaching} set out to show that, in each scenario depicted in \cref{fig:spelke_1989_fig3}, infants’ reaching behaviours show that they segment objects in accordance with the principles of no action at a distance and rigidity.

How can we infer any such thing from reaching?
\citet{spelke:1989_reaching} set things up, as depicted in \vref{fig:spelke_1989_fig1}, so that the smaller of the two objects was always closer to the infant.
The experimenters were then able to rely on the background assumption (which is not obvious but carefully justified---see \citealp[p.~186]{spelke:1989_reaching}) that infants typically reach more often for the smaller, nearer object when they represent the scenario as involving two objects than when they represent it as involving just one object.
So by comparing how often 5-month-olds reach for the smaller object, \citeauthor{spelke:1989_reaching} could determine, for each scenario, whether infants treat it as involving two objects or just one.
As predicted, they found that, overall, 5-month-olds reach for the smaller, nearer object more often when they moved in opposite directions than when they moved together.
Given the background assumption, this is evidence that infants segmented the objects in accordance with the principles of rigidity and no action at a distance.


%
\begin{figure}
\begin{center}
\includegraphics[width=0.7\textwidth]{fig/spelke_1989_fig3.png}
\caption{
  \label{fig:spelke_1989_fig3}
  Four scenarios.  Things move in opposite directions in the two scenarios depicted left, whereas in the other scenarios everything moves together.
  Source: \citet[][figure 3]{spelke:1989_reaching}.
}
\end{center}
\end{figure}
%


%
\begin{figure}
\begin{center}
\includegraphics[width=0.7\textwidth]{fig/spelke_1989_fig1.png}
\caption{
  \label{fig:spelke_1989_fig1}
  The apparatus \citet{spelke:1989_reaching} used to measure how different relative motions affect infants’ dispositions to reach.
  Source: \citet[][figure 1]{spelke:1989_reaching}.
}
\end{center}
\end{figure}
%




No action at a distance and rigidity are not the only principles we need to explain how infants segment objects.
Other principles which seem to be involved in segmenting objects are \emph{cohesion} and \emph{boundedness}.
According to the principle of cohesion, ‘two surface points lie on the same object only if the points are
linked by a path of connected surface points’ \citep[p.~49]{Spelke:1990jn}.
The principle of boundedness says, conversely, that ‘two surface points lie on distinct objects only if no path of connected surface points links them’ \citep[p.~49]{Spelke:1990jn}.
To illustrate, consider the two situations represented in the left and right parts of \vref{fig:kestenbaum_1987_fig3a}.
The principle of boundedness implies that there is one object in the left-depicted situation, and the principle of cohesion implies that there are two objects in the right-depicted situation.
\citet{kestenbaum:1987_perception} suggest that these two principles are also needed to describe how infants segment objects.


% cut description of \citet{kestenbaum:1987_perception}:
% Infants were habituated to the display.  Then either one object's position changed, or both objects’ positions changed but in such a way as to preserve the overall configuration of the two objects.  Infants could show that they perceived the configuration as a single object by looking longer when just one object's position changed.

%
\begin{figure}
\begin{center}
\includegraphics[width=0.7\textwidth]{fig/kestenbaum_1987_fig3a.png}
\caption{
  \label{fig:kestenbaum_1987_fig3a}
  According to the principles of cohesion and boundedness, the situation depicted on the left involves one object whereas the situation depicted on the right involves two.
  Source: \citet[][figure 3, part]{kestenbaum:1987_perception}.
}
\end{center}
\end{figure}
%


\section{Conclusion}
Our ultimate aim is to understand the transition humans make from knowing nothing at all about particular physical objects to having such knowledge.
Given how difficult this is,
we have taken an indirect route by first considering three abilities that are probably involved in having knowledge: abilities to segment physical objects, to represent them as persisting and to track their causal interactions (see \cref{sec:three-abilities}).
The latter two abilities are a topic for the next chapter; this chapter was about how humans first segment physical objects.

Before looking at the evidence it might have been tempting to guess that very young infants cannot segment objects at all, or else that they can only segment objects by using surface properties like shape and colour.
The truth turns out to be much more interesting.
From around four months of age, humans can indeed use shape and texture in detecting where one object ends and another begins.
But by this age they also segment objects in accordance with a set of principles which go beyond the surfaces to describe the characteristic ways objects move.
As we have seen, one consequence is that infants, like adults, can reliably form expectations about occluded parts of objects.
In fact it seems that, from as early as they have been tested, infants segment objects in much the way that adults do.

This is an exciting result, but it does not yet enable us to answer the question we started with.
The question is, How do humans first segment physical objects?
We can say that they do it in accordance with the four principles Spelke labels \emph{\gls{Principles of Object Perception}}---no action at a distance, rigidity, boundedness and cohesion.
But to say this is merely to describe patterns in their performance.
Answering our ultimate question---How do humans first come to know simple facts about particular physical objects?---will require discovering what explains these patterns.



%%% Local Variables:
%%% TeX-master: "master"
%%% End:
