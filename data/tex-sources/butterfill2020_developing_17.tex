%!TEX root = master.tex


\chapter{Joint Action}
\label{cha:joint-action}


The overarching challenge we face is to understand the developmental emergence of knowledge.
How do humans first come to know simple facts about objects, actions and minds?
So far we have seen that infants in the first year or two of life have significant abilities in each domain, and, more controversially, that these abilities do not appear to require attributing them any knowledge states at all.
In exploring evidence on development in several domains, we found no compelling reason to accept that one year olds have knowledge, rather than merely \gls{core knowledge}, in these domains.
Instead their abilities rest on what we are loosely calling \gls{core knowledge} (see \cref{sec:core-knowledge-minimal-view}), which, at least in the domains of objects and actions and perhaps even in the domain of minds too, comprises broadly perceptual and motoric representations. 

These discoveries provide us with an opportunity and a challenge.
They make it theoretically coherent to suppose that infants’ earliest abilities provide a basis for the later acquisition of knowledge.
So knowledge does not have to come from nothing: there is a representational precursor which manifests itself early in development and presumably provides some kind of basis for the later acquisition of knowledge.
But---here comes the challenge---the discoveries also make it difficult to understand how there could be any connection between core knowledge and knowledge proper.
This is because in each domain---objects, actions and minds---we have seen that the representations comprising core knowledge are not inferentially integrated with knowledge states, and that they are intentionally isolated from them.
\glsadd{intentional isolation}
\glsadd{inferential integration}
There is little prospect, then, that the transition from not knowing any simple facts about particular objects, actions or minds to knowing some such facts will involve somehow transforming core knowledge states into knowledge states. \glsadd{Assumption of Representational Connections}
The developmental emergence of knowledge must be a process of rediscovery: of coming to know things which are in some sense already encoded in core knowledge.

How could rediscovery occur?
Perhaps it rests on social interactions.
One possibility, is that the emergence of knowledge is linked to the appearance of increasingly rich forms of social interaction in the second year of life.
Could such rich forms of social interaction somehow facilitate the developmental emergence of knowledge?

 According to what Moll and Tomasello call the ‘Vygotskian Intelligence Hypothesis’,
 \begin{quote}
‘participation in cooperative \ldots\  interactions … leads children to construct uniquely powerful forms of cognitive representation’
\citep{Moll:2007gu}.
\end{quote}
 Knoblich and Sebanz arrive a similar claim based on very different considerations:
 \begin{quote}
‘perception, action, and cognition are grounded in social interaction’
\citep[p.~103]{Knoblich:2006bn}.
\end{quote}
 And \citeauthor{sinigaglia:2008_roots} likewise argue that
\begin{quote}
‘human cognitive abilities … [are] built upon social interaction’
\citep{sinigaglia:2008_roots}.
\end{quote}
I assume that they are all right, although I am less sure that I understand precisely what each duo of theorists is claiming.
To avoid a tricky exegetical detour, let us make some simplifying assumptions.
What forms of ‘cognitive representation’ should these theorists have in mind?
I shall assume that it is knowledge.
And what form of ‘cooperative’ or ‘social interaction’ should they have in mind?
I think it should be joint action.
These assumptions yield what I shall call the \emph{\gls{Joint Action Conjecture}}:
\begin{quote}
    Abilities to perform joint actions play a role in explaining the developmental emergence of knowledge, including knowledge of others’ minds.
\end{quote}
This is probably not exactly what any of the three theoretical duos quoted above had in mind, but it is inspired by their ideas and findings.

In this chapter, our primary aim is to characterise joint actions as performed by children in the first years of life.
If the Joint Action Conjecture is right, this will be a step towards understanding the developmental emergence of knowledge.
% It will also turn out to be relevant for understanding the emergence of abilities to communicate (see \cref{cha:communication}).
But first, what are joint actions?



\section{Joint Action vs Parallel but Merely Individual Actions}
\label{sec:joint-vs-parallel-action}

I have been using the term ‘joint action’ without explanation.
What anchors our thinking about joint action?
Researchers often introduce it just by giving examples.
Paradigm cases in philosophy include two people
	painting a house together \citep{Bratman:1992mi},
	lifting a heavy sofa together  \citep{Velleman:1997oo},
	preparing a hollandaise sauce together \citep{Searle:1990em},
	going to Chicago together \citep{Kutz:2000si},
	and walking together \citep{gilbert_walking_1990}.
In developmental psychology paradigm cases of joint action include  two people
	tidying up the toys together \citep{Behne:2005qh},
	cooperatively pulling handles in sequence to make a dog-puppet sing \citep{Brownell:2006gu},
	and bouncing a block on a large trampoline together \citep{Tomasello:2007gl}.
Other paradigm cases from research in cognitive psychology include two people
	lifting a two-handled basket  \citep{Knoblich:2008hy},
	putting a stick through a ring \citep{ramenzoni_joint_2011},
	and swinging their legs in phase \citep[p. 284]{schmidt_richardons:_2008}.
We need to be careful because it is not obvious that there is a single phenomenon of which all these are paradigm cases.

A slightly better way to introduce joint action is by using \glspl{contrast case}.
Contrast cases are pairs of events which are as similar as possible except that one is a joint action while the other is not.
\citet{gilbert_walking_1990} contrasts two friends out walking together in the way friends typically walk together
with two strangers who happen to be walking side by side.
The  friends’ and the strangers’ movements may happen be so similar that looking at them from above you could not say which pair was the friends and which the strangers.
Despite this, the friends’ walking together is a joint action whereas merely walking side by side is not.
Relatedly,  \citet{Searle:1990em}  contrasts an event involving several park visitors simultaneously running to a central shelter in order to perform a dance with an event involving park visitors running to the central shelter in order to escape a storm.
The first event is a joint action whereas the second is not.
But we could imagine that the two events involve very similar movements.

These \glspl{contrast case} invite the question,
How do joint actions differ from actions that are performed in parallel but are merely individual actions?
Minimally, an account of joint action needs to provide an answer to this question.

Gilbert's contrast case shows that the difference is not just  a matter of coordination.
After all, strangers who find themselves walking side by side may need to coordinate their actions in order to avoid colliding.
And Searle's contrast case shows that the difference between joint action and parallel but merely individual actions is not just that the actions have a common effect because parallel but merely individual actions can have common effects too.
So what does distinguish joint actions from parallel but merely individual actions?



\section{Shared Intention}
\label{sec:shared-intention}

Many  philosophers and some psychologists hold that all joint actions involve shared intention.
They tend to assume that characterising joint action is fundamentally a matter of characterising shared intention.
On this view, ‘the key property of joint action lies  \ldots\  in the participants’ having a \ldots\  “shared” intention’
\citep[pp.~444--5]{alonso_shared_2009}.
Although their terminology differs, many philosophers including \citet[p.~5]{Gilbert:2006wr} endorse this claim, as do developmental psychologists who have thought deeply about joint action like \citet[p.~381]{carpenter:2009_howjoint},
\citet[p.~117]{rakoczy_pretend_2006} and  \citet[p.~181]{tomasello:2008origins}.

% ‘I take a collective action to involve a collective [shared] intention.’
% \citep[p.~5]{Gilbert:2006wr}
%
% ‘The sine qua non of collaborative action is a joint goal [shared intention] and a joint commitment’
% \citep[p.~181]{tomasello:2008origins}
% ‘Shared intentionality is the foundation upon which joint action is built.’
% \citep[p.~381]{carpenter:2009_howjoint}
But what is shared intention?

The first thing to note is that  ‘shared intention’ is a term of art.
On almost any account, a shared intention is neither something literally shared nor literally an intention.

To see the point of invoking shared intention, whatever exactly it turns out to be, it is helpful to
draw a parallel with individual action.
\citet{Davidson:1971fz} opened a famous discussion of agency by asking, Which events are actions?
He contrasted actions with things that merely happen to an \gls{agent}.
To illustrate,
%we might be struck by the contrast between your arm being caused to go up by forces beyond your control, and the action you perform when you raise your arm.
% Or
we might be struck by the contrast between mere reflexes, such as the eyeblink reflex, and actions such as the blinking your eyes performed in covertly greeting a friend.

One quite standard way to answer the question Which events are actions? is by appeal to intention.
The idea is that events are actions in virtue of being
appropriately related to an intention.
On this view,
% the difference between  you arm being caused to go up and you raising is explicable by appeal to intention.
%Raising your arm involves an intention to raise your arm, which appropriately related to the movement of your arm.
intention is what distinguishes a reflex from an action.
Intention is absent from, or not appropriately related to, the eyeblink reflex but (so the view) critical for the action of blinking your eyes.

The question Which events are actions? has a counterpart for joint actions, namely
Which events are joint actions?
Invoking shared intention makes it possible to give an answer to the question about joint action that resembles the answer standardly given about action:
a joint action is an event which is appropriately related to a shared intention \citep[pp.~3--7]{pacherie:2013_lite}.
% To the extent that we are persuaded by the standard view on which events are actions,
% it may seem compelling to aim for a structurally parallel account of which events are joint actions.
% To do this we merely have to characterise shared intentions.
This parallel between intention and shared intention is important because it is
a rare point on which almost everyone will agree---even those who, like me, are not persuaded by the standard view of actions and intentions.
Shared intention is to joint action at least approximately what
ordinary, individual intention is to ordinary, individual action.

It is important to acknowledge that we have not yet said anything very informative about what shared intention is.
We asked, Which events are joint actions?
The answer was, those which stand in an appropriate relation to a shared intention.
Now we ask, What is shared intention?
Suppose we answer by saying that it is something in virtue of which events are joint actions.
Maybe this is right.
But at this point we have moved in a circle too tight to count as sufficiently informative even to a philosopher.

Beyond mostly accepting that
shared intention is to joint action what
 intention is to action,
philosophers  disagree quite wildly about what kind of thing shared intention is.
Some, like \citet{Searle:1990em}, regard it as a special kind of mental state, something that  differs from intention as belief differs from desire.
Others, like \citet{helm_plural_2008,laurence:2011_anscombian}, appear think that it requires us to postulate a special kind of \gls{agent}.
Still others, like \citet{Gold:2007zd}, think that a shared intention is just an ordinary intention that is arrived at by a special kind of reasoning, team reasoning.
Taking yet another line, \citet{gilbert:2014_book} proposes that we understand shared intention in terms of a special kind commitment,
and \citet{tuomela_we-intentions_2005} that we understand it in terms of a special mode, the ‘we-mode’.
Meanwhile, \citet{bratman:2014_book}, following a strategy introduced by \citet{tuomela_we-intentions_1988}, argues that shared intention does not require any such conceptual or ontological novelty but can be realised by a nested structure of intention and knowledge.
It seems as if each philosopher who has thought about this has ended up with a new account of her own, one that is fundamentally different from everyone else’s.

% This could be an advantage for our project.
% The aim is not to find the one true account of joint action---and given the way joint action is standardly introduced (see \cref{sec:joint-vs-parallel-action}), we should not assume any such thing exists either.
% Instead the aim is understand joint actions as performed by one- and two-year-olds.
% We can therefore select any account and investigate how well it fits what we know of joint action in the first years of life.
So which account of joint action should we start with?
Bratman’s is probably the most carefully developed and certainly the most influential among scientists.



\section{Bratman on Shared Intention}
\label{sec:Bratman-shared-intention}

We aim to understand joint actions as performed by one- and two-year-olds.
And we are assuming, for now at least, that shared intention is the distinguishing feature of all joint actions.
We therefore need an account of shared intention.

In characterising shared intention, Bratman first identifies its function.
On his account, shared intention serves to  coordinate activities,  coordinate planning and structure bargaining \citep{Bratman:1993je}.
To illustrate,
suppose you and I have a shared intention that we wash the glasses together.
Then this shared intention should structure our bargaining insofar as our different preferences may mean we need to compromise over how clean we will make the glasses;
it should coordinate our planning insofar as we need to order our washing and drying efficiently;
and it should also coordinate our activities when, for instance, you are passing me a glass to dry.

If this is what shared intentions are for, what could they be?
Bratman argues that the following are collectively sufficient conditions for you and I to have a shared intention that we J:
% %
% \footnote{
% In \citet{Bratman:1992mi}, the following were offered as jointly sufficient \textit{and individually necessary} conditions; the retreat to sufficient conditions occurs in \citet[][pp.~143-4]{Bratman:1999fr} where he notes that `for all that I have said, shared intention might be multiply realizable.'
% }
% %
\begin{quote}
`1. (a) I intend that we J and (b) you intend that we J

`2. I intend that we J in accordance with and because of la, lb, and meshing subplans of la and lb; you intend that we J in accordance with and because of la, lb, and meshing subplans of la and lb

`3. 1 and 2 are common knowledge between us'  \citep[View 4]{Bratman:1993je}.
\end{quote}
%
To illustrate, consider our shared intention that we wash the glasses together again.
According to the above account,
to have such a shared intention, it would be sufficient
that we each intend that we wash the glasses together,
that we each intend to wash the glasses together in accordance with and because of these intentions (and meshing subplans of them),
and that all of this is common knowledge between us.

What does Bratman mean by subplans?
Our shared intention to wash the glasses together may require us to make further plans.
For example, we may plan to do the water glasses before the beer glasses, and we may plan to soak the wine glasses before washing them.
These plans are \emph{subplans} just because they are plans that are components of our larger plan to wash the glasses.
And the subplans \emph{mesh} insofar as we can successfully execute all of them---they would fail to mesh if, for example, your subplan was to soak the wine glasses while washing the water glasses while my subplan was to wash the wine glasses first.

In more recent work Bratman has added these further conditions:
\begin{quote}
4. The persistence of each intention in conditions 1 and 2 is interdependent with the persistence of every other such intention
	(\citealp[p.~153]{Bratman:1999fr};
	\citealp[pp.~7--8]{bratman:2006_dynamics};
	\citealp[p.~157]{Bratman:2009lv};
	\citealp[p.~12]{Bratman:2010ats})

5. We will J ‘if but only if 1a and 1b’
	(\citealp[p.~153]{Bratman:1999fr};
	\citealp[p.~157]{Bratman:2009lv}).
\end{quote}
The common knowledge condition, \#3 above, is extended to include these further conditions, \#4 and \#5.

You could spend a long and probably very happy time studying the intricacies of Bratman’s account and various critical exchanges over it (for example, \citealp{ludwig:2015_shared,smith:2015_shared} and \citealp{bratman:2015_shareda}).
But this turns out not to be necessary for our purposes.
There is a straightforward and compelling objection to combining Bratman’s view with the \gls{Joint Action Conjecture}.


\section{An Inconsistent Triad}
\label{sec:inconsistent-triad-joint-action}

To meet the sufficient conditions Bratman gives for having a shared intention (see \cref{sec:Bratman-shared-intention}), it is not enough that we each have intentions.
In addition, we must each have intentions about these intentions.
And, thanks to the common knowledge condition, we must each know that we know that we have intentions about these intentions (see \vref{fig:bratman_escalation}).

\addFigure{bratman_escalation}{%
On Bratman’s sufficient conditions for shared intention, sharing is escalating.
Minimally, we each have to know that we know that we have intentions about our intentions.}

Suppose for a moment that all joint action involved shared intention, and that meeting Bratman’s sufficient conditions was the only way to have a shared intention.
Then performing joint actions would require having knowledge of others’ minds at close to the limits of adults’ capacities.
In that case we would have to reject the \gls{Joint Action Conjecture}, on which abilities to perform joint actions play a role in explaining the developmental emergence of knowledge including knowledge of others’ minds.
After all, if performing joint actions already requires having knowledge of others’ minds, we can hardly invoke abilities to perform joint actions to explain the emergence of such knowledge.

Nothing that involves meeting the sufficient conditions given by Bratman could explain the developmental emergence of knowledge of others’ minds because meeting these conditions requires already having knowledge of others’ minds.
% And indeed Bratman’s conditions involve knowledge of others’ minds at close to the limits of adults’ capacities.

%After all, if joint action requires shared intention and shared intention requires knowledge about knowledge about intentions about intentions, then to suppose that an infant has abilities to perform joint actions would presuppose that she was already capable of having the knowledge whose emergence in development is supposed to be explained.

One way around this issue would be to find alternative sufficient conditions for shared intention, conditions you can meet without already having knowledge of others’ minds---or any of the other knowledge we might hope to explain by invoking joint action.
Several philosophers have offered ideas that take us at least part of the way in this direction \citep[for example,][]{Tollefsen:2005vh,pacherie:2013_lite}.
But I think there is a reason why this strategy is unlikely to work.



One of the functional roles of shared intention is to coordinate planning, which is distinct from coordinating activities.
Planning occurs, for example, when we consider whether to wash the glasses in any particular order.
Although there are probably cases in which you can coordinate your plans with another without knowledge of her intentions and other mental states,
capacities to coordinate planning will in general require knowledge of others’ minds \citep{Butterfill:2011fk}.
Abilities to coordinate plans with another presuppose abilities to know things about others’ minds.
The following three claims are therefore collectively inconsistent:
%
\newcommand{\inconsistentTriad}{%
\begin{enumerate}
  \item Abilities to perform joint actions play a role in explaining the developmental emergence of knowledge, including knowledge of others’ minds. (This is the \gls{Joint Action Conjecture}.)
  \item All joint action involves shared intention.
  \item A function of shared intention is to coordinate two or more \gls{agent}s’ plans (as Bratman’s account implies).
\end{enumerate}%
}%
\inconsistentTriad
%
These three claims cannot all be true.
But which should we reject?

To answer this question, we need to consider the capacities of one- and two-year-olds.
Are children of this age already capable of coordinating their plans with others’?
If so, we may have grounds for rejecting the \gls{Joint Action Conjecture};
% Bratman book: ‘we might in the end see the capacity of young humans  for modest sociality (if such there be) as itself evidence of  the sort of psychological com­plexity at issue’
if not, there may be reason to reject one of the other claims and find an alternative way to characterise joint actions as performed by one- and two-year-olds.

Although the inconsistent triad above is not explicitly about knowledge, the underlying issue is one of knowledge.
In fact, the  inconsistent triad is analogous to the puzzles we have encountered in the attempting to understand developments about mindreading, action tracking and object cognition in earlier chapters.
In each case, there seem to be reasons to suppose that explaining infants’ early-developing abilities requires attributing them knowledge of a domain.
If decisive, these reasons would make it theoretically incoherent to appeal to these early-developing abilities in explaining the developmental emergence of knowledge in that domain.
But so far those reasons have not turned out to be decisive; there are as strong, or stronger, reasons not to attribute knowledge in explaining the early-developing capacities (see \cref{cha:linking-problem}, for example).
In the case of joint action, coordinating plans requires knowledge of other’s mental states, as we have just seen.
This is why the question of whether one- and two-year-olds’ joint actions really involve coordinating plans is matters for explaining the developmental origins of knowledge.
If these children’s actions involve coordinating plans, then they involve knowledge of mental states, and so would have to abandon the \gls{Joint Action Conjecture} and avoid appeal to joint action in explaining the developmental emergence of knowledge.




\section{Coordinating Planning}
\label{sec:coordinating-planning}

Carpenter argues that infants’ joint actions are not less sophisticated in kind than adults’:
%
\begin{quote}
     ‘Despite the common impression that joint action needs to be dumbed down   for infants due to their ‘‘lack of a robust theory of mind’’  \ldots\  all the important social-cognitive building   blocks for joint action appear to be in place:
     1-year-old infants   understand quite a bit about others’ goals and intentions and what   knowledge they share with others’   \citep[p.~383]{carpenter:2009_howjoint}.
\end{quote}
%
On her view,
which appears to be quite widely shared \citep[for example,][]{Moll:2007gu,Tomasello:2005wx},
infants are capable of joint action and all joint action involves shared intention as characterised by Bratman.
% Further, infants can meet Bratman’s sufficient conditions for shared intention.
If so, we must reject the \gls{Joint Action Conjecture}.
But first let us consider how well the evidence supports Carpenter’s position.
The hypothesis that one- and two-year-olds have shared intentions as characterised by Bratman generates a prediction:
since a function of shared intention is to coordinate planning,
children of this age should be capable, at least in some minimally demanding situations, of coordinating their plans with another’s.

What might count as evidence that you can coordinate your plans with another’s?
\citet{paulus:2016_development} created an elegantly simple task which demands minimal coordination.
Consider the apparatus in \vref{fig:paulus_2016_fig1}.
There is a box with two holes, one round and the other square, and a tool with two ends, one spherical and the other cubic.
Which hole the tool should be inserted into depends on your task.
And of course the tool will only go into the holes if it is inserted the right way around---the spherical end will not go into the square hole.
Now imagine picking up the tool and putting it into the square hole.
How should you best pick up the tool in doing this?
If you pick it up by the spherical end, you can insert the tool directly into the hole.
But if you pick the tool up by grasping the cubic end, you will have to swap it between your hands before you can insert it.
This is almost too straightforward to mention: even three-year-olds pick up the tool optimally depending on the task \citep{paulus:2016_development}.
But now imagine doing the task with another person.
You will pick up the tool and pass it to her, and she will insert it into the square hole.
How should you best pick up the tool this time?
By grasping the cubic end and passing the tool to her, she will take the spherical end and so be poised to insert the cubic end directly into the hole.
And this is what adults tend to do.
But what about children?

\addFigure{paulus_2016_fig1}{%
Materials for an investigation of the development of coordinated planning.
Source: \citet[figure 1]{paulus:2016_development}}

\citet{paulus:2016_development} found that, when acting jointly with another, three- and five-year-olds picked up the tool without any regard for what the other person was going to do with it.
It was only seven-year-olds who took into account the other’s action, and they only did so after the first few trials.
Apparently it is not until surprisingly late that children can coordinate their plans with another’s even in this minimally demanding situation.

We should not read too much into a single experiment, of course.
It is possible that some extraneous feature unrelated to demands on coordination of plans threw the younger children off in \citeauthor{paulus:2016_development}’s experiment.
But it turns out that the developmental pattern is robust across many experiments.
Apparently children under about five years of age do not coordinate their plans with others’.

\citet{warneken:2014_young} created a situation in which a pair of children needed two different tools to release a ball, a plunger and a twister.
Before the children could attempt to release the ball, first one child selected a tool and then the other child selected a tool.
 How children selected tools was therefore a good indicator of their planning.
In one condition, the task of choosing a tool was made relatively easy.
One child chose either a plunger or a twister first, and then the second child could choose whether to take a plunger or a twister.
All she needed to do was to take the kind of tool not yet chosen; otherwise they would end up with two tools of the same kind and so be unable to release the ball.
The second child’s successfully choosing a tool complementary to the one already chosen would indicate an ability to coordinate actions.
In another condition, the task of choosing a tool more clearly required coordinated planning.
The first child could see that the only tool available to the second child would be a twister.
She therefore had to anticipate this constraint on the second child’s future choice by not taking a twister herself.
Success in doing this might be taken to indicate an ability to coordinate planning.
Whereas the five-year-olds did well in both conditions, the three-year-olds showed no evidence at all of coordinating their plans with the other’s.
Some did learn over the course of the experiment to select a tool complementary to the one already chosen, suggesting an ability to coordinate actions.
But none showed evidence of coordinating plans by anticipating the constraint on the second child’s future choice.

% ‘by age 3 children are able to learn, under certain circumstances, to take account of what a partner is doing in a collaborative problem-solving context. By age 5 they are already quite skillful at attending to and even anticipating a partner’s actions’ \citep[p.~57]{warneken:2014_young}.

The systematic failure of three-year-olds to coordinate their plans with others’ suggests that there is a mismatch between our current assumptions about what joint action is and these (and younger) children’s actual abilities to engage in joint action.
On our current assumptions,
all joint action involves shared intention and Bratman is right about what shared intention is.
These assumptions are worth taking seriously because they underpin views held by \citet{carpenter:2009_howjoint} and \citet{Moll:2007gu} among others.
But they also appear to be wrong.
For Bratman’s is a planning theory of shared intentions: on his theory, shared intentions serve to coordinate plans.
The hypothesis that one- and two-year-olds have shared intentions therefore leads to the prediction that they should be capable, in some situations at least, of coordinating their plans with others’.
And all the available evidence suggests this prediction is false.
In fact coordinated planning is something humans first achieve some time after their fifth birthday.

You might object that the experiments mentioned so far all involve tools.
Could there be something especially difficult about coordinating planning when tools are involved?
If there is, it is unlikely to explain apparent deficits in coordinating planning because children show the same deficits even when no tools are involved.
\citet{meyer:2016_planning} created a simple scenario in which children passed objects to an adult.
The twist was that the children could not pass the objects straight to the adult  because of a glass screen between them. Instead children had to pass objects to the adult’s left or right.
The adult always had only one hand free to receive the object, and which hand she had free changed as the task unfolded (see \vref{fig:meyer_2016_fig1c}).
To avoid the adult having to make an awkward reach, children would ideally always pass objects to the adult’s free hand, switching sides as necessary.
They were clearly to some extent sensitive to this constraint because they mostly began by passing the object towards the adult’s free hand.
But over time, children who were three years of age or younger did not pass objects to the adult’s free hand, and even  five-year-olds ‘adjusted their action plans to a surprisingly small degree’ \citep[p.~8]{meyer:2016_planning}.
This suggests that although children under three years of age (and perhaps older children too) can anticipate and respond to others’ actions, they cannot coordinate planning with them.

\addFigure{meyer_2016_fig1c}{%
How not to pass an object.
Source: \citet[figure 1 (part)]{meyer:2016_planning}}

To be sure that the evidence we have been considering really does point to a deficit in coordinating planning, it would be helpful to compare two scenarios which are as similar as possible except that one does, whereas the other does not, require coordinating plans.

\citet[experiment 2]{gerson:2016_social} created just such a pair of contrasting scenarios.
Three-year-olds were given four egg cups, two yellow and two brown.
They then had to retrieve each of four balls from a transparent tube  and place it in an egg cup of the corresponding colour (see \vref{fig:gerson_2016_fig1B}).
One ball was half brown and half yellow, and was allowed to be in any egg cup.
But since children only had four egg cups, they needed to pay attention to the colours of the remaining balls in placing the half-brown, half-yellow ball.
Otherwise they would not have enough egg cups of the correct colours for the remaining balls.
When performing the task alone, children did well enough.
But \citeauthor{gerson:2016_social} were most interested in what would happen when children performed the task jointly with another (a puppet, in this case), taking turns to place balls.
They found that children were much worse when acting jointly.
In fact they showed no evidence at all of planning ahead when acting jointly \citep[experiment 1]{gerson:2016_social}.
But most importantly, the researchers compared the joint action scenario with a machine scenario.
In the machine scenario, children alternated with a machine which placed balls into egg cups.
The machine scenario was as similar as possible to the joint action scenario except that no coordinated planning was needed (or even possible).
Despite this, three-year-old children’s performance was better in the machine scenario, and not different from their performance when they were performing the whole task alone \citep[experiment 2]{gerson:2016_social}.
Together with the other findings we have considered, this is strong evidence that three-year-olds cannot yet coordinate their plans with others.

\addFigure{gerson_2016_fig1B}{%
Match the balls to the egg cups.
Source: \citet[figure 1B]{gerson:2016_social}}

We are working on the assumption that a function of shared intention is to coordinate two or more agents’ plans.
Given this assumption,
the hypothesis that one- and two-year-old children have shared intentions leads to the prediction that these children can coordinate their plans with others’.
At least, they should be able to do so in minimally demanding situations.
But in fact it appears that abilities to coordinate plans develop much later, perhaps between five  and seven years of age.
It is always possible, of course, that the available evidence is misleading and a new breakthrough finding will significantly alter the picture.
But the available evidence clearly best supports the hypothesis that one- and two-year-old children are not capable of having shared intentions.


\section{Joint Action in the First Years of Life}
\label{sec:joint-action-first-years-of-life}

We can summarise the position we have reached so far in this chapter with another inconsistent triad (the first was in \cref{sec:inconsistent-triad-joint-action}):
%
\begin{enumerate}
\item  One- and two-year-olds are capable of performing joint actions.
  \item All joint action involves shared intention.
  \item A function of shared intention is to coordinate two or more agents’ plans (as Bratman’s account implies).
\end{enumerate}
%
As we saw, Carpenter and others hold that all three claims are true.
But these claims lead to the incorrect prediction that one- and two-year-olds are capable of coordinating their plans with others’.
For this reason, at least one of the claims should be rejected.
But which?

Consider the first claim, which is that one- and two-year-olds are capable of performing joint actions.
So far we have been taking the truth of this claim for granted.
But a hardline response to the evidence that children show no sign of coordinating their plans with others’ until around five years of age would be to deny that children much under this age are capable of joint action at all.
Are there grounds to hold, on the contrary, that joint action is possible in the first years of life?

To set the scene, consider that the need to be fed means that infants are involved in coordinated interactions with older caregivers from birth.
In the first few months of life, infants typically appreciate dyadic interactions with others, the sort of interactions that occur when an adult spontaneously makes faces and vocalizes at an infant.
This can be shown by having the infant and a caregiver interact over a video link.
When instead of showing the caregiver live, a recording of her past interactions with the infant is played, infants are less satisfied \citep[p.~323]{Trevarthen:1980kr}.
Infants are not satisfied merely by seeing familiar gestures directed  at them: even in the first months of life, they care about some aspect of the interaction.

A key milestone in development occurs in the second half of the first year, when formerly dyadic interactions expand to include ‘objects, events and individuals outside of the dyad’ \citep[p.~197]{brownell:2011_early}.
At this stage, infants react positively to being shown or passed objects, and you can sometimes even play at teasing them by offering and then retracting an object (compare \citealp{Behne:2005dw}).
But for the most part, infants in the first year of life do not initiate joint actions.

One year olds, by contrast, may attempt to initiate a familiar joint action
by, for example, placing a cloth over her their faces and waiting until an adult plays peek-a-boo \citep{brownell:2011_early}.
They will also attempt to reengage adults who become distracted or unresponsive in a joint action.
To demonstrate this, \citet[experiment 2, Trampoline]{warneken:2007_helping} had a confederate play a game with fourteen-month-olds which involved bouncing a cube on a hand-held trampoline.
The infants held one side of the trampoline and the confederate held the other, and they bounced the cube together.
After some time, the confederate froze and become unresponsive.
How did the infants respond?
In a minority of cases, they disengaged or attempted to continue on their own.
But most of the time the infants either waited or attempted to reengage the confederate.
Their tendency to reengage the confederate in the joint action indicates that even some fourteen-month-olds are to some extent aware that what they are attempting to do requires a contribution from another.%
\footnote{%
In a later study with older children, \citet{warneken:2012_collaborative} showed that children will reengage a partner even when it would be possible for them to achieve the goal without any contribution from the partner.
}

% 2016-12-30 : p 196 (half)

At around this age, children will also spontaneously initiate joint actions.
\citet{Warneken:2006qe} created situations in which an adult clearly needed help.
For example, the adult dropped a clothespin and attempted, unsuccessfully to reach it.
Or an adult approached and gently bumped into a cabinet with a pile of papers occupying both hands, so that she could not open it to place the papers in it.
When this happened, eighteen-month-olds, who were positioned merely as bystanders, would spontaneously get up and provide help.
They would, for example, pick up the clothespin and hand it over, or open the cupboard for the papers to be placed.
(If you have access to \citealp{Warneken:2006bs}, you can see some of the videos for yourself in the Supplementary Materials.)
While \citeauthor{Warneken:2006qe}’s concern was primarily helping, many of the infants’ interventions were clearly joint actions.
Before they have even learnt to walk properly, humans will spontaneously initiate joint actions.

Joint action in the second year of life typically begins with extremely limited coordination on the part of children.
This is neatly illustrated by \citep{warneken:2007_helping}, who created an ‘Elevator’ task (see \vref{fig:warneken_2007_fig2a}).
To obtain an object hidden in the ‘Elevator’, infants had to work together with an adult.
One person was tasked with pushing up a cylinder from underneath a bench so that the other person could then remove the object from the cylinder above the bench.
When their role was simply to take the object (‘Role A’), infants were coordinated even at 14 months of age.
This suggests that they can comprehend others’ actions and perform complementary actions.
But when the infants’ role was to push up the cylinder (‘Role B’),  joint actions with the 14-month-olds nearly all failed and the 18-month-olds were uncoordinated with the adults.
Only the two-year-olds could push up the cylinder and hold it up while the adult retrieved the object (see \vref{fig:warneken_2007_fig3a}).
Coordination of action in joint action improves substantially over the second year of life.
In fact coordination of action continues to improve gradually over several years, as \citet{endedijk:2015_development} demonstrate in their observations of spontaneous coordination in drumming together by two-, three- and four-year-olds.

\addFigure{warneken_2007_fig2a}{%
To free an object, one person needs to hold the cylinder up from underneath the bench while another removes the object from the hole above the bench.
Source: \citet[figure 2 (part)]{warneken:2007_helping}}

\addFigure{warneken_2007_fig3a}{%
Coordination for joint action improves substantially over the second year of life.
Source: \citet[figure 3 (part)]{warneken:2007_helping}}


We started this section by considering an inconsistent triad of claims (\vpageref{sec:joint-action-first-years-of-life}).
% Which of the claims from the inconsistent triad at the start of this section should we reject?
Which claim should we reject?
It is clear that we cannot reject the first claim, according to which one- and two-year-olds are capable of performing joint actions.
After all, we have just considered plenty of evidence for joint action in the second and third years of life.
This leaves the second and third claims as candidates for rejection.

Some researchers suggest rejecting the third claim:
they deny that a function of shared intention is to coordinate two or more agents’ plans \citep[for example,][]{Tollefsen:2005vh,pacherie:2013_lite}.
This involves rejecting Bratman’s account of joint action (see \cref{sec:Bratman-shared-intention}) and providing an alternative account, one on which shared intention and coordinated planning are not directly related.
If we accepted one of these alternative accounts of shared intention, the claim that one- and two-year-olds have shared intentions would not generate the incorrect prediction that such young children  are capable of coordinating their plans with others’.
This strategy---rejecting the third claim and defending an alternative account of shared intention---is the right one to adopt if you are optimistic that it is possible to give a uniform account of joint action, one that encompasses everything from joint action in infancy to the most sophisticated forms of joint action in adulthood.

Lacking such optimism, I shall pursue a different approach in what follows.
I propose that we reject the second claim, according to which all joint action involves shared intention.
% The view I shall develop is analogous to one on which the category of purposive action is heterogeneous, containing both actions involving intentions and actions in which intention plays no role.
%Likewise, o
% On the view I shall develop, there are different kinds of joint action, some of which involve shared intention and some of which do not.
My hope is that characterising a form of joint action that does not involve shared intention will enable us to better understand joint actions as performed by one- and two-year-olds.


% Brownell suggests that:
% \begin{quote}
% ‘all sorts of joint activity is possible without \ldots\  complex reasoning \ldots\
% % both in other species and in our own joint behavior as adults, some of which occurs outside of reflective awareness \ldots\
% In studying its development in children the problem is how to characterize and differentiate primitive, lower levels of joint action operationally from more complex and cognitively sophisticated forms’
% \citep[p.~195]{brownell:2011_early}.
% \end{quote}
% %
% This seems like a problem where philosophical distinctions could be useful.


\section{Collective Goals vs Shared Intentions}
\label{sec:collective-goals-vs-shared-intentions}
Brownell neatly describes the problem we face in characterising joint actions as performed by one- and two-year-olds:
\begin{quote}
‘all sorts of joint activity is possible without \ldots\ complex reasoning \ldots\
% both in other species and in our own joint behavior as adults, some of which occurs outside of reflective awareness \ldots\
In studying its development in children the problem is how to characterize and differentiate primitive, lower levels of joint action operationally from more complex and cognitively sophisticated forms’
\citep[p.~195]{brownell:2011_early}.
\end{quote}
%
The evidence we have explored so far in this chapter supports her description of the problem.
One and two-year-olds perform joint actions (see \cref{sec:joint-action-first-years-of-life}).
But, as we saw in \cref{sec:coordinating-planning}, it seems they are several years away from  coordinating their plans with others’.
This indicates that Bratman’s account of joint action (see \cref{sec:Bratman-shared-intention}) does not characterise the joint actions infants perform in the second and third years of life.
An alternative is needed.

What should we look for in the alternative?
Minimally, it needs to distinguish the kind of joint actions one- and two-year-olds can perform from parallel but merely individual actions (see \cref{sec:joint-vs-parallel-action}).
And it would ideally be consistent with the \gls{Joint Action Conjecture} (see \cref{sec:inconsistent-triad-joint-action}).%
\footnote{%
In the passage just cited, Brownell identifies the need for an operational characterisation of joint action.
What follows is merely an attempt to provide a theoretical characterisation, albeit one simple enough to support the further work involved in providing an operational characterisation.
}

As a first step, it is useful to have a distinction between collective and distributive interpretations of predicates
(see \citealp{Linnebo:2005ig} for details).
Imagine Ayesha takes Zach’s glass and holds it up while Beatrice pours prosecco.
Unfortunately the prosecco misses the glass and soaks Zach’s trousers.
Here are two sentences, both true:
%
\begin{quote}
The tiny drops fell from the bottle.
\end{quote}
%
\begin{quote}
The tiny drops soaked Zach’s trousers.
\end{quote}
%
The first sentence is naturally read {distributively}. It specifies something that each drop did individually.
But the second sentence is naturally read {collectively}.
No one drop soaked Zach’s trousers; by itself, a lone drop would have no noticeable effect on his trousers at all.
Rather the soaking was something that the drops
accomplished collectively.
If the sentence is true on this reading, the tiny drops' soaking Zach’s trousers is not a matter of each drop soaking Zach’s trousers.

Compare a sentence involving actions and their outcomes:
\begin{quote}
Their thoughtless actions soaked Zach’s trousers.
\end{quote}
Taken in isolation, this sentence could be read in two ways, distributively or collectively.
We could imagine that we are talking about a sequence of actions done over a period of time, each of which soaked Zach’s trousers.
In this case the outcome, soaking Zach’s trousers, is an outcome of each of the actions.
Alternatively, we could imagine several actions which have this outcome collectively.
And this is the natural way of interpreting the sentence given that we are imagining  Ayesha holding a glass while Beatrice pours.
In this case the outcome, soaking Zach’s trousers, is not necessarily an outcome of any of the individual actions.
But it is an outcome of all of them taken together.%
\footnote{%
The collective and distributive readings are not the only possibilities, but they are the only ones that need concern us here.
% (Here I'm ignoring complications associated with the possibility that some of the actions collectively soaked Zach’s trousers while others did so distributively.)
}

Note that the distinction between distributive and collective readings involves a genuine ambiguity.
To see this, ask yourself how many times Zach’s trousers were soaked.
On the distributive reading of the sentence just above, they were soaked at least as many times as there are actions.
On the collective reading they were not necessarily soaked more than once.
% (On the distributive reading there are several outcomes of the same type and each action has a different token outcome of this type; on the collective reading there is a single token outcome which is the outcome of two or more actions.)
So whether two or more actions involving multiple \glspl{agent} have an outcome distributively or collectively is not just a matter of words: the difference concerns how the actions and outcomes are related.

Consider one last sentence:
\begin{quote}
The goal of their actions was to fill Zach’s glass. \end{quote}
Whereas the previous sentence was causal, this sentence is teleological.
Whereas the previous sentence concerned an outcome which is an actual consequence of some actions, this one concerns an outcome to which actions are merely directed.
Like the previous sentence, this sentence has both distributive and collective interpretations.
On the distributive interpretation, each of their actions was directed to an outcome, namely filling Zach’s glass.
So there were as many attempts (successful or not) to fill his glass as there were actions.
On the collective reading, by contrast, it is not necessary that any of the actions considered individually was directed to this outcome.
Rather the actions were collectively directed to this outcome.

Where two or more actions are collectively directed to an outcome, let us say that this
outcome is a \emph{\gls{collective goal}} of the actions.
Note two things.
First, a collective goal is an outcome.
It is not a mental state or a representation.
Second, this definition involves no assumptions about the intentions or other mental states of the \glspl{agent}.
% It is the actions rather than the agents which have a collective goal.
This notion of a collective goal is a narrowly logical one, not a psychological one.

As Ayesha and Beatrice’s pouring prosecco illustrates,
some actions involving two or more people are purposive in the sense that among all their actual and possible consequences, there are outcomes to which they are directed and the actions are collectively directed to this outcome.
But in virtue of what could actions involving two or more people ever have a collective goal?

Despite the simplicity of this question, there is potential for confusion.
Others have used the term ‘collective goal’ for quite different notions.
For instance, \citet[p.~30--1]{tuomela:2002_collective} defines the notion in terms of ‘persons believing or collectively accepting that the goal state \ldots\ is to be collectively achieved.’
And \citet[p.~pp. 34]{gilbert:2014_book} uses the term in connection with cases where people ‘collectively espouse a certain goal, and each one is acting \ldots\ in light of the fact that the goal is their collective goal’.
% Relatedly \citet[p.~59]{miller_social_2001} uses the term ‘collective end’ for something that agents possess only when they mutually truly believe that they have it.
Terminologically, these uses of the term ‘collective’ are unfortunate because this term is widely and consistently used in just the way I am using it in discussions about the logic of plural quantification.
Substantially, the narrowly logical notion of collective goal introduced here is more basic than the other notions in this respect: if a collective goal in Tuomela’s or Gilbert’s or anyone else’s sense were to lead to some actions, these actions would have a collective goal in the narrowly logical sense too; but the converse is untrue.
The narrowly logical notion of a collective goal allows for a clear separation between a fact that stands in need of explanation (that some actions are collectively directed to an outcome) and the things which putatively explain it (such as collective espousal, according to Gilbert).

% This separation is useful in part because it allows us to distinguish what is common to Tuomela, Gilbert, Miller and  others.
% Each provides a candidate explanation of how two or more actions involving multiple agents could have a collective goal.%
% \footnote{
% We are not claiming that these authors aim to provide such a candidate explanation, only that they do.  The authors' actual aims are more ambitious, of course.
% }
% Although they differ on the details, in each case the explanation involves shared intention, belief, knowledge or commitment.
% Our aim in what follows is to show that a full explanation of how two or more actions involving multiple agents could have a collective goal will need to invoke, in addition, a certain interagential structure of motor representations.

So in virtue of what could actions involving two or more people ever have a collective goal?
To illustrate, Ayesha might say, truthfully, ‘The collective goal of our actions was not to soak Zach's trousers in sparkling wine but only to fill this glass.’
What could make Ayesha’s statement true?

One way to answer the question is by invoking a notion of shared intention.
Suppose Ayesha and Beatrice have a shared intention that they fill the glass.
Then, on many accounts of shared intention including Bratman’s (see \cref{sec:Bratman-shared-intention}),
the shared intention involves each of them intending that they, Ayesha and Beatrice, fill the glass;
or each of them being in some other state which picks out this outcome.
The shared intention also provides for the coordination of their actions (so that, for example,
Beatrice doesn't start pouring until Ayesha is holding the glass under the bottle).
And coordination of this type would normally facilitate occurrences of the type of outcome intended.
In this way, invoking a notion of shared intention potentially explains what it is in virtue of which actions involving two or more people could have a collective goal.

But invoking shared intention is exactly what we want to avoid (see \cref{sec:joint-action-first-years-of-life}).
Are there also ways of answering the question which involve psychological structures other than shared intention?
%
%
% Suppose Ayesha’s and Beatrice’s are coordinated around the outcome of filling Zach’s glass.
% That is, their actions are coordinated and coordination of this type would normally bring about an outcome of this type.
% This is one sense in which their actions can be directed to this outcome.
% And because their actions being directed to this outcome is (in part, at least) a consequence of their coordination,
% it is not just that each of their actions is individually directed to the outcome.
% So the outcome is one to which their actions are collectively directed.
% That is, it is a collective goal of their actions.
%
% In general, where some actions are coordinated around an outcome, the actions are collectively directed to that outcome in virtue of being so coordinated.
% The fact that two people’s actions have a collective goal therefore implies very little about their states of mind.
% The coordination needed for multiple individuals’ behaviours to have a collective goal might even be provided by entirely non-psychological mechanisms,
% as popular findings about bees \citep{seeley2010honeybee} and ants
% \citep[pp.~178-83, 206-21]{hoelldobler2009superorganism} suggest.
% It is also likely that much coordination between adult humans involves relatively primitive psychological mechanisms
% (see, for example, \citealp{repp:2013_sensorimotor}, §3, and
% \citealp{Knoblich:2010fk} on emergent coordination).
% This suggests that the notion of a collective goal may turn out to be a useful starting point for understanding joint action in the first years of life.

One possibility is to invoke expectations about collective goals.


\section{Expectations about Collective Goals}
\label{sec:expectations-about-collective-goals}

Suppose that Ayesha and Beatrice each reasonably expect the actions they are about to perform to have the collective goal of filling Zach’s glass, and that they act on these expectations.
Then their expectations specify a collective goal and provide for the coordination of their actions in much the way that a shared intention would.
After all, someone who acts on an expectation that her actions and those of another will have a collective goal thereby has an incentive to coordinate her actions with the other’s insofar as this will facilitate the occurrence of the collective goal.
Apparently, then, an outcome could be a collective goal of Ayesha and Beatrice’s actions in virtue of them each acting on the expectation that this outcome will be a collective goal of their actions.

This suggests a minimally demanding sufficient condition for joint action:
%
\begin{quote}
Where an event comprises two or more \glspl{agent}’ actions and the actions have a collective goal in virtue of the agents’ acting on expectations that these actions will have this collective goal, the event is a joint action.
\end{quote}
%
This sufficient condition enables us to distinguish paradigm contrasts between joint action and parallel but merely individual actions (see \cref{sec:joint-vs-parallel-action}).
For instance, earlier we contrasted two friends out walking together in the way friends typically walk together with two strangers who happen to be walking side by side.
The friends but not the strangers meet the above sufficient condition for joint action: they are plausibly acting on an expectation that their walking together will be a collective goal of their actions.
We also contrasted several performers simultaneously running to a central shelter in order to perform a dance with an event involving park visitors running to the central shelter in order to escape a storm.
Again, the strangers do not meet the above sufficient condition whereas the performers plausibly do---plausibly the performers are each running to the shelter on the expectation that this action together and the others’ actions have the collective goal of performing the dance.%
\footnote{%
There are some tricky cases in which different performers have expectations concerning slightly different actions.
Since we are considering a merely sufficient condition for joint action, we are not required to resolve these.
}

Could one- and two-year-olds meet the proposed sufficient condition for joint action?
A collective goal is merely an outcome to which some actions are directed.
The goal-tracking abilities infants manifest in the first year of life show that they can form expectations about actions being directed to goals (see \cref{cha:action}).
It is only a small step to the further conjecture that infants can track goals to which two or more agents’ actions are collectively directed.
And there is even some evidence in support of this conjecture  \citep{fawcett:2012_observation}.
% \citep{fawcett:2012_observation}: '18-month-olds shown either a joint action or an individual action; they imitate joint actions when they observe joint actions
% \citep{warneken:2012_collaborative}: children attempt to reengage a partner even when her contribution is not necessary. They write that their results show that ‘young children are not only able to engage in joint activities with others, but they view these interactions as genuinely collaborative activities with joint goals and intentions.’ But I’m not sure how they show this.
So although we cannot be certain, it is reasonable to assume that one- and two-year-olds can form and act on expectations about collective goals.
And this is all meeting the proposed sufficient condition for joint action requires.

It may be objected that the proposed sufficient condition for joint action is not actually sufficient at all and that an additional requirement is needed, namely that the agents have common knowledge concerning their expectations.
This might make the claim that one- and two-year-olds can meet the sufficient condition for joint action less plausible.
It is one thing to have an expectation concerning a collective goal of some actions and quite another to know that someone has such an expectation.
Further, if infants’ expectations concerning collective goals involve motor and perceptual representations rather than knowledge states (see \cref{cha:theory-goal-tracking}), it might turn out that they are not in a position to have knowledge concerning these expectations.
It is therefore worth asking why the requirement on common knowledge needs to be added to the proposed sufficient condition for joint action.
Philosophers who appeal to common knowledge in characterising joint action have been strikingly reticent about why they do so.%
\footnote{%
For instance, \citet[p.~57]{bratman:2014_book} defends including a requirement on common knowledge by asserting that
‘public access to the shared intention will normally be involved in further thought that is characteristic of shared intention, as when we plan together how to carry out our shared intention.’
This assertion appears to justify the clam that common knowledge accompanies shared intention rather than the claim that common knowledge is an essential feature of shared intention.
}
And some have argued that common knowledge is not in fact necessary at all \citep[see][]{blomberg:2015_common}.
As things stand, then, there does not appear to be a compelling argument to show that common knowledge is needed in giving sufficient conditions for joint action.
Rather than being an essential feature of joint action, various kinds of common knowledge may be optional coordination enhancers.

A similar line of thought applies to the claim that commitment is an essential characteristic of all joint action \citep[][]{gilbert:2014_book}.
Children appear to first become sensitive to commitments in the context of joint action at around three years of age \citep{Grafenhain:2010zl}.
So adding requirements on commitments to a sufficient condition for joint action may well have the consequence that children in the first years of life cannot meet the sufficient condition.
But commitment, like common knowledge, is plausibly an optional enhancement rather than an essential feature of all joint action \citep[compare][pp.~118--20]{bratman:2014_book}.

Overall, then, it is not impossible that the proposed sufficient condition for joint action will turn out to be actually sufficient.
At least we do not currently have reason to assume that it needs supplementing by an appeal to common knowledge or commitment, thereby rendering it unsuitable for understanding one- and two-year-olds’ joint actions.
Let us therefore assume that the proposed sufficient condition for joint action is actually sufficient for joint action, and that it is one that one- and two-year-olds could meet.
Now consider a further question.
Does the proposed sufficient condition characterise joint actions as performed by children in the first years of life?

One point in favour of the proposed sufficient condition  is that it appears to be consistent with the evidence on joint action in the first years of life considered in \cref{sec:joint-action-first-years-of-life}.
As we saw, one-year-olds can spontaneously initiate joint actions.
If we think of joint actions as characterised by expectations about collective goals, imitating a joint action is a matter of doing things which create such expectations.
And children may initially do things which create expectations about collective goals in themselves and in others without necessarily being aware of doing so.
One-year-olds also attempt to reengage others in a joint action following an interruption.
This is consistent with their having expectations about collective goals.
A child with an expectation that hers and another’s actions will have a collective goal
may also realise that the collective goal is unlikely to be realised without the other’s actions, which would give her an incentive to reengage the other.
Overall, then, the evidence currently available on joint action in the first years of life is consistent with the claim that such joint actions are characterised by expectations about collective goals.

This argument should give us little confidence that joint actions as performed by children in the first years of life really can be characterised simply by expectations about collective goals.
The recent history of developmental psychology is all about how using more sensitive measures such as anticipatory looking or pupil dilation reveals that infant cognition is unexpectedly sophisticated
(see \cref{cha:action} for examples).
A more cautious claim would be that joint action in the first years of life is partially characterised by expectations about collective goals.
This claim is at least a reasonable starting point in attempting, over the remaining chapters, to understand how joint action may play a role in explaining the emergence of knowledge in development.



\section{Conclusion}
Our aim in this chapter was to understand joint actions as performed by children in the first years of life.
Such joint actions are a matter of two or more agents each acting on expectations that all of their actions will have a certain \gls{collective goal}.
This characterisation of the kind of joint action one- and two-year-olds perform provides some of the groundwork for investigating the possibility that, as the the \gls{Joint Action Conjecture} states,  abilities to perform joint actions play a role in explaining the developmental emergence of knowledge.

How did we arrive at the idea that joint action in the first years of life is characterised by expectations about collective goals?
We started by considering Bratman’s account of joint action (in \cref{sec:Bratman-shared-intention}).
This is the best developed and most influential account, and several developmental scientists have argued that it correctly characterises joint actions as performed by children in the first years of life.
But if they are right, we must reject the \gls{Joint Action Conjecture}.
For, as we saw (in \cref{sec:inconsistent-triad-joint-action}), the following claims form an inconsistent triad:
%
\inconsistentTriad
%
The inconsistency arises because
abilities to coordinate plans with another presuppose abilities to know things about others’ minds (see \cref{sec:inconsistent-triad-joint-action}).
Claims \#2 and \#3 therefore entail that invoking abilities to perform joint actions would be to presuppose that infants already have knowledge of minds, forcing us to reject claim \#1.

Recognising that the triad of claims is inconsistent led us to ask which claim should be rejected.
The view that Bratman’s account of joint action characterises joint actions as performed by children in the first years of life leads to the prediction that such children should be capable of coordinating their plans with others’.
But this prediction is incorrect:
the first signs of coordinated planning emerge much later in development, at around five years of age (see \cref{sec:coordinating-planning}).
We must therefore reject the view that joint action in the first years of life is characterised by Bratman’s account of shared intention.
Consequently, if we maintained both claims \#2 and \#3 of the inconsistent triad just above, then we would have to deny that children in the second and third year of life are capable of joint action.

But this we should probably not do.
A variety of evidence indicates that although they have quite limited capacities to coordinate their actions with others, even fourteen-month-olds will spontaneously initiate joint action with an adult.
Children of around this age also demonstrate awareness in the context of joint action that success requires another person’s contribution (see \cref{sec:joint-action-first-years-of-life}).
On balance, then, it does not seem plausible to deny that  children in the second and third year of life are capable of joint action.
We must therefore reject at least one of claims \#2 and \#3 of the inconsistent triad.
Doing so requires us to find an alternative to the view that Bratman’s account characterises joint action in the first years of life, one consistent with children’s capacities and their limits.

To this end I introduced the notion of a collective goal.
A \gls{collective goal} is any outcome to which the actions of two or more agents are collectively directed.
And for their actions to be {collectively} directed to an outcome is for their actions to be directed to an outcome where this is not, or not only, a matter of each action being individually directed to that outcome.
Unlike the deeply controversial and hard-to-ground notion of shared intention, the notion of a collective goal is a narrowly logical one (see \cref{sec:collective-goals-vs-shared-intentions}).
The step from the logical to the psychological
% or pheromonal, or behavioural
occurs when we ask in virtue of what some agents’ actions could have a collective goal.
One possibility is that the agents each act on an expectation that all of their actions will have a collective goal.
This suggested a minimally demanding sufficient condition for joint action involving agents acting on expectations about collective goals  (see \cref{sec:expectations-about-collective-goals}).
This sufficient condition in turn provides a candidate characterisation of joint action in the first years of life, one capable of distinguishing paradigmatic contrasts between joint actions and parallel but merely individual actions.
Perhaps one- and two-year-old’s joint actions are actions they perform in acting on expectations about collective goals.

If this is right, we should question not only \citet{carpenter:2009_howjoint}’s view that joint action in the first years of life is correctly characterised by Bratman’s account of shared intention but also broader appeals to shared intentionality.
\citet[p~124]{Tomasello:2007gl} propose that a capacity for ‘shared intentionality’ emerges ‘at around the first birthday’ from ‘a uniquely human line of development for sharing psychological states with others.’
Reflection on one- and two-year-olds’ joint actions suggests that we may not need theoretically disputed and hard-to-pin-down notions of sharing in order to characterise what they are doing.
Perhaps it is not shared intentionality, whatever exactly that turns out to be, but a set of minimally complex abilities, including abilities to form and act on expectations about collective goals, that matters for understanding cognitive development from the second and third years of life.

% \begin{quote}
% ‘the basic skills and motivations for shared intentionality typically emerge at around the first birthday from the interaction of two developmental trajectories, each representing an evolutionary adaptation from some
% different point in time.
% The first trajectory is a general primate (or perhaps great ape) line of development for understanding intentional action
% and perception, which evolved in the context of primates’ crucially important competitive interactions with one another over food, mates, and other resources (Machiavellian intelligence; Byrne \& Whiten, 1988).
% The second trajectory is a uniquely human line of development for sharing psychological states with others, which seems to be present in nascent form from very early in human ontogeny as infants share emotional states with others in turn-taking sequences (Trevarthen, 1979).
% The interaction of these two lines of development creates, at around 1 year of age, skills and motivations for sharing psychological states with others in fairly local social interactions, and then later skills and motivations for reacting to and even internalizing various kinds of social norms, collective beliefs, and cultural institutions’
% \citep[p~124]{Tomasello:2007gl}.
% \end{quote}

This characterisation of joint action, unlike Bratman’s, is consistent with the \gls{Joint Action Conjecture}.
This is because Bratman’s requires coordinating plans, which requires knowledge of other’s mental states.
By contrast, characterising joint action by appeal to collective goals places no demands on knowledge and so opens the way to 
 explaining the developmental emergence of knowledge, including knowledge of others’ minds by appeal to one- and two-year-olds’ abilities to perform joint actions.
% In the next chapter we will begin to address this question by investigating how joint action may explain the developmental emergence of referential communication.
% My plan is to show that joint action matters for nonlinguistic referential communication, which in turn matters for linguistic communication, and that linguistic communication in turn matters for the emergence of knowledge.



%%% Local Variables:
%%% TeX-master: "master"
%%% End:
