%!TEX root = master.tex


% ∞TODO : be sure that the puzzle concerns an interaction, not an age gap.  Make sure this is clear from the definition of an A-task!

% TODO : include material on replication including the special issue of Cognitive Science (?) -- papers already in Zotero

% mention Buttelmann and Kovacs, 14‐Month‐olds anticipate others’ actions based on their belief about an object's identity

% Also some stuff from JAson and KAtheryn

\chapter{Mind: the Puzzle}
\label{cha:mind}

% \section{Knowledge of Minds}


How do humans first come to know facts about others’ mental states?
How, for instance, do they come to know that Ayesha believes, falsely, that she and Beatrice will still be able to catch a bus home even if they delay leaving the party?
By the end of this chapter you should understand the facts that make answering this question difficult;
you should have a sense of how far away we all are from being able to answer it definitively; and
you should be familiar with some attempts to answer it.
You should also be able to evaluate novel answers to the question about how humans first come to know facts about others’ mental states.

The focus of this chapter is what is often called mindreading or using a theory of mind.
\emph{Mindreading} is the process of identifying a mental state as a mental state that some particular individual, another or yourself, has.
To say someone has a \emph{theory of mind} is another way of saying that she is capable of mindreading.%
\footnote{%
According to an influential definition offered by \citet[p.~515]{premack_does_1978},
for an individual to have a theory of mind its for her to ‘impute mental states to himself \emph{and} to others’ (my italics).
I have slightly relaxed their definition by changing their ‘and’ to ‘or’ in order to allow for the possibility that there are mindreaders who can identify others’ but not their own mental states.
}
As is clear from these definitions, having a theory of mind does not necessarily involve having a theory.
Similarly, talking about mindreading does not commit one to the idea that mental states can literally be read, nor to magical thinking about knowledge of others’ minds.
Despite their inappropriate connotations, we are stuck with the terms ‘theory of mind’ and ‘mindreading’ because they are so widely used.

Most research on mindreading concerns belief rather than, say, desire or intention.
This is partly because it was quite widely held that the mark of a mindreader was an ability to represent false beliefs \citep[see][]{Perner:1991xh}.
Without being committed to this claim, I shall also focus on belief and belief-like states in this chapter.
This is a practical matter: even if you don’t think there is anything special about understanding belief as opposed to understanding other mental states,
understanding the research on belief is necessary for thinking about how humans come to know facts about mental states generally.


% \section{What Are Mental States?}
% \label{sec:what-are-mental}

% [*Move to chapter 6?]

% In defining mindreading as the process of identifying a mental state as the mental state of some particular individual, we are presupposing a notion of mental state.
% Since there is longstanding controversy about which states are mental, there is a risk of parallel controversy about whether mindreading occurs.
% This is an obstacle to progress insofar as we can hardly be expected to resolve centuries of controversy about the nature of the mental before we can investigate mindreading.
% What can we do?

% One strategy, which I will adopt in what follows, is to start with a very broad conception of which states are mental.
% This makes sense given that we are interested in the origins, in development, of knowledge of minds.
% The aim will be to identify a plausible starting point for knowledge of minds, leaving open the question of whether what we identify will turn out to be something more primitive than knowledge of minds relative to some alternative, narrower conception of which states are mental.
% Given that our interest is in the origins of knowledge, controversy about which states are mental matters primarily as a source of distinctions which may be relevant to understanding how knowledge of minds emerges in development.

% *Intuitive conception of a mental state: subject, attitude, content

% *Theory of the mental: theory in which there are states with subject, attitude, content

% *Mental state: state identified by any such theory.

% *Model of minds and actions: not a theory but can be specified by a theory

% *Identify mental states: use of a model of minds and actions to track mental states.


% Creation myths with a twist: we are the creators and so can be shaped by our own stories.
% *Even a creator’s own creation myth is likely to serve many purposes in addition to that of describing the facts.




\section{All About Maxi}
\label{sec:all-about-maxi}
\citet{Wimmer:1983dz} set out to determine when humans can know facts about others’ beliefs.
They told children a story like this:
%
\begin{quote}
‘Maxi puts his chocolate in the BLUE box and leaves the room to play. While he is away (and cannot see), his mother moves the chocolate from the BLUE box to the GREEN box. Later Maxi returns. He wants his chocolate.’
\end{quote}
%
They then asked the children, ‘Where will Maxi look for his chocolate?’

The results are amazing (see \cref{fig:wimmer_1983_figX}).
At least, they are amazing to most adults who don’t already know the results of this experiment and have at least a vague sense of what four- and five-year-olds are like.
They tend to find it amazing that children under around three or four years of age systematically give the wrong answer, and that children do not typically reach adult levels of performance until at least four or five years of age.
%
\begin{figure}
\begin{center}
\includegraphics[width=0.9\textwidth]{fig/wimmer_1983_figX.png}
\caption{
  \label{fig:wimmer_1983_figX}
  The results of \citet{Wimmer:1983dz}’s false belief task.
  Source: drawn from \citet{Wimmer:1983dz}
}
\end{center}
\end{figure}
%

Given how amazing the results are, you might be tempted to think that something is wrong with the experiment.
Could it be that children are confused by the story or the question, or find it tricky for reasons that have nothing to do with their understanding of mental states?
It turns out that children’s performance is consistent across many variations of the false belief task.
Instead of asking them to predict Maxi’s action, you can ask them to predict his desire \citep{Astington:1991kk}, or you can show them how he acted and ask them to retrodict his belief \citep[for example,][]{Wimmer:1998kx}.
Or you can use tasks which actively involve children, testing their propensities to perform deceptive actions \citep[for example,][]{Chandler:1989qa} or to lie \citep[for example,][]{Polak:1999xr,talwar:2007_lying}.
You can give children nonverbal false belief tasks which involve choosing something based on another’s false belief
\citep[for example,][]{Call:1999co,krachun:2010_new} or placing a bet \citep{Ruffman:2001ng}.
You can even ask children about their own (past) false beliefs rather than somebody else’s \citep{Gopnik:1988pa}.
Whichever of these ways you test children’s understanding of false belief, you will get basically the same results Wimmer and Perner got.

In fact, Wimmer and Perner’s finding is one of the most replicated in developmental psychology.
A meta-analysis of 178 studies confirms their basic finding:
sometime around their fourth birthday, children go from systematically acting and talking as if false belief were an impossibility to being able to recognise false beliefs \citep{Wellman:2001lz}.

This is not to say that every experiment gives exactly the same results.
If you look at the results of different studies charted in \vref{fig:wellman_2001_fig2a} you can see that there is plenty of variation.
Look at what happens when children around 40 months of age are tested, for example.
Some studies find that nearly all children of this age give incorrect answers to a false belief task, whereas other studies find that nearly all children of this age give correct answers.
If you looked at one or two studies in isolation, you might think that they provide contradictory findings.
But in fact there is no contradiction.
The different studies used different versions of the false belief task.
And different versions include different combinations of features which made the task easier or harder.
For example, changing the question from ‘Where will Maxi look for his chocolate?’ to ‘Where will Maxi look \emph{first} for his chocolate?’ can improve performance \citep{siegal:1991_where}.
Similarly, changing the task so that it involves an element of deception, or so that children are more involved because they interact with Maxi (or the other protagonist), makes the task a bit easier too.
But these changes make the task easier for children of all ages.
They do not change Wimmer and Perner’s basic finding that there is a transition from systematically incorrect to systematically correct performance in typically developing children.%
\footnote{%
The restriction to {typically developing} children is necessary because there is evidence that some children either do not make this transition or do so significantly later in childhood.
Such children include individuals on the autistic spectrum and deaf children born to hearing parents \citep{peterson:2000_insights}.
}
This is what the line of best fit in \vref{fig:wellman_2001_fig2a} indicates.

%
\begin{figure}
\begin{center}
\includegraphics[width=0.9\textwidth]{fig/wellman_2001_fig2a.png}
\caption{
  \label{fig:wellman_2001_fig2a}
  How humans performed on 178 false belief tasks.
  Source: \citet[][figure 2A]{Wellman:2001lz}
}
\end{center}
\end{figure}
%


The ages at which typically developing children first pass false belief tasks can vary quite a bit.
There are differences between children in different countries, as \vref{fig:wellman_2001_fig7} shows.
Several components of a child’s linguistic and communicative abilities also appear to affect when she will first pass a false belief task \citep{milligan:2007_language,kovacs:2009_early}.
Also children who are better at inhibiting natural tendencies, switching tasks and holding things in working memory tend to pass false belief tasks at a younger age than children who are otherwise similar \citep{devine:2014_relations}.
Another predictor of when children first pass false belief tasks is the quality of their social interactions \citep{Hughes:2006fu}.
% for example, (Hughes and Leekam 2004: 607): "broad and unqualified claims for a link between theory of mind and social relations are not justified. Instead, the evidence suggests (i) a contrast between acquisition and application of theory-of-mind skills; (ii) the variety of social life (rather than any single social factor) is what matters for theory-of-mind development."
% 		Also (Hughes, Fujisawa, et al. 2006: 55): "it is the quality (rather than the simple presence) of the sibling relationship that matters."
Children with mothers who more frequently give explanations involving mental states or more frequently talk about minds tend to pass false belief tasks at a younger age \citep{peterson:2003_opening,moeller:2006_relations},
%\citet[p.~760]{moeller:2006_relations}: ‘Our results provide support for the concept that access to conversations about the mind is important for deaf children’s ToM development, in that there was a significant relationship between maternal talk about mental states and deaf children’s performance on verbal ToM tasks.’
and children with a variety of siblings (but not twins) also tend to pass at a younger age than otherwise similar children \citep{ruffman:1998_older,cassidy:2005_theory}.
It is even possible to get children to pass false belief tasks earlier than they would otherwise by giving them training relevant to understanding and reasoning about mental states \citep[for example,][]{Slaughter:1996fv,lohmann:2003_role,wellman:2013_deafness}.
%
\begin{figure}
\begin{center}
\includegraphics[width=0.8\textwidth]{fig/wellman_2001_fig7.png}
\caption{
  \label{fig:wellman_2001_fig7}
  Performance of children in different countries on false belief tasks.
  Source: \citet[][figure 7]{Wellman:2001lz}
}
\end{center}
\end{figure}
%

At this point we appear to be in a good position to start figuring out how humans first come to know facts about beliefs and other mental states.
The research just mentioned tells us that a variety of cognitive, linguistic and social factors are associated with success on false belief tasks.
Deeper exploration might allows us to disentangle correlation from causation,
perhaps leading us to the view that humans first come to know facts about mental states in roughly the way that they first come to know facts about what written words mean: they learn about mental states like belief through social interaction.
(\citet{heyes:2014_cultural} develop a version of this view.)

But there is another body of evidence which appears to conflict dramatically with this line of thought.



\section{Infants track false beliefs}
\label{sec:infants-track-false}
Recall Wimmer and Perner’s experiment with the story about Maxi.
In this experiment children are asked to say where Maxi (who manifestly has a false belief) will look for his chocolate.
What happens if, before asking this question, the experimenter simply muses, as if thinking aloud, ‘I wonder where Maxi will look for his chocolate’?
When given such a prompt, children (and adults) tend to
look to the location where they expect something to happen.
Where children look in response to the prompt can therefore reveal what they expect to happen.
And when \citet{Clements:1994cw} measured children’s anticipatory looking,
they found that three year olds would look to where Maxi should go if he had a false belief.
(Actually they used a different scenario with a mouse and some cheese, but it has the same structure as the Maxi story.)
Yet when asked the question about where Maxi will go, these same children would systematically, and confidently, predict that Maxi will act as if he had a true belief \citep{Garnham:2001ql,Ruffman:2001ng}, as you can see in \vref{fig:clements_1994_fig1}.
%
% --- *redraw graph: just show 2;11-3;2 as bars (to highlight that there are conflicting predictions).
%
\begin{figure}
\begin{center}
\includegraphics[width=0.8\textwidth]{fig/clements_1994_fig1.png}
\caption{
  \label{fig:clements_1994_fig1}
  Anticipatory looking and verbal predictions provide apparently contradictory information about whether three year olds are tracking beliefs.
Source: \citet[figure 1]{Clements:1994cw}
}
\end{center}
\end{figure}
%

Why are these results exciting and puzzling?
It isn’t primarily about age—it isn’t merely that they show that three-year-olds can track false beliefs.
Rather the exciting feature is that for individual children observing a single scenario, anticipatory looking and verbal prediction give contradictory indications about what they expect to happen next.
Their anticipatory looking suggests they expect Maxi to act on a false belief and go one way,
whereas their verbal predictions suggest that they expect Maxi to act as if false belief were impossible and so to go the other way.
Since Maxi can’t go both ways, Clements and Perner’s three-year-olds appear to have contradictory expectations.


The case for accepting that eye movements reveal an ability to anticipate actions based on false beliefs is strengthened by evidence that both two-year-olds’ \citep{Southgate:2007js} and also adults’ \citep{Low:2012_identity} anticipatory looking shows a similar pattern.
So three-year-olds’ anticipatory looking is not a quirk:
individuals of all ages look as if they are anticipating actions based on false beliefs.

And it is not only in their anticipatory looking that children younger than three years of age manifest abilities to form expectations concerning actions based on false beliefs.
\citet{Onishi:2005hm} made a discovery that has transformed our understanding of how humans first come to know facts about minds.
They created a \gls{violation-of-expectation} version of the false belief task which enabled them to test fourteen-month-olds.
(The \gls{violation-of-expectation} method was explained in \cref{sec:segmentation}.)
As in Wimmer and Perner’s task, children observe a sequence of events in which a protagonist (like Maxi) manifestly acquires a false belief about the location of an object.
Of course, the young age of the subjects means that the sequence is acted out for them rather than told as a story.
And instead of asking their infants a question, Onishi and Baillargeon had their protagonist continue to the end of the story, either by reaching to the wrong location or by reaching to the correct location.
If infants expect the protagonist to act in accordance with her false belief, then they should look longer when the protagonist manifestly has a false belief but nevertheless acts as if she had a true belief.
And this is exactly what the infants did.
As you can see in \vref{fig:onishi_fig}, fourteen-month-olds show the converse pattern of looking when the protagonist has a true belief.
%
\addFigure{onishi_fig}{Fourteen-month-olds expect people to act in line with their beliefs, whether true or false.
Source: drawn from \citet{Onishi:2005hm}}

Following Onishi and Baillargeon’s groundbreaking experiment, many have replicated and extended their findings.
You can change the scenario so that it involves false belief about the contents of a container rather than about the location of an object \citep{he:2011_false},
or so that it involves relating beliefs to expressions of emotion rather than to instrumental actions \citep{scott:2017_surprise},
or so that it involves inferring belief from verbal communication \citep{scott:2012_verbal_fb} or from action \citep{traeuble_early_2010}.
Despite these quite radical changes to the tasks, essentially the same results are found in each case:
children in their second or third year of life can form different expectations depending on what another believes.
Taking these results together with experiments measuring anticipatory looking, we have evidence that even one-year-olds can form expectations about actions based on false beliefs.

Individual two- and three-year-old children observing a single scenario involving a protagonist like Maxi acting on a false belief appear to have inconsistent expectations concerning what the protagonist will do.
Whereas some responses robustly indicate that the child expects the protagonist to act as if false belief were impossible (see \cref{sec:all-about-maxi}), other responses no less robustly indicate that the child expects the protagonist to act in accordance with her false belief (as we have seen in this section).
I shall eventually suggest that these apparently contradictory responses give rise to a deep and difficult puzzle about the nature of mindreading.
But before I say what the puzzle is, let us first be sure that its foundations are solid.
Do two- and three-year-olds  really make contradictory responses to scenarios involving false beliefs?
% OLD: These contradictory responses give rise to a puzzle about the nature of mindreading.


\section{A Replication Challenge}
\label{sec:replication-challenge}

One challenge concerns whether studies of infant false belief tracking can be replicated,
and indeed whether corresponding studies with adults replicate.
A surprising number of findings have turned out to be inexplicably hard to replicate, while other findings have been replicated \citep[for example,][]{kulke:2017_replication,kulke:2018_implicit,kulke:2018_implicita,powell:2017_replications,crivello:2017_infants,dorrenberg:2018_how}.
Even more confusingly, some findings have been both successfully and unsuccessfully replicated (for example, see \citealp{kulke:2018_implicit} on \citealp{Southgate:2007js}).

Attempts to explain why some findings can be replicated and others not (and why some findings sometimes but not always replicate) are currently quite varied.
Some argue that particular failed replications are due to methodological differences (\citealp[for example,][]{buttelmann:_interpreting}; see \citealp{poulin-dubois:2018_infants} for responses).
Others suggest that particular measures, most prominently anticipatory looking, may not be reliable indicators of belief tracking at all \citep[p.~14]{kulke:2019_why}.
But I think most would agree that any systematic explanation for the pattern of successes and failures in replication attempts is some way off.

Despite the uncertainty, two challenges seem particularly important.
First, when various tasks are supposed to measure a single ability, we would normally expect to find signs of convergence in performance across the tasks: that is, those and only those subjects who pass one of these tasks will tend to pass other tasks.
\citet[][p.~2]{kulke:2017_replication} observe that whereas performance on false belief tasks used to test older children is convergent in this sense, 
there is little evidence of convergence for false belief tasks suitable for infants;
and \citet{poulin-dubois:_probing} find evidence for divergence.%
\footnote{%
There are, however, studies which find a relation between performance on tasks suitable for infants and tasks used with older children \citep[for example,][]{meristo:2016_early}.
}
Second \citet[p.~741]{wellman:2018_theorya} notes that in tasks typically used with older children, measures of belief tracking are predictive of social skills, whereas there is as yet little evidence that performance on belief tracking tasks used with infants predicts social abilities.

Perhaps future research will overcome these challenges by uncovering evidence of convergence and predictive value.
Alternatively, it may be that the challenges are based on an incorrect premise.
Given that there is individual (and cultural) variability in the ability measured by false beliefs tasks designed for use with older children, it makes sense to expect convergence and predictive value.
But perhaps the ability measured by false beliefs tasks suitable for use with infants does not vary much between individuals or across development.%
\footnote{%
\citet{meristo:2012_belief,meristo:2016_early}
}
If this turned out to be right, we would not necessarily expect performance on tasks suitable for infants to show signs of convergence, nor should it have much predictive value concerning social skills.
But this is speculation.
Until these challenges are resolved, there is surely room for doubt about whether one-year-olds really can form expectations about actions based on false beliefs.

What follows rests on a guess.
My guess is that even two- and three-year-olds really can track beliefs.
I thought there was already a case for this guess twenty years before I wrote this chapter \citep{Butterfill:2001fn}.
And even taking seriously challenges raised by patterns of success and failure in replication studies,
on balance the evidence in favour of this guess has grown since then.

If we accept this guess, we are immediately confronted with a complementary challenge to the idea that 
two- and three-year-olds make contradictory responses to scenarios involving false beliefs.
Perhaps the tasks on which two- and three-year-olds fail to exhibit belief tracking are just badly designed?

\section{Methodological Defects or Truly Contradictory Responses?}
\label{sec:truly-contr-resp}

We have just been considering a challenge to the idea that performance tasks suitable for infants really provide evidence for belief tracking in two- and three-year-olds.
The next, converse challenge concerns whether the tasks on which two- and three-year-olds fail to exhibit belief tracking really indicate an absence of belief-tracking abilities.

Consider the responses children make which indicate that they expect Maxi (or whoever) to act as if false belief were impossible.
These are typically (but not always) responses to a question directed to the child.
Asking children a question invites them to consider multiple possibilities, and it requires them to engage conversationally with an experimenter who is not part of the story.
By contrast, anticipatory looking and violation-of-expectation paradigms do not interrupt children’s engagement with the scenario in these ways.
Perhaps, then, children are not really making contradictory responses.
Instead it might be that anticipatory looking and violation-of-expectation paradigms provide a sensitive measure of children’s expectations concerning how Maxi will act,
whereas the responses which appear to indicate that they expect Maxi (or whoever) to act as if false belief were impossible are merely a reflection of poor experimental design resulting in children becoming distracted.%
\footnote{%
\citet{Rubio-Fernandez:2012} and \citet{rubio-fernandez:2013_perspective} offer a view along these lines.
}

But is this right?
Can we explain away two-and three-year-olds apparently contradictory responses by saying that anticipatory looking and violation-of-expectation paradigms simply use better measures?
Two considerations jointly suggest that we cannot.
% – first consideration
The first consideration is that three-year-old children succeed on tasks that are very similar to false belief tasks but concern desire, perception or pretence rather than belief \citep{custer:1996ht,Gopnik:1991db,Gopnik:1994cb}.
In fact they can even succeed on tasks which require contrasting their own desires with another’s incompatible desires \citep[Study 2]{rakoczy:2007_desire}.
%Compare Gopnik, Slaughter and Meltzoff (1994: 178): “changing conceptual content can have robust effects on tasks that are otherwise similar to the false belief task.”
This shows that, by three years of age and probably earlier, children can competently make verbal predictions in response to questions even where doing so involves taking into account facts about others’ mental states.
% Rather, whatever exactly explains why children perform as they do must include factors specific to reasoning about belief (not about desire, pretence or perception).
% %
% \footnote{%
% These considerations are also an obstacle to \citet{rubio-fernandez:2013_perspective}’s suggestion that the apparent discrepancy is merely due to the fact that some false belief tasks involve interrupting subjects’ tracking of a belief.
% }
% – second consideration
The second consideration is that children between their second and fourth birthdays do not usually answer at random when asked about false beliefs.
Instead they systematically and confidently give answers that contradict those ordinary adults usually provide: they answer questions as if there were no such thing as false belief.
This consideration shows that children whose anticipatory looking manifests expectations about false beliefs
are not all merely confused when they are asked a question.%
\footnote{%
What happens if you attempt to make answering a question about what Maxi (say) believes easier by having Maxi say aloud what he believes before asking the question?
When \citet[Experiment~2]{riggs:1995_what} did this, they found it made no difference.
Even though  three- and four-year-olds only had to repeat something they had just heard to count as passing, and even though they could report what said when asked about it, they continued to answer questions about belief as if there was no such thing as false belief.
}


Taken together, these two considerations suggest that typically developing two- and three-year-olds’ answers to questions involving belief are based on a sensible and coherent model of minds and actions in which belief does not feature.
This is a model on which how people act is a consequence of what they want (or perhaps of what is desirable for someone in their position), of how the world is, of which parts of it they have engaged with, and perhaps also of their emotions.
Someone who wants to eat the chocolate will go to where the chocolate actually is (unless prevented from doing so by never having engaged with the chocolate); and someone who wants to feed a mouse will fetch the cheese.
For our purposes, the key feature of these explanations is that they imply two- and three-year-olds rely on a model of minds and actions which does not incorporate beliefs.
What determines how people act on such a model is not what they believe to be the case but rather what is the case.
Several philosophers have argued that a model along these lines is theoretically coherent and enables accurate predictions and explanations in a wide variety of situations \citep{Gordon:2000cx,Stout:1996le}.
% Stout, Things That Happen Because They Should [1996, p. 3]: “it is possible to explain people’s behaviour teleologically in terms of facts without making any reference at all to their beliefs and intentions.”  Stout also argues that explanations of action in terms of psychological properties have only derivative value, but of course this conclusion doesn’t follow from the mere existence of other kinds of explanation.
And several developmental psychologists have independently argued that children’s early reasoning about actions is based on such a model.%
\footnote{%
See \citet{Wellman:1994vo,Wellman:2000vx,doherty:2011_engagement,roessler:2013_teleology}.
Doherty, Perner and Roessler all favour the view that two- or three-year-olds have an entirely non-mental model of action.
However there is evidence that children understand something of perception, desire, emotion, and guesses and their interactions before they can pass standard false belief tasks \citep[for example,][]{Wellman:2004qt,Wellman:2000cd,wellman:2011_sequentiala,rakoczy:2010_executive}.
}

On the face of it, then, what prevents most two- and three-year-old children from answering questions about false beliefs as adults typically do is that, in answering these questions, they rely on a model of minds and actions in which belief does not feature.

This view is supported by several further considerations.
One is that, as already mentioned, such children appear to shift to giving belief-based answers following training that is specifically relevant to understanding and reasoning about mental states \citep[for example,][]{Slaughter:1996fv,lohmann:2003_role}.
Another consideration is that children’s abilities to answer questions about false beliefs predict aspects of social competence, and do so independently of factors like their abilities to inhibit natural responses to things and to hold several things in mind  \citep{razza:2009_associations}.
These facts about the causes and consequences of typically developing children’s performance on the kinds of false belief task which they usually fail until around four years of age are hard to reconcile with the view that their performance  is an artefact of poor experimental design.
But the facts do make perfect sense if children shift from answering questions as if false belief were impossible to giving answers that take into account false beliefs  because they shift from relying on a belief-free model of minds and actions to a model which does incorporate beliefs.

There seems to be little prospect of finding a methodological defect in the false belief tasks that typically developing two- and three-year-olds tend to fail.
% But what about the false belief tasks that children of this age pass?
% Could they be methodologically defective?
% Taken individually, several studies do involve features which might in principle explain children’s responses independently of their tracking false beliefs \citep{heyes:2014_false}.
% And, further, it is almost inevitable that not all of the reported results will be successfully replicated.
% Indeed some failures to replicate have already been published (see the supplementary materials to \citealp{grossewiesmann:2016_implicit}).
% But there is converging evidence from a variety of labs involving several different measures and a good range of scenarios.
% So although there is much yet to learn about infants’
%  abilities to track false beliefs, it seems unlikely that they never have nonaccidentally correct expectations about any scenarios involving false beliefs.

At this point I suggest we have sufficient evidence to conclude that typically developing two- and three-year-olds’ responses to scenarios involving false beliefs really do manifest contradictory expectations.
The appearance of contradiction is genuine and not merely an  artefact of experimental design.

Note that accepting this modest claim leaves open the much harder question of what underpins these conflicting responses.
I have been suggesting that, on the face of it,
those responses which indicate an expectation that a protagonist will act in accordance with a false belief involve the use of a model of minds and actions incorporating beliefs.
As we will see (in \cref{sec:can-we-reject}), there are challenges to this claim.
%For all we have said so far, these responses might not even be a consequence of mindreading at all.
I have also been suggesting that, on the face of it,
those responses which indicate an expectation that a protagonist will act as if false belief were impossible involve the use of a model of minds and actions not incorporating beliefs.
There are significant challenges to this claim too—some hold that the responses reflect difficulties in applying a model of minds and actions to certain tasks, and do not reveal features of the model itself (see \crefrange{sec:task-analysis}{sec:too-much-mindreading}).
There is, then, plenty of controversy to come.
But one thing should not be controversial at all: two- and three-year-olds really do tend to make contradictory responses to scenarios involving false beliefs.
This leaves us with a puzzle.

% But these considerations are not decisive.
% What’s the alternative to my guess?
% The alternative guess has two parts.
% First, the only model (or models) of minds and actions that two- and three-year-olds use already incorporates beliefs.
% But then why do these children tend to answer questions as if false belief were impossible?
% Because—this is the second part of the alternative guess—some factor other than using a belief-free model of minds and actions causes two- and three-year-olds to reason as if false belief were impossible on a wide range of false belief tasks.
% The challenge, of course, is to identify this factor.
% We’ll come back to this later (in \cref{sec:inhibition-and-selection}).
% But first, What evidence favours this alternative guess?

% As we have already seen, two- and three-year-olds’ anticipatory looking manifests an ability to track false beliefs.
% The most straightforward way to explain this is to postulate that their anticipatory looking involves a model of minds and actions which incorporates beliefs.

% ** cut here


\section{Models}
\label{sec:models}

I have been talking about children relying on a model of minds and actions without explaining what this means.

For simplicity, start by thinking about a physical model of a house.
This model could be used in various ways.
Perhaps initially you use the model to imagine your dream house,
but then you win the lottery and use the model as a plan for building an actual house.
Later you display the model in the hallway and your house guests use it to find their way around.

A model is something that can serve different purposes.
Having a model does not commit you to using it for any particular purpose.
(If your dreams are more extravagant than your lottery winnings, you could use a different model when building the actual house.)
The model’s usefulness does not depend only on its accuracy: the ease with which it can be used to imagine, build or navigate matters.
The best model for a given set of purposes may not be the most accurate.
Further, it can be advantageous to have multiple models of a single thing.
For example, building a house can involve creating multiple models.
There is nothing wrong with having and using more than one model.

In talking about someone relying on a model of minds and actions, there is clearly no suggestion that they have a physical model.
In general, a model is a way some part of aspect of the world could be.
So a \gls{model of minds and actions} is just a way mental aspects of the world could be.

A model is distinct from a theory. 
A model can be used to make claims about the world, of course (‘this is the bathroom’), but the model itself entails nothing about how the world actually is.
By contrast, a theory does \citep{godfrey-smith:2005_folk}.
Models are easily confused with theories because we can use theories to specify models.
To describe the model of minds and actions a child or adult relies on, we might give a theory of the mental.
Similarly, in stating the \gls{Principles of Object Perception} we partially characterised a model of the physical that infants (and adults) rely on (see \cref{sec:principles-object-perception}). 
But of course to say that someone relies on a model is not to say that they have or endorse a theory.
The theory is what we as theorists use to characterise the model.%
\footnote{%
There is more on using theories to characterise models of minds and actions in \cref{sec:minim-models-ment,sec:minimal-models-of-the-mental}.
}

But why do we need models at all?
The answer is easier to see by thinking about the physical case.
Suppose we discover that infants in the first four months of life can track briefly occluded physical objects.
This discovery raises the question, what are physical objects like from the point of view of the infant?
We can answer this question by identifying a model of the physical.
The model tells us what the physical world is like from the infant’s point of view (or from the point of view of some process in the infant).
Similarly, to say that infants in the first or second year of life can \gls{track a belief} leaves entirely open the question of how the mental appears to the infant (if it appears to the infant at all).
Identifying a model of minds and actions allows us to characterise how minds are from the infants’ point of view (or from the point of view of a process in the infant).

Consider a two- or three-year-old who, when asked where Maxi will look for his chocolate (see \cref{sec:all-about-maxi}), confidently tells you that Maxi will look where his chocolate actually is (so not where Maxi believes his chocolate is).
Should we therefore conclude that such a child is not relying on a model of minds and actions?
Not at all.
The child’s answer may be based on a coherent model of minds and actions, one on which facts guide actions.
Although adults are often surprised that children answer questions on the basis of facts rather than beliefs, there is nothing obviously wrong with the strategy of using facts to predict and explain actions.
Indeed, it is possible to develop a sophisticated and coherent theory of minds and actions along these lines \citep[see, for example,][]{Stout:1996le}.
Children who systematically make predictions about actions as if agents always acted on the relevant facts are plausibly relying on a coherent model of minds and actions which does not incorporate beliefs.
This is a model on which minds are windows onto facts.

When one child systematically passes and another systematically fail a certain kind of false belief task, the difference is not necessarily that one lacks a model of minds and actions.
It may be that one child is relying, in this kind of task at least, on a model of minds and actions not incorporating beliefs; whereas the other child relies on a model incorporating beliefs.





\section{The Mindreading Puzzle}
\label{sec:mindreading-puzzle}

% Our immediate question is, When can humans first know facts about others’ beliefs?
% % In \cref{sec:all-about-maxi} we saw that, when tested in some ways, typically developing humans appear unable to manifest knowledge of any such facts until around four years of age.
% In this section we have seen that infants’ anticipatory looking and performance on a wide range of violation-of-expectation tasks
% indicates that humans are able  to form expectations concerning actions based on false beliefs from some time in the second year of life or earlier.
% Taken in isolation, it might be natural to interpret these findings as evidence for the hypothesis that humans can know facts about others’ beliefs at around one year of age or earlier.
% But things are not quite so straightforward:
% the hypothesis generates incorrect predictions.
% In particular, the hypothesis generates predictions that typically developing two- and three-year-olds should be able to talk about others’ beliefs, to predict how someone with a false belief will act roughly as well as they can predict actions based on true beliefs, and to explain failures of action by invoking true beliefs.
% But as we saw in \cref{sec:all-about-maxi}, these predictions are all incorrect.


% % If these abilities are based on knowledge of others’ beliefs, then we must conclude that humans can first
% %know facts about others’ beliefs

%The findings introduced so far leave us with a puzzle.
On the face of it, it seems that for many children, there is an age at which:
%
\begin{enumerate}
\item in performing a false belief task of the kind two- and three-year-olds tend to fail,%
\footnote{%
But which false belief tasks are ‘of the kind two- and three-year-olds tend to fail’?
This turns out to be an unexpectedly tricky question.
See  \cref{sec:task-analysis,sec:dual-process}.
}
 the child relies on a model of minds and actions not incorporating beliefs;
\item in anticipatory looking and on violation-of-expectation tasks (or in other false belief tasks which one- or two-year-olds tend to pass—see \cref{sec:can-we-reject}), the child relies on a model of minds and actions incorporating beliefs;
and
\item the child has a single model of minds and actions.
\end{enumerate}
%
These claims are jointly inconsistent, so one of them must be false.
But which claim should we reject?
Taken individually, each seems at least superficially plausible.
The first claim was discussed in \cref{sec:truly-contr-resp}, and the second claim in \cref{sec:infants-track-false}.
% We have just seen that there is good evidence for the second claim from a variety of experiments involving both anticipatory looking and the violation-of-expectation method.
% And a moment ago I outlined some of the evidence for the first claim, too.
% (The evidence indicates that Wimmer and Perner’s task is of a kind that children can perform well when it doesn’t concern belief; and when tasks of this kind do concern beliefs, there is an age at which children confidently and systematically give exactly the answers that someone who had a model of minds and actions not incorporating beliefs would give.)
And while I haven’t discussed the third claim, rejecting this claim and supposing that children have multiple, incompatible models of minds and actions would surely be a desperate measure.

Solving \gls{the Mindreading Puzzle} requires us to work out which of these three claims, 1–3, we should reject.
In what follows we will consider attempts to solve it.



% \section{Why the Mindreading Puzzle Matters}
% \label{sec:why-mindr-puzzle}

% **Reject the first claim and you have nativism.
% Reject the second claim and you have empiricism.
% Reject the third claim and you have the awkward hybrid account that reflection on knowledge of objects and on knowledge of colour appeared to leave us with.



%%% Local Variables:
%%% TeX-master: "master"
%%% End:
