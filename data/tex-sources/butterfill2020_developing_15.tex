%!TEX root = master.tex


% ∞TODO : be sure that the puzzle concerns an interaction, not an age gap.  Make sure this is clear from the definition of an A-task!


\chapter{Three Levels of Analysis}
\label{cha:mind02}



This chapter introduces two attempts to solve \gls{the Mindreading Puzzle} introduced in \cref{cha:mind} (see \cref{sec:mindreading-puzzle}).
The first attempt involves rejecting the claim \#1 of the Puzzle; and the second two are complementary attempts to reject  claim \#2 of the Puzzle.
(The possibility of rejecting  claim \#3 is the topic of \cref{cha:mind-solution}.)
The hope is that solving the Mindreading Puzzle will take us one step closer to understanding how humans first come to know facts about others’ beliefs, and perhaps about mental states more generally.

\section{Tracking Beliefs without Representing Them?}
\label{sec:can-we-reject}


Can we reject the second of the three claims behind the Mindreading Puzzle, the claim that younger children rely on a model of minds and action incorporating beliefs?
Any attempt to reject this claim will hinge on distinguishing tracking beliefs from having a model of minds and actions which incorporates beliefs.
Let me explain.

\glsadd{track a belief}
For a process to \emph{track} someone’s belief that $p$ is for it to nonaccidentally  depend in some way on whether she believes that $p$.
Relatedly, to say that someone tracks beliefs is to say that there are processes in her which track some beliefs.
By contrast, to say that someone’s responses rely on a model of minds and actions incorporating
%the possibility of false
beliefs implies that making these responses involves identifying certain beliefs as the beliefs of particular individuals and using such identifications in predicting or explaining those individual’s actions.
What many experiments actually measure is whether certain subjects can track beliefs:
the question is whether changes in what another believes are reflected in the subjects’ anticipatory looking, looking times or predictions about action (say).
In principle it is possible to track beliefs without explaining or predicting anything, and without relying on any model of minds and actions.


The distinction between tracking beliefs and relying on a model incorporating beliefs is usually ignored, perhaps because it is so natural to assume that anyone tracking beliefs is likely to be doing so by using a model incorporating beliefs.
But we need to recognise that tracking beliefs does not necessarily entail representing them if we are to solve \gls{the Mindreading Puzzle} by rejecting the second claim—the claim that younger children rely on a model of minds and action incorporating beliefs.
And
some recent discoveries provide us with an interesting reason not to ignore the possibility that infants are tracking beliefs without relying on any model of minds and actions.


\section{Altercentric Interference}
\label{sec:altercentric}
The discoveries concern something called \emph{\gls{altercentric interference}}.
An adult’s ability to track or report facts about a physical object can be influenced (and sometimes impaired) when another manifestly has a false belief concerning the object.
Importantly, this can occur even when the adult is not required to think about the other or her beliefs \citep{Wel:2013uq}.

To illustrate, suppose you are observing a scene involving a ball and an occluder. The ball moves around, sometimes stopping behind the occluder and sometimes leaving the scene.
At some point the ball is not visible and you are asked to press a button if you think it is behind the screen.
This is a simple task which, apparently, tests only your ability to detect an object’s location: no mindreading is required at all.
Or so it seems.

Suppose that we complicate the scene slightly by introducing a protagonist who observes some but not all of the ball’s movements.
The protagonist never acts at all, but merely observes.
Now it can happen that the protagonist is likely to have a false belief about the ball’s location, as she did not observe some of the ball’s movements.
Of course this false belief is completely irrelevant to your task, which is just to press a button if you think the ball is behind the screen.
And yet it turns out that how quickly you perform your task is influenced by the other’s beliefs.%
\footnote{%
Versions of this task have been implemented by \citet{kovacs_social_2010,Wel:2013uq,edwards:2017_reaction}.
}

This  is \gls{altercentric interference}: your performance is influenced by another’s belief, although their belief is not relevant to your task.

\citet{kovacs_social_2010} provided evidence that altercentric interference occurs in seven-month-old infants and not just adults.
Impressively their paradigm allowed them to use the same stimuli with 7-month-old infants and adults.%
\footnote{%
\citet{phillips:2015_second} offer a detailed challenge to the methods of this study.
Although some find the critique convincing \citep[for example,][p.~27]{schneider:2017_current},
it is striking that follow-up work has successfully found converging results \citep{edwards:2017_reaction,Wel:2013uq},  
and that a direct attempt to test the challenger account reports findings that support \citeauthor{kovacs_social_2010}’s original interpretation of their findings
\citep{elkaddouri:2019_measuring}.
On balance, \citet{kovacs_social_2010} seems to be passing the replication test with flying colours.
}




What explains altercentric interference?
We should distinguish two possibilities.
One possibility is that you represent the protagonist’s belief, and this belief representation somehow interferes with your representation of the object.
Another possibility is that you do not represent the other’s belief at all.
Instead, you believe what they believe.

As the second possibility can be hard to take seriously,%
\footnote{%
I once spent an evening explaining it to Celia Heyes, who told me it was the kind of idea only a philosopher would come up with. I’m not sure this was intended as a complement.
} 
recall the \gls{Motor Theory of Goal Tracking}  from \cref{sec:motor-theory-goal-tracking}.
Observing another’s actions can cause motor representations in you as if it were you, not her, who was acting (see \cref{cha:action}).
And this ‘mirroring’ not only interferes with your own actions \citep[for example,][]{kilner:2003_interference}
but also plays a role in enabling you to predict her actions  \citep[for example,][]{Costantini:2012fk}.
Importantly, the Motor Theory does not entail that you have beliefs about, or represent, another’s motor representations.
The theory is entirely about first-order motor representations of outcomes.

It is possible that mirroring might occur with respect to representations other than motor representations, including spatial and perceptual representations \citep[for example,][]{Furlanetto:2015_altercentric,Zwickel:2011,kampis:2015_neural}. And, of course, beliefs.

Observing another person experiencing events that cause her to acquire the belief that the melon is in the box (say) may also cause you to believe this too, and thereby enable you to predict her actions.
This is more like contagion than ascription.
It not a matter of representing her belief.
It requires no understanding of mental states.
It is simply a matter of mirroring her beliefs---of believing what she believes.

Mirroring does not involve any model of minds or actions.
You can mirror without having the faintest idea of what someone else believes, and without any capacity to represent beliefs.
Yet mirroring would enable you to track beliefs. \glsadd{track a belief}

Infants’ (and adults’) performance on false belief tasks which measure altercentric interference demonstrate that they can track beliefs.
But this performance could in principle be explained by mirroring beliefs, and therefore without postulating any representations of mental states at all.
Could the same be true of infants’ performance on other false belief tasks?


% The notion of \emph{co-representation} is useful here.
% Two or more individuals \emph{co-represent} something if they each individually represent it and their representations are of the same kind.
% For example, if you believe that $p$ and I believe that $p$, then we co-represent $p$ in virtue of having these beliefs.


% This point about altercentric interference is potentially confusing.
% Suppose I believe that there is cocaine in Urma’s bag.
% For you to suffer altercentric interference, you have to represent what I believe: that is, you have to represent that there is cocaine in Urma’s bag.
% You also have to represent this because it is what I believe.
% But you do not have to represent that I believe it.
% Nor do you have to represent any other mental state.
% So altercentric interference need not require mindreading.
% % Don’t say this: it suggests altercentric interference involves me acquiring your beliefs
% %It is contagion, not ascription.





Consider \citeauthor{Onishi:2005hm}’s experiment again.
They had infants observe as a protagonist manifestly acquired a false belief.
The protagonist then either performed an action which made sense given the false belief or else acted as if she had a true belief.
\citeauthor{Onishi:2005hm}’s key finding was that infants looked longer when the protagonist acted as if she had a true belief (see \vref{fig:onishi_fig}).
This might be because the infants relied on a model of minds and actions to predict the protagonist’s actions.
But it might also be because, like \citeauthor{kovacs_social_2010}’s infants, their expectations about the object’s actual location were influenced by the protagonist’s false belief.
In this case, the effect would be a consequence of altercentric interference and need not involve infants ascribing a belief or any mental state.

Altercentric interference could in principle explain
performance on any experiment in which subjects simply observe a protagonist with a false belief.
This includes experiments that rely on anticipatory looking or \gls{violation-of-expectation}.
They do not obviously require mindreading.
So experiments with infants using these paradigms do not, by themselves, give us sufficient reason to suppose infants’ responses rely on a model of minds and actions incorporating beliefs.
Can we therefore answer the Mindreading Puzzle by rejecting the second of the three claims that give rise to it?%
\footnote{%
Although they do express their view in terms of belief mirroring,
it is possible to interpret \citet{perner:2012_infants} as making a suggestion along roughly these lines.
}


\section{Mirroring beliefs?}
\label{sec:mirroring-beliefs}


Things are not quite so simple.
False belief tasks designed for older children typically ask questions both about belief and about reality, effectively ruling out the possibility of belief mirroring.
You would not get very far asking one year olds questions, of course.
But the same can be achieved by having a task which requires infants to act on the basis of both information about what a protagonist falsely believes and information about what is actually the case.
Several tasks measuring how one-year-olds interact with a false-believing adult impose just this requirement.%
\footnote{%
Suitable tasks include those which require infants to interpret a pointing gesture \citep{southgate:2010fb,Carpenter:2002gc} or a request \citep{buttelmann:2015_what}; but we should be cautious at present in using these findings as some attempts to replicate these findings  have failed (see \citealp{dorrenberg:2018_how} and \citealp{kulke:2018_implicit}).
}

\citet{Knudsen:2011fk} claim that one-year-olds manifest an ability to track beliefs in pointing to warn others of danger.
To support this claim, they exploited infants’ tendency to spontaneously point to inform others about things. 
They put eighteen-month-olds in a situation where an adult manifestly had a false belief that would probably cause her to reach into a container which, unknown to her, contained something disgusting.
They wondered how often infants would anticipate the adults’ action and point to this container to warn the adult with the false belief.
To ensure that infants were not motivated to point merely by the presence of an unexpected disgusting thing, Knudsen and Liszkowski set things up so that there were two disgusting things and anyone who lacked the adults’ false belief would be equally likely to accidentally encounter either of them.
And in order to isolate the effect of the false belief, Knudsen and Liszkowski compared infants’ pointing responses in another scenario that was as similar as possible except that the adult did not have a false belief specifying the location of an object but instead was completely ignorant.
Would infants point more often to one of the disgusting things when the adult falsely believed that an object of hers was at that location?
As you can see in \vref{fig:knudsen_2012_fig2a},
they did.
The part labelled ‘Experiment 1’ shows that infants pointed more often to the disgusting thing that was where the adult falsely believed that her object was, and the other part (labelled ‘Experiment 3’) shows that they pointed equally to the two disgusting things when the adult was ignorant and had no location-specifying belief.
This is evidence that infants’ abilities to track false beliefs enable them to intervene effectively in others’ actions (\citealp{knudsen:2012_eighteen} provide further evidence).
%
\begin{figure}
\begin{center}
\includegraphics[width=0.6\textwidth]{fig/knudsen_2012_fig2a.png}
\caption{
  \label{fig:knudsen_2012_fig2a}
 Eighteen-month-olds manifest abilities to track beliefs in their pointing actions.
Both the target and distractor box contain something disgusting.
In Experiment 1, an adult falsely believes that something of hers is in the target box; in Experiment 3 the adult is completely ignorant about the location of her object.
Source: \citet{Knudsen:2011fk}, figure 2 (part).
}
\end{center}
\end{figure}
%

Belief mirroring could not straightforwardly explain one-year-olds’ performance on tasks which, like Knudsen and Liszkowski’s, involve interaction.
On the face of it, in pointing to warn the adult, infants are combining information about a false belief (which enables them to predict the adult’s action) with information about the actual contents of a container (which is what makes warning relevant).
This indicates that the pointing is not explained by belief mirroring: to appeal to belief mirroring in explaining a process is to suppose that the process fails to distinguish information about the false belief from information about the container’s contents.

% This is not decisive, of course.
% It is possible in principle that \citeauthor{Knudsen:2011fk}’s findings
% are consequences of belief mirroring.
% % novelty -> point at disgusting things
% % belief mirroring -> adult is going there
% % adult is going there -> this is the more relevant location
% % this is the more relevant location -> point there
% It may even be possible in principle that the influences of what others believe on infants’ pointing actions, on the ways they comprehend adults, and on they ways they offer to help are all a consequence of belief mirroring.
% But it would be surprising if belief mirroring were responsible for such a wide range of responses on so many quite different false belief tasks.

What can we conclude?
We have been considering whether it is possible to resolve \gls{the Mindreading Puzzle} by rejecting its second claim on the grounds that infants are mirroring beliefs rather than representing them.
For all we know, some belief tracking in infancy (and adulthood too) may be a consequence of belief mirroring.
 But infants manifest abilities to track beliefs in a wide variety of ways.
And the results of experiments involving pointing, helping and other kinds of social interaction make it  unlikely that belief mirroring is what explains the full range of one- and two-year-olds’ belief tracking.
%
% Is there any hope for the idea that we can resolve the puzzle by showing that infants track beliefs but do not a rely on a model of minds and actions incorporating beliefs?
% The main obstacle is the wide variety of false belief tasks on which infants manifest belief-tracking abilities.
% What could explain their performance if it is not that they have a model of minds and actions incorporating beliefs?
%
This suggests that at least some of infants’ responses really are underpinned by a model of minds and actions incorporating beliefs,
just as the second claim in \gls{the Mindreading Puzzle} has it.


If you reject this suggestion you will need to explain why infants manifest belief-tracking abilities  on such a wide variety of false belief tasks if it is not because they have a model of minds and actions incorporating beliefs.%
\footnote{
For an ambitious, relatively detailed attempt involving behavioural patterns, see \citet{ruffman:2014_belief}.
An alternative approach is suggested by \citet{heyes:2014_false}, who conjectures that infants’ performance on a range of false belief tasks is driven by effects such as retroactive interference and so does not involve any model of minds and actions.

}
If, on the other hand, the suggestion is right, then we must look elsewhere for a solution to the Mindreading Puzzle.






\section{Three Levels of Analysis}
\label{sec:three-levels-of-analysis}

% If we can’t reject the second of the claims comprising the Mindreading Puzzle, we have to reject one of the other claims.
% But before going further it is necessary to switch levels of analysis.

If we can’t reject the second of the claims comprising \gls{the Mindreading Puzzle}, what about rejecting the first claim?
According to this first claim, the children we are concerned with fail false belief tasks like Wimmer and Perner’s because, in performing these tasks, they rely on a model of minds and actions not incorporating beliefs.
Suppose this claim is false.
Why else might these children fail some false belief tasks?

Answering this question involves two steps.
We must first identify a feature, or a set of features, common to all the false belief tasks that the children fail.
Then, second, we must show that this feature means that these false belief tasks impose an extraneous demand on children.
This demand should be extraneous in the sense that it is not related, or only indirectly related, to ascribing false beliefs.
Showing this will allow us to conclude that the children fail some false belief tasks not because they rely on a model of minds and actions not incorporating beliefs (as \gls{the Mindreading Puzzle} has it) but because these tasks involve an extraneous demand.

%What feature, or set of features, is common to all the false belief tasks that the children we are concerned with fail?
Pursuing this idea involves switching levels of analysis.
Let me explain.
Fully understanding mindreading will eventually require at least three levels of analysis (see \vref{table:levels-of-analysis}).
First, it requires us to analyse the model or models of minds and actions children and adults use in performing various tasks.
This has been our focus until now: the Mindreading Puzzle is primarily a puzzle about which model of minds and actions are used in false belief tasks.
Second, it requires us to analyse how the model relates to the child’s or adults’ cognition and action.
What kind of mechanism realises a given model of minds and actions?
Is it, for instance, that the model is something known which is explicitly used in inferences?
Or should we rather think of the model as describing the operations of a \gls{core system}, or in some other way?
We do not yet have to face up to these questions about mechanisms.
Instead our current concern is with a third level of analysis, the task analysis.

\begin{table}

\begin{center}
\footnotesize	%shrink for better spacing

% add space between rows
\extrarowsep=7pt

\begin{tabu} to 0.8\linewidth {X[1,l] X[3,l]}

\toprule

Models & Which model or models of minds and actions do children use in mindreading?
\\
Mechanisms & How does a particular model relate to the child’s cognition and action?  Is the model something the child knows, does the model describe the operation of a core system, or what?
\\
Tasks & What features determine whether a particular false belief tasks is an \gls{A-task}?
\\
%
\bottomrule
%
\end{tabu}
\caption{Fully understanding mindreading requires at least three levels of analysis.}
\label{table:levels-of-analysis}
\end{center}	%careful -- position of this affects distance between table and caption(!)


\end{table}

\normalsize


What is task analysis?
We know that there are at least two kinds of false belief task.
There is a kind that children tend to fail until they are somewhere around three to five years old,
 and there is a kind—at least one, maybe more—that children tend to pass in their first or second year of life.
We also have good evidence that all the false belief tasks children tend to fail until they are somewhere around three to five years old are measuring a single factor \citep{Wellman:2001lz}.
% **TODO : fix! (interaction, not age!) [added paragraph below]
Let us therefore call any false belief task that children tend to fail until around three to five years of age an  \emph{\gls{A-task}}.
I’ll refer to any false belief task that infants tend to pass in their first or second year of life simply as a \emph{\gls{non-A-task}}.
(I’m  tempted to call them ‘B-tasks’ but won’t because we don’t yet know whether different tasks in this category measure different things, as \citet{yott:2016_are,poulin-dubois:_probing} stress.)

While we have used age to distinguish between A-tasks and non-A-tasks, note that age is not the fundamental issue here.
(Researchers sometimes write, mistakenly in my view, as if the developmental puzzle about false belief were that children pass A-tasks relatively late in life.)
What really matters is the interaction between performance and task type: there are individuals whose performance on non-A-tasks appears to conflict with their performance on A-tasks. 
Even if this interaction could be observed for just a day in each child’s life, rather than for several years, its theoretical interest would hardly be less.





\section{Task Analysis}
\label{sec:task-analysis}
Which features determine whether a particular false belief task is an A-task, a non-A-task, or neither?

An adequate answer to this question should make it possible to determine \textit{a priori} whether or not a completely novel false belief task is an A-task.
That is, the task analysis should enable us to predict in advance of actually testing any subjects how different groups of subjects will perform on the task.


% It may be tempting to suppose that this question, which has been widely neglected, is easy to answer or has already been adequately answered.
One initially tempting suggestion is that A-tasks are distinct from other false belief tasks in that they involve language in some way.
Things are not so simple, however.
Some A-tasks are entirely non-verbal \citep[for example,][]{Call:1999co},
some barely involve language at all \citep[for example,][]{krachun:2010_new},
and many involve non-verbal responses
\citep[for example,][]{Chandler:1989qa,custer:1996ht,low:2010_preschoolers}.
Conversely, some non-A-tasks depend on verbal narratives \citep[for example,][]{scott:2012_verbal_fb}.
% ***check: is this right?  I’ve re-read it a couple of times and it’s not completely clear.  I think some of their tasks involve low levels of language.  could maybe extend the point to Wellman et al metaanalysis: A-tasks with varying linguistic demands all produce the same basic result (there is an interaction)!
A further complication is that a meta-analysis of the relation between false belief and language found no significant difference in the relation between language and belief tracking for different kinds of A-task despite these differing in the linguistic demands they impose  \citep[p.~637]{milligan:2007_language}.
This suggests we probably cannot distinguish A-tasks by appeal to the way they involve language.




If it is not simply language, what else could distinguish A-tasks from other false belief tasks?
A clue is given by experiments in which two or more kinds of response to the same stimuli were recorded.
For instance, recall Clements and Perner’s task (which was introduced in \cref{sec:mindreading-puzzle}).
They showed children a scenario in which someone acquired a false belief and then, for each child, they measured both anticipatory looking and verbal predictions.
\citet{low:2014_quack} went even further by measuring
anticipatory looking, looking time during a critical window, and verbal responses to a question.
Two- and three-year-olds seem to manifest abilities to track false beliefs in their looking times and anticipatory looking but not in responding to a question.
This indicates that an A-task and a non-A-task may differ merely in the kind of response measured.
But what is it about the response that determines whether a false belief task is an A-task or not?

One hypothesis is that A-tasks involve responses elicited by communicative actions directed to the subject, whereas non-A-tasks do not \citep{Baillargeon:gx,he:2012_2}.
A related hypothesis is  that A-tasks are those which involve features that somehow disrupt the process of tracking another’s perspective \citep{Rubio-Fernandez:2012,rubio-fernandez:2013_perspective, rubio-fernandez:2016_don}.

Neither of these hypothesis seems to be right.
One objection to both hypotheses is that 18-month-olds appear to perform well on some false belief tasks in which their response is elicited by a communicative action directed to them \citep[for example,][p.~98]{buttelmann:2015_what}.
There is also a converse objection.
Performance on A-tasks is related to children’s use of mental state terms in ordinary conversation \citep{hughes:1998_understanding},
and children appear not to offer unprompted comments on others’ false beliefs until around the age they pass some A-tasks (\citealp{Bartsch:1995vl}; \citealp{ruffman:2002_relation}; \citealp[p.~810]{pyers:2009_language});
%\citet[p.~810]{pyers:2009_language}: ‘we found that mental-state language preceded false-belief understanding in 6 of the 8 first-cohort signers, and co-occurred with false-belief understanding in the remaining 2 first-cohort signers. There was no case in which false-belief understanding came first.’
both facts suggest that A-tasks could in principle include some without disruption and without responses elicited by communicative actions directed to the subject.%
\footnote{%
\citet{helming:2015:making} attempt to explain why children fail A-tasks by appeal to pragmatic considerations.
The considerations that some non-A-tasks include communicative prompts and that not all A-tasks need include them is a problem for their attempted explanation.
}

\citet[p.~430]{Garnham:2001jm} offer a hybrid hypothesis which overcomes these difficulties.
Their guess is that a task which involves a ‘declarative expression’ about a belief or belief-based action or emotion will be an A-task;%
\footnote{%
The term ‘declarative expression about belief’ should be understood broadly to include verbal predictions of actions and emotions where the prediction involves ascribing a false belief, and also to include predictions which are expressed by, say, moving a pointer, lifting a flap or pointing to a picture.
So making a declarative expression about belief need not  involve directly mention of beliefs nor need it involve any verbal or linguistic declaration.
}
a task which involves a response prompted by a communicative action directed to the subject will also be an A-task;
and no other tasks will be A-tasks.

Even this hybrid hypothesis is probably untrue.
Consider three tasks.
In each task, subjects are shown a scenario in which a protagonist manifestly acquires a false belief.
In each task, subjects are required to choose an object.
And things are contrived in such a way that subjects have to take into account the protagonist’s false belief if they are to reliably make the best choice.
In the first task, subjects are choosing an object for themselves \citep{krachun:2010_new}.
In the second task, subjects are choosing an object to help someone else \citep{buttelmann:2015_what}.
And in the third task, subjects are choosing an object in accordance with a verbal request \citep{southgate:2010fb}.
Which, if any, of these tasks are A-tasks?
According to any of the hypotheses we have just been considering, tasks which involve responses prompted by communicative actions directed to the subject should be A-tasks.
So all three of these tasks should be A-tasks.
In fact, the first is an A-task and the other two are non-A-tasks (see \vref{table:task-analysis-is-hard}).
% \footnote{%
% If you predicted this, please publish your hypothesis about which false belief tasks are A-tasks.
% }
This suggests that Garnham and Perner’s hypothesis is incorrect.
It also illustrates that task analysis is surprisingly hard.
It’s not too difficult to make up a story about these three false belief tasks after you know which are A-tasks, of course.
The tricky thing is being able to tell the story in advance of knowing which false belief tasks are A-tasks.


\begin{table}

% add space between rows
\extrarowsep=5pt

% add space between rows
%\renewcommand{\arraystretch}{1.5}

\begin{center}
\footnotesize	%shrink for better spacing
\begin{tabu} to \linewidth {X[2,l] X[2,l] X[cm]}

\toprule
Task & Type of response & Is an A-task?  \\
\midrule
%
\citet{krachun:2010_new}; also \citet{Call:1999co} & choose an object for oneself & yes
\\
\citet{buttelmann:2015_what} & choose an object for someone with a false belief & no
\\
\citet{southgate:2010fb} & choose an object in accordance with a request from someone with a false belief & no
\\
%
\bottomrule
%
\end{tabu}
\caption{Identifying features that would allow us to predict whether a false belief task is an A-task is hard. Note that some of the findings mentioned here are subject to unsuccessful replication attempts (see \cref{sec:replication-challenge}) or have yet to be replicated.}
\label{table:task-analysis-is-hard}
\end{center}	%careful -- position of this affects distance between table and caption(!)

%reset to default value
\renewcommand{\arraystretch}{1}

\end{table}

\normalsize



My guess is that we are all a long way from understanding what determines whether a false belief task is an A-task.
But in the absence of a better supported hypothesis, for now I am going to work on the assumption that Garnham and Perner’s hypothesis is approximately correct.
We will eventually consider an explanation for the difficulty of task analysis, one which makes sense of the failure of existing proposals (see \cref{sec:Task Analysis Revisited}).

Why do we care about the task analysis?
Our current aim is to solve \gls{the Mindreading Puzzle} by identifying extraneous demands imposed by A-tasks but not other false belief tasks.
%This would allow us to reject the first claim of the Puzzle, the claim that two- and three-year-olds’ failure on A-tasks reflects their reliance on a model of minds and actions not incorporating beliefs.
In pursuing this aim, we would ideally  show that A-tasks are different from other false belief tasks in having some feature (or set of features) which imposes extraneous demands on children, demands that are unrelated, or only indirectly related, to ascribing false beliefs.
The point of the task analysis is that it tells us (approximately) which tasks are A-tasks.
Our next challenge is to consider whether the distinguishing features of A-tasks do somehow mean that these tasks impose extraneous demands.
If they do, it will be possible to solve the Mindreading Puzzle by rejecting the claim that children who can track false beliefs fail A-tasks because they rely on a model of minds and actions not incorporating the possibility of false belief.
(This is the first claim in \gls{the Mindreading Puzzle}.)

%this is worth including because it’s a potential source of confusion when students read the literature
It is striking that, in attempting to solve the Mindreading Puzzle, researchers appear to focus on just one level of analysis.
Some write as if task analysis alone were sufficient—as if
identifying features  which distinguish A-tasks from non-A-tasks amounts to answering the questions about models and mechanisms too (compare \citealp[p.~6]{Rubio-Fernandez:2012} and
% offer their hypothesis as an alternative to the view that the discrepancy between A-tasks and non-A-tasks reflects the operation of different belief-tracking systems in humans \citep{Apperly:2009ju}, and
\citealp[pp.~26ff]{he:2012_2}).
Others appear to neglect the need for a task analysis and so risk mischaracterising what solving the Mindreading Puzzle requires \citep[for example,][]{carruthers:2013_mindreading,helming:2015:making}.
Both positions are mistaken.
Task analysis is both necessary and hard, but a task analysis isn’t a solution to the Mindreading Puzzle all by itself.
Solving the Mindreading Puzzle requires not only identifying distinguishing features of A-tasks but also understanding how these features affect subjects’ performance in the ways they do.
One false belief task involves a communicative prompt  or a declarative expression about belief, whereas another false belief task does not.
Why does this apparently minor difference transform the relation between performance on the task and the subject’s age (among other factors)?%
\footnote{%
Note that this question is not equivalent to asking why there is a gap of months or years between success on non-A-tasks and success on A-tasks.
The question arises not simply because A-tasks are in some sense more difficult than other false belief tasks.
The question arises because typically developing children’s performance on A-tasks undergoes a striking change over months or years, whereas there appears to be no corresponding age-related change in performance on other false belief tasks.
Put statistically, if you compare performance on an A-task with performance on a non-A-task you should observe an interaction of age and task type on performance.
% * should this go somewhere else?
}

% — * above : isn’t the issue that individuals simultaneously make inconsistent predictions in response to a single stimulus?

\section{Selection and Inhibition}
\label{sec:inhibition-and-selection}

Take someone and ask her to count some sheep.
Suppose she fails.
Can you conclude that she can’t count?
Maybe, but not if you were whispering random numbers in her ear while she was attempting to count the sheep.
In that case, her failure might be a consequence of your distracting ruse rather than of her inability to count.
Or suppose that while attempting to count some sheep your subject is also attempting to count the camels and the snakes as well.
In this case, too, her failure on your task is not evidence that she cannot count because it could equally be explained by the conflicting demands she is under.
She can count alright, she just can’t count three things at once.
One premise of \gls{the Mindreading Puzzle} is that some children fail A-tasks because they cannot count—or, rather, because they are relying on a model of minds and actions not incorporating beliefs.
But might it be instead that A-tasks involve extraneous demands?
Could the communicative prompt or declarative expression about belief which distinguishes A-tasks somehow function like whispering numbers in someone’s ear while she tries to count?
Or could it function like asking her to do something else while counting?

In this section we will investigate the first alternative;%
\footnote{%
A better but more complex analogy for this alternative is \citet{eriksen:1974_effects}’s flanker task.
}
the following section explores the second alternative.
If either alternative (or some combination of them) is correct, we will have understood why the apparently minor difference between A-tasks and other false belief tasks transforms the relation between performance and age:
it is because using a communicative prompt or requiring declarative expression about belief imposes demands extraneous to ascribing beliefs.

Why does this matter?
If A-tasks involve an extraneous demand, we have no reason to suppose children who fail A-tasks are relying on a model of minds and actions not involving beliefs.
This would mean that we can solve the Mindreading Puzzle by rejecting the first of the three claims that comprise it.
(\Gls{the Mindreading Puzzle} was introduced in \cref{sec:mindreading-puzzle}).
This in turn would support a straightforwardly nativist explanation of how humans first come to know facts about beliefs, and perhaps of the mental generally.


% The first claim  in the Mindreading Puzzle is the claim that children who can pass non-A false belief tasks but fail A-tasks do so because, in completing the A-tasks, they are relying on a model of minds and actions not incorporating beliefs.
% (I have formulated this claim in terms of our task analysis from the previous section.)
% Imagine this claim is false.
% What else could explain why such children fail A-tasks?





The most sustained attempt to show that A-tasks involve demands extraneous to ascribing beliefs is due to Leslie and his collaborators.%
\footnote{%
You can find this view in several papers, including any of \citet{Leslie:1998nq,leslie:2000_theory,Leslie:2005ti}.
For the application of these ideas to \gls{the Mindreading Puzzle}, see Scott, Baillargeon, Song and Leslie (\citeyear{scott:2010_attributing}), \citet{Scott:2010oh}, \citet{Baillargeon:gx} or \citet[§1-V]{baillargeon:2015_psychological}.
I am ignoring some potentially confusing differences between these authors.
For instance, Scott, Baillargeon, Song and Leslie (\citeyear{scott:2010_attributing}, pp.~390–1) and \citet[p.~1176]{Scott:2010oh} write as if response selection and inhibition are distinct processes whereas \citet[p.~50]{Leslie:2005ti} clearly regard inhibition as a mechanism by which selection is achieved.
My aim here is to reconstruct the most defensible, best supported version of their view.
But their work is so full of ideas that if you read these papers yourself you might end up with a different interpretation of their view.
}
They start from the observation that
humans often exhibit a tendency to suppose that others believe what they do.
Unless this tendency is inhibited, they are unable to make use of information about false beliefs.
And inhibiting this tendency requires mental effort.
This means that, even when you as an adult are confronted with a scenario in which it should be obvious that someone has a false belief, you might sometimes act as if they had a true belief \citep[p.~194]{Apperly:2009cc}.
%This is probably one reason why adults sometimes fail to take into account others’ false beliefs even when it should be obvious that they have false beliefs \citep[p.~194]{Apperly:2009cc}.

Extending these ideas, \citet{Leslie:2005ti} offer a hypothesis about why many children under four or five years of age fail false belief tasks.
When a child observes a scenario in which someone comes to have a false belief, there is a representation of the content of this false belief in this child; perhaps the content is \emph{Maxi’s chocolate is in the blue box}.
But, in addition, there is also a representation of the content of a corresponding true belief; perhaps this content is \emph{Maxi’s chocolate is in the green box}.
When asked to say where Maxi thinks his chocolate is, the child has to select one of these contents as the content of Maxi’s belief.
What determines which content the child will select?
According to \citet[p.~49]{Leslie:2005ti} this depends on which content has more of a quality they label ‘salience’.
Their hypothesis is that the content of the true belief initially has greater salience.
According to Leslie et al, this is why younger children and adults under pressure tend to act as if they were unaware of the possibility of  false belief.
Although they are representing the content of another’s actual false belief, they end up ascribing her a true belief because the content of this true belief is more salient.%
\footnote{%
You might wonder whether the selection operation that Leslie et al focus on is really extraneous to ascribing beliefs.
\citet{doherty:1999_selecting} pursues this question.
}

If this is correct, what is involved in ascribing a false belief and passing a false belief task?
In order to ascribe a false belief, it is necessary to inhibit the true belief content.
This reduces its salience, which means that the false belief content is selected when the child (or adult) ascribes a belief.
So why do children under four or five years of age typically fail false belief tasks?
It is because they lack the inhibitory control necessary to ascribe a false belief.

One consequence of Leslie et al’s view is that whether children can pass false belief tasks should be related to how well they can inhibit tendencies to act.
For example, suppose we measured how well children can follow an instruction to avoid peeking at a present or how well they can inhibit a tendency to follow a verbal instruction.
Are children who are better at exercising inhibitory control in these cases also better at ascribing false beliefs?
It turns out that they are \citep{carlson:2004_individual,devine:2014_relations}, just as Leslie et al’s view requires.

The evidence is not entirely consistent with Leslie et al’s view, however.
One difficulty is that inhibitory control appears to be only one among several factors which explain whether children pass or fail A-tasks \citep{devine:2014_relations}.
Further, although factors like inhibitory control can vary between cultures in the sense that two children of the same age from different cultures tend to differ in how well they can exercise inhibitory control, these differences are not reflected in the children’s performance on false belief \citep{sabbagh:2006_development}.
Conversely, deaf children with hearing parents tend to pass A-tasks years later than hearing children (and than deaf children with deaf parents), and this delay cannot be explained by a corresponding deficit in inhibitory control \citep{devilliers:2012_deception,schick:2007_language}.%
\footnote{%
These point are often neglected.
For example, \citet{baillargeon:2015_psychological} assert, incorrectly, that ‘limited executive-function skills … explain young children’s failure at these tasks [i.e.\ A-tasks].’
}

A related problem for Leslie et al’s view arises from studies of the consequences of varying whether a false belief task is likely to require selecting one of several possible contents of belief.
In some A-tasks,  children are given a scenario in which they know where a protagonist falsely believes an object to be, and in which the children themselves do not know where the object is \citep[for example,][]{wellman:1990_simple}. %this isn’t the best example; find a better one?
To illustrate, Wimmer and Perner’s false belief task with Maxi and his chocolate introduced in \cref{sec:all-about-maxi} might be modified so that after Maxi leaves his chocolate is moved but the child does not know where to.
On Leslie et al’s view, it seems that the need for selection should not arise because the child is in no position to represent the content of a true belief corresponding to the false belief.
Despite this, children’s performance on these ‘selection-less’ false belief tasks is not relevantly different from their performance on any other A-task \citep[p.~704]{wellman:2001_theory}.
% – fine but mentioned just below
% Conversely,
% an attempt to identify why performing false belief tasks places demands on selection and inhibitory control found evidence that such demands are not specifically related to tracking others’ mental states; instead they may arise from the need to follow a narrative and answer a question about it, irrespective of whether the narrative concerns mental states
% \citep{bull:2008_role}.
%‘Stories tasks make high demands on EFs, irrespective of their mental-state content’ \citep[p.~670]{bull:2008_role}

A more subtle difficulty for Leslie et al’s view concerns the role that inhibition plays in ascribing false beliefs.
On their view, inhibition is required every time someone ascribes a false belief (without inhibition, the content of the false belief is not selected).
But an attempt to directly investigate demands on inhibitory control during belief ascription found no evidence that reasoning about belief specifically required inhibitory control:
instead demands on inhibitory control may arise from the need to follow a narrative and answer a question about it, irrespective of whether the narrative concerns mental states \citep{bull:2008_role}.
This suggests that sometimes—perhaps even often—inhibitory control is not required to ascribe beliefs.
If that is right, the reason why children do not pass some false belief tasks until they have a certain level of inhibitory control may be that inhibitory control (or something related to it) is necessary for acquiring an ability to think about false beliefs \citep{benson:2013_individual,devine:2014_relations}.
This suggests that Leslie et al’s hypothesis may need refining.
I shall assume, for the sake of argument at least, that the view can be refined to accommodate this apparently conflicting evidence.%
\footnote{%
One way of refining Leslie et al’s view might be to drop the claim that inhibition is required for selecting between two contents (the content of the false belief and the content a corresponding true belief would have).
Of course, this would then leave open the question of what selection (assuming it is required at all) demands that typical children do not possess until around four or five years of age.
}
% $question: can Leslie et al’s view be refined to accommodation the evidence which appears to conflict with it?

There is a more pressing issue.
So far I have presented Leslie et al’s view as if it were an explanation of why children tend to fail false belief tasks until they are around four or five years of age.
This is also how Leslie et al presented their view before Onishi and Baillargeon’s groundbreaking discoveries about infants’ abilities to track false beliefs (these discoveries were introduced in \cref{sec:mindreading-puzzle}).
But, thanks to these discoveries, we now know that a view which explains why children tend to fail false belief tasks until they are around four or five years of age would explain too much.
What we need to explain is why they fail all and only those false belief tasks which are A-tasks until around this age.
How can Leslie et al’s view be extended to explain this?

Their key idea is that selection is required on A-tasks but not on other false belief tasks \citep{scott:2010_attributing,Baillargeon:gx,Scott:2010oh}.%
\footnote{%
One complication in interpreting these authors’ views is that they appear to talk about different kinds of selection without explicitly distinguishing them or explaining their relation.
There is selection between different contents beliefs can have (this is the focus of earlier papers by Leslie et al) and there is selection between different possible responses to a question or communicative prompt.
Here I focus on selection between contents.
The idea that children who can track beliefs fail some A-tasks because they have difficulty selecting among possible responses to a question relates to Carruthers’ view; this will be discussed later, in \cref{sec:too-much-mindreading}.
}
What does this mean?
Recall that on A-tasks, subjects are required to ascribe a false belief and this is held to involve selecting between the false belief content (for example, \emph{Maxi’s chocolate is in the blue box}) and the initially more salient content of a corresponding true belief (for example, \emph{Maxi’s chocolate is in the green box}).
On non-A-tasks, such selection is not required.
Or so these authors claim.
But why accept this?

In principle, there are two possible reasons why selection would  not be required on non-A-tasks.
One is that non-A-tasks do not involve ascribing beliefs at all.
If this were right, infants’ performance on non-A-tasks would be merely a consequence of the fact that they represent the content of a false belief as well as the content of a corresponding true belief.
This might explain the results of false belief tasks in which infants merely distinguish between situations involving true and false beliefs \citep[for example,][]{kovacs_social_2010,southgate:2014_belief-based}.
But this cannot explain cases in which infants appear to have specific expectations concerning how someone with a false belief will act \citep[for example,][]{Southgate:2007js,Knudsen:2011fk}.
For this reason I take the idea that selection is not involved in non-A-tasks because they do not involve belief ascription to be insufficient by itself.

Another possible reason why selection is not involved (or does not require inhibition) in non-A-tasks is that, in these tasks, infants only represent the content of a false belief and do not also represent the content of a corresponding true belief \citep{scott:2010_attributing,Baillargeon:gx,Scott:2010oh}.
This idea coheres with a more general idea about selection processes: they are needed only because multiple potential targets are represented or attended to \citep{buetti:2014_flanker}.
% compare \citep{buetti:2014_flanker}: what creates a problem of response selection is that attention is drawn to several things.
% Alternatively, selection might in principle not require inhibition in non-A-tasks because although infants do represent the content of the true belief, this representation lacks salience.  %no: notational variant
The challenge here is to understand why non-A-tasks in particular should trigger a representation of the content of the false belief without also triggering a representation of the content of a corresponding true belief.
What is it about non-A-tasks that could explain this?
Our attempt at task analysis (in \cref{sec:task-analysis}) yielded as a working hypothesis the view that A-tasks differ from non-A-tasks in that A-tasks involve either a declarative expression about belief or belief-based action or emotion or else a response to a communicative prompt.
It is unclear how such differences could be responsible for whether those performing the task represent only the content of a false belief or also represent the content of a true belief.
This is not to say that it might be possible to develop Leslie et al’s idea further.
But to offer it as an explanation for why some children who can track beliefs nevertheless fail A-tasks amounts to invoking something more deeply mysterious than the thing to be explained ever was.


% This is initially confusing insofar as we have just seen that response selection is necessary for ascribing a false belief (or so Leslie et al claim).
% Is the idea that infants can pass non-A-tasks because these do not require ascribing beliefs?
% Not quite.

% changed my mind about this.
% The processing load account is opposed to alternatives involving implicit knowledge (\citealp[p.~74]{Leslie:2005ti}; \citealp[pp.~24–7]{he:2012_2}), but really these accounts are offering compatible accounts to the Mindreading Puzzle.
% (Essentially, Leslie et al are offering a view about the processes in virtue of which altercentric interference might occur.)


% Let us step back from the details to get a sense of the difficulty of our current approach to solving the Mindreading Puzzle.
% We are currently working on the assumption that infants’ performance on non-A-tasks is underpinned by a model of minds and actions incorporating beliefs.
% (This is the second claim in the Mindreading Puzzle, which we considered in \cref{sec:can-we-reject}.)
% Suppose infants are given a false belief task involving anticipatory looking, a task like \citeauthor{Clements:1994cw}’s task discussed in \cref{sec:mindreading-puzzle}.
% Infants’ anticipatory looking is supposed to be a consequence of their ascribing beliefs and making predictions about actions on the basis of these ascriptions.
% How does this anticipatory looking task differ from an A-task in which the measure is not anticipatory looking but verbal prediction of the action?
% Given our current assumptions, both tasks involve ascribing beliefs and making predictions on the basis of these ascriptions.
% The difference between the tasks is just that in one case the prediction results in looking whereas in the other case it results in talking.
% How could this make a difference to the inhibitory control required?


% One challenge is that, on the face of it at least, the extent to which inhibitory control is required should not vary dramatically between A-tasks and carefully matched non-A-tasks.
% To see why not,
% consider again Clements and Perner’s contrast between anticipatory looking and verbal predictions.
% If subjects’ anticipatory looking reflects a prediction based on a representation of a false belief, then the subjects must have inhibited any tendency to make use of their own knowledge in deriving a prediction (and they must have made use of their representation of the false belief in deriving the prediction).
% But Leslie et al’s view requires that inhibitory control beyond what two- and three-year-olds are capable of is needed only when a prediction results in a declarative expression and not when it results in anticipatory looking.
% Why suppose this?
% Unfortunately neither Leslie nor others associated with this view offer neither theoretical motivation nor evidence for this assumption.

If you bought Leslie et al’s claims about why some children who can track beliefs nevertheless fail A-tasks, I think there’s an excellent chance you can get your money back.
Conversely, being able to turn these claims into a genuine explanation should make you famous.


\section{Too Much Mindreading?}
\label{sec:too-much-mindreading}

Our current question is whether we can solve \gls{the Mindreading Puzzle} by showing that A-tasks impose extraneous demands on children.
We have just been exploring the idea that A-tasks involve not just ascribing beliefs but also an extraneous operation of selecting between the possible contents of a belief.
As we saw, this idea may eventually be important to understanding the Mindreading Puzzle.
But even if it can overcome the various objections, it almost certainly does not identify extraneous demands that are present in all and only those false belief tasks which are A-tasks.
Can we do better?


Let us step back from the details to get a sense of the difficulty of our current approach to solving the Mindreading Puzzle.
Compare Onishi and Baillargeon’s \gls{violation-of-expectation} false belief task with an A-task which, instead of requiring children to predict actions based on false beliefs, instead involves children watching someone perform an action based on a false belief and then requires them to explain this action \citep[for example,][]{Wimmer:1998kx}.
Leslie et al’s view requires that inhibitory control beyond what two- and three-year-olds are capable of is needed only for the A-task.
But on our current assumptions, performing either task—the non-A-task or the A-task—involves using a model of minds and actions incorporating beliefs to identify a causal relation between a false belief and an observed action.
Why might the difficulty involved in  identifying such a relation vary dramatically depending on factors such as whether it eventually results in looking longer or in giving a verbal response?


% While there can be little doubt that inhibitory control is important for developing mindreading abilities, Leslie et al’s proposal that inhibitory control explains why many children who can track false beliefs nevertheless fail A-task faces substantial challenges.

\citet{carruthers:2013_mindreading,carruthers:2015_two} has suggested an answer.
He proposes that any false belief task which involves a verbal response involves performing three mindreading tasks simultaneously or in close succession.
Indeed, he suggests that this is true of all and only A-tasks.%
\footnote{%
\citet[§1-V.1]{baillargeon:2015_psychological} make a similar suggestion.
They appear to be endorsing a disjunctive view, on which difficulties of selection and inhibition explains why some children fail some A-tasks (see \cref{sec:inhibition-and-selection}) whereas something like Carruthers’ view explains performance on other A-tasks.
\citet[§1.2]{carruthers:2015_two} describes his own view as ‘consistent with, but somewhat broader than’ that of \citet{Baillargeon:gx}, whereas I am presenting his view as an alternative to \citeauthor{Baillargeon:gx}’s.
This is because \citeauthor{Baillargeon:gx}’s and Carruthers’ core ideas are importantly different, even if potentially complementary.
(Carruthers also appears not to have considered \citeauthor{Baillargeon:gx}’s views in detail, for he misdescribes them as concerned with  ‘language-involving false belief-tasks’.)
}
Mindreading is involved in ascribing belief, in processing a communicative prompt and in producing speech or some other communicative response.
So what prevents some children who can ascribe beliefs from manifesting their competence on A-tasks is that they cannot (yet) ascribe beliefs while also doing the mindreading needed to understand and respond to a communicative prompt.
Or so Carruthers suggests.

% One potential difficulty for this view is a fact already mentioned.
% Different A-tasks differ with respect to how much language they require.
% While not decisive, a meta-analysis of the relation between false belief and language found no significant difference in the relation between language and belief tracking for different kinds of A-task with differing linguistic demands \citep[p.~637]{milligan:2007_language}.
% This is the opposite of what Carruthers says his view predicts.



An immediate problem with Carruthers’ suggestion as I have just presented it is that it generates incorrect predictions.
There are many tasks which are almost identical to an A-task but involve something other than belief;
these include tasks about desire, perception, pretence and speech.
If Carruthers is right that A-tasks involve performing three mindreading tasks simultaneously, then these related tasks likewise involve performing three mindreading tasks.
And if Carruthers is right that children who fail A-tasks despite passing other false belief tasks fail A-tasks because these involve performing three mindreading tasks simultaneously,
then we would expect that children should fail other tasks which involve performing three mindreading tasks simultaneously.
But they do not.
As already mentioned, children who fail A-tasks succeed on almost identical tasks which concern not belief but another mental or semantic state (see \cref{sec:truly-contr-resp}).

Since this is not merely a problem for Carruthers but a major obstacle to solving \gls{the Mindreading Puzzle}, let’s pause to consider some of the tasks.
\citet{Gopnik:1988pa} created a false belief task which is about children’s own beliefs rather than someone else’s.
They showed children a smarties box and asked them what they thought was in it; as expected, most asserted that it contains smarties.
They then opened the box to reveal that it actually contained pencils.
Children were then asked what they first thought was in the box.
Here is an illustrative protocol from \citet[p.~195]{Astington:1988lg} illustrating how a typical three-year-old responds:
\begin{quote}
{\sc Experimenter}: Look!  Here’s a box
\\ {\sc Subject}: Smarties!
\\ E:  Let’s look inside.
\\ S:  Okay.
\\ E:  Let’s open it and look inside.
\\ S:  Oh … holy moly … pencils!
\\ E:  Now I’m going to put them back and close it up again.  [does so]
\\ E:  Now … when you first saw the box, before we opened it, what did you think was inside it?
\\ S:  Pencils.
\\ E:  Nicky [friend of the subject] hasn’t seen inside this box.  When Nicky comes in and sees it ..  When Nicky sees the box, what will he think is inside it?
\\ S:  Pencils.
\end{quote}
This is an A-task: children’s performance is correlated with their performance on other variations of Wimmer and Perner’s false belief tasks \citep{Wellman:2001lz}.
\citet{Gopnik:1991db} developed a variety of tasks which are structurally identical but concern changes not in what a child believes  but in what she pretends, desires, imagines or sees.
Strikingly, three-year-olds who misreport changes in their own beliefs can nevertheless answer similar questions about changes in their own pretence, desires and perceptions.
Likewise, \citet{riggs:1995_what} contrasted children’s ability to report what people said with their ability to report what people think.
When reporting their own past speech,
children who failed to correctly report what they thought also misreported what they said (see also \citealp{Wimmer:1991zt}).
But when reporting another’s speech,
\citeauthor{riggs:1995_what} found that children who could correctly answer the question ‘What did Iain say?’ failed to answer the question ‘What did Iain think?’
%(Strikingly, having Iain tell the children what he thinks does not improve children’s performance.)

How is all of this relevant to Carruthers’ suggestion about why some children who can track beliefs nevertheless fail A-tasks?
If Carruthers is right that A-tasks require performing three mindreading tasks simultaneously,
then the counterpart tasks involving not belief but desire, pretence, perception and speech also involve performing three mindreading tasks simultaneously.
So children who fail A-tasks can readily perform multiple simultaneous mindreading tasks.
The mere fact (if it is a fact) that A-tasks involve multiple simultaneous mindreading tasks therefore cannot explain why children’s performance on these tasks is related to their age in the way it is.
% Contrast the first claim of the Mindreading Puzzle: children fail A-tasks because in performing these tasks they rely on a model of minds and actions not incorporating beliefs.
% If it is to solve the Mindreading Puzzle, Carruthers’ suggestion must provide an alternative to this claim that does at least as well in explaining the patterns in children’s performance.
Minimally, Carruthers’ suggestion needs supplementing with an account of why multiple mindreading tasks involving beliefs are different from other multiple mindreading tasks.%
\footnote{%
There is a temptation to suppose that belief is special because only representing  beliefs inconsistent with your own current beliefs involves switching perspectives.
But switching perspectives is no less involved in representing other mental states like desires that are incompatible with your own  \citep[compare][]{rakoczy:2007_desire}.
}

Suppose this could be done.
Would  we have solved the Mindreading Puzzle?
Not yet.
Carruthers’ suggestion simplifies things by assuming that A-tasks involve both a communicative prompt and a communicative response.
In fact, some A-tasks involve just one of these two features.
For instance, there are tasks in which you have to respond by selecting an object for yourself \citep[for example,][]{krachun:2010_new}, or by pointing to a picture \citep[for example,][]{custer:1996ht}.%
\footnote{%
It would be a mistake to take for granted that A-tasks which rely more heavily on language or communication are somehow more demanding.
Given the currently available evidence, the opposite assumption is equally plausible (see \citealp[p.~329]{hollebrandse:2012_childrens} and \citealp[p.~757]{moeller:2006_relations}).
% — Does Low (2010) make the opposite claim? \citep[p.~329]{hollebrandse:2012_childrens} also make the opposite claim: ‘language supports explicit reasoning about beliefs’
% (‘Low argues that cognitive flexibility in explicit FB reasoning may be more effective if it is carried out in linguistic formulations, and the mediation of language may help children understand the representational nature of different perspectives and behavioural rules.’)
% — Hearing children did worse on a low-verbal false belief task than a standard one.  ‘It appears that the stripping away of language demands made this task confusing for the hearing children’ \citep[p.~757]{moeller:2006_relations}.
}
Conversely, consider spontaneous comments on beliefs.
When children’s talk about minds is recorded, spontaneously commenting on beliefs is closely related to being able to pass A-tasks \citep{Bartsch:1995vl,ruffman:2002_relation}.
So although there are no A-tasks which clearly do not involve a communicative prompt (as far as I know),
it seems likely that a sufficiently patient experimenter could devise one.
So we must modify Carruthers’ view slightly.
His claim must be not that all A-tasks involve three mindreading tasks simultaneously, but that they involve at least two.

This seems like a minor modification but it creates a further problem.
Take a look at \vref{table:a-tasks-communication}.
This table shows that some false belief tasks which are not A-tasks require subjects to respond to a communicative prompt (see row 2), and that at least one non-A-task involves subjects producing a communicative response (see row 4).
Apparently, then, children who fail A-tasks are nevertheless capable of performing two mindreading tasks where one involves belief.

\begin{table}

% add space between rows
\renewcommand{\arraystretch}{1.5}

\begin{center}
\footnotesize	%shrink for better spacing
\begin{tabu} to \linewidth {X[l] X[8,l] X[4,cm] X[4,cm] X[4,cm]}

\toprule
& Task/paradigm  & Involves communicative prompt? & Requires communicative response? & Is an A-task?  \\
\midrule
%
1 & non- and low-verbal tasks \citep[for example,][]{custer:1996ht,low:2010_preschoolers,krachun:2010_new}  & yes & no & yes
\\
2 & \citet{southgate:2010fb}; \citet{buttelmann:2015_what}; \citet{Carpenter:2002gc} & yes & no & no
\\
3 & spontaneous talk about beliefs \citep[for example,][]{Bartsch:1995vl,ruffman:2002_relation} & no & yes & yes-ish*
\\
4 & \citet{Knudsen:2011fk}   & no & yes & no
\\
%
\bottomrule
%
\end{tabu}
\caption{Whether a task involves a communicative prompt or requires a communicative response is not straightforwardly related to whether it is an A-task.
Note that some of the findings mentioned here are subject to unsuccessful replication attempts (see \cref{sec:replication-challenge}) or have yet to be replicated. 
(*Children’s spontaneous talk about belief is related to their performance on A-tasks.)}
\label{table:a-tasks-communication}
\end{center}	%careful -- position of this affects distance between table and caption(!)

%reset to default value
\renewcommand{\arraystretch}{1}

\normalsize

\end{table}





Here, then, is a second objection to Carruthers’ suggestion.
He claims that children fail A-tasks because these tasks involve three mindreading tasks.
This is probably false because some A-tasks do not involve three mindreading tasks (at least not for the reasons Carruthers identifies).
We might then revise the claim to say that children fail  A-tasks because these tasks involve two or three mindreading tasks.
But the revised claim conflicts with evidence that children pass some tasks which (according to Carruthers) involve two mindreading tasks.

We have examined two attempts to solve \gls{the Mindreading Puzzle} by identifying extraneous demands that distinguish A-tasks from false belief tasks.
In each case the aim was to provide an alternative to the claim that children fail A-tasks because in performing them they rely on a model of minds and actions not incorporating beliefs.
The alternatives are that children fail A-tasks because they require extraneous demands in the form of selecting between multiple possible contents of a belief, or because they require  performing multiple mindreading tasks simultaneously.
As we have seen, both alternatives face  objections.
While these objections are not decisive, they do show that the Mindreading Puzzle is more interesting, and harder to solve, than is commonly appreciated.
These objections also provide a reason to consider alternative ways of solving the Mindreading Puzzle.


The objections we have considered generalise beyond the specific proposals we have been considering.
The broader idea behind Leslie et al’s and Carruthers’ approaches is that two- and three-year-olds who can track false beliefs nevertheless fail A-tasks because of performance factors \citep[compare][p.~74]{Leslie:2005ti}.
As we have seen, a wide variety of findings indicate that two- and three-year-olds’ responses on A-tasks are unlikely to be a consequence of performance limits.
And correlations  between changes in performance on A-tasks and improvements in things typically considered to be performance factors, such as capacities for selection and inhibitory control, are most likely a consequence of the fact that these capacities are involved in acquiring a model of minds and actions which incorporates the possibility of false belief (as \citealp{devine:2014_relations}  suggest).

% it seems that changes in performance factors play only a relatively modest role in explaining the change children undergo when they transition from failing A-tasks to passing them.
% Furthermore, manipulations that ought to affect the impact of performance factors do not transform the relation between age and performance on A-tasks.
% For example, instead of having children answer a question about false beliefs straight way you can let them stop and think before answering \citep{Wimmer:1983dz}.
% You can even attempt to make A-tasks easier for children to answer a question about what someone thinks by having her say aloud what she thinks \citep{riggs:1995_what}.
% *not sure where this is going

% So far we have only one plausible candidate answer to this question.
% Children fail A-tasks because they do not use a model of minds and actions which incorporates beliefs when they are performing A-tasks.





% \citet[p.~611]{low:2010_preschoolers}: ‘Performance on those low-verbal false belief tasks was, however, strongly correlated with verbal answers provided for traditional false-belief tasks (i.e., change in location, unexpected contents, and deceptive appearance).’

% — Does Low (2010) make the opposite claim? \citep[p.~329]{hollebrandse:2012_childrens} also make the opposite claim: ‘language supports explicit reasoning about beliefs’
% (‘Low argues that cognitive flexibility in explicit FB reasoning may be more effective if it is carried out in linguistic formulations, and the mediation of language may help children understand the representational nature of different perspectives and behavioural rules.’)

% — Hearing children did worse on a low-verbal false belief task than a standard one.  ‘It appears that the stripping away of language demands made this task confusing for the hearing children’ \citep[p.~757]{moeller:2006_relations}.

% — problem from \citet{Ruffman:2001ng} that younger failers are certain about their (incorrect) answers whereas younger passers are uncertain about their (correct) answers.  ‘In contrast, for older children who passed the verbal measure there was no difference between betting in the false and true belief tasks.” (218)’

% — problem from \citet{pyers:2009_language}: adults with low linguistic competence fail low-verbal false belief tasks


% Suppose I believe, falsely, that the cocaine is in Urma’s bag and that you know I believe this.
% Does it follow that if asked what I believe, you will report what you know?
% \citet[pp.~113–4]{Baillargeon:gx} note two factors which may prevent your report from reflecting your knowledge.
% One factor is a tendency to act based on your own knowledge; this tendency must be inhibited if your report is to reflect what you know of my belief.
% The other factor is that you must be able to make use of your knowledge in talking about what I believe.
% To illustrate, you may know a lot about formal logic and yet in the first tense moments of an exam find yourself unable to draw on that knowledge.
% Similarly, you might know that I believe the cocaine is in Urma’s bag but be unable to draw on this knowledge when you want to report what I believe.
% \citeauthor{Baillargeon:gx} suggest that many children who can track false beliefs nevertheless  fail A-tasks because of one or both of these factors.

% Is this suggestion potentially explanatory?
% There are two problems.
% The first is that both factors seem to be required both by A-tasks and by other false belief tasks.
% Consider again Clements and Perner’s contrast between anticipatory looking and verbal predictions.
% If subjects’ anticipatory looking reflects a prediction based
% on a representation of a false belief, then the subjects must have inhibited any tendency to make use of their own knowledge in deriving a prediction and they must have made use of their representation of the false belief in deriving the prediction.
% We are missing an account of why a prediction that is verbally expressed

% Consider the first factor.
% To act in accordance with facts about what others believe, any conflicting tendencies to act based on what you know must be suppressed.
% Sustained attempts to show that children fail A-tasks because of this factor have failed (see \citealp[p.~667]{Wellman:2001lz} on real presence).


% The suggestion requires that at least one of the two factors is distinctive of A-tasks; that is, it is present in A-tasks and absent in false belief tasks which are not A-tasks.
% Consider again Clements and Perner’s contrast between anticipatory looking and verbal predictions.
% Given that anticipatory looking is caused by a representation of another’s false belief (rather than, say, by altercentric interference),
% it seems that subjects




\section{What Now?}
\label{sec:what-now}
Our aim is to understand how humans first come to know facts about mental states and beliefs in particular.
The obstacle to understanding this is what I have been calling \gls{the Mindreading Puzzle} (which was introduced in \cref{sec:mindreading-puzzle}).
This puzzle arises because of some apparently conflicting evidence.

There is strong evidence that infants, from sometime in their first or second year of life onwards, can use a model of minds and actions incorporating beliefs in predicting actions and understanding others (as we saw in \cref{sec:can-we-reject}).
Taken in isolation, this evidence would support a straightforwardly nativist answer to our question about the developmental emergence of knowledge of minds.
Humans are born with what \citet{kovacs_social_2010} call the ‘social sense’.
Their coming to know what someone believes is essentially a matter of directing their social sense at the person.

But there is also strong evidence that typically developing children can first use a model of minds and actions incorporating beliefs in order to predict actions and understand others sometime around their fourth or fifth birthday.
% — *explain constructivist?
Acquiring this model of minds and actions appears to take them many months, and to depend on opportunities for discussion about minds and actions as well as on rich forms of social interactions (as we saw in \cref{sec:all-about-maxi}).
Taken in isolation, this evidence would support a straightforwardly constructivist account of how humans first come to know about minds.
Our understanding of minds is like our understanding of economic value or of astronomy.
We acquire it through interactions, particularly interactions with people who, like human adults, are already relative experts, and perhaps also through reflection.

Faced with this apparently conflicting evidence, the obvious strategy for reconciliation is to attempt to discover what is wrong with one or another of these positions.
That is what we have been trying to do (and it is what most researchers have been trying to do, too).
We have considered whether infants’ performance on some false belief tasks, the non-A-tasks, might be explained without supposing that they are relying on a model of minds and actions incorporating beliefs (see \cref{sec:can-we-reject}).
And we explored whether the patterns in children’s performance on some false belief tasks, A-tasks, might be explained other than by supposing that in performing these tasks they are relying on a model of minds and actions not incorporating beliefs (see \crefrange{sec:three-levels-of-analysis}{sec:too-much-mindreading}).
We have seen that both approaches to reconciling the apparently conflicting evidence and solving the Mindreading Puzzle face major obstacles.
My guess is that neither approach will succeed.
Even if you don’t agree, the obstacles are substantial enough to justify investigating an alternative strategy.

%%% Local Variables:
%%% TeX-master: "master"
%%% End:


