\maketitle


\noindent
\textit{I had to shorten this review slightly for publication; this is the longer version.}

\ %blank line


\noindent
Sometimes when trying to sprint through an airport there is one sizeable individual blocking your way, while at other times it is several people ambling side-by-side who hold you up.
If the several’s blocking your way is not a matter of each individually blocking your way, then they are \emph{collectively} blocking your way.
As this illustrates, some properties permit both singular and plural, collective predication.
Likewise, commitments can be predicated both of individuals and of pluralities of individuals.
Andrea can be committed to making a pizza (say);
but it is also possible, writes Gilbert, that Jessica and Heinrich can be collectively committed.
To be collectively committed—to have a \emph{joint commitment}—is not a matter of each individually having a \emph{personal commitment} like Andrea’s any more than several amblers’ collectively blocking your way is a matter of them each individually blocking your way.
Thus a joint commitment is simply a commitment, just as a collective blocking is simply a blocking.
%Joint commitments differ from personal commitment in whether there are several or just one parties to the commitment.

% A nearby crocodile is a potentially life-threatening danger to humans, and so is an approaching colony of army ants.
% Of course, the ants are not individually life-threateningly dangerous; they have this property collectively but not distributively.
% Likewise, argues Gilbert, 

How can people come to have joint commitments?
One possibility is that they make an agreement of some kind.
But it is also possible, argues Gilbert, to have joint commitments without agreements.
If several people each openly (but not necessarily verbally) express readiness to commit, together with the others, to something, and this is common knowledge among them, then ‘the joint commitment is in place’ (p.~311).

This book,
which handily brings together a thematically coherent and important selection of journal articles, book chapters and two previously unpublished essays,
hinges on three striking claims about joint commitments.
The first is that all joint commitments have a certain form. 
They are all commitments to emulate, as far as possible, a single body which does something (e.g.~pp.~311, 348, 400).
Joint commitments can differ only in what the thing is. 
Possibilities include believing a proposition, intending to $\phi$, upholding a decision to $\phi$, accepting a certain fiat, feeling guilty over a particular action, and attending to a certain object.
Thus Andrea and Heinrich can have a joint commitment to emulate a single body which washes the dishes, but they cannot have a joint commitment simply to wash the dishes.
(We aren’t told why not.%
% \footnote{%
% Unless perhaps in a footnote I skipped.
% }
)

Gilbert’s second striking claim about joint commitments is that they are a source of directed obligations.
Those who have a joint commitment thereby each have an obligation to each of the parties to that commitment, and they each do so in virtue of these being the people who created that joint commitment.
This obligation is to ‘do whatever fits best with whichever of the possible conforming combinations of actions the others are doing their part in’ (p.~402).

The third striking claim is that joint commitment underlies a wide variety of social phenomena including acting together, having a collective belief, sharing values, having a social convention, feeling collective guilt, promising, jointly attending to an object, 
constituting a polity,
acting patriotically,
and having a political obligation.
Gilbert’s analyses of these social phenomena are ‘regimented’ (fn.~45, p.~218) in such a way that most can be specified in \vref{table:analyses}.%
\footnote{%
This table ignores some variations between accounts given in different chapters.
It also omits the clause ‘with respect to themselves’ in Gilbert’s analysis of social convention (because I am not fully confident I understand it). %the problem is how it fits with emulating the *single* body
}
In addition, acting together is having a collective intention (\cref{table:analyses}, line 1) and acting accordingly in the light of it (pp.~34, 70).%
\footnote{%
Gilbert explains that she uses ‘collective goal’ and ‘shared intention’ interchangeably (p.~12).
}
Promises are analysed as collective agreements (\cref{table:analyses}, line 5) that the promisor is to do something (p.~317–8).
To constitute a polity is to be jointly committed to emulate a single body that upholds a particular set of political institutions (p.~364).
Acting patriotically is being party to the joint commitments that constitute a country and acting in the light of those commitments in order to conform to them (p.~366).
And political obligations arise in virtue of joint commitments that constitute a polity (pp.~402–3).
Note that in each case the social phenomenon involves creating, in imagination at least, ‘a third thing’ where ‘each of us is one of the parts’ (p.~269).
So to collectively feel guilt is not to feel guilt but to emulate something distinct from us, of which we are proper parts, which feels guilt.




\begin{table}

% add space between rows
\renewcommand{\arraystretch}{1.5}

\begin{center}
%shrink for better spacing:
\footnotesize	
\begin{tabu} to \linewidth {X[m,l]X[m,l]X[-1,m,l]} 

\toprule
For two or more agents to collectively:  & … is for each to be jointly committed to emulate a single body that:  & see chapters
\\
\midrule
%
intend to $\phi$ & intends to $\phi$ & 1,4,5,10
\\
believe that $p$ & believes that $p$ & 6,7
\\
value item $I$ & believes that $I$ has a certain value & 8
\\
\vspace{0.5em}
have a social convention concerning regularity $R$ & accepts the fiat that $R$ is to be conformed to & 9
\\
make an agreement that $p$ & upholds a decision that $p$ & 9,13,18
\\
feel guilt over action $A$ & feels guilt over action  $A$ & 10
\\
attend to X & attends to X & 14
\\

%
\bottomrule
%
\end{tabu}
\caption{Gilbert’s analyses of seven social phenomena.}
\label{table:analyses}
\end{center}	%careful -- position of this affects distance between table and caption(!)

%reset to default value
\renewcommand{\arraystretch}{1}

\normalsize

\end{table}


A consequence of these three striking claims is that many social phenomena, both those involving just two individuals as well as those involving whole countries, create obligations which are neither legal nor moral.


Gilbert’s pioneering, far-reaching work provokes many questions.
One concerns the form all joint commitments have.
If, as Gilbert holds, joint commitments are all commitments to emulate a single body which does something, then the thing to which there is commitment involves nothing collective.
Joint commitment thus serves, for Gilbert, as a device which transforms ordinary, singular phenomena (intention, belief or whatever) into collective phenomena with added commitments.
To this extent Gilbert’s programme is reductionist: 
shared values, collective beliefs and the rest are reduced to joint commitments plus ordinary, individual values, beliefs and the rest.
Why does everything but joint commitment require reduction?%
\footnote{%
Gilbert herself suggests there is no reason to treat individual guilt or individual belief as primary in giving analyses of guilt (p.~234) or belief (pp.~163ff).
}
%And why take collective phenomena such as collective belief or collective intention to involve commitments which their non-collective counterparts do not?


Reduction comes at a cost.
Emulating as far as possible a single body that intends to wash up is not generally the most efficient way for several people to get the washing up done—Andrea and Heinrich had better exploit the fact that they are two than pretend to be an aggregate animal.
The ‘emulating a single body’ form also seems to exclude a shared intention to make out after washing up.
%our racing each other to the tree,
And if it doesn’t preclude shared intentions to tango outright, it has unfortunate stylistic consequences in implying that those with such intentions are jointly committed to emulate a single body that intends to tango.
As well as limiting Andrea’s and Heinrich’s social life in these respects, insistence on this form may conflict with Gilbert’s aim to ‘describe the phenomena to which everyday statements refer’ (p.~3).%
\footnote{%
Is this really Gilbert’s aim? 
She invokes neither semantic analysis nor experimental investigations but relies on ‘informal observation including self-observation’ (p.~24) and her ‘own sense of the matter’ (p.~358).
Since what ordinary people say in everyday life can be quite surprising to philosophers, and working out their views is certainly not straightforward (for an illustration, see \citet{nagel:2013_authentic,nagel:2013_lay,starmans:2012_folk,starmans:2013_taking}),
it may be more charitable to interpret Gilbert’s project as having other aims. %conceptual analysis?
}

Another question is whether the shared reduces to the collective.
If two actors collectively emulate a cantering horse, neither should cantor; but if they share the role (alternating between scenes), each should cantor.
Likewise, concerning blame, there is a distinction between our being collectively to blame (which does not entail our each being to blame, see Chapter 3) and our sharing blame (which does).
Gilbert analyses shared values as joint commitments to emulate a single body which believes that $I$ has a certain value.
But an optimal way for two or more individuals to emulate a single body which believes that $I$ has a certain value may sometimes involve neither individually acting in ways that conform with this belief (like two actors collectively emulating a cantering horse).
Our sharing a value may therefore mean each of us sometimes has an obligation to act not in conformity with this value.
Is this more plausibly an account of collective than shared value?
And given that joint commitment is a collective notion, do we not need, in addition, a notion of shared commitment?

There are also questions about Gilbert’s claim that certain directed obligations—specifically, obligations on the part of each person who has a joint commitment to each of the others who have it—exist in virtue of a joint commitment (p.~402).
Why accept this?%
\footnote{%
\citet[p.~361]{Roth:2004ki} asks a related question.
% ‘But why think that the intention is somehow attributable to me in
% particular, so that I have this participatory commitment? Perhaps you and I constitute a group, and it is committed to the walk, but this
% needn't entail that I myself am so committed.’
}
One consideration is that it is ‘hard to deny’ in particular cases (unless perhaps, like me, you are quite good at denying things).
Another consideration is a parallel with personal commitment:
%
\begin{quote}
  ‘just as—in the case of a personal commitment—you are in a position to berate yourself for failing to do what you committed yourself to do, all of those who are parties with you to a given \emph{joint} commitment are in a position to berate you for failing to act according to that joint commitment’ (p.~401).
\end{quote}
%
Isn’t this a more accurate parallel: parties to a joint commitment are in a position to jointly berate themselves for failing to act according to the joint commitment?
It doesn’t follow from this, of course, that any individual has the standing to berate, nor that any individual can be berated.
Aren’t shared commitments more plausibly a source of directed obligations than  joint commitments?


% \begin{quote}
%   ‘Philosophers tend to be modest about the relevance or effect of
%   their work in relation to science—the tendency nowadays is to
%   think of philosophy as informed by the sciences rather than the
%   other way round. Many social scientists, meanwhile, are keen to
%   avail themselves of the results of philosophical
%   investigations. Philosophical investigations in collective
%   epistemology, then, are likely to constitute an important
%   contribution to the social sciences.’ (p.~168)
% \end{quote}

Gilbert aims to make ‘an important contribution to the social sciences’ (p.~168; cf.~p.~277) and her ideas are influential.
Their implications for psychology are startling.
On Gilbert’s view, participating in any of the social phenomena she analyses involves 
expressing readiness to commit and therefore 
thinking thoughts about joint commitment plus other notions used in the analysis (p.~334).
Thus acting together requires thinking about intentions and about joint commitments, and jointly attending to an object requires thinking about joint attention and about joint commitments.
Consider two-year-old toddlers who cannot yet think about joint commitments and appear insensitive to the possibility of them \citep{Grafenhain:2010zl,hamann:2012_children}.
% \footnote{%
% See \citet{Grafenhain:2010zl}. 
% While many recent findings have shown that toddlers’ social cognition is surprisingly sophisticated, it would be a precocious toddler indeed that was sensitive to whether she had the standing to rebuke another party to a joint commitment.
% }
On Gilbert’s view, it is impossible for there to be mutual recognition between toddler and adult, and it is impossible to act together with a toddler, by, for instance, looking at a book together, sharing a smile, or walking together.
This is unexpected given evidence that toddlers appear to spontaneously initiate, and to repair, such joint activities \citep[e.g.][]{Warneken:2006qe,warneken:2013_young}.
%If parents and carers routinely talk about doing things together with two-year-old children—who, after all, can be fun to play with—then Gilbert’s account will fail relative to her one of own standards for it.
% Importantly, the same applies to Gilbert’s account of joint attention.
% On her account, our jointly attending to something is a matter of our being jointly committed to attend to it as a body (p.~337).
% This implies that to jointly attend, we must each have, and be able to use in expressing readiness, the concepts of joint attention and joint commitment.
Joint attention with the toddler is also impossible,
%On almost any view of infant cognitive development, 
contradicting hypotheses about joint attention playing a role in infant pointing and facilitating early stages in the acquisition of language.
% As Gilbert notes, philosophers tend to be cautious about asserting that their views have scientific implications.
% Perhaps with good reason?
Can we hold on to Gilbert’s core insights without making such bold conjectures?

Yes.
Drop the claim that acting together and the other social phenomena Gilbert analyses inherently involve commitment (which is based only on personal observation and reflection).
Recognise instead that for acting together, as for other collective phenomena probably including joint commitment itself, there is significant diversity—two or more agents’ actions may be collectively directed to a single outcome in virtue of various things, not all of which involve commitment.
Then, given that commitments are valuable because they add stability (p.~123), treat joint commitment as important not because it ‘underlies’  social phenomena (p.~400) but because it is an ingredient which can enhance them.
Doing this would require abandoning two premises which many philosophers working on collective intentionality require, namely that 
we have an adequate pre-theoretical grip on things like acting together,
and that there is just one form of acting together, of collective belief, or of shared value which philosophers are permitted to offer analyses of.
But recognising the diversity of collective phenomena would enable us to avoid opposing experimentation with personal observation while 
holding on to Gilbert’s core insights.

% ***Why propose and defend an analysis with such unexpectedly strong scientific implications?
% Many philosophers working on collective phenomena require the premise that there is just one form of acting together, of collective belief, or of shared value which philosophers may offer competing analyses of.
% Gilbert is frequently explicit that her arguments aim to show, at most, that her analyses capture one notion of collective belief or shared value: she must also recognise that fully capturing ‘everyday understandings’ probably requires considerably more diversity.




%closing paragraph:
Suppose it turned out that 
not everything collective involves commitment,
and shared, not joint, commitments are the ground of directed obligations, 
and not all joint commitments are commitments to emulate a single body.
Much of value would remain.
For no one has come close to Gilbert in presenting such a systematic view of such wide a range of social phenomena with such clarity and originality.
This is a collection of beautifully crafted, wonderfully readable essays full of questions, challenges and insights, and an indispensable source for a trailblazing philosopher’s view on collective intentionality. 



%%% Local Variables:
%%% TeX-master: "master"
%%% End: